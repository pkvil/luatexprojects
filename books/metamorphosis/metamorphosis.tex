%%
%% Franz Kafka - Metamorphosis
%%
%% Typeset with LuaTeX
%%
%% Source text: https://www.gutenberg.org/files/5200/5200-0.txt
%% Source fonts: 
%%

%%%%%%%%% F O N T S %%%%%%%%%%%%%%%%%%%%%%%%%%%%%%%%%%%%%%%%%%%%%%%%%%%%%%%%%%%%

% Use font loader from ConTeXt
\input luaotfload.sty

% Font names
\font\elevenrm "Cochineal-Roman":+onum; at 11pt\relax
\font\elevenit "Cochineal-Italic" at 11pt\relax

\font\fourteenrm "Cochineal-Roman":+onum; at 14pt\relax
\font\fourteenit "Cochineal-Italic" at 14pt\relax

% Semantic names for direct use
\def\normalsize{%
	\gdef\rm{\elevenrm}%
	\gdef\it{\elevenit}%
}

\def\titlesize{%
	\gdef\rm{\fourteenrm}%
	\gdef\it{\fourteenit}%
}

% Use italics for emphasis
\let\emph\it

% Start with normalsize roman
\normalsize\rm

%%%%%%%%% P A R A G R A P H   S E T T I N G S %%%%%%%%%%%%%%%%%%%%%%%%%%%%%%%%%%

% Default distance between lines
\baselineskip=3ex

% No extra space after periods.
\frenchspacing

% Vertical distance between paragraphs
\parskip=0pt

% Paragraph indentation
\parindent=1em

% Allow some looseness if needed
\emergencystretch=8pt

% \firstnoindent -- Don't indent next paragraph
\def\firstnoindent{\global\everypar={\wipeeverypar\setbox7=\lastbox}}
\def\wipeeverypar{\global\everypar={}}

% No penalties for widows and orphans, these are dealt with using \adjustpage
\widowpenalty=0
\clubpenalty=0

%%%%%%%%% C O N T E N T %%%%%%%%%%%%%%%%%%%%%%%%%%%%%%%%%%%%%%%%%%%%%%%%%%%%%%%


% T I T L E P A G E



\centerline{I}
\vskip3\baselineskip
\firstnoindent


One morning, when Gregor Samsa woke from troubled dreams, he found
himself transformed in his bed into a horrible vermin. He lay on his
armour-like back, and if he lifted his head a little he could see his
brown belly, slightly domed and divided by arches into stiff sections.
The bedding was hardly able to cover it and seemed ready to slide off
any moment. His many legs, pitifully thin compared with the size of the
rest of him, waved about helplessly as he looked.

“What’s happened to me?” he thought. It wasn’t a dream. His room, a
proper human room although a little too small, lay peacefully between
its four familiar walls. A collection of textile samples lay spread out
on the table—Samsa was a travelling salesman—and above it there hung a
picture that he had recently cut out of an illustrated magazine and
housed in a nice, gilded frame. It showed a lady fitted out with a fur
hat and fur boa who sat upright, raising a heavy fur muff that covered
the whole of her lower arm towards the viewer.

Gregor then turned to look out the window at the dull weather. Drops of
rain could be heard hitting the pane, which made him feel quite sad.
“How about if I sleep a little bit longer and forget all this
nonsense”, he thought, but that was something he was unable to do
because he was used to sleeping on his right, and in his present state
couldn’t get into that position. However hard he threw himself onto his
right, he always rolled back to where he was. He must have tried it a
hundred times, shut his eyes so that he wouldn’t have to look at the
floundering legs, and only stopped when he began to feel a mild, dull
pain there that he had never felt before.

“Oh, God”, he thought, “what a strenuous career it is that I’ve chosen!
Travelling day in and day out. Doing business like this takes much more
effort than doing your own business at home, and on top of that there’s
the curse of travelling, worries about making train connections, bad
and irregular food, contact with different people all the time so that
you can never get to know anyone or become friendly with them. It can
all go to Hell!” He felt a slight itch up on his belly; pushed himself
slowly up on his back towards the headboard so that he could lift his
head better; found where the itch was, and saw that it was covered with
lots of little white spots which he didn’t know what to make of; and
when he tried to feel the place with one of his legs he drew it quickly
back because as soon as he touched it he was overcome by a cold
shudder.

He slid back into his former position. “Getting up early all the time”,
he thought, “it makes you stupid. You’ve got to get enough sleep. Other
travelling salesmen live a life of luxury. For instance, whenever I go
back to the guest house during the morning to copy out the contract,
these gentlemen are always still sitting there eating their breakfasts.
I ought to just try that with my boss; I’d get kicked out on the spot.
But who knows, maybe that would be the best thing for me. If I didn’t
have my parents to think about I’d have given in my notice a long time
ago, I’d have gone up to the boss and told him just what I think, tell
him everything I would, let him know just what I feel. He’d fall right
off his desk! And it’s a funny sort of business to be sitting up there
at your desk, talking down at your subordinates from up there,
especially when you have to go right up close because the boss is hard
of hearing. Well, there’s still some hope; once I’ve got the money
together to pay off my parents’ debt to him—another five or six years I
suppose—that’s definitely what I’ll do. That’s when I’ll make the big
change. First of all though, I’ve got to get up, my train leaves at
five.”

And he looked over at the alarm clock, ticking on the chest of drawers.
“God in Heaven!” he thought. It was half past six and the hands were
quietly moving forwards, it was even later than half past, more like
quarter to seven. Had the alarm clock not rung? He could see from the
bed that it had been set for four o’clock as it should have been; it
certainly must have rung. Yes, but was it possible to quietly sleep
through that furniture-rattling noise? True, he had not slept
peacefully, but probably all the more deeply because of that. What
should he do now? The next train went at seven; if he were to catch
that he would have to rush like mad and the collection of samples was
still not packed, and he did not at all feel particularly fresh and
lively. And even if he did catch the train he would not avoid his
boss’s anger as the office assistant would have been there to see the
five o’clock train go, he would have put in his report about Gregor’s
not being there a long time ago. The office assistant was the boss’s
man, spineless, and with no understanding. What about if he reported
sick? But that would be extremely strained and suspicious as in five
years of service Gregor had never once yet been ill. His boss would
certainly come round with the doctor from the medical insurance
company, accuse his parents of having a lazy son, and accept the
doctor’s recommendation not to make any claim as the doctor believed
that no-one was ever ill but that many were workshy. And what’s more,
would he have been entirely wrong in this case? Gregor did in fact,
apart from excessive sleepiness after sleeping for so long, feel
completely well and even felt much hungrier than usual.

He was still hurriedly thinking all this through, unable to decide to
get out of the bed, when the clock struck quarter to seven. There was a
cautious knock at the door near his head. “Gregor”, somebody called—it
was his mother—“it’s quarter to seven. Didn’t you want to go
somewhere?” That gentle voice! Gregor was shocked when he heard his own
voice answering, it could hardly be recognised as the voice he had had
before. As if from deep inside him, there was a painful and
uncontrollable squeaking mixed in with it, the words could be made out
at first but then there was a sort of echo which made them unclear,
leaving the hearer unsure whether he had heard properly or not. Gregor
had wanted to give a full answer and explain everything, but in the
circumstances contented himself with saying: “Yes, mother, yes,
thank-you, I’m getting up now.” The change in Gregor’s voice probably
could not be noticed outside through the wooden door, as his mother was
satisfied with this explanation and shuffled away. But this short
conversation made the other members of the family aware that Gregor,
against their expectations was still at home, and soon his father came
knocking at one of the side doors, gently, but with his fist. “Gregor,
Gregor”, he called, “what’s wrong?” And after a short while he called
again with a warning deepness in his voice: “Gregor! Gregor!” At the
other side door his sister came plaintively: “Gregor? Aren’t you well?
Do you need anything?” Gregor answered to both sides: “I’m ready, now”,
making an effort to remove all the strangeness from his voice by
enunciating very carefully and putting long pauses between each,
individual word. His father went back to his breakfast, but his sister
whispered: “Gregor, open the door, I beg of you.” Gregor, however, had
no thought of opening the door, and instead congratulated himself for
his cautious habit, acquired from his travelling, of locking all doors
at night even when he was at home.

The first thing he wanted to do was to get up in peace without being
disturbed, to get dressed, and most of all to have his breakfast. Only
then would he consider what to do next, as he was well aware that he
would not bring his thoughts to any sensible conclusions by lying in
bed. He remembered that he had often felt a slight pain in bed, perhaps
caused by lying awkwardly, but that had always turned out to be pure
imagination and he wondered how his imaginings would slowly resolve
themselves today. He did not have the slightest doubt that the change
in his voice was nothing more than the first sign of a serious cold,
which was an occupational hazard for travelling salesmen.

It was a simple matter to throw off the covers; he only had to blow
himself up a little and they fell off by themselves. But it became
difficult after that, especially as he was so exceptionally broad. He
would have used his arms and his hands to push himself up; but instead
of them he only had all those little legs continuously moving in
different directions, and which he was moreover unable to control. If
he wanted to bend one of them, then that was the first one that would
stretch itself out; and if he finally managed to do what he wanted with
that leg, all the others seemed to be set free and would move about
painfully. “This is something that can’t be done in bed”, Gregor said
to himself, “so don’t keep trying to do it”.

The first thing he wanted to do was get the lower part of his body out
of the bed, but he had never seen this lower part, and could not
imagine what it looked like; it turned out to be too hard to move; it
went so slowly; and finally, almost in a frenzy, when he carelessly
shoved himself forwards with all the force he could gather, he chose
the wrong direction, hit hard against the lower bedpost, and learned
from the burning pain he felt that the lower part of his body might
well, at present, be the most sensitive.

So then he tried to get the top part of his body out of the bed first,
carefully turning his head to the side. This he managed quite easily,
and despite its breadth and its weight, the bulk of his body eventually
followed slowly in the direction of the head. But when he had at last
got his head out of the bed and into the fresh air it occurred to him
that if he let himself fall it would be a miracle if his head were not
injured, so he became afraid to carry on pushing himself forward the
same way. And he could not knock himself out now at any price; better
to stay in bed than lose consciousness.

It took just as much effort to get back to where he had been earlier,
but when he lay there sighing, and was once more watching his legs as
they struggled against each other even harder than before, if that was
possible, he could think of no way of bringing peace and order to this
chaos. He told himself once more that it was not possible for him to
stay in bed and that the most sensible thing to do would be to get free
of it in whatever way he could at whatever sacrifice. At the same time,
though, he did not forget to remind himself that calm consideration was
much better than rushing to desperate conclusions. At times like this
he would direct his eyes to the window and look out as clearly as he
could, but unfortunately, even the other side of the narrow street was
enveloped in morning fog and the view had little confidence or cheer to
offer him. “Seven o’clock, already”, he said to himself when the clock
struck again, “seven o’clock, and there’s still a fog like this.” And
he lay there quietly a while longer, breathing lightly as if he perhaps
expected the total stillness to bring things back to their real and
natural state.

But then he said to himself: “Before it strikes quarter past seven I’ll
definitely have to have got properly out of bed. And by then somebody
will have come round from work to ask what’s happened to me as well, as
they open up at work before seven o’clock.” And so he set himself to
the task of swinging the entire length of his body out of the bed all
at the same time. If he succeeded in falling out of bed in this way and
kept his head raised as he did so he could probably avoid injuring it.
His back seemed to be quite hard, and probably nothing would happen to
it falling onto the carpet. His main concern was for the loud noise he
was bound to make, and which even through all the doors would probably
raise concern if not alarm. But it was something that had to be risked.

When Gregor was already sticking half way out of the bed—the new method
was more of a game than an effort, all he had to do was rock back and
forth—it occurred to him how simple everything would be if somebody
came to help him. Two strong people—he had his father and the maid in
mind—would have been more than enough; they would only have to push
their arms under the dome of his back, peel him away from the bed, bend
down with the load and then be patient and careful as he swang over
onto the floor, where, hopefully, the little legs would find a use.
Should he really call for help though, even apart from the fact that
all the doors were locked? Despite all the difficulty he was in, he
could not suppress a smile at this thought.

After a while he had already moved so far across that it would have
been hard for him to keep his balance if he rocked too hard. The time
was now ten past seven and he would have to make a final decision very
soon. Then there was a ring at the door of the flat. “That’ll be
someone from work”, he said to himself, and froze very still, although
his little legs only became all the more lively as they danced around.
For a moment everything remained quiet. “They’re not opening the door”,
Gregor said to himself, caught in some nonsensical hope. But then of
course, the maid’s firm steps went to the door as ever and opened it.
Gregor only needed to hear the visitor’s first words of greeting and he
knew who it was—the chief clerk himself. Why did Gregor have to be the
only one condemned to work for a company where they immediately became
highly suspicious at the slightest shortcoming? Were all employees,
every one of them, louts, was there not one of them who was faithful
and devoted who would go so mad with pangs of conscience that he
couldn’t get out of bed if he didn’t spend at least a couple of hours
in the morning on company business? Was it really not enough to let one
of the trainees make enquiries—assuming enquiries were even
necessary—did the chief clerk have to come himself, and did they have
to show the whole, innocent family that this was so suspicious that
only the chief clerk could be trusted to have the wisdom to investigate
it? And more because these thoughts had made him upset than through any
proper decision, he swang himself with all his force out of the bed.
There was a loud thump, but it wasn’t really a loud noise. His fall was
softened a little by the carpet, and Gregor’s back was also more
elastic than he had thought, which made the sound muffled and not too
noticeable. He had not held his head carefully enough, though, and hit
it as he fell; annoyed and in pain, he turned it and rubbed it against
the carpet.

“Something’s fallen down in there”, said the chief clerk in the room on
the left. Gregor tried to imagine whether something of the sort that
had happened to him today could ever happen to the chief clerk too; you
had to concede that it was possible. But as if in gruff reply to this
question, the chief clerk’s firm footsteps in his highly polished boots
could now be heard in the adjoining room. From the room on his right,
Gregor’s sister whispered to him to let him know: “Gregor, the chief
clerk is here.” “Yes, I know”, said Gregor to himself; but without
daring to raise his voice loud enough for his sister to hear him.

“Gregor”, said his father now from the room to his left, “the chief
clerk has come round and wants to know why you didn’t leave on the
early train. We don’t know what to say to him. And anyway, he wants to
speak to you personally. So please open up this door. I’m sure he’ll be
good enough to forgive the untidiness of your room.” Then the chief
clerk called “Good morning, Mr. Samsa”. “He isn’t well”, said his
mother to the chief clerk, while his father continued to speak through
the door. “He isn’t well, please believe me. Why else would Gregor have
missed a train! The lad only ever thinks about the business. It nearly
makes me cross the way he never goes out in the evenings; he’s been in
town for a week now but stayed home every evening. He sits with us in
the kitchen and just reads the paper or studies train timetables. His
idea of relaxation is working with his fretsaw. He’s made a little
frame, for instance, it only took him two or three evenings, you’ll be
amazed how nice it is; it’s hanging up in his room; you’ll see it as
soon as Gregor opens the door. Anyway, I’m glad you’re here; we
wouldn’t have been able to get Gregor to open the door by ourselves;
he’s so stubborn; and I’m sure he isn’t well, he said this morning that
he is, but he isn’t.” “I’ll be there in a moment”, said Gregor slowly
and thoughtfully, but without moving so that he would not miss any word
of the conversation. “Well I can’t think of any other way of explaining
it, Mrs. Samsa”, said the chief clerk, “I hope it’s nothing serious.
But on the other hand, I must say that if we people in commerce ever
become slightly unwell then, fortunately or unfortunately as you like,
we simply have to overcome it because of business considerations.” “Can
the chief clerk come in to see you now then?”, asked his father
impatiently, knocking at the door again. “No”, said Gregor. In the room
on his right there followed a painful silence; in the room on his left
his sister began to cry.

So why did his sister not go and join the others? She had probably only
just got up and had not even begun to get dressed. And why was she
crying? Was it because he had not got up, and had not let the chief
clerk in, because he was in danger of losing his job and if that
happened his boss would once more pursue their parents with the same
demands as before? There was no need to worry about things like that
yet. Gregor was still there and had not the slightest intention of
abandoning his family. For the time being he just lay there on the
carpet, and no-one who knew the condition he was in would seriously
have expected him to let the chief clerk in. It was only a minor
discourtesy, and a suitable excuse could easily be found for it later
on, it was not something for which Gregor could be sacked on the spot.
And it seemed to Gregor much more sensible to leave him now in peace
instead of disturbing him with talking at him and crying. But the
others didn’t know what was happening, they were worried, that would
excuse their behaviour.

The chief clerk now raised his voice, “Mr. Samsa”, he called to him,
“what is wrong? You barricade yourself in your room, give us no more
than yes or no for an answer, you are causing serious and unnecessary
concern to your parents and you fail—and I mention this just by the
way—you fail to carry out your business duties in a way that is quite
unheard of. I’m speaking here on behalf of your parents and of your
employer, and really must request a clear and immediate explanation. I
am astonished, quite astonished. I thought I knew you as a calm and
sensible person, and now you suddenly seem to be showing off with
peculiar whims. This morning, your employer did suggest a possible
reason for your failure to appear, it’s true—it had to do with the
money that was recently entrusted to you—but I came near to giving him
my word of honour that that could not be the right explanation. But now
that I see your incomprehensible stubbornness I no longer feel any wish
whatsoever to intercede on your behalf. And nor is your position all
that secure. I had originally intended to say all this to you in
private, but since you cause me to waste my time here for no good
reason I don’t see why your parents should not also learn of it. Your
turnover has been very unsatisfactory of late; I grant you that it’s
not the time of year to do especially good business, we recognise that;
but there simply is no time of year to do no business at all, Mr.
Samsa, we cannot allow there to be.”

“But Sir”, called Gregor, beside himself and forgetting all else in the
excitement, “I’ll open up immediately, just a moment. I’m slightly
unwell, an attack of dizziness, I haven’t been able to get up. I’m
still in bed now. I’m quite fresh again now, though. I’m just getting
out of bed. Just a moment. Be patient! It’s not quite as easy as I’d
thought. I’m quite alright now, though. It’s shocking, what can
suddenly happen to a person! I was quite alright last night, my parents
know about it, perhaps better than me, I had a small symptom of it last
night already. They must have noticed it. I don’t know why I didn’t let
you know at work! But you always think you can get over an illness
without staying at home. Please, don’t make my parents suffer! There’s
no basis for any of the accusations you’re making; nobody’s ever said a
word to me about any of these things. Maybe you haven’t read the latest
contracts I sent in. I’ll set off with the eight o’clock train, as
well, these few hours of rest have given me strength. You don’t need to
wait, sir; I’ll be in the office soon after you, and please be so good
as to tell that to the boss and recommend me to him!”

And while Gregor gushed out these words, hardly knowing what he was
saying, he made his way over to the chest of drawers—this was easily
done, probably because of the practise he had already had in bed—where
he now tried to get himself upright. He really did want to open the
door, really did want to let them see him and to speak with the chief
clerk; the others were being so insistent, and he was curious to learn
what they would say when they caught sight of him. If they were shocked
then it would no longer be Gregor’s responsibility and he could rest.
If, however, they took everything calmly he would still have no reason
to be upset, and if he hurried he really could be at the station for
eight o’clock. The first few times he tried to climb up on the smooth
chest of drawers he just slid down again, but he finally gave himself
one last swing and stood there upright; the lower part of his body was
in serious pain but he no longer gave any attention to it. Now he let
himself fall against the back of a nearby chair and held tightly to the
edges of it with his little legs. By now he had also calmed down, and
kept quiet so that he could listen to what the chief clerk was saying.

“Did you understand a word of all that?” the chief clerk asked his
parents, “surely he’s not trying to make fools of us”. “Oh, God!”
called his mother, who was already in tears, “he could be seriously ill
and we’re making him suffer. Grete! Grete!” she then cried. “Mother?”
his sister called from the other side. They communicated across
Gregor’s room. “You’ll have to go for the doctor straight away. Gregor
is ill. Quick, get the doctor. Did you hear the way Gregor spoke just
now?” “That was the voice of an animal”, said the chief clerk, with a
calmness that was in contrast with his mother’s screams. “Anna! Anna!”
his father called into the kitchen through the entrance hall, clapping
his hands, “get a locksmith here, now!” And the two girls, their skirts
swishing, immediately ran out through the hall, wrenching open the
front door of the flat as they went. How had his sister managed to get
dressed so quickly? There was no sound of the door banging shut again;
they must have left it open; people often do in homes where something
awful has happened.

Gregor, in contrast, had become much calmer. So they couldn’t
understand his words any more, although they seemed clear enough to
him, clearer than before—perhaps his ears had become used to the sound.
They had realised, though, that there was something wrong with him, and
were ready to help. The first response to his situation had been
confident and wise, and that made him feel better. He felt that he had
been drawn back in among people, and from the doctor and the locksmith
he expected great and surprising achievements—although he did not
really distinguish one from the other. Whatever was said next would be
crucial, so, in order to make his voice as clear as possible, he
coughed a little, but taking care to do this not too loudly as even
this might well sound different from the way that a human coughs and he
was no longer sure he could judge this for himself. Meanwhile, it had
become very quiet in the next room. Perhaps his parents were sat at the
table whispering with the chief clerk, or perhaps they were all pressed
against the door and listening.

Gregor slowly pushed his way over to the door with the chair. Once
there he let go of it and threw himself onto the door, holding himself
upright against it using the adhesive on the tips of his legs. He
rested there a little while to recover from the effort involved and
then set himself to the task of turning the key in the lock with his
mouth. He seemed, unfortunately, to have no proper teeth—how was he,
then, to grasp the key?—but the lack of teeth was, of course, made up
for with a very strong jaw; using the jaw, he really was able to start
the key turning, ignoring the fact that he must have been causing some
kind of damage as a brown fluid came from his mouth, flowed over the
key and dripped onto the floor. “Listen”, said the chief clerk in the
next room, “he’s turning the key.” Gregor was greatly encouraged by
this; but they all should have been calling to him, his father and his
mother too: “Well done, Gregor”, they should have cried, “keep at it,
keep hold of the lock!” And with the idea that they were all excitedly
following his efforts, he bit on the key with all his strength, paying
no attention to the pain he was causing himself. As the key turned
round he turned around the lock with it, only holding himself upright
with his mouth, and hung onto the key or pushed it down again with the
whole weight of his body as needed. The clear sound of the lock as it
snapped back was Gregor’s sign that he could break his concentration,
and as he regained his breath he said to himself: “So, I didn’t need
the locksmith after all”. Then he lay his head on the handle of the
door to open it completely.

Because he had to open the door in this way, it was already wide open
before he could be seen. He had first to slowly turn himself around one
of the double doors, and he had to do it very carefully if he did not
want to fall flat on his back before entering the room. He was still
occupied with this difficult movement, unable to pay attention to
anything else, when he heard the chief clerk exclaim a loud “Oh!”,
which sounded like the soughing of the wind. Now he also saw him—he was
the nearest to the door—his hand pressed against his open mouth and
slowly retreating as if driven by a steady and invisible force.
Gregor’s mother, her hair still dishevelled from bed despite the chief
clerk’s being there, looked at his father. Then she unfolded her arms,
took two steps forward towards Gregor and sank down onto the floor into
her skirts that spread themselves out around her as her head
disappeared down onto her breast. His father looked hostile, and
clenched his fists as if wanting to knock Gregor back into his room.
Then he looked uncertainly round the living room, covered his eyes with
his hands and wept so that his powerful chest shook.

So Gregor did not go into the room, but leant against the inside of the
other door which was still held bolted in place. In this way only half
of his body could be seen, along with his head above it which he leant
over to one side as he peered out at the others. Meanwhile the day had
become much lighter; part of the endless, grey-black building on the
other side of the street—which was a hospital—could be seen quite
clearly with the austere and regular line of windows piercing its
façade; the rain was still falling, now throwing down large, individual
droplets which hit the ground one at a time. The washing up from
breakfast lay on the table; there was so much of it because, for
Gregor’s father, breakfast was the most important meal of the day and
he would stretch it out for several hours as he sat reading a number of
different newspapers. On the wall exactly opposite there was photograph
of Gregor when he was a lieutenant in the army, his sword in his hand
and a carefree smile on his face as he called forth respect for his
uniform and bearing. The door to the entrance hall was open and as the
front door of the flat was also open he could see onto the landing and
the stairs where they began their way down below.

“Now, then”, said Gregor, well aware that he was the only one to have
kept calm, “I’ll get dressed straight away now, pack up my samples and
set off. Will you please just let me leave? You can see”, he said to
the chief clerk, “that I’m not stubborn and I like to do my job; being
a commercial traveller is arduous but without travelling I couldn’t
earn my living. So where are you going, in to the office? Yes? Will you
report everything accurately, then? It’s quite possible for someone to
be temporarily unable to work, but that’s just the right time to
remember what’s been achieved in the past and consider that later on,
once the difficulty has been removed, he will certainly work with all
the more diligence and concentration. You’re well aware that I’m
seriously in debt to our employer as well as having to look after my
parents and my sister, so that I’m trapped in a difficult situation,
but I will work my way out of it again. Please don’t make things any
harder for me than they are already, and don’t take sides against me at
the office. I know that nobody likes the travellers. They think we earn
an enormous wage as well as having a soft time of it. That’s just
prejudice but they have no particular reason to think better of it. But
you, sir, you have a better overview than the rest of the staff, in
fact, if I can say this in confidence, a better overview than the boss
himself—it’s very easy for a businessman like him to make mistakes
about his employees and judge them more harshly than he should. And
you’re also well aware that we travellers spend almost the whole year
away from the office, so that we can very easily fall victim to gossip
and chance and groundless complaints, and it’s almost impossible to
defend yourself from that sort of thing, we don’t usually even hear
about them, or if at all it’s when we arrive back home exhausted from a
trip, and that’s when we feel the harmful effects of what’s been going
on without even knowing what caused them. Please, don’t go away, at
least first say something to show that you grant that I’m at least
partly right!”

But the chief clerk had turned away as soon as Gregor had started to
speak, and, with protruding lips, only stared back at him over his
trembling shoulders as he left. He did not keep still for a moment
while Gregor was speaking, but moved steadily towards the door without
taking his eyes off him. He moved very gradually, as if there had been
some secret prohibition on leaving the room. It was only when he had
reached the entrance hall that he made a sudden movement, drew his foot
from the living room, and rushed forward in a panic. In the hall, he
stretched his right hand far out towards the stairway as if out there,
there were some supernatural force waiting to save him.

Gregor realised that it was out of the question to let the chief clerk
go away in this mood if his position in the firm was not to be put into
extreme danger. That was something his parents did not understand very
well; over the years, they had become convinced that this job would
provide for Gregor for his entire life, and besides, they had so much
to worry about at present that they had lost sight of any thought for
the future. Gregor, though, did think about the future. The chief clerk
had to be held back, calmed down, convinced and finally won over; the
future of Gregor and his family depended on it! If only his sister were
here! She was clever; she was already in tears while Gregor was still
lying peacefully on his back. And the chief clerk was a lover of women,
surely she could persuade him; she would close the front door in the
entrance hall and talk him out of his shocked state. But his sister was
not there, Gregor would have to do the job himself. And without
considering that he still was not familiar with how well he could move
about in his present state, or that his speech still might not—or
probably would not—be understood, he let go of the door; pushed himself
through the opening; tried to reach the chief clerk on the landing who,
ridiculously, was holding on to the banister with both hands; but
Gregor fell immediately over and, with a little scream as he sought
something to hold onto, landed on his numerous little legs. Hardly had
that happened than, for the first time that day, he began to feel
alright with his body; the little legs had the solid ground under them;
to his pleasure, they did exactly as he told them; they were even
making the effort to carry him where he wanted to go; and he was soon
believing that all his sorrows would soon be finally at an end. He held
back the urge to move but swayed from side to side as he crouched there
on the floor. His mother was not far away in front of him and seemed,
at first, quite engrossed in herself, but then she suddenly jumped up
with her arms outstretched and her fingers spread shouting: “Help, for
pity’s sake, Help!” The way she held her head suggested she wanted to
see Gregor better, but the unthinking way she was hurrying backwards
showed that she did not; she had forgotten that the table was behind
her with all the breakfast things on it; when she reached the table she
sat quickly down on it without knowing what she was doing; without even
seeming to notice that the coffee pot had been knocked over and a gush
of coffee was pouring down onto the carpet.

“Mother, mother”, said Gregor gently, looking up at her. He had
completely forgotten the chief clerk for the moment, but could not help
himself snapping in the air with his jaws at the sight of the flow of
coffee. That set his mother screaming anew, she fled from the table and
into the arms of his father as he rushed towards her. Gregor, though,
had no time to spare for his parents now; the chief clerk had already
reached the stairs; with his chin on the banister, he looked back for
the last time. Gregor made a run for him; he wanted to be sure of
reaching him; the chief clerk must have expected something, as he leapt
down several steps at once and disappeared; his shouts resounding all
around the staircase. The flight of the chief clerk seemed,
unfortunately, to put Gregor’s father into a panic as well. Until then
he had been relatively self controlled, but now, instead of running
after the chief clerk himself, or at least not impeding Gregor as he
ran after him, Gregor’s father seized the chief clerk’s stick in his
right hand (the chief clerk had left it behind on a chair, along with
his hat and overcoat), picked up a large newspaper from the table with
his left, and used them to drive Gregor back into his room, stamping
his foot at him as he went. Gregor’s appeals to his father were of no
help, his appeals were simply not understood, however much he humbly
turned his head his father merely stamped his foot all the harder.
Across the room, despite the chilly weather, Gregor’s mother had pulled
open a window, leant far out of it and pressed her hands to her face. A
strong draught of air flew in from the street towards the stairway, the
curtains flew up, the newspapers on the table fluttered and some of
them were blown onto the floor. Nothing would stop Gregor’s father as
he drove him back, making hissing noises at him like a wild man. Gregor
had never had any practice in moving backwards and was only able to go
very slowly. If Gregor had only been allowed to turn round he would
have been back in his room straight away, but he was afraid that if he
took the time to do that his father would become impatient, and there
was the threat of a lethal blow to his back or head from the stick in
his father’s hand any moment. Eventually, though, Gregor realised that
he had no choice as he saw, to his disgust, that he was quite incapable
of going backwards in a straight line; so he began, as quickly as
possible and with frequent anxious glances at his father, to turn
himself round. It went very slowly, but perhaps his father was able to
see his good intentions as he did nothing to hinder him, in fact now
and then he used the tip of his stick to give directions from a
distance as to which way to turn. If only his father would stop that
unbearable hissing! It was making Gregor quite confused. When he had
nearly finished turning round, still listening to that hissing, he made
a mistake and turned himself back a little the way he had just come. He
was pleased when he finally had his head in front of the doorway, but
then saw that it was too narrow, and his body was too broad to get
through it without further difficulty. In his present mood, it
obviously did not occur to his father to open the other of the double
doors so that Gregor would have enough space to get through. He was
merely fixed on the idea that Gregor should be got back into his room
as quickly as possible. Nor would he ever have allowed Gregor the time
to get himself upright as preparation for getting through the doorway.
What he did, making more noise than ever, was to drive Gregor forwards
all the harder as if there had been nothing in the way; it sounded to
Gregor as if there was now more than one father behind him; it was not
a pleasant experience, and Gregor pushed himself into the doorway
without regard for what might happen. One side of his body lifted
itself, he lay at an angle in the doorway, one flank scraped on the
white door and was painfully injured, leaving vile brown flecks on it,
soon he was stuck fast and would not have been able to move at all by
himself, the little legs along one side hung quivering in the air while
those on the other side were pressed painfully against the ground. Then
his father gave him a hefty shove from behind which released him from
where he was held and sent him flying, and heavily bleeding, deep into
his room. The door was slammed shut with the stick, then, finally, all
was quiet.




II


It was not until it was getting dark that evening that Gregor awoke
from his deep and coma-like sleep. He would have woken soon afterwards
anyway even if he hadn’t been disturbed, as he had had enough sleep and
felt fully rested. But he had the impression that some hurried steps
and the sound of the door leading into the front room being carefully
shut had woken him. The light from the electric street lamps shone
palely here and there onto the ceiling and tops of the furniture, but
down below, where Gregor was, it was dark. He pushed himself over to
the door, feeling his way clumsily with his antennae—of which he was
now beginning to learn the value—in order to see what had been
happening there. The whole of his left side seemed like one, painfully
stretched scar, and he limped badly on his two rows of legs. One of the
legs had been badly injured in the events of that morning—it was nearly
a miracle that only one of them had been—and dragged along lifelessly.

It was only when he had reached the door that he realised what it
actually was that had drawn him over to it; it was the smell of
something to eat. By the door there was a dish filled with sweetened
milk with little pieces of white bread floating in it. He was so
pleased he almost laughed, as he was even hungrier than he had been
that morning, and immediately dipped his head into the milk, nearly
covering his eyes with it. But he soon drew his head back again in
disappointment; not only did the pain in his tender left side make it
difficult to eat the food—he was only able to eat if his whole body
worked together as a snuffling whole—but the milk did not taste at all
nice. Milk like this was normally his favourite drink, and his sister
had certainly left it there for him because of that, but he turned,
almost against his own will, away from the dish and crawled back into
the centre of the room.

Through the crack in the door, Gregor could see that the gas had been
lit in the living room. His father at this time would normally be sat
with his evening paper, reading it out in a loud voice to Gregor’s
mother, and sometimes to his sister, but there was now not a sound to
be heard. Gregor’s sister would often write and tell him about this
reading, but maybe his father had lost the habit in recent times. It
was so quiet all around too, even though there must have been somebody
in the flat. “What a quiet life it is the family lead”, said Gregor to
himself, and, gazing into the darkness, felt a great pride that he was
able to provide a life like that in such a nice home for his sister and
parents. But what now, if all this peace and wealth and comfort should
come to a horrible and frightening end? That was something that Gregor
did not want to think about too much, so he started to move about,
crawling up and down the room.

Once during that long evening, the door on one side of the room was
opened very slightly and hurriedly closed again; later on the door on
the other side did the same; it seemed that someone needed to enter the
room but thought better of it. Gregor went and waited immediately by
the door, resolved either to bring the timorous visitor into the room
in some way or at least to find out who it was; but the door was opened
no more that night and Gregor waited in vain. The previous morning
while the doors were locked everyone had wanted to get in there to him,
but now, now that he had opened up one of the doors and the other had
clearly been unlocked some time during the day, no-one came, and the
keys were in the other sides.

It was not until late at night that the gaslight in the living room was
put out, and now it was easy to see that his parents and sister had
stayed awake all that time, as they all could be distinctly heard as
they went away together on tip-toe. It was clear that no-one would come
into Gregor’s room any more until morning; that gave him plenty of time
to think undisturbed about how he would have to re-arrange his life.
For some reason, the tall, empty room where he was forced to remain
made him feel uneasy as he lay there flat on the floor, even though he
had been living in it for five years. Hardly aware of what he was doing
other than a slight feeling of shame, he hurried under the couch. It
pressed down on his back a little, and he was no longer able to lift
his head, but he nonetheless felt immediately at ease and his only
regret was that his body was too broad to get it all underneath.

He spent the whole night there. Some of the time he passed in a light
sleep, although he frequently woke from it in alarm because of his
hunger, and some of the time was spent in worries and vague hopes
which, however, always led to the same conclusion: for the time being
he must remain calm, he must show patience and the greatest
consideration so that his family could bear the unpleasantness that he,
in his present condition, was forced to impose on them.

Gregor soon had the opportunity to test the strength of his decisions,
as early the next morning, almost before the night had ended, his
sister, nearly fully dressed, opened the door from the front room and
looked anxiously in. She did not see him straight away, but when she
did notice him under the couch—he had to be somewhere, for God’s sake,
he couldn’t have flown away—she was so shocked that she lost control of
herself and slammed the door shut again from outside. But she seemed to
regret her behaviour, as she opened the door again straight away and
came in on tip-toe as if entering the room of someone seriously ill or
even of a stranger. Gregor had pushed his head forward, right to the
edge of the couch, and watched her. Would she notice that he had left
the milk as it was, realise that it was not from any lack of hunger and
bring him in some other food that was more suitable? If she didn’t do
it herself he would rather go hungry than draw her attention to it,
although he did feel a terrible urge to rush forward from under the
couch, throw himself at his sister’s feet and beg her for something
good to eat. However, his sister noticed the full dish immediately and
looked at it and the few drops of milk splashed around it with some
surprise. She immediately picked it up—using a rag, not her bare
hands—and carried it out. Gregor was extremely curious as to what she
would bring in its place, imagining the wildest possibilities, but he
never could have guessed what his sister, in her goodness, actually did
bring. In order to test his taste, she brought him a whole selection of
things, all spread out on an old newspaper. There were old, half-rotten
vegetables; bones from the evening meal, covered in white sauce that
had gone hard; a few raisins and almonds; some cheese that Gregor had
declared inedible two days before; a dry roll and some bread spread
with butter and salt. As well as all that she had poured some water
into the dish, which had probably been permanently set aside for
Gregor’s use, and placed it beside them. Then, out of consideration for
Gregor’s feelings, as she knew that he would not eat in front of her,
she hurried out again and even turned the key in the lock so that
Gregor would know he could make things as comfortable for himself as he
liked. Gregor’s little legs whirred, at last he could eat. What’s more,
his injuries must already have completely healed as he found no
difficulty in moving. This amazed him, as more than a month earlier he
had cut his finger slightly with a knife, he thought of how his finger
had still hurt the day before yesterday. “Am I less sensitive than I
used to be, then?”, he thought, and was already sucking greedily at the
cheese which had immediately, almost compellingly, attracted him much
more than the other foods on the newspaper. Quickly one after another,
his eyes watering with pleasure, he consumed the cheese, the vegetables
and the sauce; the fresh foods, on the other hand, he didn’t like at
all, and even dragged the things he did want to eat a little way away
from them because he couldn’t stand the smell. Long after he had
finished eating and lay lethargic in the same place, his sister slowly
turned the key in the lock as a sign to him that he should withdraw. He
was immediately startled, although he had been half asleep, and he
hurried back under the couch. But he needed great self-control to stay
there even for the short time that his sister was in the room, as
eating so much food had rounded out his body a little and he could
hardly breathe in that narrow space. Half suffocating, he watched with
bulging eyes as his sister unselfconsciously took a broom and swept up
the left-overs, mixing them in with the food he had not even touched at
all as if it could not be used any more. She quickly dropped it all
into a bin, closed it with its wooden lid, and carried everything out.
She had hardly turned her back before Gregor came out again from under
the couch and stretched himself.

This was how Gregor received his food each day now, once in the morning
while his parents and the maid were still asleep, and the second time
after everyone had eaten their meal at midday as his parents would
sleep for a little while then as well, and Gregor’s sister would send
the maid away on some errand. Gregor’s father and mother certainly did
not want him to starve either, but perhaps it would have been more than
they could stand to have any more experience of his feeding than being
told about it, and perhaps his sister wanted to spare them what
distress she could as they were indeed suffering enough.

It was impossible for Gregor to find out what they had told the doctor
and the locksmith that first morning to get them out of the flat. As
nobody could understand him, nobody, not even his sister, thought that
he could understand them, so he had to be content to hear his sister’s
sighs and appeals to the saints as she moved about his room. It was
only later, when she had become a little more used to everything—there
was, of course, no question of her ever becoming fully used to the
situation—that Gregor would sometimes catch a friendly comment, or at
least a comment that could be construed as friendly. “He’s enjoyed his
dinner today”, she might say when he had diligently cleared away all
the food left for him, or if he left most of it, which slowly became
more and more frequent, she would often say, sadly, “now everything’s
just been left there again”.

Although Gregor wasn’t able to hear any news directly he did listen to
much of what was said in the next rooms, and whenever he heard anyone
speaking he would scurry straight to the appropriate door and press his
whole body against it. There was seldom any conversation, especially at
first, that was not about him in some way, even if only in secret. For
two whole days, all the talk at every mealtime was about what they
should do now; but even between meals they spoke about the same subject
as there were always at least two members of the family at home—nobody
wanted to be at home by themselves and it was out of the question to
leave the flat entirely empty. And on the very first day the maid had
fallen to her knees and begged Gregor’s mother to let her go without
delay. It was not very clear how much she knew of what had happened but
she left within a quarter of an hour, tearfully thanking Gregor’s
mother for her dismissal as if she had done her an enormous service.
She even swore emphatically not to tell anyone the slightest about what
had happened, even though no-one had asked that of her.

Now Gregor’s sister also had to help his mother with the cooking;
although that was not so much bother as no-one ate very much. Gregor
often heard how one of them would unsuccessfully urge another to eat,
and receive no more answer than “no thanks, I’ve had enough” or
something similar. No-one drank very much either. His sister would
sometimes ask his father whether he would like a beer, hoping for the
chance to go and fetch it herself. When his father then said nothing
she would add, so that he would not feel selfish, that she could send
the housekeeper for it, but then his father would close the matter with
a big, loud “No”, and no more would be said.

Even before the first day had come to an end, his father had explained
to Gregor’s mother and sister what their finances and prospects were.
Now and then he stood up from the table and took some receipt or
document from the little cash box he had saved from his business when
it had collapsed five years earlier. Gregor heard how he opened the
complicated lock and then closed it again after he had taken the item
he wanted. What he heard his father say was some of the first good news
that Gregor heard since he had first been incarcerated in his room. He
had thought that nothing at all remained from his father’s business, at
least he had never told him anything different, and Gregor had never
asked him about it anyway. Their business misfortune had reduced the
family to a state of total despair, and Gregor’s only concern at that
time had been to arrange things so that they could all forget about it
as quickly as possible. So then he started working especially hard,
with a fiery vigour that raised him from a junior salesman to a
travelling representative almost overnight, bringing with it the chance
to earn money in quite different ways. Gregor converted his success at
work straight into cash that he could lay on the table at home for the
benefit of his astonished and delighted family. They had been good
times and they had never come again, at least not with the same
splendour, even though Gregor had later earned so much that he was in a
position to bear the costs of the whole family, and did bear them. They
had even got used to it, both Gregor and the family, they took the
money with gratitude and he was glad to provide it, although there was
no longer much warm affection given in return. Gregor only remained
close to his sister now. Unlike him, she was very fond of music and a
gifted and expressive violinist, it was his secret plan to send her to
the conservatory next year even though it would cause great expense
that would have to be made up for in some other way. During Gregor’s
short periods in town, conversation with his sister would often turn to
the conservatory but it was only ever mentioned as a lovely dream that
could never be realised. Their parents did not like to hear this
innocent talk, but Gregor thought about it quite hard and decided he
would let them know what he planned with a grand announcement of it on
Christmas day.

That was the sort of totally pointless thing that went through his mind
in his present state, pressed upright against the door and listening.
There were times when he simply became too tired to continue listening,
when his head would fall wearily against the door and he would pull it
up again with a start, as even the slightest noise he caused would be
heard next door and they would all go silent. “What’s that he’s doing
now”, his father would say after a while, clearly having gone over to
the door, and only then would the interrupted conversation slowly be
taken up again.

When explaining things, his father repeated himself several times,
partly because it was a long time since he had been occupied with these
matters himself and partly because Gregor’s mother did not understand
everything the first time. From these repeated explanations Gregor
learned, to his pleasure, that despite all their misfortunes there was
still some money available from the old days. It was not a lot, but it
had not been touched in the meantime and some interest had accumulated.
Besides that, they had not been using up all the money that Gregor had
been bringing home every month, keeping only a little for himself, so
that that, too, had been accumulating. Behind the door, Gregor nodded
with enthusiasm in his pleasure at this unexpected thrift and caution.
He could actually have used this surplus money to reduce his father’s
debt to his boss, and the day when he could have freed himself from
that job would have come much closer, but now it was certainly better
the way his father had done things.

This money, however, was certainly not enough to enable the family to
live off the interest; it was enough to maintain them for, perhaps, one
or two years, no more. That’s to say, it was money that should not
really be touched but set aside for emergencies; money to live on had
to be earned. His father was healthy but old, and lacking in self
confidence. During the five years that he had not been working—the
first holiday in a life that had been full of strain and no success—he
had put on a lot of weight and become very slow and clumsy. Would
Gregor’s elderly mother now have to go and earn money? She suffered
from asthma and it was a strain for her just to move about the home,
every other day would be spent struggling for breath on the sofa by the
open window. Would his sister have to go and earn money? She was still
a child of seventeen, her life up till then had been very enviable,
consisting of wearing nice clothes, sleeping late, helping out in the
business, joining in with a few modest pleasures and most of all
playing the violin. Whenever they began to talk of the need to earn
money, Gregor would always first let go of the door and then throw
himself onto the cool, leather sofa next to it, as he became quite hot
with shame and regret.

He would often lie there the whole night through, not sleeping a wink
but scratching at the leather for hours on end. Or he might go to all
the effort of pushing a chair to the window, climbing up onto the sill
and, propped up in the chair, leaning on the window to stare out of it.
He had used to feel a great sense of freedom from doing this, but doing
it now was obviously something more remembered than experienced, as
what he actually saw in this way was becoming less distinct every day,
even things that were quite near; he had used to curse the ever-present
view of the hospital across the street, but now he could not see it at
all, and if he had not known that he lived in Charlottenstrasse, which
was a quiet street despite being in the middle of the city, he could
have thought that he was looking out the window at a barren waste where
the grey sky and the grey earth mingled inseparably. His observant
sister only needed to notice the chair twice before she would always
push it back to its exact position by the window after she had tidied
up the room, and even left the inner pane of the window open from then
on.

If Gregor had only been able to speak to his sister and thank her for
all that she had to do for him it would have been easier for him to
bear it; but as it was it caused him pain. His sister, naturally, tried
as far as possible to pretend there was nothing burdensome about it,
and the longer it went on, of course, the better she was able to do so,
but as time went by Gregor was also able to see through it all so much
better. It had even become very unpleasant for him, now, whenever she
entered the room. No sooner had she come in than she would quickly
close the door as a precaution so that no-one would have to suffer the
view into Gregor’s room, then she would go straight to the window and
pull it hurriedly open almost as if she were suffocating. Even if it
was cold, she would stay at the window breathing deeply for a little
while. She would alarm Gregor twice a day with this running about and
noise making; he would stay under the couch shivering the whole while,
knowing full well that she would certainly have liked to spare him this
ordeal, but it was impossible for her to be in the same room with him
with the windows closed.

One day, about a month after Gregor’s transformation when his sister no
longer had any particular reason to be shocked at his appearance, she
came into the room a little earlier than usual and found him still
staring out the window, motionless, and just where he would be most
horrible. In itself, his sister’s not coming into the room would have
been no surprise for Gregor as it would have been difficult for her to
immediately open the window while he was still there, but not only did
she not come in, she went straight back and closed the door behind her,
a stranger would have thought he had threatened her and tried to bite
her. Gregor went straight to hide himself under the couch, of course,
but he had to wait until midday before his sister came back and she
seemed much more uneasy than usual. It made him realise that she still
found his appearance unbearable and would continue to do so, she
probably even had to overcome the urge to flee when she saw the little
bit of him that protruded from under the couch. One day, in order to
spare her even this sight, he spent four hours carrying the bedsheet
over to the couch on his back and arranged it so that he was completely
covered and his sister would not be able to see him even if she bent
down. If she did not think this sheet was necessary then all she had to
do was take it off again, as it was clear enough that it was no
pleasure for Gregor to cut himself off so completely. She left the
sheet where it was. Gregor even thought he glimpsed a look of gratitude
one time when he carefully looked out from under the sheet to see how
his sister liked the new arrangement.

For the first fourteen days, Gregor’s parents could not bring
themselves to come into the room to see him. He would often hear them
say how they appreciated all the new work his sister was doing even
though, before, they had seen her as a girl who was somewhat useless
and frequently been annoyed with her. But now the two of them, father
and mother, would often both wait outside the door of Gregor’s room
while his sister tidied up in there, and as soon as she went out again
she would have to tell them exactly how everything looked, what Gregor
had eaten, how he had behaved this time and whether, perhaps, any
slight improvement could be seen. His mother also wanted to go in and
visit Gregor relatively soon but his father and sister at first
persuaded her against it. Gregor listened very closely to all this, and
approved fully. Later, though, she had to be held back by force, which
made her call out: “Let me go and see Gregor, he is my unfortunate son!
Can’t you understand I have to see him?”, and Gregor would think to
himself that maybe it would be better if his mother came in, not every
day of course, but one day a week, perhaps; she could understand
everything much better than his sister who, for all her courage, was
still just a child after all, and really might not have had an adult’s
appreciation of the burdensome job she had taken on.

Gregor’s wish to see his mother was soon realised. Out of consideration
for his parents, Gregor wanted to avoid being seen at the window during
the day, the few square meters of the floor did not give him much room
to crawl about, it was hard to just lie quietly through the night, his
food soon stopped giving him any pleasure at all, and so, to entertain
himself, he got into the habit of crawling up and down the walls and
ceiling. He was especially fond of hanging from the ceiling; it was
quite different from lying on the floor; he could breathe more freely;
his body had a light swing to it; and up there, relaxed and almost
happy, it might happen that he would surprise even himself by letting
go of the ceiling and landing on the floor with a crash. But now, of
course, he had far better control of his body than before and, even
with a fall as great as that, caused himself no damage. Very soon his
sister noticed Gregor’s new way of entertaining himself—he had, after
all, left traces of the adhesive from his feet as he crawled about—and
got it into her head to make it as easy as possible for him by removing
the furniture that got in his way, especially the chest of drawers and
the desk. Now, this was not something that she would be able to do by
herself; she did not dare to ask for help from her father; the sixteen
year old maid had carried on bravely since the cook had left but she
certainly would not have helped in this, she had even asked to be
allowed to keep the kitchen locked at all times and never to have to
open the door unless it was especially important; so his sister had no
choice but to choose some time when Gregor’s father was not there and
fetch his mother to help her. As she approached the room, Gregor could
hear his mother express her joy, but once at the door she went silent.
First, of course, his sister came in and looked round to see that
everything in the room was alright; and only then did she let her
mother enter. Gregor had hurriedly pulled the sheet down lower over the
couch and put more folds into it so that everything really looked as if
it had just been thrown down by chance. Gregor also refrained, this
time, from spying out from under the sheet; he gave up the chance to
see his mother until later and was simply glad that she had come. “You
can come in, he can’t be seen”, said his sister, obviously leading her
in by the hand. The old chest of drawers was too heavy for a pair of
feeble women to be heaving about, but Gregor listened as they pushed it
from its place, his sister always taking on the heaviest part of the
work for herself and ignoring her mother’s warnings that she would
strain herself. This lasted a very long time. After labouring at it for
fifteen minutes or more his mother said it would be better to leave the
chest where it was, for one thing it was too heavy for them to get the
job finished before Gregor’s father got home and leaving it in the
middle of the room it would be in his way even more, and for another
thing it wasn’t even sure that taking the furniture away would really
be any help to him. She thought just the opposite; the sight of the
bare walls saddened her right to her heart; and why wouldn’t Gregor
feel the same way about it, he’d been used to this furniture in his
room for a long time and it would make him feel abandoned to be in an
empty room like that. Then, quietly, almost whispering as if wanting
Gregor (whose whereabouts she did not know) to hear not even the tone
of her voice, as she was convinced that he did not understand her
words, she added “and by taking the furniture away, won’t it seem like
we’re showing that we’ve given up all hope of improvement and we’re
abandoning him to cope for himself? I think it’d be best to leave the
room exactly the way it was before so that when Gregor comes back to us
again he’ll find everything unchanged and he’ll be able to forget the
time in between all the easier”.

Hearing these words from his mother made Gregor realise that the lack
of any direct human communication, along with the monotonous life led
by the family during these two months, must have made him confused—he
could think of no other way of explaining to himself why he had
seriously wanted his room emptied out. Had he really wanted to
transform his room into a cave, a warm room fitted out with the nice
furniture he had inherited? That would have let him crawl around
unimpeded in any direction, but it would also have let him quickly
forget his past when he had still been human. He had come very close to
forgetting, and it had only been the voice of his mother, unheard for
so long, that had shaken him out of it. Nothing should be removed;
everything had to stay; he could not do without the good influence the
furniture had on his condition; and if the furniture made it difficult
for him to crawl about mindlessly that was not a loss but a great
advantage.

His sister, unfortunately, did not agree; she had become used to the
idea, not without reason, that she was Gregor’s spokesman to his
parents about the things that concerned him. This meant that his
mother’s advice now was sufficient reason for her to insist on removing
not only the chest of drawers and the desk, as she had thought at
first, but all the furniture apart from the all-important couch. It was
more than childish perversity, of course, or the unexpected confidence
she had recently acquired, that made her insist; she had indeed noticed
that Gregor needed a lot of room to crawl about in, whereas the
furniture, as far as anyone could see, was of no use to him at all.
Girls of that age, though, do become enthusiastic about things and feel
they must get their way whenever they can. Perhaps this was what
tempted Grete to make Gregor’s situation seem even more shocking than
it was so that she could do even more for him. Grete would probably be
the only one who would dare enter a room dominated by Gregor crawling
about the bare walls by himself.

So she refused to let her mother dissuade her. Gregor’s mother already
looked uneasy in his room, she soon stopped speaking and helped
Gregor’s sister to get the chest of drawers out with what strength she
had. The chest of drawers was something that Gregor could do without if
he had to, but the writing desk had to stay. Hardly had the two women
pushed the chest of drawers, groaning, out of the room than Gregor
poked his head out from under the couch to see what he could do about
it. He meant to be as careful and considerate as he could, but,
unfortunately, it was his mother who came back first while Grete in the
next room had her arms round the chest, pushing and pulling at it from
side to side by herself without, of course, moving it an inch. His
mother was not used to the sight of Gregor, he might have made her ill,
so Gregor hurried backwards to the far end of the couch. In his
startlement, though, he was not able to prevent the sheet at its front
from moving a little. It was enough to attract his mother’s attention.
She stood very still, remained there a moment, and then went back out
to Grete.

Gregor kept trying to assure himself that nothing unusual was
happening, it was just a few pieces of furniture being moved after all,
but he soon had to admit that the women going to and fro, their little
calls to each other, the scraping of the furniture on the floor, all
these things made him feel as if he were being assailed from all sides.
With his head and legs pulled in against him and his body pressed to
the floor, he was forced to admit to himself that he could not stand
all of this much longer. They were emptying his room out; taking away
everything that was dear to him; they had already taken out the chest
containing his fretsaw and other tools; now they threatened to remove
the writing desk with its place clearly worn into the floor, the desk
where he had done his homework as a business trainee, at high school,
even while he had been at infant school—he really could not wait any
longer to see whether the two women’s intentions were good. He had
nearly forgotten they were there anyway, as they were now too tired to
say anything while they worked and he could only hear their feet as
they stepped heavily on the floor.

So, while the women were leant against the desk in the other room
catching their breath, he sallied out, changed direction four times not
knowing what he should save first before his attention was suddenly
caught by the picture on the wall—which was already denuded of
everything else that had been on it—of the lady dressed in copious fur.
He hurried up onto the picture and pressed himself against its glass,
it held him firmly and felt good on his hot belly. This picture at
least, now totally covered by Gregor, would certainly be taken away by
no-one. He turned his head to face the door into the living room so
that he could watch the women when they came back.

They had not allowed themselves a long rest and came back quite soon;
Grete had put her arm around her mother and was nearly carrying her.
“What shall we take now, then?”, said Grete and looked around. Her eyes
met those of Gregor on the wall. Perhaps only because her mother was
there, she remained calm, bent her face to her so that she would not
look round and said, albeit hurriedly and with a tremor in her voice:
“Come on, let’s go back in the living room for a while?” Gregor could
see what Grete had in mind, she wanted to take her mother somewhere
safe and then chase him down from the wall. Well, she could certainly
try it! He sat unyielding on his picture. He would rather jump at
Grete’s face.

But Grete’s words had made her mother quite worried, she stepped to one
side, saw the enormous brown patch against the flowers of the
wallpaper, and before she even realised it was Gregor that she saw
screamed: “Oh God, oh God!” Arms outstretched, she fell onto the couch
as if she had given up everything and stayed there immobile. “Gregor!”
shouted his sister, glowering at him and shaking her fist. That was the
first word she had spoken to him directly since his transformation. She
ran into the other room to fetch some kind of smelling salts to bring
her mother out of her faint; Gregor wanted to help too—he could save
his picture later, although he stuck fast to the glass and had to pull
himself off by force; then he, too, ran into the next room as if he
could advise his sister like in the old days; but he had to just stand
behind her doing nothing; she was looking into various bottles, he
startled her when she turned round; a bottle fell to the ground and
broke; a splinter cut Gregor’s face, some kind of caustic medicine
splashed all over him; now, without delaying any longer, Grete took
hold of all the bottles she could and ran with them in to her mother;
she slammed the door shut with her foot. So now Gregor was shut out
from his mother, who, because of him, might be near to death; he could
not open the door if he did not want to chase his sister away, and she
had to stay with his mother; there was nothing for him to do but wait;
and, oppressed with anxiety and self-reproach, he began to crawl about,
he crawled over everything, walls, furniture, ceiling, and finally in
his confusion as the whole room began to spin around him he fell down
into the middle of the dinner table.

He lay there for a while, numb and immobile, all around him it was
quiet, maybe that was a good sign. Then there was someone at the door.
The maid, of course, had locked herself in her kitchen so that Grete
would have to go and answer it. His father had arrived home. “What’s
happened?” were his first words; Grete’s appearance must have made
everything clear to him. She answered him with subdued voice, and
openly pressed her face into his chest: “Mother’s fainted, but she’s
better now. Gregor got out.” “Just as I expected”, said his father,
“just as I always said, but you women wouldn’t listen, would you.” It
was clear to Gregor that Grete had not said enough and that his father
took it to mean that something bad had happened, that he was
responsible for some act of violence. That meant Gregor would now have
to try to calm his father, as he did not have the time to explain
things to him even if that had been possible. So he fled to the door of
his room and pressed himself against it so that his father, when he
came in from the hall, could see straight away that Gregor had the best
intentions and would go back into his room without delay, that it would
not be necessary to drive him back but that they had only to open the
door and he would disappear.

His father, though, was not in the mood to notice subtleties like that;
“Ah!”, he shouted as he came in, sounding as if he were both angry and
glad at the same time. Gregor drew his head back from the door and
lifted it towards his father. He really had not imagined his father the
way he stood there now; of late, with his new habit of crawling about,
he had neglected to pay attention to what was going on the rest of the
flat the way he had done before. He really ought to have expected
things to have changed, but still, still, was that really his father?
The same tired man as used to be laying there entombed in his bed when
Gregor came back from his business trips, who would receive him sitting
in the armchair in his nightgown when he came back in the evenings; who
was hardly even able to stand up but, as a sign of his pleasure, would
just raise his arms and who, on the couple of times a year when they
went for a walk together on a Sunday or public holiday wrapped up
tightly in his overcoat between Gregor and his mother, would always
labour his way forward a little more slowly than them, who were already
walking slowly for his sake; who would place his stick down carefully
and, if he wanted to say something would invariably stop and gather his
companions around him. He was standing up straight enough now; dressed
in a smart blue uniform with gold buttons, the sort worn by the
employees at the banking institute; above the high, stiff collar of the
coat his strong double-chin emerged; under the bushy eyebrows, his
piercing, dark eyes looked out fresh and alert; his normally unkempt
white hair was combed down painfully close to his scalp. He took his
cap, with its gold monogram from, probably, some bank, and threw it in
an arc right across the room onto the sofa, put his hands in his
trouser pockets, pushing back the bottom of his long uniform coat, and,
with look of determination, walked towards Gregor. He probably did not
even know himself what he had in mind, but nonetheless lifted his feet
unusually high. Gregor was amazed at the enormous size of the soles of
his boots, but wasted no time with that—he knew full well, right from
the first day of his new life, that his father thought it necessary to
always be extremely strict with him. And so he ran up to his father,
stopped when his father stopped, scurried forwards again when he moved,
even slightly. In this way they went round the room several times
without anything decisive happening, without even giving the impression
of a chase as everything went so slowly. Gregor remained all this time
on the floor, largely because he feared his father might see it as
especially provoking if he fled onto the wall or ceiling. Whatever he
did, Gregor had to admit that he certainly would not be able to keep up
this running about for long, as for each step his father took he had to
carry out countless movements. He became noticeably short of breath,
even in his earlier life his lungs had not been very reliable. Now, as
he lurched about in his efforts to muster all the strength he could for
running he could hardly keep his eyes open; his thoughts became too
slow for him to think of any other way of saving himself than running;
he almost forgot that the walls were there for him to use although,
here, they were concealed behind carefully carved furniture full of
notches and protrusions—then, right beside him, lightly tossed,
something flew down and rolled in front of him. It was an apple; then
another one immediately flew at him; Gregor froze in shock; there was
no longer any point in running as his father had decided to bombard
him. He had filled his pockets with fruit from the bowl on the
sideboard and now, without even taking the time for careful aim, threw
one apple after another. These little, red apples rolled about on the
floor, knocking into each other as if they had electric motors. An
apple thrown without much force glanced against Gregor’s back and slid
off without doing any harm. Another one however, immediately following
it, hit squarely and lodged in his back; Gregor wanted to drag himself
away, as if he could remove the surprising, the incredible pain by
changing his position; but he felt as if nailed to the spot and spread
himself out, all his senses in confusion. The last thing he saw was the
door of his room being pulled open, his sister was screaming, his
mother ran out in front of her in her blouse (as his sister had taken
off some of her clothes after she had fainted to make it easier for her
to breathe), she ran to his father, her skirts unfastened and sliding
one after another to the ground, stumbling over the skirts she pushed
herself to his father, her arms around him, uniting herself with him
totally—now Gregor lost his ability to see anything—her hands behind
his father’s head begging him to spare Gregor’s life.




III


No-one dared to remove the apple lodged in Gregor’s flesh, so it
remained there as a visible reminder of his injury. He had suffered it
there for more than a month, and his condition seemed serious enough to
remind even his father that Gregor, despite his current sad and
revolting form, was a family member who could not be treated as an
enemy. On the contrary, as a family there was a duty to swallow any
revulsion for him and to be patient, just to be patient.

Because of his injuries, Gregor had lost much of his mobility—probably
permanently. He had been reduced to the condition of an ancient invalid
and it took him long, long minutes to crawl across his room—crawling
over the ceiling was out of the question—but this deterioration in his
condition was fully (in his opinion) made up for by the door to the
living room being left open every evening. He got into the habit of
closely watching it for one or two hours before it was opened and then,
lying in the darkness of his room where he could not be seen from the
living room, he could watch the family in the light of the dinner table
and listen to their conversation—with everyone’s permission, in a way,
and thus quite differently from before.

They no longer held the lively conversations of earlier times, of
course, the ones that Gregor always thought about with longing when he
was tired and getting into the damp bed in some small hotel room. All
of them were usually very quiet nowadays. Soon after dinner, his father
would go to sleep in his chair; his mother and sister would urge each
other to be quiet; his mother, bent deeply under the lamp, would sew
fancy underwear for a fashion shop; his sister, who had taken a sales
job, learned shorthand and French in the evenings so that she might be
able to get a better position later on. Sometimes his father would wake
up and say to Gregor’s mother “you’re doing so much sewing again
today!”, as if he did not know that he had been dozing—and then he
would go back to sleep again while mother and sister would exchange a
tired grin.

With a kind of stubbornness, Gregor’s father refused to take his
uniform off even at home; while his nightgown hung unused on its peg
Gregor’s father would slumber where he was, fully dressed, as if always
ready to serve and expecting to hear the voice of his superior even
here. The uniform had not been new to start with, but as a result of
this it slowly became even shabbier despite the efforts of Gregor’s
mother and sister to look after it. Gregor would often spend the whole
evening looking at all the stains on this coat, with its gold buttons
always kept polished and shiny, while the old man in it would sleep,
highly uncomfortable but peaceful.

As soon as it struck ten, Gregor’s mother would speak gently to his
father to wake him and try to persuade him to go to bed, as he couldn’t
sleep properly where he was and he really had to get his sleep if he
was to be up at six to get to work. But since he had been in work he
had become more obstinate and would always insist on staying longer at
the table, even though he regularly fell asleep and it was then harder
than ever to persuade him to exchange the chair for his bed. Then,
however much mother and sister would importune him with little
reproaches and warnings he would keep slowly shaking his head for a
quarter of an hour with his eyes closed and refusing to get up.
Gregor’s mother would tug at his sleeve, whisper endearments into his
ear, Gregor’s sister would leave her work to help her mother, but
nothing would have any effect on him. He would just sink deeper into
his chair. Only when the two women took him under the arms he would
abruptly open his eyes, look at them one after the other and say: “What
a life! This is what peace I get in my old age!” And supported by the
two women he would lift himself up carefully as if he were carrying the
greatest load himself, let the women take him to the door, send them
off and carry on by himself while Gregor’s mother would throw down her
needle and his sister her pen so that they could run after his father
and continue being of help to him.

Who, in this tired and overworked family, would have had time to give
more attention to Gregor than was absolutely necessary? The household
budget became even smaller; so now the maid was dismissed; an enormous,
thick-boned charwoman with white hair that flapped around her head came
every morning and evening to do the heaviest work; everything else was
looked after by Gregor’s mother on top of the large amount of sewing
work she did. Gregor even learned, listening to the evening
conversation about what price they had hoped for, that several items of
jewellery belonging to the family had been sold, even though both
mother and sister had been very fond of wearing them at functions and
celebrations. But the loudest complaint was that although the flat was
much too big for their present circumstances, they could not move out
of it, there was no imaginable way of transferring Gregor to the new
address. He could see quite well, though, that there were more reasons
than consideration for him that made it difficult for them to move, it
would have been quite easy to transport him in any suitable crate with
a few air holes in it; the main thing holding the family back from
their decision to move was much more to do with their total despair,
and the thought that they had been struck with a misfortune unlike
anything experienced by anyone else they knew or were related to. They
carried out absolutely everything that the world expects from poor
people, Gregor’s father brought bank employees their breakfast, his
mother sacrificed herself by washing clothes for strangers, his sister
ran back and forth behind her desk at the behest of the customers, but
they just did not have the strength to do any more. And the injury in
Gregor’s back began to hurt as much as when it was new. After they had
come back from taking his father to bed Gregor’s mother and sister
would now leave their work where it was and sit close together, cheek
to cheek; his mother would point to Gregor’s room and say “Close that
door, Grete”, and then, when he was in the dark again, they would sit
in the next room and their tears would mingle, or they would simply sit
there staring dry-eyed at the table.

Gregor hardly slept at all, either night or day. Sometimes he would
think of taking over the family’s affairs, just like before, the next
time the door was opened; he had long forgotten about his boss and the
chief clerk, but they would appear again in his thoughts, the salesmen
and the apprentices, that stupid teaboy, two or three friends from
other businesses, one of the chambermaids from a provincial hotel, a
tender memory that appeared and disappeared again, a cashier from a hat
shop for whom his attention had been serious but too slow,—all of them
appeared to him, mixed together with strangers and others he had
forgotten, but instead of helping him and his family they were all of
them inaccessible, and he was glad when they disappeared. Other times
he was not at all in the mood to look after his family, he was filled
with simple rage about the lack of attention he was shown, and although
he could think of nothing he would have wanted, he made plans of how he
could get into the pantry where he could take all the things he was
entitled to, even if he was not hungry. Gregor’s sister no longer
thought about how she could please him but would hurriedly push some
food or other into his room with her foot before she rushed out to work
in the morning and at midday, and in the evening she would sweep it
away again with the broom, indifferent as to whether it had been eaten
or—more often than not—had been left totally untouched. She still
cleared up the room in the evening, but now she could not have been any
quicker about it. Smears of dirt were left on the walls, here and there
were little balls of dust and filth. At first, Gregor went into one of
the worst of these places when his sister arrived as a reproach to her,
but he could have stayed there for weeks without his sister doing
anything about it; she could see the dirt as well as he could but she
had simply decided to leave him to it. At the same time she became
touchy in a way that was quite new for her and which everyone in the
family understood—cleaning up Gregor’s room was for her and her alone.
Gregor’s mother did once thoroughly clean his room, and needed to use
several bucketfuls of water to do it—although that much dampness also
made Gregor ill and he lay flat on the couch, bitter and immobile. But
his mother was to be punished still more for what she had done, as
hardly had his sister arrived home in the evening than she noticed the
change in Gregor’s room and, highly aggrieved, ran back into the living
room where, despite her mothers raised and imploring hands, she broke
into convulsive tears. Her father, of course, was startled out of his
chair and the two parents looked on astonished and helpless; then they,
too, became agitated; Gregor’s father, standing to the right of his
mother, accused her of not leaving the cleaning of Gregor’s room to his
sister; from her left, Gregor’s sister screamed at her that she was
never to clean Gregor’s room again; while his mother tried to draw his
father, who was beside himself with anger, into the bedroom; his
sister, quaking with tears, thumped on the table with her small fists;
and Gregor hissed in anger that no-one had even thought of closing the
door to save him the sight of this and all its noise.

Gregor’s sister was exhausted from going out to work, and looking after
Gregor as she had done before was even more work for her, but even so
his mother ought certainly not to have taken her place. Gregor, on the
other hand, ought not to be neglected. Now, though, the charwoman was
here. This elderly widow, with a robust bone structure that made her
able to withstand the hardest of things in her long life, wasn’t really
repelled by Gregor. Just by chance one day, rather than any real
curiosity, she opened the door to Gregor’s room and found herself face
to face with him. He was taken totally by surprise, no-one was chasing
him but he began to rush to and fro while she just stood there in
amazement with her hands crossed in front of her. From then on she
never failed to open the door slightly every evening and morning and
look briefly in on him. At first she would call to him as she did so
with words that she probably considered friendly, such as “come on
then, you old dung-beetle!”, or “look at the old dung-beetle there!”
Gregor never responded to being spoken to in that way, but just
remained where he was without moving as if the door had never even been
opened. If only they had told this charwoman to clean up his room every
day instead of letting her disturb him for no reason whenever she felt
like it! One day, early in the morning while a heavy rain struck the
windowpanes, perhaps indicating that spring was coming, she began to
speak to him in that way once again. Gregor was so resentful of it that
he started to move toward her, he was slow and infirm, but it was like
a kind of attack. Instead of being afraid, the charwoman just lifted up
one of the chairs from near the door and stood there with her mouth
open, clearly intending not to close her mouth until the chair in her
hand had been slammed down into Gregor’s back. “Aren’t you coming any
closer, then?”, she asked when Gregor turned round again, and she
calmly put the chair back in the corner.

Gregor had almost entirely stopped eating. Only if he happened to find
himself next to the food that had been prepared for him he might take
some of it into his mouth to play with it, leave it there a few hours
and then, more often than not, spit it out again. At first he thought
it was distress at the state of his room that stopped him eating, but
he had soon got used to the changes made there. They had got into the
habit of putting things into this room that they had no room for
anywhere else, and there were now many such things as one of the rooms
in the flat had been rented out to three gentlemen. These earnest
gentlemen—all three of them had full beards, as Gregor learned peering
through the crack in the door one day—were painfully insistent on
things’ being tidy. This meant not only in their own room but, since
they had taken a room in this establishment, in the entire flat and
especially in the kitchen. Unnecessary clutter was something they could
not tolerate, especially if it was dirty. They had moreover brought
most of their own furnishings and equipment with them. For this reason,
many things had become superfluous which, although they could not be
sold, the family did not wish to discard. All these things found their
way into Gregor’s room. The dustbins from the kitchen found their way
in there too. The charwoman was always in a hurry, and anything she
couldn’t use for the time being she would just chuck in there. He,
fortunately, would usually see no more than the object and the hand
that held it. The woman most likely meant to fetch the things back out
again when she had time and the opportunity, or to throw everything out
in one go, but what actually happened was that they were left where
they landed when they had first been thrown unless Gregor made his way
through the junk and moved it somewhere else. At first he moved it
because, with no other room free where he could crawl about, he was
forced to, but later on he came to enjoy it although moving about in
that way left him sad and tired to death, and he would remain immobile
for hours afterwards.

The gentlemen who rented the room would sometimes take their evening
meal at home in the living room that was used by everyone, and so the
door to this room was often kept closed in the evening. But Gregor
found it easy to give up having the door open, he had, after all, often
failed to make use of it when it was open and, without the family
having noticed it, lain in his room in its darkest corner. One time,
though, the charwoman left the door to the living room slightly open,
and it remained open when the gentlemen who rented the room came in in
the evening and the light was put on. They sat up at the table where,
formerly, Gregor had taken his meals with his father and mother, they
unfolded the serviettes and picked up their knives and forks. Gregor’s
mother immediately appeared in the doorway with a dish of meat and soon
behind her came his sister with a dish piled high with potatoes. The
food was steaming, and filled the room with its smell. The gentlemen
bent over the dishes set in front of them as if they wanted to test the
food before eating it, and the gentleman in the middle, who seemed to
count as an authority for the other two, did indeed cut off a piece of
meat while it was still in its dish, clearly wishing to establish
whether it was sufficiently cooked or whether it should be sent back to
the kitchen. It was to his satisfaction, and Gregor’s mother and
sister, who had been looking on anxiously, began to breathe again and
smiled.

The family themselves ate in the kitchen. Nonetheless, Gregor’s father
came into the living room before he went into the kitchen, bowed once
with his cap in his hand and did his round of the table. The gentlemen
stood as one, and mumbled something into their beards. Then, once they
were alone, they ate in near perfect silence. It seemed remarkable to
Gregor that above all the various noises of eating their chewing teeth
could still be heard, as if they had wanted to show Gregor that you
need teeth in order to eat and it was not possible to perform anything
with jaws that are toothless however nice they might be. “I’d like to
eat something”, said Gregor anxiously, “but not anything like they’re
eating. They do feed themselves. And here I am, dying!”

Throughout all this time, Gregor could not remember having heard the
violin being played, but this evening it began to be heard from the
kitchen. The three gentlemen had already finished their meal, the one
in the middle had produced a newspaper, given a page to each of the
others, and now they leant back in their chairs reading them and
smoking. When the violin began playing they became attentive, stood up
and went on tip-toe over to the door of the hallway where they stood
pressed against each other. Someone must have heard them in the
kitchen, as Gregor’s father called out: “Is the playing perhaps
unpleasant for the gentlemen? We can stop it straight away.” “On the
contrary”, said the middle gentleman, “would the young lady not like to
come in and play for us here in the room, where it is, after all, much
more cosy and comfortable?” “Oh yes, we’d love to”, called back
Gregor’s father as if he had been the violin player himself. The
gentlemen stepped back into the room and waited. Gregor’s father soon
appeared with the music stand, his mother with the music and his sister
with the violin. She calmly prepared everything for her to begin
playing; his parents, who had never rented a room out before and
therefore showed an exaggerated courtesy towards the three gentlemen,
did not even dare to sit on their own chairs; his father leant against
the door with his right hand pushed in between two buttons on his
uniform coat; his mother, though, was offered a seat by one of the
gentlemen and sat—leaving the chair where the gentleman happened to
have placed it—out of the way in a corner.

His sister began to play; father and mother paid close attention, one
on each side, to the movements of her hands. Drawn in by the playing,
Gregor had dared to come forward a little and already had his head in
the living room. Before, he had taken great pride in how considerate he
was but now it hardly occurred to him that he had become so thoughtless
about the others. What’s more, there was now all the more reason to
keep himself hidden as he was covered in the dust that lay everywhere
in his room and flew up at the slightest movement; he carried threads,
hairs, and remains of food about on his back and sides; he was much too
indifferent to everything now to lay on his back and wipe himself on
the carpet like he had used to do several times a day. And despite this
condition, he was not too shy to move forward a little onto the
immaculate floor of the living room.

No-one noticed him, though. The family was totally preoccupied with the
violin playing; at first, the three gentlemen had put their hands in
their pockets and come up far too close behind the music stand to look
at all the notes being played, and they must have disturbed Gregor’s
sister, but soon, in contrast with the family, they withdrew back to
the window with their heads sunk and talking to each other at half
volume, and they stayed by the window while Gregor’s father observed
them anxiously. It really now seemed very obvious that they had
expected to hear some beautiful or entertaining violin playing but had
been disappointed, that they had had enough of the whole performance
and it was only now out of politeness that they allowed their peace to
be disturbed. It was especially unnerving, the way they all blew the
smoke from their cigarettes upwards from their mouth and noses. Yet
Gregor’s sister was playing so beautifully. Her face was leant to one
side, following the lines of music with a careful and melancholy
expression. Gregor crawled a little further forward, keeping his head
close to the ground so that he could meet her eyes if the chance came.
Was he an animal if music could captivate him so? It seemed to him that
he was being shown the way to the unknown nourishment he had been
yearning for. He was determined to make his way forward to his sister
and tug at her skirt to show her she might come into his room with her
violin, as no-one appreciated her playing here as much as he would. He
never wanted to let her out of his room, not while he lived, anyway;
his shocking appearance should, for once, be of some use to him; he
wanted to be at every door of his room at once to hiss and spit at the
attackers; his sister should not be forced to stay with him, though,
but stay of her own free will; she would sit beside him on the couch
with her ear bent down to him while he told her how he had always
intended to send her to the conservatory, how he would have told
everyone about it last Christmas—had Christmas really come and gone
already?—if this misfortune hadn’t got in the way, and refuse to let
anyone dissuade him from it. On hearing all this, his sister would
break out in tears of emotion, and Gregor would climb up to her
shoulder and kiss her neck, which, since she had been going out to
work, she had kept free without any necklace or collar.

“Mr. Samsa!”, shouted the middle gentleman to Gregor’s father,
pointing, without wasting any more words, with his forefinger at Gregor
as he slowly moved forward. The violin went silent, the middle of the
three gentlemen first smiled at his two friends, shaking his head, and
then looked back at Gregor. His father seemed to think it more
important to calm the three gentlemen before driving Gregor out, even
though they were not at all upset and seemed to think Gregor was more
entertaining than the violin playing had been. He rushed up to them
with his arms spread out and attempted to drive them back into their
room at the same time as trying to block their view of Gregor with his
body. Now they did become a little annoyed, and it was not clear
whether it was his father’s behaviour that annoyed them or the dawning
realisation that they had had a neighbour like Gregor in the next room
without knowing it. They asked Gregor’s father for explanations, raised
their arms like he had, tugged excitedly at their beards and moved back
towards their room only very slowly. Meanwhile Gregor’s sister had
overcome the despair she had fallen into when her playing was suddenly
interrupted. She had let her hands drop and let violin and bow hang
limply for a while but continued to look at the music as if still
playing, but then she suddenly pulled herself together, lay the
instrument on her mother’s lap who still sat laboriously struggling for
breath where she was, and ran into the next room which, under pressure
from her father, the three gentlemen were more quickly moving toward.
Under his sister’s experienced hand, the pillows and covers on the beds
flew up and were put into order and she had already finished making the
beds and slipped out again before the three gentlemen had reached the
room. Gregor’s father seemed so obsessed with what he was doing that he
forgot all the respect he owed to his tenants. He urged them and
pressed them until, when he was already at the door of the room, the
middle of the three gentlemen shouted like thunder and stamped his foot
and thereby brought Gregor’s father to a halt. “I declare here and
now”, he said, raising his hand and glancing at Gregor’s mother and
sister to gain their attention too, “that with regard to the repugnant
conditions that prevail in this flat and with this family”—here he
looked briefly but decisively at the floor—“I give immediate notice on
my room. For the days that I have been living here I will, of course,
pay nothing at all, on the contrary I will consider whether to proceed
with some kind of action for damages from you, and believe me it would
be very easy to set out the grounds for such an action.” He was silent
and looked straight ahead as if waiting for something. And indeed, his
two friends joined in with the words: “And we also give immediate
notice.” With that, he took hold of the door handle and slammed the
door.

Gregor’s father staggered back to his seat, feeling his way with his
hands, and fell into it; it looked as if he was stretching himself out
for his usual evening nap but from the uncontrolled way his head kept
nodding it could be seen that he was not sleeping at all. Throughout
all this, Gregor had lain still where the three gentlemen had first
seen him. His disappointment at the failure of his plan, and perhaps
also because he was weak from hunger, made it impossible for him to
move. He was sure that everyone would turn on him any moment, and he
waited. He was not even startled out of this state when the violin on
his mother’s lap fell from her trembling fingers and landed loudly on
the floor.

“Father, Mother”, said his sister, hitting the table with her hand as
introduction, “we can’t carry on like this. Maybe you can’t see it, but
I can. I don’t want to call this monster my brother, all I can say is:
we have to try and get rid of it. We’ve done all that’s humanly
possible to look after it and be patient, I don’t think anyone could
accuse us of doing anything wrong.”

“She’s absolutely right”, said Gregor’s father to himself. His mother,
who still had not had time to catch her breath, began to cough dully,
her hand held out in front of her and a deranged expression in her
eyes.

Gregor’s sister rushed to his mother and put her hand on her forehead.
Her words seemed to give Gregor’s father some more definite ideas. He
sat upright, played with his uniform cap between the plates left by the
three gentlemen after their meal, and occasionally looked down at
Gregor as he lay there immobile.

“We have to try and get rid of it”, said Gregor’s sister, now speaking
only to her father, as her mother was too occupied with coughing to
listen, “it’ll be the death of both of you, I can see it coming. We
can’t all work as hard as we have to and then come home to be tortured
like this, we can’t endure it. I can’t endure it any more.” And she
broke out so heavily in tears that they flowed down the face of her
mother, and she wiped them away with mechanical hand movements.

“My child”, said her father with sympathy and obvious understanding,
“what are we to do?”

His sister just shrugged her shoulders as a sign of the helplessness
and tears that had taken hold of her, displacing her earlier certainty.

“If he could just understand us”, said his father almost as a question;
his sister shook her hand vigorously through her tears as a sign that
of that there was no question.

“If he could just understand us”, repeated Gregor’s father, closing his
eyes in acceptance of his sister’s certainty that that was quite
impossible, “then perhaps we could come to some kind of arrangement
with him. But as it is ...”

“It’s got to go”, shouted his sister, “that’s the only way, Father.
You’ve got to get rid of the idea that that’s Gregor. We’ve only harmed
ourselves by believing it for so long. How can that be Gregor? If it
were Gregor he would have seen long ago that it’s not possible for
human beings to live with an animal like that and he would have gone of
his own free will. We wouldn’t have a brother any more, then, but we
could carry on with our lives and remember him with respect. As it is
this animal is persecuting us, it’s driven out our tenants, it
obviously wants to take over the whole flat and force us to sleep on
the streets. Father, look, just look”, she suddenly screamed, “he’s
starting again!” In her alarm, which was totally beyond Gregor’s
comprehension, his sister even abandoned his mother as she pushed
herself vigorously out of her chair as if more willing to sacrifice her
own mother than stay anywhere near Gregor. She rushed over to behind
her father, who had become excited merely because she was and stood up
half raising his hands in front of Gregor’s sister as if to protect
her.

But Gregor had had no intention of frightening anyone, least of all his
sister. All he had done was begin to turn round so that he could go
back into his room, although that was in itself quite startling as his
pain-wracked condition meant that turning round required a great deal
of effort and he was using his head to help himself do it, repeatedly
raising it and striking it against the floor. He stopped and looked
round. They seemed to have realised his good intention and had only
been alarmed briefly. Now they all looked at him in unhappy silence.
His mother lay in her chair with her legs stretched out and pressed
against each other, her eyes nearly closed with exhaustion; his sister
sat next to his father with her arms around his neck.

“Maybe now they’ll let me turn round”, thought Gregor and went back to
work. He could not help panting loudly with the effort and had
sometimes to stop and take a rest. No-one was making him rush any more,
everything was left up to him. As soon as he had finally finished
turning round he began to move straight ahead. He was amazed at the
great distance that separated him from his room, and could not
understand how he had covered that distance in his weak state a little
while before and almost without noticing it. He concentrated on
crawling as fast as he could and hardly noticed that there was not a
word, not any cry, from his family to distract him. He did not turn his
head until he had reached the doorway. He did not turn it all the way
round as he felt his neck becoming stiff, but it was nonetheless enough
to see that nothing behind him had changed, only his sister had stood
up. With his last glance he saw that his mother had now fallen
completely asleep.

He was hardly inside his room before the door was hurriedly shut,
bolted and locked. The sudden noise behind Gregor so startled him that
his little legs collapsed under him. It was his sister who had been in
so much of a rush. She had been standing there waiting and sprung
forward lightly, Gregor had not heard her coming at all, and as she
turned the key in the lock she said loudly to her parents “At last!”.

“What now, then?”, Gregor asked himself as he looked round in the
darkness. He soon made the discovery that he could no longer move at
all. This was no surprise to him, it seemed rather that being able to
actually move around on those spindly little legs until then was
unnatural. He also felt relatively comfortable. It is true that his
entire body was aching, but the pain seemed to be slowly getting weaker
and weaker and would finally disappear altogether. He could already
hardly feel the decayed apple in his back or the inflamed area around
it, which was entirely covered in white dust. He thought back of his
family with emotion and love. If it was possible, he felt that he must
go away even more strongly than his sister. He remained in this state
of empty and peaceful rumination until he heard the clock tower strike
three in the morning. He watched as it slowly began to get light
everywhere outside the window too. Then, without his willing it, his
head sank down completely, and his last breath flowed weakly from his
nostrils.

When the cleaner came in early in the morning—they’d often asked her
not to keep slamming the doors but with her strength and in her hurry
she still did, so that everyone in the flat knew when she’d arrived and
from then on it was impossible to sleep in peace—she made her usual
brief look in on Gregor and at first found nothing special. She thought
he was laying there so still on purpose, playing the martyr; she
attributed all possible understanding to him. She happened to be
holding the long broom in her hand, so she tried to tickle Gregor with
it from the doorway. When she had no success with that she tried to
make a nuisance of herself and poked at him a little, and only when she
found she could shove him across the floor with no resistance at all
did she start to pay attention. She soon realised what had really
happened, opened her eyes wide, whistled to herself, but did not waste
time to yank open the bedroom doors and shout loudly into the darkness
of the bedrooms: “Come and ’ave a look at this, it’s dead, just lying
there, stone dead!”

Mr. and Mrs. Samsa sat upright there in their marriage bed and had to
make an effort to get over the shock caused by the cleaner before they
could grasp what she was saying. But then, each from his own side, they
hurried out of bed. Mr. Samsa threw the blanket over his shoulders,
Mrs. Samsa just came out in her nightdress; and that is how they went
into Gregor’s room. On the way they opened the door to the living room
where Grete had been sleeping since the three gentlemen had moved in;
she was fully dressed as if she had never been asleep, and the paleness
of her face seemed to confirm this. “Dead?”, asked Mrs. Samsa, looking
at the charwoman enquiringly, even though she could have checked for
herself and could have known it even without checking. “That’s what I
said”, replied the cleaner, and to prove it she gave Gregor’s body
another shove with the broom, sending it sideways across the floor.
Mrs. Samsa made a movement as if she wanted to hold back the broom, but
did not complete it. “Now then”, said Mr. Samsa, “let’s give thanks to
God for that”. He crossed himself, and the three women followed his
example. Grete, who had not taken her eyes from the corpse, said: “Just
look how thin he was. He didn’t eat anything for so long. The food came
out again just the same as when it went in”. Gregor’s body was indeed
completely dried up and flat, they had not seen it until then, but now
he was not lifted up on his little legs, nor did he do anything to make
them look away.

“Grete, come with us in here for a little while”, said Mrs. Samsa with
a pained smile, and Grete followed her parents into the bedroom but not
without looking back at the body. The cleaner shut the door and opened
the window wide. Although it was still early in the morning the fresh
air had something of warmth mixed in with it. It was already the end of
March, after all.

The three gentlemen stepped out of their room and looked round in
amazement for their breakfasts; they had been forgotten about. “Where
is our breakfast?”, the middle gentleman asked the cleaner irritably.
She just put her finger on her lips and made a quick and silent sign to
the men that they might like to come into Gregor’s room. They did so,
and stood around Gregor’s corpse with their hands in the pockets of
their well-worn coats. It was now quite light in the room.

Then the door of the bedroom opened and Mr. Samsa appeared in his
uniform with his wife on one arm and his daughter on the other. All of
them had been crying a little; Grete now and then pressed her face
against her father’s arm.

“Leave my home. Now!”, said Mr. Samsa, indicating the door and without
letting the women from him. “What do you mean?”, asked the middle of
the three gentlemen somewhat disconcerted, and he smiled sweetly. The
other two held their hands behind their backs and continually rubbed
them together in gleeful anticipation of a loud quarrel which could
only end in their favour. “I mean just what I said”, answered Mr.
Samsa, and, with his two companions, went in a straight line towards
the man. At first, he stood there still, looking at the ground as if
the contents of his head were rearranging themselves into new
positions. “Alright, we’ll go then”, he said, and looked up at Mr.
Samsa as if he had been suddenly overcome with humility and wanted
permission again from Mr. Samsa for his decision. Mr. Samsa merely
opened his eyes wide and briefly nodded to him several times. At that,
and without delay, the man actually did take long strides into the
front hallway; his two friends had stopped rubbing their hands some
time before and had been listening to what was being said. Now they
jumped off after their friend as if taken with a sudden fear that Mr.
Samsa might go into the hallway in front of them and break the
connection with their leader. Once there, all three took their hats
from the stand, took their sticks from the holder, bowed without a word
and left the premises. Mr. Samsa and the two women followed them out
onto the landing; but they had had no reason to mistrust the men’s
intentions and as they leaned over the landing they saw how the three
gentlemen made slow but steady progress down the many steps. As they
turned the corner on each floor they disappeared and would reappear a
few moments later; the further down they went, the more that the Samsa
family lost interest in them; when a butcher’s boy, proud of posture
with his tray on his head, passed them on his way up and came nearer
than they were, Mr. Samsa and the women came away from the landing and
went, as if relieved, back into the flat.

They decided the best way to make use of that day was for relaxation
and to go for a walk; not only had they earned a break from work but
they were in serious need of it. So they sat at the table and wrote
three letters of excusal, Mr. Samsa to his employers, Mrs. Samsa to her
contractor and Grete to her principal. The cleaner came in while they
were writing to tell them she was going, she’d finished her work for
that morning. The three of them at first just nodded without looking up
from what they were writing, and it was only when the cleaner still did
not seem to want to leave that they looked up in irritation. “Well?”,
asked Mr. Samsa. The charwoman stood in the doorway with a smile on her
face as if she had some tremendous good news to report, but would only
do it if she was clearly asked to. The almost vertical little ostrich
feather on her hat, which had been a source of irritation to Mr. Samsa
all the time she had been working for them, swayed gently in all
directions. “What is it you want then?”, asked Mrs. Samsa, whom the
cleaner had the most respect for. “Yes”, she answered, and broke into a
friendly laugh that made her unable to speak straight away, “well then,
that thing in there, you needn’t worry about how you’re going to get
rid of it. That’s all been sorted out.” Mrs. Samsa and Grete bent down
over their letters as if intent on continuing with what they were
writing; Mr. Samsa saw that the cleaner wanted to start describing
everything in detail but, with outstretched hand, he made it quite
clear that she was not to. So, as she was prevented from telling them
all about it, she suddenly remembered what a hurry she was in and,
clearly peeved, called out “Cheerio then, everyone”, turned round
sharply and left, slamming the door terribly as she went.

“Tonight she gets sacked”, said Mr. Samsa, but he received no reply
from either his wife or his daughter as the charwoman seemed to have
destroyed the peace they had only just gained. They got up and went
over to the window where they remained with their arms around each
other. Mr. Samsa twisted round in his chair to look at them and sat
there watching for a while. Then he called out: “Come here, then. Let’s
forget about all that old stuff, shall we. Come and give me a bit of
attention”. The two women immediately did as he said, hurrying over to
him where they kissed him and hugged him and then they quickly finished
their letters.

After that, the three of them left the flat together, which was
something they had not done for months, and took the tram out to the
open country outside the town. They had the tram, filled with warm
sunshine, all to themselves. Leant back comfortably on their seats,
they discussed their prospects and found that on closer examination
they were not at all bad—until then they had never asked each other
about their work but all three had jobs which were very good and held
particularly good promise for the future. The greatest improvement for
the time being, of course, would be achieved quite easily by moving
house; what they needed now was a flat that was smaller and cheaper
than the current one which had been chosen by Gregor, one that was in a
better location and, most of all, more practical. All the time, Grete
was becoming livelier. With all the worry they had been having of late
her cheeks had become pale, but, while they were talking, Mr. and Mrs.
Samsa were struck, almost simultaneously, with the thought of how their
daughter was blossoming into a well built and beautiful young lady.
They became quieter. Just from each other’s glance and almost without
knowing it they agreed that it would soon be time to find a good man
for her. And, as if in confirmation of their new dreams and good
intentions, as soon as they reached their destination Grete was the
first to get up and stretch out her young body.




\bye
