%%
%% Personal plain TeX macros.
%% Can only be used with LuaTeX, requires luaotfload.sty in a searchable path.
%%


%%%%%%%%% F O N T S %%%%%%%%%%%%%%%%%%%%%%%%%%%%%%%%%%%%%%%%%%%%%%%%%%%%%%%%%%%%

% Use font loader from ConTeXt
\input luaotfload.sty

% Font names

\font\twelverm 	"DanteMTStd-Regular":+onum;+pnum;+tlig; at 12pt
\font\twelveit 	"DanteMTStd-Italic" at 12pt
\font\twelvesc 	"DanteMTStd-Regular":+smcp;+c2sc;letterspace=10; at 12pt
\font\twelvetit "DanteMTStd-Regular":+onum;+titl;letterspace=10; at 12pt
\font\twelvetitb "DanteMTStd-Regular":+onum;-titl;letterspace=10; at 12pt


\font\thirteenrm "DanteMTStd-Regular":+lnum;+pnum;letterspace=10; at 14pt
\font\thirteensc "DanteMTStd-Regular":+onum;+pnum;+smcp;+c2sc;letterspace=10; at 14pt

\font\fourteentit "DanteMTStd-Regular":+onum;+titl;letterspace=10; at 14pt
\font\fourteensc "DanteMTStd-Regular":+smcp;+onum;letterspace=10; at 14pt
\font\fourteenit "DanteMTStd-Italic":+onum;+pnum; at 14pt

\font\dccapfont "DanteMTStd-Regular":+titl at 54pt
\font\dcquotefont "DanteMTStd-Regular":+titl at 25pt

\def\secIfontA{\thirteenrm}
\def\secIfontB{\eighteenrm}

\def\secIIfont{\fourteenit}

% Semantic names for direct use

\def\normalsize{%
	\gdef\rm{\twelverm}%
	\gdef\it{\twelveit}%
	\gdef\sc{\twelvesc}%
	\gdef\tit{\twelvetit}%
}

% Use italics for emphasis
\def\emph{\it}

% Start with normalsize roman
\normalsize\rm


%%%%%%%%% P A R A G R A P H   S E T T I N G S %%%%%%%%%%%%%%%%%%%%%%%%%%%%%%%%%%

% Default distance between lines
\baselineskip=14.06pt

% Static height of first line
\topskip=\baselineskip

% No extra space after periods.
\frenchspacing

% Vertical distance between paragraphs
\parskip=0pt

% Paragraph indentation
\parindent=1em

% Lineskip & lineskiplimit
\lineskiplimit=-1000pt

% Badness
\emergencystretch=0pt
\pretolerance=150
\tolerance=9999
\hbadness=150
\hfuzz=0pt

% \firstnoindent -- Don't indent next paragraph
\def\firstnoindent{\global\everypar={\wipeeverypar\setbox7=\lastbox}}
\def\wipeeverypar{\global\everypar={}}

% No penalties for widows and orphans, these are dealt with using \adjustpage
\widowpenalty=0
\clubpenalty=0

%%%%%%%%% P A G E   L A Y O U T %%%%%%%%%%%%%%%%%%%%%%%%%%%%%%%%%%%%%%%%%%%%%%%%

% Paper size
\pagewidth=150mm	
\pageheight=225mm

% Undo legacy 1in origin
\pdfvariable horigin 0in	
\pdfvariable vorigin 0in

% Typeset area distance from top left corner
\hoffset=\dimexpr(\pagewidth/9)\relax	
\voffset=\dimexpr(\pageheight/9)\relax
\advance\voffset by -\topskip
\advance\voffset by 1ex


% Ensure that first baseline starts at the same distance from the top
\topskip=\baselineskip % I had an idea with this but forgot, leaving it in for now.

% Typeset area size
\hsize=\dimexpr(6\pagewidth/9)
\vsize=\topskip \advance\vsize by 30\baselineskip

% Backup original vsize 
\newdimen\originalvsize \originalvsize=\vsize

% Defines a macro \ap:<pageno>, expands to how many lines should be adjusted.
\def\adjustpage#1#2{%
	\expandafter\xdef\csname ap:#1\endcsname{#2}
	\ifnum #1=1
		\setvsize
	\fi
}
% Example: \adjustpage{15}{+2} adjusts page 15 to have 2 extra lines

% This macro gets expanded last in the output routine, don't use directly.
\def\setvsize{%
	\global\vsize=\originalvsize
	\ifcsname ap:\the\pageno\endcsname
		\global\advance \vsize by \csname ap:\the\pageno\endcsname \baselineskip
	\fi
}


%%%%%%%%% H E A D E R %%%%%%%%%%%%%%%%%%%%%%%%%%%%%%%%%%%%%%%%%%%%%%%%%%%%%%%%%%

\def\makeheadlineplain{%
	\vbox to0pt{%
		\vskip-22.5pt
  		\line{%
  			\vbox to8.5pt{}%
  			\the\headline
  		}%
  		\vss
  	}
	\nointerlineskip
}

\def\makeheadline{%
    \vbox to0pt{%
        \vss
        \hbox to\hsize{%
            \the\headline
        }%
        \vskip-\topskip
        \vskip0.5\baselineskip % raise or lower from default
        \hbox{}%
    }%
    \nointerlineskip
}

\newif\ifheadline \headlinetrue

\headline{%
	\ifheadline
		\ifodd\pageno
			\hss \normalsize\sc \defaultcolor 
			\lowercase{Name of the book} \hss% recto
		\else
			\hss \normalsize\sc \defaultcolor 
			\lowercase{Author Authorsson}\hss% verso	
		\fi
	\else
		\hss
		\global\headlinetrue	
	\fi
}

\headline{}

% Turn off headline on current page
\def\thispagenoheadline{\global\headlinefalse}


%%%%%%%%% F O O T E R %%%%%%%%%%%%%%%%%%%%%%%%%%%%%%%%%%%%%%%%%%%%%%%%%%%%%%%%%%

\def\makefootline{%
	\baselineskip=14pt
	\lineskiplimit=0pt
	\line{%
		\the\footline
	}%
}

\newif\iffootline \footlinetrue

\footline{%
	\iffootline
		\ifodd\pageno
			\hss{\normalsize\rm \defaultcolor \folio\quad}% recto
		\else
			{\quad \normalsize\rm \defaultcolor \folio}\hss% verso	
		\fi
	\else
		\hss
		\global\footlinetrue	
	\fi
}

% Turn off footline on current page
\def\thispagenofootline{\global\footlinefalse}


%%%%%%%%%% C O L O R S %%%%%%%%%%%%%%%%%%%%%%%%%%%%%%%%%%%%%%%%%%%%%%%%%%%%%%%%%

% Token for \aftergroup to reset color
\def\colorstackpop{\pdfextension colorstack 0 pop}

% Switch to CMYK color in current group. Expects four values between 0 and 1
\def\setcmykcolor#1{%
   	\pdfextension colorstack 0 push {#1 k #1 K}% 
   	\aftergroup\colorstackpop
}

% Switch to RGB color in current group. Expects three values between 0 and 1
\def\setrgbcolor#1{%
   	\pdfextension colorstack 0 push {#1 rg #1 RG}% 
   	\aftergroup\colorstackpop
}

% Color names
\def\black{\setcmykcolor{0 0 0 1}}
\def\blue{\setcmykcolor{1 1 0 0}}
\def\red{\setrgbcolor{0.8 0 0}}

% Semantic names for direct use
\def\defaultcolor{\black}
\def\hyperlinkcolor{\blue}
\def\rubricationcolor{\red}

% Example: {\hyperlinkcolor This is blue text}


%%%%%%%%% I M A G E S %%%%%%%%%%%%%%%%%%%%%%%%%%%%%%%%%%%%%%%%%%%%%%%%%%%%%%%%%%

\newdimen\imagewidth    \imagewidth=0pt 	% 0pt gives natural size of image
\newdimen\imageheight   \imageheight=0pt
\newtoks\imagedir		\imagedir{} 		% Must end with /, e.g. mypics/

\def\image#1{%
	\hbox{%
		\saveimageresource 
			\ifdim\imagewidth=0pt \else width\imagewidth\fi 
            \ifdim\imageheight=0pt \else height\imageheight\fi 
            {\the\imagedir#1}%
      	\useimageresource\lastsavedimageresourceindex
	}%
}

% Example: {\imageheight=4cm \image{mypicture.png}}

%%%%%%%%% O U T P U T   R O U T I N E %%%%%%%%%%%%%%%%%%%%%%%%%%%%%%%%%%%%%%%%%%



\output{\beforeoutput \plainoutput \afteroutput}

% For two page layout, \shiftoffset is how much verso pages should move
\newdimen\shiftoffset
\shiftoffset = \pagewidth
\advance \shiftoffset by -\hsize
\advance \shiftoffset by -2\hoffset

\def\beforeoutput{%
	% Two page layout
	\ifodd\pageno
	\else
		\advance \hoffset by \shiftoffset
	\fi
}

\def\prepghook{\vbox to0pt{\kern-\voffset 
   \hbox to0pt{\kern-\hoffset\background\hss}\vss}% 
   \nointerlineskip 
} 

\def\background{%
    \saveimageresource width\pagewidth height\pageheight {grid.pdf}%
    \useimageresource\lastsavedimageresourceindex
}

% Turn off background image
\def\background{}

\def\plainoutput{\shipout\vbox{\prepghook\makeheadline\pagebody\makefootline}%
  \advancepageno
  \ifnum\outputpenalty>-20000 \else\dosupereject\fi}

\def\afteroutput{%
	% Manually overriding vsize to solve widows through \adjustpage
	\setvsize
}

%%%%%%%%% S E C T I O N I N G %%%%%%%%%%%%%%%%%%%%%%%%%%%%%%%%%%%%%%%%%%%%%%%%%%

\def\secI#1#2{%
	\vfil\break
	\ifodd\pageno
	\else
		\null\thispagenofootline
		\vfill\eject
	\fi
	\centerline{\noindent\fourteensc \lowercase{#1}}
	\vskip0.8\baselineskip
	\centerline{\noindent\fourteentit \uppercase{#2}}
	\vskip2.2\baselineskip
}


\def\secII#1#2{%
	\vskip\baselineskip
	\vskip-1ex
	{\secIIfont\red
	 \raggedright\noindent 
	 \if 😀#1😀\else
	 	Kapitel #1\hskip0.33em \char"00B7\hskip0.33em
	 \fi
	 #2\par%
	}%
	%\vskip1\baselineskip
	\vskip1ex
	\firstnoindent
}


%%%%%%%%% D R O P C A P S %%%%%%%%%%%%%%%%%%%%%%%%%%%%%%%%%%%%%%%%%%%%%%%%%%%%%%

\def\dropcapD{%
\vskip2\baselineskip
\smash{%
    \hskip0pt
    \raise 21pt \llap{\dcquotefont \hskip0pt}%
    {\dccapfont D}%
}%
\vskip-3\baselineskip
\dimen0=41pt
\dimen1=45pt
\dimen2=45pt
\parshape 4
    \dimen0 \dimexpr(\hsize-\dimen0)
    \dimen1 \dimexpr(\hsize-\dimen1)
    \dimen2 \dimexpr(\hsize-\dimen2)
    0pt \hsize
\noindent
}

\def\dropcapS{%
\vskip2\baselineskip
\smash{%
    \hskip-2pt
    \raise 21pt \llap{\dcquotefont \hskip0pt}%
    {\dccapfont S}%
}%
\vskip-3\baselineskip
\dimen0=23pt
\dimen1=28pt
\dimen2=28pt
\parshape 4
    \dimen0 \dimexpr(\hsize-\dimen0)
    \dimen1 \dimexpr(\hsize-\dimen1)
    \dimen2 \dimexpr(\hsize-\dimen2)
    0pt \hsize
\noindent
}
%%%%%%%%%%%%%%%%%%%%%%%%%%%%%%%%%%%%%%%%%%%%%%%%%%%%%%%%%%%%%%%%%%%%%%%%%%%%%%%%

%         C O N T E N T
          
%%%%%%%%%%%%%%%%%%%%%%%%%%%%%%%%%%%%%%%%%%%%%%%%%%%%%%%%%%%%%%%%%%%%%%%%%%%%%%%%

%\adjustpage{2}{+2}
%\adjustpage{5}{-1}


%%% Frontmatter

\parindent0pt

\def\ToCPartLine#1{%
	\vskip0.5\baselineskip
    {\twelvesc\uppercase{#1}}
    \vskip0.5\baselineskip
}

    \newcount\testpg \testpg=1

\def\TocChapLine#1#2{{%
	\advance\leftskip by 2em
	\noindent\leavevmode
	\it
	\llap{%
		\hbox to 1em{\hfil #1}\quad
	}%
	\global\advance\testpg by 3\relax
	#2\hfill \the\testpg\par
}}

\thispagenoheadline
\thispagenofootline

\secII{}{Innehållsförteckning}
\vskip2\baselineskip

\TocChapLine{}{Till läsaren}

\ToCPartLine{Första delen: Världsskapelsen}

\TocChapLine{1}{Kaos. Världsträdet.}
\TocChapLine{2}{Urvarelserna.}
\TocChapLine{3}{Underjorden. Burs söner. Urtidskonstnärerna. Odens\hfill\break själfuppoffring.}
\TocChapLine{4}{Rimtursarnes undergång. Vakten vid Hvergelmer.}
\TocChapLine{5}{Världsskapelsen fortsättes.}
\TocChapLine{6}{Världskvarnen.}
\TocChapLine{7}{Världsskapelsen fullbordas.}

\ToCPartLine{Andra delen: Urtidens fridsålder}

\TocChapLine{8}{Fredsförbundet. Gudaklenoderna.}
\TocChapLine{9}{Människoskapelsen.}
\TocChapLine{10}{Heimdall kulturbringaren.}
\TocChapLine{11}{De af Heimdall utlärda heliga runorna.}
\TocChapLine{12}{Asagudarne.}
\TocChapLine{13}{Tors och Egils vänskap. Ivaldeslägten.}

\ToCPartLine{Tredje delen: Synden och ofridens tidsålder}

\TocChapLine{14}{Trolldomsrunorna.}
\TocChapLine{15}{Jättarne vilja underrätta sig om Tors styrka.}
\TocChapLine{16}{Tors färd till jätten Hymer.}
\TocChapLine{17}{Tors fälttåg mot jätten Geirrud.}
\TocChapLine{18}{Tors hammare stulen.}
\TocChapLine{19}{Huru Asgård fick sin vallgördel och Oden sin häst.}
\TocChapLine{20}{Täflingen mellan urtidskonstnärerna.}
\TocChapLine{21}{Järnhammaren pröfvas.}
\TocChapLine{22}{Mjödet i Byrgers källa. Oden hos Fjalar.}
\TocChapLine{23}{Domen över Valands konstvärk.}
\TocChapLine{24}{Försöken att försona Valand. Ivaldesönernas flykt.}
\TocChapLine{25}{Den förste fimbulvintern. Barnen i odödlighetsängden.\hfill\break Valands hämdesvärd.}
\TocChapLine{26}{Balders död.}
\TocChapLine{27}{Fimbulvinterns höjdpunkt. Folkutvandringen från\hfill\break norden. Sköld-Borgar och Halfdan.}
\TocChapLine{28}{Valands senare öden och död.}
\TocChapLine{29}{Od-Svipdag och Fröja.}
\TocChapLine{30}{Halfdans tåg mot norden. Egils död.}
\TocChapLine{31}{Svipdag hämtar hämdesvärdet. Halfdans död.}
\TocChapLine{32}{Svipdag kommer till Asgård. Hans färd till Balder.}
\TocChapLine{33}{Fröj friar till Gerd.}
\TocChapLine{34}{Brytning mellan Asar och Vaner.}
\TocChapLine{35}{Världskriget. Vanerna intaga Asgård.}
\TocChapLine{36}{Världskriget i Midgård. Haddings ungdomsäfventyr.}
\TocChapLine{37}{Världskrigets slut.}
\TocChapLine{38}{Nornorna Domen öfver de döde. Salighets- och\hfill\break strafforterna.}
\TocChapLine{39}{Högbebyggarne.}
\TocChapLine{40}{Ragnarök. Världsförnyelsen.}
\vskip\baselineskip
\TocChapLine{}{Mytologisk namnförteckning}

\thispagenoheadline
\thispagenofootline

\parindent1em

\vfil\break

\secII{}{Till läsaren}

För första gången öfverlämnas härmed i den svenska ungdomens händer
vår folkstams myter som ett sammanhängande helt. En närmare undersökning
har visat, att de bildat en fortlöpande saga, hvari hvarje myt, äfven om
den i sin uppkomst var fristående, blifvit införlifvad som länk i en
kedja, hvilken börjar med världens, gudarnes och människoslägtets upphof
och ändar med Ragnarök och världsförnyelsen.

Dessa kedjans yttersta länkar sammanknytas genom myterna om världens
lefnadsåldrar, om urfäderna och de händelser, som i deras dagar timade.
Det första människoparet och de närmaste slägtleden efter det lefva i en
kulturlös oskulds- och fridsålder. En af vanagudarne sändes
förmänskligad till Midgard för att undervisa människorna. Han blir
germanfolkets förste präst och lärare. Han efterträdes af den förste
domaren (Sköld-Borgar), och denne af den förste konungen
(Halfdan-Mannus). Dennes tre söner åter varda de nu i tre grenar sig
skiljande germanfolkens förste stamkonungar. Med deras öden afslutas den
mytiska urhistorien, men för att fortsättas i hjältedikter om deras afkomlingar, hvilka dikter utlöpa i
den egentliga historien.

Alla mytologier, i hvilka berättelser om urfäder ingå, hafva ställt
desse i närmaste förbindelse med gudarne. Himmelens och underjordens
heliga makter värna skapelsen och människorna, sina skyddslingar, mot
fiendtliga väsen och länka, i umgänge med urfäderna, händelsernas gång.
Därigenom få de vigtigare mytiska tilldragelserna med nödvändighet en
kronologisk ordning, emedan en sådan med nödvändighet råder i
berättelsen om de hvarandra följande urfäderna. Gudarnes gärningar och
öden under Sköld-Borgars styrelse måste hafva timat före deras gärningar
och öden under Halfdans, och dessa åter före dem, som tilldrogo sig i
Halfdans söners dagar. På samma sätt hafva de indiske arierna skilt
mällan de händelser i guda- och människovärlden, som egde rum före,
under och efter deras urfäder Manus' och Brigus lefnad; de iraniske
arierna mällan dem i urpatriarken Jimas och i följande patriarkers tid;
de helleniske arierna mällan dem i guldåldern under Kronos' styrelse,
dem i silfveråldern, då världsspiran öfvergår till Zeus, samt dem i
koppar- och järnåldrarne, likasom åter inom dessa senare mällan de
mytiska tilldragelserna i Kadmos' och Jasons tider och dem, som följa
hvarandra intill det trojanska krigets. När Ovidius ville besjunga alla
förvandlingar den antika mytologien omtalar, kunde han fördenskull göra
detta i episkkronologisk ordning, börjande med kaos och slutande med
Cæsar.

Öfver allt, där ett folk är i besittning af sagor med heligt anseende,
ärfda från slägte till slägte, sammanknyta sig dessa efter hand och
likasom af sig själfva, på grund af fantasi- och föreställningslifvets
behof af ordning och sammanhang. Det kräfves icke hög bildning för att
detta förlopp skall ega rum. Hos finnar och bulgarer ha, likaväl som hos
arier, arfsägner sammanlänkat sig i episka kedjor. Det fordras ej häller
en utvecklad kultur i detta ords moderna mening, för att religiösa och
sedliga föreställningar af öfverraskande höghet yppa sig i ett folks
mytiska sägner vid sidan af föreställningar, som synas oss barnsliga.
Det är en etnologisk erfarenhet, gjord hos olika folkraser, hos
minkopéer och indianer såväl som hos Rigveda-arier och germaner, och bör
minst förvåna oss hos desse senare, hvilkas hedniska kultur var en
poetisk. Det är gifvet, att det högsta och bästa de förnummo fick
uttryck i deras diktkonst, särskildt i den religiösa. Det torde knappt
behöfva tilläggas, att ett folks religion, när den icke föråldrats, står
högre än massan af den religionens bekännare, emedan den afspeglar den
idealare sidan af folkets lif.

De germaniska myterna äro af mycket olika ålder. Några af de här
meddelade härstamma, såsom den jämförande mytgranskningen kan bevisa,
från den fornariska tiden. Till dem höra myterna om en brytning mällan
gudarne och naturkonstnärerna, en häraf uppkommen fimbulvinter och
åtgärder, som vidtogos för att undan denna rädda hvad skapelsen hade
bäst för ett kommande saligt världstidehvarf. Andra myter
tillhöra en långt senare tid, några möjligen hedendomens sista
århundraden. Men allteftersom de uppkommit, hafva de som nya länkar
knutits till den redan för handen varande episka kedjan.

Med afseende på skälen och bevisen för riktigheten af den framställning
jag här lämnar af vår mytologis epos, har jag att hänvisa till
»Undersökningar i germanisk Mytologi», hvaraf första delen utkom år
1886, och hvars andra del är under utarbetning.

Stockholm i November 1887.

\rightline{\it Viktor Rydberg.}


\secI{Första delen}{Världsskapelsen}

\secII{1}{Kaos. Världsträdet.}

\dropcapD en värld, i hvilken vi lefva, är icke af evighet. Den har haft en
början och skall få ett slut. Det har varit en tidernas morgon. Då
»{\it 
fanns ej sand, ej sjö,
ej svala böljor,
jord var icke,
ej himmel där ofvan}»,
men rymden fanns, ett tomt ofantligt svalg -- Ginnungagap.

Och en trefald af krafter fanns, som inträdde i Ginnungagap och värkade
där: kölden, värmen och den skapande visdomen.

De uppstego ur källor af okändt ursprung. Norr om Ginnungagap upprann
köldens källa, Hvergelmer, och inhöljde rymden åt detta håll i frostiga
töcken. Den näjden kallas derför Nifelheim (töckenvärlden). Söder om
Ginnungagap upprann värmens källa, som sedan blifvit kallad Urds. Mellan
dem båda, midt under Ginnungagap, upprann den skapande visdomens källa,
om hvilken blifvit sagdt, att hennes djup är sådant, att icke ens Odens
tanke kan mäta det. Den källan blef sedan kallad Mimers.

Ur Hvergelmer brusade dimmsvepta köldvågor, in i Ginnungagap och mötte
der värmeböljorna från söder. Genom deras strid och blandning uppstodo
de grundämnen, af hvilka den första världen bildades.

I rymdens afgrund, där den skapande visdomens källa upprinner, låg fröet
till Yggdrasil, världsträdet. Det spirade och sände ut rötter genom de
tre krafternas källor, och dess sammanflätade oräkneliga rottrådar vordo
stommen till den grund, hvarpå underjorden hvilar. Under långa
världsåldrar lyfte sig trädets stam allt högre och sköt grenknippor
öfver hvarandra, på hvilka de olika världarna, allt eftersom de
skapades, fingo sina underlag. För mänskliga ögon är Yggdrasil osynligt.
Det kallas äfven Mimerträdet.

\secII{2}{Urvarelserna.}

\dropcapS kaparkraften ur visdomskällan genomträngde grundämnena och ur deras
jäsning uppkom ett lefvande väsen, urkon Audumla. Hon närde sig af de
lifsfrön, som i grundämnena förefunnos, och frigjorde ur deras stelnade
föreningar en i dem dold gudomlig lifsform. Detta tillgick så, att när
Audumla slickade deras rimfrost och sälta, framkom så småningom, likasom
löst af hennes tunga, en mänsklig skepnad ur den steniga grunden; den
var skön att skåda och af väldig storlek. En gudomsande bodde i honom,
och han vardt stamfader till de gudar, som intill denna tidsålders slut
styra och skydda människornas värld. Denne, den älste af tidehvarfvens
gudar, hette Bure och blef fader till Bur och farfader till Oden och
hans bröder.

Men det uppkom i urvärlden ännu en varelse: den ofantlige jätten Ymer.
Etterdroppar, som utslungades af Hvergelmers köldvågor, samlade sig,
växte och vordo till honom. Med fyra munnar sög han fyra mjölkströmmar
ur Audumlas spenar. De närde honom med lifsfrön af olika art, hvilka
gjorde honom till fader åt olika ätter, framspirande ur olika lemmar af
hans kropp. Så framväxte under hans vänstra arm man och mö, från hvilka
en vacker och gudavänlig jätteätt har kommit; men hans fötter födde med
hvarandra de
fördärflige rimtursarnes fader Trudgelmer, ett vidunder med tre
hufvuden. Trudgelmer födde -- likasom sin fader utan kvinna -- sonen
Bergelmer.

Bland de ädla väsen, som härstamma från Ymer, äro den vise Mimer och
hans syster Beisla, samt de tre ödesdiserna, systrarna Urd, Verdande och
Skuld de förnämsta.

\secII{3}{Underjorden. Burs söner. Urtidskonstnärerna. Odens själfuppoffring.}

Underjorden är den älsta delen af skapelsen och var bebodd och smyckad,
långt innan jorden inrättades till bostad åt människoslägtet, ja innan
Asgard vardt bebygdt af asagudarne.

Underjorden består af två mycket olika områden, åtskilda af en hög
bärgskedja med stupande väggar. Bärgskedjan kallas Nidafjället. Norr om
henne ligger det dystra, kalla, gyttjiga och töckenhöljda Nifelhel, där
förut Nifelheim var i Kaos. Uppe på fjällslätten återfinnes källan
Hvergelmer, ur hvilken en mängd floder strömma ned mot norr och söder.

Söder om Nidafjället utbreder sig ett land af obeskriflig härlighet med
blomsterfält och lundar, aldrig härjade af frost och vinter. Detta land
delas i två riken, Mimers rike och ödesdisernas. Mimers är beläget öfver
det forna Ginnungagap rundt omkring den skapande visdomens källa, som i
honom fick sin vårdare och väktare.

Det andra riket är beläget kring värmekällan, och denna har fått till
sina väktare de tre nornorna eller ödesdiserna, Urd och hennes systrar.

Tillsammans kallas Mimers och Urds riken för Hel. Underjorden utgöres
således af Hel och Nifelhel.

Midt öfver underjorden hvilar på Yggdrasils grenar jordens skifva. Men
det dröjde länge, innan hon blef till en värld, beboelig för människor.
Där dvaldes under långa tidsåldrar jätten Ymer och hans fötters flerhöfdade barn. Det var en värld uppfylld af vanskapliga och vilda jätteväsen.

Medan det såg så ut på jordens yta, bodde i underjordens evigt gröna
riken de älste gudarne fredligt och lyckligt tillsammans med Mimer,
ödesdiserna och andra medlemmar af samma ädla jätteslägt. Bures son Bur
vardt där gift med Mimers syster Beisla, som födde med honom tre söner.
Den älste af de tre är Oden, känd äfven under många andra namn. Han blef
stamfader för asarnes gudaslägt. Den andre är den milde Höner, känd
äfven under namnet Vee. Den tredje är Lodur, »uppflammaren», som äfven
kallas Vile. Höner och Lodur blefvo stamfäder för vanernas gudaslägt.

Mimer, väktaren vid den skapande visdomens källa, drack hvarje morgon af
dess kraft- och kunskapsförlänande mjöd, den dyrbaraste och ljufvaste
saften i skapelsen, och begåfvades därigenom med danareförmåga och djup
visdom. Han födde sju söner, som fingo åtnjuta samma dryck och blefvo,
likasom fadern och i faderns tjänst, personliga skaparkrafter,
naturvärkmästare och storsmeder, som smidde gräs och örter och underbara
smycken af eld eller guld eller andra grundämnen.

Det var af ödet bestämdt, att Oden och hans ättlingar icke skulle stanna
i underjorden; de voro utsedde att beherrska världar där ofvanför. Oden
uppsteg i världsträdet för att från dess höjd öfverskåda de riken, som
blifvit honom tilldelade, och han såg, huru jordens rund innehades af
den vanskaplige Ymer och dennes fötters ohyggliga afkomma. Ung och
oerfaren var Oden ännu, och han kände, att han af egen kraft ej förmådde
ränsa jorden från dessa vidunder och uppfylla de andra plikter, hvilka
väntade honom som världars styresman. En dryck ur visdomskällan kunde
förhjälpa honom härtill; men Mimer, källans djupsinnige väktare, vägrade
honom drycken, innan han bevisat sig den värdig genom själfuppoffring.
Då gaf Oden sig själf till offer åt sin lifsuppgift. Nio dygn dvaldes
han i det vindfarna trädet utan föda och dryck, sårad med spjut, offrad
åt Oden, offrad åt sig själf, blickande nedåt, bedjande under tårar om den styrka han
saknade, och lyssnande till den runosång, som förnams ur djupet, tills
han vanmäktig nedsjönk ur Yggdrasils krona. Då gaf honom Mimer den
efterlängtade drycken och lärde honom nio underbart värkande sånger. Och
Odens krafter utvecklade sig sedan rikligen; han vardt stor i kunskap
och skaparemakt.

\secII{4}{Rimtursarnes undergång. Vakten vid Hvergelmer.}

Nu kände han sig i stånd att utföra det värk han hade i sinnet: att
befria jorden från urgiganterna. Hans bröder Höner och Lodur förenade
sig med honom till företagets fullgörande. De dödade Ymer genom att
öppna hans hals-ådror. Ur såret frambrusade en blodström, som växte till
ett haf, hvari urjättens vanskapliga och onda afföda drunknade.

Tillintetgjorda vordo dessa vidunder, rimtursarne, likväl icke. Döden
tillintetgör intet. Deras själar fortlefde och nedstego till
underjorden. Där funno de, norr om Nidafjället, i Nifelhel, ett land,
som med sitt mörker, sina frostiga dimmor och stinkande träsk anstod
dem. Där byggde de sig en sal och vordo under tidernas lopp farliga
grannar till Mimers rike. Till rimtursarnes slägt höra sjukdomsandarne.
De bo i Nifelhel, äfven de. Bland dem äro tramarne (onda vättar), Morn
(själavåndan), Tope (vanvettet), Ope (den skakande gråten) och Otåle
(den rastlösa oron).

Till skydd mot rimtursarne uppställde gudarne och Mimer en vakt vid
Hvergelmers källa på gränsen mot Nifelhel. Till höfdingar för vakten
utvaldes alfen Ivalde och hans tre söner, sedan de svurit gudarne
trohetsed. Ivaldes söner voro tappre stridsmän och skicklige konstnärer,
som af Mimer själf fått lära smideskonstens hemligheter. En af de tre,
Valand (Völund), var i synnerhet en utmärkt värkmästare, jämförlig
med den ypperste smeden bland Mimers söner, Sindre. Valands bröder hette
Egil och Slagfinn. De voro förträfflige bågskyttar och skidlöpare.


\secII{5}{Världsskapelsen fortsättes.}

Ur det blodhaf, som betäckte jorden, lyfte Oden och hans bröder de
inunder liggande grunderna så, att i midten uppstod det land, som kallas
Midgard, och på randen af jordskifvan det land, hvars norra och östra
delar heta Jotunheim. Mällan Midgard och Jotunheim ligger en bred
bottenfördjupning, hvari blodböljorna samlade sig och bildade det kring
Midgard på alla sidor sig utbredande hafvet.

De fördränkte urgiganterna hade icke alla varit af samma vanskaplighet.
Denna hade minskats med hvarje slägte. Bland Bergelmers barn och
barnbarn funnos äfven sådana, som voro välskapade och syntes mindre
vilda än deras fäder. Mot desse yngre jättar visade sig Oden miskundsam
och lät dem rädda sig ur böljorna upp på kusten af Jotunheim, som de och
deras afkomlingar därefter bebyggde. En af dem var jättemön Gullveig.
För henne fingo gudinnorna vänskap. En annan var jätteynglingen Loke,
till hvilken Oden fann välbehag, emedan han var vacker, snarfyndig,
kvick och munter. Gudarne anade icke, huru farliga vänner Gullveig och
Loke voro.

Ymers lemmar blefvo byggnadsämnen till världsskapelsens fortsättande.
Men därtill kräfdes många och flitiga händer. Då samlade sig de älste
höghelige gudarne och Mimer och de andre naturvärkmästarne till möten
och satte sig å tingsstolar och rådgjorde såväl om byggnadsplanen som om
många och goda arbetares anskaffande. Och vordo de enige därom, att
Mimer skulle vara höfdingen för alla dessa, samt att han med bistånd af
eldväsendet Durin, som äfven kallas Surt, skulle gifva tillvaro åt en
skara svartalfer och dvärgar,
som hade att biträda vid arbetets utförande. Så skedde äfven. De
lifsfrön, ur hvilka Mimer och Durin med sin skaparekonst framkallade
dvärgarnes och svartalfernas slöjdskickliga slägten, förefunnos i Ymers
blod och lemmar. Mimer skapade dvärgarne; Durin-Surt skapade
svartalferna, som därför kallas hans söner.

Och nu vardt en liflig värksamhet på underjordens Idaslätter. Gudar och
urtidskonstnärer, dvärgar och svartalfer, byggde ässjor, samt smidde och
slöjdade de värktyg, som till arbetets fullbordande erfordrades, och de
konstvärk och smycken, med hvilka skapelsen skulle prydas.

Enligt den rådgärd för världsinrättningen, hvarom gudarne och
naturvärkmästarne blifvit enige, skulle en himmel hvälfva sig öfver
världs-alltet, och under honom skulle dag och natt, sol och måne vandra
i fastställd ordning utstakade banor, för att mäta tiden och fördela den
mellan värksamhet och hvila. Himlahvalfvet gjordes af Ymers hufvudskål
och af hans hjärna de stormmoln, som drifva därunder.

Mimer hade en dotter Natt, mörk till hyn men skön till dragen och huld
till lynnet, värdig att varda hvad hon blef: gudars moder. Med
morgonrodnadsalfen Delling, som är höfding i östra delen af Mimers rike,
födde hon sonen Dag, som blef ljusalfernas stamfader. Åt Natt och Dag
gjorde urtidskonstnärerna charer, ty det var på makternas tingsplats
beslutadt, att den mörka Mimerdottern och hennes ljuse, fagre son skulle
skiftevis fara genom underjordens östra hästdörrar upp på himmelen och
återvända genom dess portar i väster, där aftonrodnadens herre, Billing,
fick sig land anvisadt. Dags char infattades med ädelstenar. Hans häst
heter Skinfaxe (»den med skinande mank»). Natts häst heter Rimfaxe (»den
med rim-mank»). Om dagen betar han underjordsfältens gräs, beströdt med
honungsdagg, som faller från Yggdrasils nedersta, rikt lummiga
grenknippa. Fördenskull tuggar han under färden en fradga, som, när den
om morgonen droppar från hans mule, blir till en äringsgifvande dagg,
befruktande jordens dalar.

Med Lodur hade Natt två sköna barn, Måne och Sol, om hvilka var å
makternas rådstämma beslutadt, att de skulle föra öfver himmelen de af
eld och guld smidda charer, som tillvärkats i underjordssmedjorna för
att lysa världen. Sinnrika inrättningar voro af urtidskonstnärerna
uppfunna, för att skydda soldisen, när hon åker, mot charens brännande
strålar, och för att svalka hästarne, som draga den. Soldisens skydd är
en vid charen fästad sköld, som heter Svalin. Fölle han bort, skulle
soldisen förbrännas och charen från sin väg störta ned mot jorden och
antända den. Å hästarnes remtyg är under deras bogar fästad en
inrättning, som blåser svalka öfver dem. Hästarne heta Arvak (»den
tidigt vakne») och Allsvinn (»den mycket raske»). På Arvaks öron och
Allsvinns hofvar ristade naturvärkmästarne goda runor.

Af Ymers benbyggnad bildade makterna jordens bärg. Af hans kött, som var
mättadt med Audumlas näringsrika safter, beslöt man göra fruktbärande
mull och utbreda den öfver Midgards kala stenmarker.


\secII{6}{Världskvarnen.}

För detta ändamål -- och äfven för andra -- byggde urtidskonstnärerna en
ofantlig kvarn, Grotte. »Skärens kvarn», »stormarnes kvarn» kallas hon
äfven. Dess bjälklag restes i underjorden på Nidafjället rundt omkring
källan Hvergelmer, som är moderkällan till alla världens vatten, till
underjordens, jordens och himmelens. Från Hvergelmer komma de, och dit,
efter fullbordadt kretslopp, återvända de. Medels en från hafvets botten
genom jorden gående ränna underhålles den ständiga förbindelsen mellan
Hvergelmer och oceanen. Under denna ränna lades kvarnstenarne på sitt
bjälklag så, att den rörlige stenens öga står midt öfver källan. Genom
kvarnstensögat brusa fördenskull vattnen till och från Hvergelmer. Ebb
uppkommer i hafvet, när vattnen nedstörta
genom kvarnstensögat; flod i hafvet uppkommer, när vattnen genom samma
öppning stötas upp igen. Den kringgående kvarnstenen förorsakar den af
seglare fruktade malströmmen, en hvirfvel i oceanen, som pilsnabbt suger
skeppen till sig och drager dem ned i djupet, stundom äfven slungar dem
ur djupet tillbaka.

Grottes rörliga kvarnsten kringvrides af nio jättekvinnor, som vandra
utefter jordens rand och skjuta kvarnens ofantliga vridstång framför
sig.

På denna kvarn lades Ymers lemmar och maldes. Det slags mjöl, som däraf
uppkom, var den mull och sand, som hafvet allt sedan för till kusterna
af Midgard, och hvarmed vikarna och strandbräddarna uppgrundas, för att
förr eller senare varda grönskande fält. Af Ymers kött skapades de älsta
lagren af mull, de som betäckte Midgards stengrund, när Oden och hans
bröder upplyfte dem ur Ymers blod. Mullen är bördig, emedan den härleder
sig ifrån Audumlas mjölk, och sålunda kunde Midgard betäckas med
växtlighet.

Detta skedde, då soldisen på sin char för första gången körde upp på den
nyskapade himmelen och höljde jorden i välgörande strålars ljus. Då
spirade örter ur mullen, och efter hand fick jorden skogarnes och
blomsterfältens gröna skrud och vardt sådan, att den längre fram kunde
bebos af människor. Såsom bostad för vårt slägte kallas Midgard dess
»sal», dess »hus», bygdt åt oss af Burs söner,

{»de som det härliga}
{Midgard skapat;}
{sol sken sunnan}
{å »salens» stenar;}
{då vardt grunden grodd}
{med gröna örter.»}

Till kvarnens vårdare utsågo gudarne två af sin krets, vanaguden Fröj,
skördarnes herre, fruktbarhetens beskyddare, samt Lodur, »uppflammaren.»
Å Fröjs vägnar ombesörjes mälden af hans tjenare Byggver och Bejla.
Lodur öfvervakar kvarnens regelmässiga gång, och under honom stå de
nio jättekvinnor, som skjuta framför sig kvarnens vridstång. Lodur
kallas fördenskull äfven Mundelföre (»han som låter kringföra
vridstången»).

Sedan Ymers kött blifvit förvandladt till mull, kom ordningen till
Trudgelmers och därefter till Bergelmers. Det är Bergelmers lemmar, som
i vår nuvarande tidsålder malas af världskvarnen. En af den yngre
jätteslägtens äldre män, den mångkunnige Vaftrudner sade i en
visdomstäflan, som han hade med Oden, att han minns, när Bergelmer lades
å kvarnen -- det var hans älsta minne; men Trudgelmer och Ymer minns han
icke. Dem hade han blott hört omtalas.

Kvarnen tjenar äfven ett annat ändamål. Med hennes vridstång kringföres
ej endast hennes rörliga sten, utan äfven stjärnehvalfvet. Det är
stjärnehvalfvets rörelse, som Lodur, Månes och Sols fader, har att
öfvervaka.

Ännu ett annat och mycket vigtigt ändamål har världskvarnen uppfyllt.
Hon är den heliga, med hemlighetsfulla krafter begåfvade gnideldens
upphof. Eld hade funnits dessförinnan, och alltifrån begynnelsen; men
elden är af många slag, och den renaste och förträffligaste uppkom icke,
förrän Grottes stenar gnedos mot hvarandra. Ditintills hade denna rena
eld varit fördold i grundämnena, utan att visa sig för gudarnes ögon;
men nu framlockades han af gnidningen.

Och i och med honom framföddes underbart den gud, som är den heliga
elden personliggjord, nämligen »den hvitaste», den mest skinande af
gudar, Heimdall. Han föddes i skepnaden af ett det skönaste barn. Hans
öden som barn och yngling skola sedan omtalas. Han kallas son af nio
mödrar, emedan de nio jättekvinnor, som draga världskvarnen,
frambringade honom genom sitt arbete med vridstången, som således var
världens första eldborr.


\secII{7}{Världsskapelsen fullbordas.}

Det var bestämdt, att Oden och hans ättlingar, asagudarne, skulle ega
rike och borgar i den öfversta grenknippan af Yggdrasil, medan
vanagudarne skulle fortfarande bebygga underjorden. Denna bestämmelse
hade sin grund i de båda gudaslägternas olika skaplynne. Oden var född
med krigisk håg och lust för ingripande, äfventyrliga och
kraftansträngande gärningar, och detta hans lynne öfvergick också på
hans ättlingar, som alla, äfven den milde Balder, voro burne till
strids- och segergudar. Dagar kunde komma, när världsbyggnaden hotades
af förstörelsens makter, och det vore då nödigt att i Yggdrasils höga
krona, med utsigt åt alla håll, hvarifrån fara vore att vänta, äga en
världsbeskyddande vakt af slagfärdige hjältegudar. Fördenskull beslöto
makterna att uppe i Yggdrasil, högt öfver den väg, som Måne på sina
rundresor i rymden färdas, grundlägga och ordna den värld, som efter
sina bebyggare kallas Asgard.

Vanagudarne, den blide Höners slägt, äro äfven sin fader like och hafva
hans håg för fredlig värksamhet. Deras kall är att upprätthålla den
regelmässiga lagstadgade gången af världsförloppet; asarnes är att
försvara den mot fiender. Däri ligger den egentliga skilnaden mällan
dessa gudaslägters uppgifter. Det är fördenskull vanagudar, som
ombesörja stjärnehimmelens och tidvattnets regelbundna rörelser, samt
det jämna mellan år, måneskiften, natt och dag fördelade tidsförloppet.
Det är vanagudar, som sörja för såddens framgång och lycklig årsväxt,
och det är de, som med kärleksband knyta man vid kvinna och tillse, att
till slägtenas kedja fogas länk efter länk. Men där ett kraftigt
ingripande till skydd och värn är behöfligt, där uppträda asagudarne.
Dock visa äfven vanagudarne hjältemod, när det fordras.

Eftersom nu asarne borde hafva bostäder sådana, att de från dem kunde
öfverskåda och försvara världen, så gällde
det att skapa det högt belägna Asgard och en säker förbindelseled mellan
Asgard och underjorden. Urtidskonstnärerna byggde fördenskull bron
Bifrost, som från norr till söder beskrifver en stor båge genom rymden.
Bifrosts ene broände är lagd på den nordliga randen af underjorden; den
andre på den sydliga. Från dessa sina brohufvuden, som äro väl befästade
och förskansade och tecknade med skyddsrunor, går Bifrost förbi
jordskifvans rand och utanför densamma upp till Yggdrasils högsta
grenknippa, hvari Asgards mark med sina gyllne lundar hvilar. Här uppe
byggde urtidskonstnärerna Valhalls oerhördt stora och praktfulla sal åt
Oden, samt vidsträckta och härliga borgar äfven åt de andre asagudarne.
Till Valhall hör ett utkikstorn, Lidskjalf, hvarifrån asarne hafva
utsigt öfver Midgard och Jotunheim och öfver underjordens randbälte, ty
underjordens skifva är mycket större än jordens och skjuter på alla
sidor långt fram om henne ut i världsrymden. Underjordens randbälte
ligger fördenskull under öppen himmel. Det är på det bältet vanagudarne
och alferna bo, vanerna på den västra sidan, alferna på den östra.
Underjorden kallas också Jormungrund (storgrunden), emedan hon är större
än jorden, som, sedd från Lidskjalf, skymmer endast hennes mellersta
del. Men hvad i den skymda delen försiggår får Oden veta genom sina
kloka korpar Hugin och Munin, som dagligen flyga öfver Jormungrund.

Bron Bifrost är för gudarne behöflig. Väl ega de hästar, hvilka kunna
simma igenom lufthafvet, som är »gudarnes fjärd;» men en häst simmar ej
så snabbt och ledigt som han springer, och lufthafvets bredd är stor och
dess strömningar starka. Fördenskull begagna sig gudarne dagligen af den
märkvärdiga bron.

Öfver Asgard utbreder Yggdrasil på ett för gudarne synligt sätt sina
öfversta bladrika grenar, behängda med frukter, hvilkas beskaffenhet
längre fram skall omtalas. Än högre upp är det ställe i världsrymden,
där himmelens alla vatten -- utdunstningarna från hafvet och sjöarne och
från Yggdrasils
krona -- samla sig. Vattnen, medan de befinna sig därstädes, äro mättade
med ett ämne, som kallas vaferämne eller »svart skräckglans» och är
detsamma som ger åt åskmolnen deras mörka metalliska färg. Det kan
antändas och blir då till snabba fladdrande zigzaglågor, som träffa sitt
mål med medveten säkerhet. Stället, der de vaferladdade vattnen samlas,
kallas Eiktyrner, »ekstingaren», emedan de därifrån utgående åskmolnens
blixtar ofta slå ned i Midgards ekar. Från Eiktyrner nedbrusar en älf,
som med sina vafer-utdunstande böljor slår en skyddande gördel kring
Asgard. Om vaferdimmorna öfver älfven antändas, liknar den en hvirflande
eldflod. Öfver älfven leder en fallbrygga till den underbara
Asgardsporten, ett af urtidskonstnärernas mästervärk.

Högst i världsträdet dväljes den guldglänsande hanen Gullenkamme, ett
vartecken af asagudarnes vaksamhet och vård om världens trygghet.




\secI{Andra delen}{URTIDENS FRIDSÅLDER}

\secII{8}{Fredsförbundet. Gudaklenoderna.}

Det nyskapade, med växtlighet prydda Midgard låg nu i sitt vårlifs
fägring och var en syn, som fröjdade äfven gudarnes blickar.
Jotunheimsmakterna, som hemsöka jorden med frost och torka, med
hvirfvelvindar och öfversvämningar, höllo sig ännu stilla och lämnade
Midgard i ro; de voro icke starka nog att våga ett angrepp på gudarnes
skapelse. Jätteslägten, som ur Ymers blodhaf fått rädda sig upp på
kusten af Jotunheim, var ännu fåtalig och bemöttes af gudarne med
vänlighet. Ingenting störde världsinrättningens regelbundna gång.
Årstiderna aflöste hvarandra i stadgad tid, världskvarnen stod upprätt
på sitt bjälklag; mullen, som hon malde, var rikligt blandad med guld,
och hon kringvreds under välsignelsebringande sånger. Polstjärnan stod i
den världsåldern öfverst på himmelen, och himlahvalfvet hade icke den
sneda ställning det sedan fick.

Till betryggande af världens frid beslöts, att alla varelseslägter
skulle ingå förbund och gifva asarne gisslan. Från Vanaheim sändes som
gisslan till Asgard vanaguden Njord, som upptogs i asarnes slägt och
alltsedan bor bland dem, men ofta besöker sin gamla odalborg, Noatun,
belägen bakom västerhafvet vid en strand, där svanar sjunga. Oden gifte
sig med hans syster Frigg. Mimerdottern Natt är äfven Njords och Friggs
moder.

Med Mimer utbytte Oden panter på ömsesidig vänskap, som ej häller
någonsin vardt bruten. Af alferna, bland hvilka Ivalde och hans söner
voro höfdingar, kräfde och fick han trohets-ed. Äfven Jotunheim hade
ställt gisslan genom Gullveig, jättemön, och Loke, jätteynglingen. Båda
fingo vistas i Asgard, där Gullveig blef upptagen i Friggs hofstat. För
den kvicke och snabbtänkte Loke, som var en genomlistig hycklare och
låtsade största tjänstvillighet, fattade Oden sådan vänskap, att han
blandade blod med honom och lofvade att icke smaka mjöd, utan att det
frambures till dem båda. Å sin sida fick Jotunheim gisslan från asarne.
Med en vacker och gudavänlig jättekvinna -- äfven sådana funnos i
Jotunheim -- hade Oden födt en son, Tyr, som sedan blef stridsmännens
gud. Tyrs moder vardt gift med jätten Hymer, och Oden lät Tyr under hans
uppväxtår stanna som fosterson i Hymers gård. När därefter Oden fick med
Frigg sonen Tor, lät han äfven denne uppfostras i Jotunheim.
Fosterföräldrarna voro jätten Vingner och hans hustru Lora.

Det var i fridsålderns dagar Mimers söner och alferna smidde åt gudarne
de härliga klenoder, som pryda Asgard och lända gudarne till gagn och
värn. Allt hvad gudarne behöfde eller önskade af gyllene smycken och
vapen och husgeråd smiddes åt dem i Mimersönernas eller Ivaldesönernas
smedjor. Fridsåldern var en guldålder tillika. Det föddes väl icke ett
asa- eller vanabarn, som icke fick någon dyrbar klenod eller gagnerikt
konstvärk från Mimers söner eller Ivaldes. Nog är det från dem som Tor,
Odens och Friggs starkaste son, fick charen, hvari han åker bland
åskmolnen, och starkhetsbältet, som han spänner kring sin midja -- och
från dem som Balder, Odens och Friggs älskligaste son, fick skeppet
Ringhorne. Bland vanerna var det tillåtligt, att broder gifte sig med
syster. Vanaguden Njord födde med en syster två barn: Fröj, som blef
årsväxtens gud, och Fröja, som blef kärlekens och fruktsamhetens och är
den skönaste af alla gudinnor. Af Mimers söner fick Fröj en
gåfva och af Ivaldes söner en annan, hvilka sedan skola omtalas. Åt
Fröja smidde fyra underjordskonstnärer det vackraste kvinnosmycke i
världen, bröstsmycket Brisingamen. Åt Njord smiddes den bästa af alla
stridsyxor; åt flere af gudinnorna och diserna falk- och svanehammar; åt
Heimdall en stridslur, som kan höras öfver hela världen, men som Mimer
tagit i förvar och gömt i världsträdets djupaste skugga, tills den af
honom motsedda dagen kommer, då den varder behöflig. Märkvärdigt bland
klenoder var ock det tafvelspel af guld, hvarmed gudarne under
fridsåldern lekte; tafvelspelet kunde vara motspelare själf och flytta
sina brickor. Men den vigtigaste af urtidssmedernas gåfvor till gudarne
voro föryngringsäpplena, »asarnes läkemedel mot ålderdom.» De förärades
till Idun, Ivaldes dotter, som vardt upptagen i Asgard. Då de förvaras
af henne, ega de sin kraft, men annars icke.


\secII{9}{Människoskapelsen.}

Oden, Höner och Lodur kommo till det härliga Midgard, som de skapat, och
vandrade där. Mycket var där att se och fröjda sig åt; men något
saknades ändå: varelser, som kunde glädja sig åt Midgard, som åt ett
välprydt hem, och tacka dem, som förlänat dem hemmet. Gudarne gingo
utefter stranden af det nordliga hafvet, där detta drager sig söder ut
kring Aurvangalandet (»Lerslättlandet», sydligaste delen af
skandinaviska halfön). Där sågo de på stranden två träd och beslöto
omdana dem till likhet med den skepnad de själfva buro och att göra dem
till medvetna varelser. Det ena trädet kallas Ask, det andra Embla.
Lodur lossade dem från deras förbindelse med jorden och gaf dem förmåga
att röra och föra sig efter inre driffjädrar; han förvandlade deras
kalla safter till varmt blod och omformade dem till afbilder af gudarne.
Höner gaf dem mänskligt jag med medvetande och vilja. Oden gaf dem den
yppersta gåfvan: anden.

Så skapades det första människoparet: Ask, mannen och Embla, kvinnan.

Vackra att skåda, men nakna och blyga stodo de inför gudarne. Då
afklädde sig Oden sina egna praktfulla kläder och klädde med dem Ask och
Embla. Och de tyckte sig ståtliga, när de fått drägter.

Hvad som skedde vid det första människoparets skapelse upprepas i viss
mån vid hvarje människas. De båda träden Ask och Embla hade uppspirat ur
ollon, som världs-asken Yggdrasil fällt till jorden. Och äfven med deras
afkomlingar är förhållandet detsamma. Grundämnet till hvarje människa
skjuter skott, blommar och mognar till frukt å den världsomskyggande
askens grenar. Yggdrasil är »människoberedaren.» När en sådan frukt
mognat, faller den ned i Fensalar (»de gyttjiga dammarnes salar»), som
är Höners land och hans dotter Friggs odalmarker. Där ligga dock
frukterna icke ouppmärksammade. Storkarne, som äro Höners fåglar --
själf kallas han fördenskull »Långben» och »Träskkonung» -- ser dem och
flyger med dem till kvinnor, som trängta att smekas af små barnahänder.
Lodur, uppflammaren och eldborrens herre, bär dem å eld i modersskötet
och ger dem där hvad han gaf Ask och Embla: rörelseförmågan, det varma
blodet och gudaskepnaden. Höner sänder dem själen, Oden anden. Dock
sänder dem Höner själen icke efter eget godtfinnande. Oräkneliga själar
invänta sin födelse till tidslifvet, och det är att välja bland dem och
välja mödrar åt dem. Valet tillkommer ödesdisen Urd, som, emedan de, som
bida att varda mödrar, äro många och de väntande ofödda själarna än
flere, måste hafva många underordnade nornor, som förrätta åt henne
denna tjänst. Några af dessa nornor äro af asaslägt, andra af alfernas
stam, andra äro döttrar af Mimersonen Dvalin. Till den moder, som en
sådan norna utkorat åt en barnasjäl, sändes denna genom Höner. Så kommer
hvarje människa till världen: en frukt från världsträdet, omdanad af en
trefald gudomliga makter och utkårad af Urd till det moderssköte hon
fått, den lefnadsställning,
hvari hon hamnar, och de öden hon har att genomlefva. Urd ger henne
äfven en osynlig skyddsande genom lifvet, en underordnad norna, som
kallas fylgia eller haminga.

Ask och Embla födde barn, och deras afkomlingar förökade sig i det
fruktbara Aurvangalandet. De kände icke eldens bruk, de hade inga
sädeskorn att så, de förstodo ej att skaffa fram malmer ur jorden, än
mindre att smida sådana. Samhällsband och lagar kände de icke; ej häller
andra gudar än de tre, som skapat deras första föräldrar. Lagar behöfde
de dock till en början icke, ty de voro rättsinnade och godhjärtade. Men
de voro också lätt förledda, och det kom en tid, då en frestarinna till
det onda uppträdde ibland dem. För att odla och stärka deras anlag till
det goda, för att upplysa dem och binda dem vid sig med heliga band,
fattade fördenskull gudarne det beslut att sända dem en uppfostrare och
lärare.

\bye

\paragraph{X.}

\paragraph{HEIMDALL KULTURBRINGAREN.}

Till människornas lärare utsågs den rena och heliga eldens gud Heimdall,
kort efter det att han blifvit född. Huru det tillgick med hans födelse
är förut omtaladt. Han räknas till vanernas slägt, emedan han genom de
nio mödrarna vid världskvarnens vridstång bragtes i dagen å underjordens
västra randbälte, på andra sidan världshafvet, där vanerna bo.

För sitt vigtiga kall måste barnet utrustas med styrka, visdom och
härdighet. Man gaf det fördenskull att dricka samma tre safter, som
vattna världsträdets rötter, nämligen safterna i de tre
underjordskällorna.

Urds källa har ett silfverhvitt vatten, hvari svanar simma och öfver
hvilket Yggdrasil hvälfver evigt gröna bladmassor. Allt, som kommer i
beröring med den källans vatten, får en skinande hvit färg och stärkes
underbart, ty vattnet ger världsalltet dess fysiska lif och kallas
därför »jordens kraft». Barnet fick först en dryck därur.

\protect\hypertarget{lb1625905.xhtmlux5cux23start21}{}{}\protect\hypertarget{lb1625905.xhtmlux5cux23start21-a}{}{}\protect\hypertarget{lb1625905.xhtmlux5cux23start21-b}{}{}\protect\hypertarget{lb1625905.xhtmlux5cux23start21-c}{}{}\protect\hypertarget{lb1625905.xhtmlux5cux23start21-d}{}{}

Därefter fördes det till Mimers källa. Denna är sirad med en sjufaldig
infattning af guld, och diktens sådd växer kring dess säfomkransade
rand. Mimer gaf åt barnet en dryck af sin källas mjöd, som gömmer
skaparkraft, skaldeingifvelse och visdom.

Barnet fördes därifrån till Hvergelmers källa och fick dricka dess
kyliga härdande vatten.

Vid stranden af Vanaheim gjordes en båt i ordning för Heimdalls färd
till Midgard. Båten sirades med guldringar och andra prydnader, och i
den lades gossen, medan han efter de undfångna dryckerna slumrade. Hans
hufvud hvilade på en sädeskärfve. Det är från den som alla sädeskorn i
Midgard hafva sitt upphof. Bredvid honom lades eldborren, med hvilken
den heliga gnidelden kan vinnas, och rundt omkring honom de värktyg, som
kräfvas för all slags slöjd, samt vapen och smycken.

Därefter sköts båten ut i hafvet, för att själf söka den väg, som
ödesdiserna för honom utstakat. Några säga, att den drogs af ett
svanepar, och att svanar icke funnos i Midgard före dem, som kommo dit
med Heimdall. Alla svanar härstamma från det par, som simmar i Urds
källa.

Det hände en dag, att människor, som stodo på stranden af
Aurvangalandet, sågo en båt nalkas utan att vara drifven af åror. (Segel
voro den tiden okända.) Båten gick in i en liten vik vid stranden, och
det befanns, att en späd gosse sof i den, och att den innehöll en mängd
märkvärdiga och vackra ting, som voro för folket okända. Barnet upptogs
och vårdades med största kärlek och ömhet; kärfven tillvaratogs, och så
äfven de öfriga tingen, ehuru det bland dem var sådana, till hvilkas
bruk ej var lätt att gissa.

Hvem gossen var och hvarifrån han kom visste människorna icke. Emedan
han kommit med en sädeskärfve till deras land, kallade de honom Skef
(»kärfve»). Så växte han upp hos detta folk och blef ännu i unga år
deras undervisare och lärare i åkerbruk och alla slöjder, i tidsrunor
och evighetsrunor (världslig kunskap och religiös). Sedan han
\protect\hypertarget{lb1625905.xhtmlux5cux23start22}{}{}\protect\hypertarget{lb1625905.xhtmlux5cux23start22-a}{}{}\protect\hypertarget{lb1625905.xhtmlux5cux23start22-b}{}{}\protect\hypertarget{lb1625905.xhtmlux5cux23start22-c}{}{}\protect\hypertarget{lb1625905.xhtmlux5cux23start22-d}{}{}
vuxit upp till yngling, gjorde de honom till sin styresman och kallade
honom Rig (»höfding»).

Heimdall lärde människorna att plöja och baka, slöjda och smida, spinna
och väfva, rista runor och läsa. Han lärde dem att tämja husdjur och att
rida, att uppföra timrade boningar och knyta familjeband och
samhällsband. Han lärde dem att nyttja bågen, yxan och spjutet till jagt
och till värn mot urtidens vilddjur. Han undervisade dem i nornornas
stadgar för ett rättrådigt lif och i hvad de skulle göra för att vinna
ynnest hos de mäktige, välvillige gudarne. Af Heimdall fingo människorna
veta gudarnes namn och olika värf. Han lät dem åt gudarne uppföra
altaren och tempel, framkallade med eldborren den rena eld, som ensam är
värdig att brinna i deras tjänst, och förestafvade de böner och heliga
sånger, som alltsedan från människoläppar ljudit till makternas lof.

En gång då Heimdall vandrade gröna stigar utefter hafvets strand, kom
han till en stuga, bebodd af Ae och Edda. Han mottogs gästvänligt och
stannade i tre dygn. Nio månader därefter föddes hos hjonelaget sonen
Träl. Heimdall besökte sedan Aves och Ammas välburgna hem; när nio
månader förflutit, föddes där sonen Karl. Därifrån gick Heimdall till en
prydlig gård; där bodde stamfadern och stammodern till de sedan så
ryktbara slägter, som kallas skilfingar eller ynglingar, sköldungar,
hildingar och budlungar. Heimdall stannade tre dygn; nio månader
därefter föddes i det hemmet en son med ljusa lockar, röda kinder och
djärft blickande ögon. Han vardt höfding med tiden och folkets domare.
Folket kallade honom Borgar (»beskyddaren») och Sköld (»den värnande»).

Sålunda helgade och stadfäste Heimdall de tre stånden, de ofries, de
fribornes och de högbornes. Alla tre ärades med gudomlig börd, på samma
gång som de hafva mänsklig. De gjordes af honom till halfsyskon och
ålades därmed att vänligt och billigt bemöta hvarandra. Fördenskull
kallas de tre stånden »Heimdalls heliga ätter».

\protect\hypertarget{lb1625905.xhtmlux5cux23start23}{}{}\protect\hypertarget{lb1625905.xhtmlux5cux23start23-a}{}{}\protect\hypertarget{lb1625905.xhtmlux5cux23start23-b}{}{}\protect\hypertarget{lb1625905.xhtmlux5cux23start23-c}{}{}\protect\hypertarget{lb1625905.xhtmlux5cux23start23-d}{}{}

Heimdall lefde länge som människa bland våra urfäder och underkastade
sig den allmänna människolotten att åldras och dö. Han hade anordnat,
att hans lik skulle bäras ned till den lilla vik, där han som barn
landat. Det var om vintertiden. När det sörjande folket kom dit ned med
den döde, såg det med förvåning, att samma farkost, hvarmed Heimdall som
barn blifvit förd till Aurvangalandet, återkommit. Den ringprydde båten
låg där glänsande af rimfrost och is och ifrig att få sätta ut på hafvet
igen. Heimdalls lik nedlades i båten, och de tacksamma människorna
hopade omkring honom dyrbara smiden, ej färre än dem, med hvilka han
anländt. När allt var färdigt, gick farkosten ut i hafvet och försvann
vid synranden.

Den återvände till Vanaheim. Här afkläddes Heimdall sin åldrade
människoform och vardt till en strålande gudayngling. Oden upptog honom
i Asgard och i sin familjekrets.

Sköld-Borgar efterträdde honom som höfding och domare i Aurvangalandet.

\paragraph{XI.}

\paragraph{DE AF HEIMDALL UTLÄRDA HELIGA RUNORNA.}

Den heliga runkunskap, hvaraf Heimdall lärt människorna så mycket, som
är dem nyttigt att veta, var först i Mimers ego. Han hade den likväl
icke af sig själf, utan hade hämtat den ur visdomskällan, som han vaktar
under världsträdets mällersta rot. Genom själfuppoffring i sin ungdom
hade, såsom redan är omtaladt, Oden fått en dryck därur, samt nio
runosånger, som innehålla hemliga välgörande krafter och kallas
fimbulsånger. Bland dem må följande omtalas:

hugrunornas fimbulsång, som förhjälper till kunskap och visdom;

hjälprunornas, som underlätta barnens inträde i världen och häfva sorger
och bekymmer;

läkerunornas, som förläna läkande kraft;

brimrunornas, som ränsa luften från skadelystna väsen
\protect\hypertarget{lb1625905.xhtmlux5cux23start24}{}{}\protect\hypertarget{lb1625905.xhtmlux5cux23start24-a}{}{}\protect\hypertarget{lb1625905.xhtmlux5cux23start24-b}{}{}\protect\hypertarget{lb1625905.xhtmlux5cux23start24-c}{}{}\protect\hypertarget{lb1625905.xhtmlux5cux23start24-d}{}{}
och gifva makt öfver vind och våg, när det gäller att frälsa seglare ur
sjönöd, samt makt öfver elden, när han antändt människoboningar;

målrunornas, som återgifva talförmåga åt stumme och förstummade;

segerrunornas fimbulsång, som uppstämmes, när man går en fiendtlig här
till mötes. Då lyfta stridsmännen sina sköldar till jämnhöjd med
öfverläppen och sjunga med dämpad stämma, så att de mångas röster
sammansmälta till ett doft brus, likt bränningarnas i hafvet, och om de
då höra Oden blanda sin stämma däri, då veta de, att han förlänar dem
seger. Då »sjunger han under deras sköldar», och de varda herrar på
slagfältet.

Hugrunorna äro af mångfaldigt slag och innehålla både jordisk och andlig
visdom. De hafva varit eftersträfvade af män och kvinnor, högättade och
lågättade. Så äfven läkerunorna, som gått i arf hos somliga
höfdingeätter. En bön om dessa slags runor ljuder så: »Hel eder,
asagudar! Hel eder, asynjor! Hel dig, månggagnande Fold (jordgudinnan
Frigg)! Måtten I gifva oss ordets makt och andens odling och läkande
händer, medan vi lefva!»

Om Heimdalls sonson, Borgar-Skölds son Halfdan, har det sjungits, att
han, den unge,

{»lärde sig runor,}\\
{evighetsrunor}\\
{och jordelifsrunor;}\\
{han lärde sig sedan}\\
{att människor frälsa,}\\
{döfva svärds-egg,}\\
{lugna hafvet,}\\
{fågelsång tyda,}\\
{dämpa eldar,}\\
{sona och döfva,}\\
{sorger häfva.»}\\

Åtgärder vordo i urtiden träffade att sprida dessa välgörande runor
bland alla varelseslägter. Oden utbredde dem
\protect\hypertarget{lb1625905.xhtmlux5cux23start25}{}{}\protect\hypertarget{lb1625905.xhtmlux5cux23start25-a}{}{}\protect\hypertarget{lb1625905.xhtmlux5cux23start25-b}{}{}\protect\hypertarget{lb1625905.xhtmlux5cux23start25-c}{}{}\protect\hypertarget{lb1625905.xhtmlux5cux23start25-d}{}{}
bland asarne; alfen Dain och Mimersonen Dvalin, som fått lära dem af
Mimer, spredo dem bland alfer och dvärgar. Genom Heimdall kommo de till
människorna. Den goda gåfvan, blandad i heligt mjöd, sändes vida vägar,
»och sedan är den bland asar, bland alfer, bland visa vaner och bland
människors barn». Ej häller jättarne vordo lottlösa. De voro ju
upptagne, äfven de, i urtidens fridsförbund, och Mimer sände dem runor,
hvilka grundlagt den kunskap, som finnes hos Jotunheimsslägten och sedan
af den blifvit illa brukad.

\paragraph{XII.}

\paragraph{ASAGUDARNE.}

Den vakt af stridbare gudar, som tagit bostäder på Yggdrasils öfre
grenknippa i Asgard, för att därifrån öfvervaka och skydda världen, äro
desse:

1) \emph{Oden}, allfadern och tidsåldrarnas fader. Sedan ofridsåldern
kommit i världen, kallas han äfven valfader (»de på valplatsen fallnes
fader»), härfader, segerfader, och han bär därjämte många andra namn.
Hans växt är majestätisk, hans panna hög, hans ögonbryn starkt tecknade,
hans anletsdrag ädla, hans blick tankfull och grubblande. Sedan han,
drifven af kunskaps-trängtan, försänkte sitt ena öga i Mimers
visdomsbrunn, är han enögd, och då han uppenbarat sig bland människor,
har han visat sig med det lytet, när han velat låta förstå hvem han är.
Då uppträder han vanligen ock i sid hatt och höljd i en vid blå kappa.
Men i sin egen sal, bland gudar och einheriar (döde och till Oden
upptagne hjältar), synes han skön och utan lyte och, ehuru tankediger,
så blid, att alla med glädje skåda upp till hans vördnadsbjudande
anlete. Hans skägg faller ned öfver hans bröst.

Han eger en åttafotad häst Sleipner, den ypperste af alla springare.
Denne hade han icke i begynnelsen, utan fick den genom en händelse, som
sedan skall omtalas. I sadeln på Sleipner och följd af sina ulfhundar
Gere och Freke
\protect\hypertarget{lb1625905.xhtmlux5cux23start26}{}{}\protect\hypertarget{lb1625905.xhtmlux5cux23start26-a}{}{}\protect\hypertarget{lb1625905.xhtmlux5cux23start26-b}{}{}\protect\hypertarget{lb1625905.xhtmlux5cux23start26-c}{}{}\protect\hypertarget{lb1625905.xhtmlux5cux23start26-d}{}{}
jagar han stundom nattetid genom luftrymderna, för att ränsa dem från de
skadliga väsen, som kallas gifer och tunridor och äro af rimtursarnes
och sjukdomsandarnes slägt. Oden är världens beskyddare samt folkens och
samhällenas högste styresman. Men han är icke världshändelsernas
enväldige länkare. Och ehuru han är den kunnigaste bland gudar, dröjde
det länge, innan hans tanke trängde ned i tingens upphof och tingens
slutmål. Det var fördenskull han så länge var den djupe grubblaren. Ödet
har delat världsmakten med honom och förbehållit den större andelen åt
sig själf. Denna okända makt har sina ombud i de tre höga ödesdiserna,
Urd och hennes systrar. Det är de, icke Oden, som kåra till lif och
gifva lagar och bestämma lefnadshändelserna och lifslängden för dem, som
de till lif kårat. Dock äro åt Oden förbehållna stora
företrädesrättigheter, förutan hvilka han icke skulle vara den herrskare
han är. Han, men icke nornorna, har att bestämma öfver fältslagens
utgång, att gifva seger och utse dem, som skola falla i striden. Ty utan
denna rätt kunde han icke vara staternas och folkens herre, sedan
stridsåldern kom i världen. Och han är öfverdomaren i den domstol, som
dagligen sitter vid Urds källa och dömer de döda, allt efter deras
sinnelag och gärningar, till salighet eller osalighet. Svaghetssynder
dömer han mildt. Han är böjd att ursäkta mans och kvinnas felsteg, när
icke äktenskapstrohetens band slitas, och ehuru han varnar mot
öfverflödigt drickande, är han öfverseende mot dem, som skåda flitigt i
dryckeshornet, när intet annat faller dem till last. Men illvillige
lögnare, förrädare, nidingar, äktenskapsbrytare, helgedomsskändare
fördömer han till rättvisa straff. Han älskar mildhet och hatar grymhet,
och under sina vandringar bland människor har han pröfvat dem och
utdelat redan i detta lifvet åt den hårdhjärtade näpst och åt den
barmhärtige belöning. Goda sedeläror har han gifvit människornas barn.
»Hel de gifvande!» har han ropat till dem, utanför hvilkas dörrar
vandraren står hungrig och förfrusen. Själf en visdomssökare, som
genomletat världarna och gästat fiendtliga jätteväsen, för att finna
\protect\hypertarget{lb1625905.xhtmlux5cux23start27}{}{}\protect\hypertarget{lb1625905.xhtmlux5cux23start27-a}{}{}\protect\hypertarget{lb1625905.xhtmlux5cux23start27-b}{}{}\protect\hypertarget{lb1625905.xhtmlux5cux23start27-c}{}{}\protect\hypertarget{lb1625905.xhtmlux5cux23start27-d}{}{}
sanningar, uppmanar han enhvar att odla sitt vett. Det gläder honom, när
de dödlige stifta vänskap och troget bevara den. Han varnar mot
öfvermod, och han vill, att en man skall föredraga fattigdom med
själfständighet framför välmåga med beroende. Feghet hatar han, mannamod
älskar han. »Glad och god skall en man vara, medan han går sin död till
mötes.»

Många bekymmer har Oden att bära för världens och människornas skull.
Men hans tyngste tanke är väl den, att han icke är fullkomligt tadelfri.
Dess större hugnad blef det honom, att han vardt fader till den syndfrie
Balder. Så länge denne oskyldige gud blomstrade i ungdomskraft, fruktade
Oden icke för sin makts och sin världs framtid; men när Balder tynade,
intogs han af dystra aningar, och när Balder dött, hade han fått visshet
om en bidande undergång, men ock fått tröst därför. Ingen vet hvad han
hviskade i den döde Balders öra; kanske var det den tröstens ord, ty
Balder skall återkomma och ett rättfärdighetens rike grundläggas. För
goda ändamål har Oden tillgripit medel, som han själf beklagat.
Tillfredsstäld inom sig är han därför icke.

Så mild asafadern är, synes han dock för mången förskräckande. Han, som,
sårad med spjut, offrade sig själf, för att kunna varda Midgards danare
och människoslägtets fader, har därmed gifvit de dödlige ett föredöme,
och han fordrar, att den enskilde skall, då fara hotar, gifva sitt lif
för folket. Däraf kommer, att man offrat människor åt honom för
samhällets bästa och ej velat alldeles afstå därifrån, ehuru han själf
förkunnat, att bättre är att icke bedja alls, än att blota öfver höfvan,
och att han ser till den offrandes hjärtelag, ej till offergåfvans
beskaffenhet.

Hög och gåtfull har han alltid stått för sina tillbedjare. Han kan
skämta vid mjödet, ja äfven inför människorna skämta med sig själf; men
detta är krusningar på ytan af en ande, hvars djup griper med aningsfull
vördnad.

2) \emph{Tor}, son af Oden och Frigg. I hans skaplynne ligger ingen
dunkel gåta. Det skaplynnet är klart som en genomskinlig källas vatten.

\protect\hypertarget{lb1625905.xhtmlux5cux23start28}{}{}\protect\hypertarget{lb1625905.xhtmlux5cux23start28-a}{}{}\protect\hypertarget{lb1625905.xhtmlux5cux23start28-b}{}{}\protect\hypertarget{lb1625905.xhtmlux5cux23start28-c}{}{}\protect\hypertarget{lb1625905.xhtmlux5cux23start28-d}{}{}

Tor har breda skuldror och ståtlig kroppsbyggnad, hvars resning och
lemmar nogsamt visa hans ofantliga styrka. Bland gudarne finnes ingen så
stark som han. På denna väldiga kropp är ett ungdomligt hufvud med
blondt hår och ljusa skäggfjun, omgifvande ett ansigte, som uttrycker
öppenhet och ärlighet, trofasthet och, oaktadt den skarpa blicken, äfven
mycken godlynthet, när han icke vredgas och brusar upp, hvilket lätt
händer, om någonting lågt och elakt retar honom. Tor är den luftränsande
och välgörande åskans gud. När rymden är kvalmig och den torkande jorden
längtar efter rägn, far han ut på sin af bockarne Tanngnyst och
Tanngrisner dragna vagn genom de från Eiktyrner ulseglande med
vaferdimmor och vatten laddade molnen. Då tändas vaferdimmorna och varda
till blixtar, och vattnet nedstörtar i befruktande skurar.

Om Tors hammare och starkhetsbälte och hans många äfventyr med jättarne
skall längre fram talas. Han är Midgards värnare mot Jotunheims-makterna
och är den flitige plöjarens gode vän. Till trälen, som arbetar på åkern
och i skogen, ser Tor med blida ögon. Missaktas den ofrie och betungade
af andra, så är Tor honom alltid bevågen och i domen öfver de döde talar
han väl för honom och förordar, att hans mödor må lönas med salighetens
lott. Därför säges det, att »Tor eger trälarne, men Oden jarlarne». Oden
älskar honom högt, och han var alltid sin faders trogne son. Stor och
stark som han är, kräfver han mycket både af mat och dryck, och hans
dryckeshorn är det största i Asgard. Han är gift med Sif, som är af
alfernas stam. Han har två söner, Mode och Magne, som brås på sin fader.
Innan han vardt gift med Sif, födde han Magne med jättekvinnan Jernsaxa,
som också är en af Heimdalls nio mödrar.

Den omisstänksamme Tor trodde länge, att Loke var hederlig och pålitlig,
ty denne var ju Odens ungdomsvän och upptagen af honom i Asgard. Tor
var, som de andre gudarne, road af Lokes infall och upptåg, och tog
honom stundom med sig på sina utfärder. Men sedan Tor blifvit några
gånger
\protect\hypertarget{lb1625905.xhtmlux5cux23start29}{}{}\protect\hypertarget{lb1625905.xhtmlux5cux23start29-a}{}{}\protect\hypertarget{lb1625905.xhtmlux5cux23start29-b}{}{}\protect\hypertarget{lb1625905.xhtmlux5cux23start29-c}{}{}\protect\hypertarget{lb1625905.xhtmlux5cux23start29-d}{}{}
grundligt lurad af Loke, förlorade han allt förtroende för honom och
tålde honom illa.

3) \emph{Balder}, en ljusgud, »solskifvans mäktige främjare», hvars
uppgift i världsordningen var att å sin häst eller sitt luftskepp
Ringhorne vara en skyddsvakt för Sol och Måne på deras färder öfver
himmelen. Balder är son af Oden och Frigg. Han är den mäst högvuxne
bland asarne och den ojämförligt skönaste af alla gudar. Balder älskade
icke kriget; han sörjde öfver, att det någonsin uppkommit. Det oaktadt
var han aldrig sen att draga vapen mot den ohjälpliga ondskan. Så skön
han är, så god och ädelsinnad är han äfven. Därför vardt han lofvad och
älskad af alla väsen, som äro i stånd att fatta tillgifvenhet för hvad
som är hjärtegodt och rent. Försonlig mot sina fiender och böjd att
förlåta, var han världens fredsstiftare och milde domare. Dömde
\emph{han} strängt, då stod det med den dömdes sak så illa, att domen
icke \emph{kunde} jämkas eller förmildras. Människorna hafva uppkallat
de hvitaste blommor efter honom, emedan de i dessas hvithet trott sig
finna något, som liknade Balders pannas glans. Han blef gift med
måndisen Nanna, dotter af månguden Nöckve (»skeppsegaren»), och hade med
henne sonen Forsete.

4) \emph{Had}, en vacker yngling, till lynnet obetänksam, häftig,
lättrörd, lättförledd, lidelsefull och med stormande känslor. Hans
uppförande berodde af det inflytande, hvarunder han kom. En tid lät han
sig leda af Balder och gjorde sig då förtjent af allas loford; därefter
lät han leda sig af Gullveig och Loke och begick då gärningar, som han
djupt ångrat. Had var rikt begåfvad och en stor idrottsman. Som sångare
och harpospelare kunde han röra allas hjärtan och väcka de mäst skilda
känslor hos åhörarne: glädje, hat, sorg och medlidande. Ett särskildt
sångsätt är uppkalladt efter honom. Som simmare och näfkämpe var han
oöfverträfflig och förde alla vapen väl, äfven bågen, i hvars skötande
han öfverträffades endast af Valands broder Egil, väktaren vid
Hvergelmers och Jotunheims vatten, samt af Egils son Ull.

\protect\hypertarget{lb1625905.xhtmlux5cux23start30}{}{}\protect\hypertarget{lb1625905.xhtmlux5cux23start30-a}{}{}\protect\hypertarget{lb1625905.xhtmlux5cux23start30-b}{}{}\protect\hypertarget{lb1625905.xhtmlux5cux23start30-c}{}{}\protect\hypertarget{lb1625905.xhtmlux5cux23start30-d}{}{}

5) \emph{Forsete}, Balders son och afbild, rättvisans gud, skarpsinnig i
vanskliga rättsfrågors afgörande och mer än någon i stånd att förlika
tvistande.

6) \emph{Brage}, den långskäggige, Valhalls skald och harpolekare, son
af Oden och gift med Idun, som eger föryngringens äpplen. Ståtlig tager
han sig ut, där han sitter vid dryckesbordet i Valhall, helt nära Odens
högsäte, och han dväljes gärna vid mjödhornet, hvarför också den
försmädlige Loke en gång kallade honom en »bänkprydare», dugligare att
vara det, än att vara stridsman. Brage är vis och vältalig. Vid
offergillen i Midgard pläga många egna fjärde skålen åt honom, sedan man
förut druckit till ära för Oden, Njord och Fröj. Bägaren, som egnas
Brage, kallas löftets bägare, ty vid den aflägges löfte om en eller
annan storbragd, värdig att af skalder besjungas.

7) \emph{Vidar}, son af Oden och jättekvinnan Grid. Han är tystlåten,
intager blygsamt det lägsta sätet vid gudabordet i Valhall och stiger på
Odens tillsägelse upp och fyller hornet åt andra, äfven åt den
föraktlige Loke, utan att visa sig kränkt af sådan tjänstgöring. Han har
ett kraftigt utseende; men få äro de som ana, huru stark han är: näst
Tor den starkaste af alla asagudar. Än färre anade, att det är han, den
hjältemodige tigaren, som skall i Ragnarök-striden hämnas sin fallne
fader.

8) \emph{Tyr}, son af Oden och den fagra jättekvinna, som vardt jätten
Hymers maka, har varit Tors ledsagare på mer än en af dennes
äfventyrliga färder i Jotunheim, och säkert är, att Tor icke kunde haft
en modigare och pålitligare. Tyr är de i fylkingarnes led kämpande
stridsmännens (ej själfva krigets) gud och en förebild för dem i
uppoffring för det helas bästa. Sin djärfhet och offervillighet visade
han särskildt, då han, såsom längre fram skall omtalas, lade sin högra
hand som pant i Fenrersulfvens gap och fick den afbiten vid handleden.
Han kämpar sedan med den vänstra. Det kan sägas, att en här utan
fältherre saknar sin högra hand och kämpar vänsterhändt. Fältherrarnes
förebild är
\protect\hypertarget{lb1625905.xhtmlux5cux23start31}{}{}\protect\hypertarget{lb1625905.xhtmlux5cux23start31-a}{}{}\protect\hypertarget{lb1625905.xhtmlux5cux23start31-b}{}{}\protect\hypertarget{lb1625905.xhtmlux5cux23start31-c}{}{}\protect\hypertarget{lb1625905.xhtmlux5cux23start31-d}{}{}
Oden, som uppfunnit den viggformiga slagordningen. Soldaternas förebild
är Tyr. Oden och Tyr äro som två händer, som i förening tillvinna sig
segern.

9) \emph{Vale}, son af Oden och aftonrodnads-alfen Billings dotter, den
»solhvita» Rind. Föddes, för att på Had hämnas Balders död, som annars
skulle förblifvit ohämnad. Nattgammal uppfyllde han detta kall.

Dessa gudar äro de egentlige asarne. Till dem komma såsom upptagne i
Valhall tre vanagudar, Njord, Fröj och Heimdall, samt en af alfernas
stam, Ull.

10) \emph{Njord}, broder till Odens maka Frigg. Han är segelvägarnes,
sjöfartens, fiskets, handelns och rikedomens gud och som sådan flitigt
dyrkad. Han födde med sin syster guden Fröj och nio döttrar, bland
hvilka den underbart sköna Fröja. De andra åtta äro Fröjas diser. Bland
dem äro läkedomsgudinnan Eir, samt Bjärt (»den lysande»), Blid (»den
blida») och Frid (»den fagra»), hvilkas namn tyda på deras älsklighet.
Njord är, som alla vanagudar, vänlig och gagnerik. Sänd i oskuldstiden
som gisslan från vanerna till Oden, skall han mot världsålderns slut
återvända »till vise vaner», där han allt fortfarande har en odalborg i
Noatun (»skeppens stad»). Han vardt längre fram i tiden gift med
skidlöparedisen Skade, dotter af jätten Tjasse. Tjasse är densamme som
urtidskonstnären Valand, Ivaldesonen, som genom det oblidkeliga och
grymma hat han fick till gudarne förvandlades till sin natur och vardt
ett fördärfligt jätteväsen.

11) \emph{Fröj}, årsväxtens människovänlige herre. Hans födelse hälsades
med glädje af alla ätter, utom jättarnes, och gudarne skänkte honom
Alfheim i tandgåfva. Fröj vardt därigenom furste öfver alferna, bland
hvilka Ivaldes tre söner voro de märkligaste. Och eftersom bruket var,
att konungasöner sändes att uppfostras af deras fäders underordnade
höfdingar och förtroendemän, så sändes Fröj att uppfostras hos Ivaldes
söner -- den blifvande skördeguden hos de naturvärkmästare, som smidde
världsgagnande gudaklenoder och växtlighetssmycken. Huru Valand svek
sina plikter mot
fostersonen\protect\hypertarget{lb1625905.xhtmlux5cux23start32}{}{}\protect\hypertarget{lb1625905.xhtmlux5cux23start32-a}{}{}\protect\hypertarget{lb1625905.xhtmlux5cux23start32-b}{}{}\protect\hypertarget{lb1625905.xhtmlux5cux23start32-c}{}{}\protect\hypertarget{lb1625905.xhtmlux5cux23start32-d}{}{},
skall förtäljas sedan; likaledes huru Fröj friade till jättedottern
Gerd. -- Om Fröjs skaplynne kan det vara nog att upprepa det vitsord,
som Tyr i ett skaldestycke gifvit honom: »Fröj är (efter Balders död)
den bäste af bålde ryttare i asars gårdar; han kommer ingen mö och ingen
mans hustru att begråta förlusten af sina kära, och han vill lösa enhvar
ur band och bojor».

12) \emph{Heimdall}, den heliga eldens gud. Om hans födelse och lefnad
bland människorna är förut taladt. Sedan han fulländat sitt vigtiga kall
i Midgard och blifvit upptagen i Asgard, har Oden anförtrott åt honom
vakten på bron Bifrost. Emedan den norra broänden, som ligger utanför
Nifelhel i rimtursarnes grannskap, är mäst hotad af gudafiender, är där
en stark borg uppförd, som kallas Himmelsvärnet. Där är Heimdalls
bostad. Evig natt rufvar öfver denna näjd, och de vilda vinterstormarne
hafva där sitt tillhåll och sin utflyktsort. Också gäckade Loke en gång
Heimdall därför, att han fått boning i ett land, där han är så utsatt
för oblid väderlek. Men Heimdalls borg är väl inrättad, till borgvaktens
trefnad; och »i dess behagliga sal dricker Heimdall glad det goda mjöd».
Och gudarne glädjas åt sin trogne väktare, när de se honom, själf
lysande med hvitaste glans, rida på sin häst Gulltopp i den yttersta
Nordens svarta mörker. Heimdall behöfver mindre sömn än en fågel; han
ser lika väl, natt som dag, hundra raster omkring sig, och hans hörsel
är så fin, att han hör gräset växa på jorden. Mot Loke har han ständigt
hyst ovilja, och genom sin skarpsynthet och oaflåtna vaksamhet har han
mer än en gång omintetgjort dennes listiga anslag.

13) \emph{Ull.} Innan Sif, den guldlockiga disen, vardt Tors maka, hade
hon varit gift med Ivaldesonen Egil, den store bågskytten och
skidlöparen, och med honom födt sonen Ull, som blef en mästare i sin
faders idrotter och därför kallas bågeguden, jagtguden, skidguden. När
Sif giftes med Tor, vardt äfven Ull upptagen i Asgard och fick gudarang.
Ulls skidor äro ett konstvärk från Ivaldesönernas värkstad; de
\protect\hypertarget{lb1625905.xhtmlux5cux23start33}{}{}\protect\hypertarget{lb1625905.xhtmlux5cux23start33-a}{}{}\protect\hypertarget{lb1625905.xhtmlux5cux23start33-b}{}{}\protect\hypertarget{lb1625905.xhtmlux5cux23start33-c}{}{}\protect\hypertarget{lb1625905.xhtmlux5cux23start33-d}{}{}
kunna, i likhet med hans faders, begagnas å vatten, såväl som å land,
och de kunna äfven begagnas som sköld. Ull är vacker, smärt och smidig
och i allo en utmärkt stridsman. Därtill kommer, att han har en
herrskares och styresmans egenskaper. När i ofredens tidsålder ett krig
uppkom mällan asarne och vanerna och desse senare vordo ensamme herrar i
Asgard, satte de fördenskull Ull i Odens högsäte och gåfvo honom Odens
namn och herrskarerättigheter.

Som redan är omtaladt, vordo i oskuldstiden två varelser af jättestam
upptagna i Asgard: Loke och Gullveig.

Loke och hans bröder Helblinde och Byleist äro söner af orkan- och
åskjätten Farbaute (»den med fördärf slående»). Orkanens skyfall och
störtskurar, som genom de uppsvällda floderna förmäla sig med hafvet,
gåfvo upphof åt Helblinde, som blef gift med hafsdjupets jättinna Ran.
Orkanens hvirfvelvind gaf upphof åt Byleist (»den östlige stormaren»).
Ett ljungeldsslag från orkanen födde Loke till världen. Hans moder är
Laufey (»blad ön», d. v. s. den af orkanblixten träffade trädkronan).
Det våldsamma i Farbautes skaplynne visade sig ej på ytan af hans son
Lokes, så länge han gick fritt omkring i Asgard, utan först sedan han
vardt i fängsel nedlagd i sin pinohåla. Därifrån åstadkommer han
jordskalfven. Det var på smygvägar han förberedde en kommande
världsbrand, och till utseendet var han förledande och till laterna
inställsam. Loke var en fin iakttagare af andras skaplynne, och när han
upptäckt deras svaga punkter, visste han att utlägga frestelser i
enlighet med dem. Vid sina planers uppgörande tänkte han långt framför
sig och förstod inrätta dem så, att hans allra värsta såge ut att vara
de fördelaktigaste och bästa. Kvickhet, lastbarhet och bottenlös elakhet
förenades i hans natur.

Gullveig är Loke lik, ja värre och, i grunden af sitt väsen,
förfärligare än han. De arbetade i endrägt tillsammans på gudarnes och
världens fördärf. Gullveig, som först framföddes af rimtursen Rimner,
vardt tre gånger född till världen af olika jätteföräldrar och tre
gånger af gudarne bränd. Det
\protect\hypertarget{lb1625905.xhtmlux5cux23start34}{}{}\protect\hypertarget{lb1625905.xhtmlux5cux23start34-a}{}{}\protect\hypertarget{lb1625905.xhtmlux5cux23start34-b}{}{}\protect\hypertarget{lb1625905.xhtmlux5cux23start34-c}{}{}\protect\hypertarget{lb1625905.xhtmlux5cux23start34-d}{}{}
oaktadt lefver hon ännu. Hon är en och tre på samma gång, och det är på
henne den Ragnaröksiande sången (Völuspa) syftar, när den talar om »de
tre förskräckliga tursamöar», hvilkas ankomst från Jotunheim gjorde slut
på frids- och guldåldern.

Bland de gudinnor, som bo i Asgard, hafva här ofvanför redan omtalats
Frigg, Sif, Nanna, Idun, Skade, Gerd, Fröja och Eir. Om Frigg må nämnas,
att ehuru hon är god maka, likasom hon också är öm moder, har hon dock
ofta sin vilja för sig och gynnar andra gunstlingar bland människors
barn än Oden. Hon ingriper gärna i sin makes befattning med folkens och
samhällenas öden, och det har händt, att när ett folk, hvars motståndare
haft Odens ynnest, vändt sig med böner till Frigg, hon förstått ställa
det så, att Oden sett sig föranlåten att skänka hennes skyddslingar
segern. Oden har berömt hennes förutseende och tystlåtna klokhet. Frigg
har en syster Fulla, som är hennes förtrogna och förvarar hennes
smycken. Fulla är nära vän till Nanna, Balders maka. Emedan Fulla är
himladrottningens syster, bär hon, till tecken af sin värdighet, ett
diadem.

Asgard har många präktiga gudaborgar. Förnämst bland dem är Odens borg
Valhall. Det namnet, som betyder »de på slagfälten kårades hall» hade
borgen icke i begynnelsen, utan fick det under ofredens världsålder,
sedan krig uppkommit i Midgard. Den kallas äfven
Bilskirner{\protect\hyperlink{lb1625905.xhtmlux5cux23n1}{1}}. Det
närmaste området kring Valhall heter Gladsheim, och där växer lunden
Glaser, »som står med gyllne löf utanför Odens portar». Vidsträckt och
guldstrålande står Valhall på Gladsheims mark. Salen, där asagudarne
samlas kring sin fader och där de kårade få säte kring dryckesborden, är
bjälkad med spjutskaft, taklagd med silfversköldar och bonad med
guldbrynjor och vapen. Valhall har 540 stockvärk och 540 portar, dessa
så stora, att när en gång Ragnarökstriden kommer, skola genom hvarje
port 800 einheriar (kårade hjältar) på en gång
\protect\hypertarget{lb1625905.xhtmlux5cux23start35}{}{}\protect\hypertarget{lb1625905.xhtmlux5cux23start35-a}{}{}\protect\hypertarget{lb1625905.xhtmlux5cux23start35-b}{}{}\protect\hypertarget{lb1625905.xhtmlux5cux23start35-c}{}{}\protect\hypertarget{lb1625905.xhtmlux5cux23start35-d}{}{}
tåga ut. Och dock är det endast halfva antalet einheriar, som bor i
Valhall; den andra hälften kåras af Fröja och bor hos henne. Einheriarne
tillbringa sina dagar dels med fäster, vid hvilka sköna stridsdiser,
valkyriorna, räcka dem de fyllda dryckeshornen, dels med stridslekar på
Asgards slätter. Drycken, som de dricka, framprässas ur världsträdets
kronas saftiga blad, som innehålla mjödet från Mimers källa, blandadt
med safterna från Urds källa och Hvergelmer. Vid måltiderna framsättes
maten såväl för Oden, som för alla hans bordvänner, men han ger sin
andel till Gere och Freke, som ligga vid hans fötter. Han lefver själf
allena af mjöd, när han vistas i Valhall och icke underkastar sig andra
lefnadsvilkor.

Fröjas område i Asgard heter Folkvang, och den stora praktfulla sal,
hvari de kämpar, som hon kårat, vistas, heter Sessrymner. Den borg, som
Balder i Asgard bebodde, är Glitner; den har guldbjälkar och silfvertak.
Efter Balders förflyttning till underjorden är hans son Forsete herre i
Glitner, och han själf bor i Breidablik på Glansfälten i underjorden.
Minst af alla områden i världen har detta blifvit fläckadt med ondt.
Tors område i Asgard heter Trudheim och Trudvang.

Flere gudar och gudinnor hafva sina borgar och odalmarker belägna
annanstädes än i Asgard. Det är redan förut berättadt, att Heimdalls
borg Himmelsvärnet är byggd i yttersta Norden vid Bifrosts brohufvud
därstädes, och att Njords odalgård Noatun är väster om världshafvet i
Vanaheim. Skade bor hälst i sin faders land Trymheim, på hvars
fjällsidor och slätter hon går på skidor och jagar. Fröj vistas mäst i
Asgard; men det rike han beherrskar är Alfheim. Ulls odalmarker,
Ydalarne, äro belägna kring hans fader Egils borg, Ysäter, vid det
vatten, kalladt Elivågor, Raun och Gandvik, som skiljer det stora
Svitiod från Jotunheim. Dessa marker kallas Ydalarne (»bågarnes dalar»)
emedan Egil och Ull äro de ypperste bågskyttarne. Vidar den tyste har
också ett landområde, Vide, bevuxet med ris och högt gräs; det utgör en
del af Vigrids stora slätt, på hvilken Ragnarökstriden skall utkämpas.

\protect\hypertarget{lb1625905.xhtmlux5cux23start36}{}{}\protect\hypertarget{lb1625905.xhtmlux5cux23start36-a}{}{}\protect\hypertarget{lb1625905.xhtmlux5cux23start36-b}{}{}\protect\hypertarget{lb1625905.xhtmlux5cux23start36-c}{}{}\protect\hypertarget{lb1625905.xhtmlux5cux23start36-d}{}{}

\paragraph{XIII.}

\paragraph{TORS OCH EGILS VÄNSKAP. IVALDESLÄGTEN.}

Såsom den gamle jätten Vaftrudner själf sagt, ha jättarne ett ondt gry i
sig, och det tjenade i längden till föga, att de hade blifvit upptagne i
det fredsförbund, som makterna i oskuldstiden slöto med hvarandra.
Jättarne tillväxte och förökade sig i Jotunheim, och sedan de blifvit
talrika, och deras manstarka slägter fått sådana höfdingar som Hymer,
Geirraud och Gymer, dröjde det icke länge, innan de lade i dagen sitt
onda sinnelag mot gudarne och mot människorna, som under gudarnes hägn
bebygga Midgard.

Det första förebudet till fiendskap mällan gudarne och jättarne
inträffade i jätten Vingners gård. Det är förut omtaladt, att Oden hade
anförtrott åt Vingner och hans hustru Lora att uppfostra Tor. Gossen
visade tidigt ovanliga kroppskrafter och var redan vid tolf års ålder
väldigt stark. Vanligen råder ett godt och trofast förhållande mellan
föräldrar och fosterbarn; men här slutade förhållandet så, att Tor
ihjälslog både Vingner och Lora. De forntidssånger, som besjungit denna
tilldragelse och omtalat dess orsak, äro förlorade; men när man
betänker, huru i grunden välvillig Tor är, och att han aldrig lönat godt
med ondt, samt att de allra fleste jättar visat ett otillförlitligt
sinnelag, så är det troligt, att Vingner och Lora, anande i sin
fosterson en framtida farlig jättefiende, ville förrädiskt bringa honom
om lifvet, och att han till billigt straff härför drap dem. Säkert är,
att han från Vingners gård medförde som segerbyte den stenhammare han
sedan alltid nyttjat mot jättarne, utom under den korta tid, då han egde
en järnhammare, som Mimersonen Sindre smidde åt honom.

Stor syntes likväl den från Jotunheim hotande faran icke vara, så länge
höfdingarne för den gränsvakt, som var uppställd mot Jotunheim, nämligen
Ivalde och hans söner, höllo den trohetsed de svurit gudarne. Desse
alffurstar, som fått sig vakten anförtrodd längs hela det vattendrag,
som från
\protect\hypertarget{lb1625905.xhtmlux5cux23start37}{}{}\protect\hypertarget{lb1625905.xhtmlux5cux23start37-a}{}{}\protect\hypertarget{lb1625905.xhtmlux5cux23start37-b}{}{}\protect\hypertarget{lb1625905.xhtmlux5cux23start37-c}{}{}\protect\hypertarget{lb1625905.xhtmlux5cux23start37-d}{}{}
Hvergelmers källa rinner upp till jordytan och skiljer Midgard från
Jotunheim, hade växtlighetsdiser till systrar och hustrur och voro
således äfven genom slägtskapsband knutna till de makter, som skydda
världsträdets grönska och gifva Midgardsboarne skördar.

Bland de tre Ivaldesönerna sysslade Valand mäst i sin smedja, där han
smidde gudasmycken och växtlighetsklenoder och därmed gagnade den sak
han svurit att tjena. Fröj, hans fosterson, vistades ofta hos honom. Det
var därför de båda andra bröderna, som bevakningen af Midgards norra
utkant närmast ålåg, och Egil utmärkte sig särskildt för oförtruten och
tapper vakttjänst. Hans första hustru hette Groa. Han bodde med henne å
Ysäter, en väl omgärdad, behagligt inredd och guldprydd borg vid
Elivågor. Där fanns en under honom lydande besättning af krigiske alfer.
Enär Egil och Groa länge voro utan barn, upptogo de som fosterbarn en
gosse, Tjalve, och en flicka, Raskva, som de hittat i en damm.

Emedan Egil var den ypperste af alla bågskyttar, kallades han också
Örvandel (»pilskötaren»). Tor, hvilken alltsom oftast besökte
gränslandet, för att ha ett öga på jättarne, var Egils gode vän och
plägade alltid vid sådana besök taga in hos honom. Ofta plägade också
Groa vistas i Tors hem, Trudvang, när Egil, ensam eller med sina alfer,
gjorde ströftåg på Elivågor och å jättekusten.

Landet, Svitiod det stora, där bröderna bodde, var guldrikt. Dess
nordligaste älfvar flöto på bäddar af guldsand. Ivalde, brödernas fader,
var den förste herrskaren öfver det stora Svitiod, hvaraf svearnes land
är den sydligaste delen. Norr om dem beherrskade Ivalde ett
skidlöparefolk, skridfinnarne. Han kallas fördenskull äfven Finnkonung.
Han var en väldig dryckeskämpe och lika skicklig att handtera spjutet
som Egil bågen.

{\_\_\_\_\_\_\_\_}

\protect\hypertarget{lb1625905.xhtmlux5cux23start38}{}{}\protect\hypertarget{lb1625905.xhtmlux5cux23start38-a}{}{}\protect\hypertarget{lb1625905.xhtmlux5cux23start38-b}{}{}\protect\hypertarget{lb1625905.xhtmlux5cux23start38-c}{}{}\protect\hypertarget{lb1625905.xhtmlux5cux23start38-d}{}{}

\paragraph{III.}

\paragraph{SYNDENS OCH OFRIDENS TIDSÅLDER.}

\paragraph{XIV.}

\paragraph{TROLLDOMSRUNORNA.}

Ofvan är berättadt, huru gudarne sände Heimdall till urmänniskorna, för
att lära dem seder, nyttiga och prydande konster och heliga runor och
runosånger. Heimdallsrunorna -- så kunna vi kalla dem -- hade det
ändamål att göra människorna gudfruktiga och visa, samt rättfärdiga och
hjälpsamma mot hvarandra, och att förläna dem makt att kufva eld och
vind, när dessa uppträdde skadligt. De voro till för att »människor
frälsa, lugna hafvet, fågelsång tyda, dämpa eldar, sona och döfva
(smärtor), sorger häfva». De skänkte läkekraft och tröst för
jordelifvets kval och gåfvo läror för den vandel, som leder genom döden
till saligheten.

Men redan i urtiden uppträdde en ond makt, för att motvärka och förstöra
de goda inflytelser, som Heimdallsrunorna utöfvade på människoslägtet.
Det var en jättekvinna, Gullveig, tursen Rimners dotter, en ättelägg af
frostmakterna, skapelsens fiender. Hon var vacker, hon som Loke, och
upptogs likasom han i Asgard som gisslan från jättevärlden. Ur
hemlighetsfulla djup i sitt eget väsen hade hon hämtat ingifvelserna
till de konster, med hvilka hon kom att förföra gudar och människor. Hon
förstod att gifva skadliga krafter åt föremål, som i sig själfva voro
oskadliga; hon uppfann den onda säjden, hvarmed man kan trolla ofall,
sjukdom och död på människor. Från sin säjdstol utlärde hon de
trollspråk,
\protect\hypertarget{lb1625905.xhtmlux5cux23start39}{}{}\protect\hypertarget{lb1625905.xhtmlux5cux23start39-a}{}{}\protect\hypertarget{lb1625905.xhtmlux5cux23start39-b}{}{}\protect\hypertarget{lb1625905.xhtmlux5cux23start39-c}{}{}\protect\hypertarget{lb1625905.xhtmlux5cux23start39-d}{}{}
hvarmed skadefröjden alltsedan velat vålla ofred och olycka, slå med
sinnesförvirring, sjukdom och död. Hon hade Lokes inställsamma later,
och det dröjde länge, innan gudarne trodde henne om ondt. Hon var Friggs
tjenarinna och lyckades tillnarra sig Fröjas förtroende. Slutligen
upptäcktes hvilka onda konster hon påfunnit, och att hon sökt locka
Fröja att säjda. Gudarne förfärades öfver de onda följder, som hennes
trollkonster skulle få, ifall de spredes. De dömde henne att dö i eld
och värkställde domen på henne. Det var svårt för lågorna att beröra
henne. Och hennes hjärta, hvari hennes själ låg förborgad, kunde elden
endast halfsveda, ej förbränna. Det var »urkallt» som etterdropparne,
hvaraf Ymer bildats.

Loke, som åsett förbränningen, letade i hemlighet i askan, fann där det
halfsvedda hjärtat och slukade det. Det hade en underlig värkan på
honom. Han kom därigenom att föda en orm till världen. Ormen kastades i
hafvet och växte där med stor hast. Redan en icke lång tid därefter var
han hafvets största vidunder. Och under århundradenas lopp växer han så,
att han kommer att ligga i ring på hafsbottnen rundt omkring Midgard.
Han kallas därför Midgardsormen. Han växer i kapp med ondskan på jorden
och är fullvuxen, när fördärfvet nått sin höjd och världsförstörelsen
nalkas.

Gullveigs första förbränning inträffade vid fridsålderns slut. Hvad som
sedan hände henne skulle, för ordningens skull, längre fram berättas;
men redan här må dock något däraf meddelas. Den döda Gullveigs själ
skulle ha stannat i Nifelhel bland hennes stamfäder rimtursarne; men så
stor var hennes trängtan att göra ondt, att hon väl däraf fick kraft att
komma tillbaka, och hon föddes på nytt af andra jätteföräldrar. Hon
uppträdde då under annat namn, och att hon var Gullveig visste utanför
Jotunheim ingen mer än hon själf och Loke. Efter sin andra födelse
kringvandrade hon i Midgard, där hon kallades Heid (ett namn, som i
aflägsen forntid betydt »trollkvinna»). Likasom Heimdall gick med heliga
runor från gård till gård i Aurvangalandet, gick hon från gård till gård
därstädes med sina oheliga, »alltid efterlängtad af elakt
kvinnfolk\protect\hypertarget{lb1625905.xhtmlux5cux23start40}{}{}\protect\hypertarget{lb1625905.xhtmlux5cux23start40-a}{}{}\protect\hypertarget{lb1625905.xhtmlux5cux23start40-b}{}{}\protect\hypertarget{lb1625905.xhtmlux5cux23start40-c}{}{}\protect\hypertarget{lb1625905.xhtmlux5cux23start40-d}{}{}».
Hvad ondt hon därmed gjorde kan icke mätas och icke skildras. Äfven i
Asgard lyckades hon innästla sig. När gudarne upptäckt hvem hon var,
brände de henne för andra gången med samma påföljd som den förra. Loke
efterletade och fann hennes hjärta obrändt i askan. Han slukade det och
födde till världen Fenrersulfven.

Hon pånyttföddes, och hon brändes tredje gången. Äfven då fann och
slukade Loke hennes hjärta. Då födde han till världen pästjättinnan
Leikin (med orätt kallad Hel), som blef drottning öfver rimtursarne och
sjukdomsandarne i Nifelhel.

\paragraph{XV.}

\paragraph{JÄTTARNE VILJA UNDERRÄTTA SIG OM TORS STYRKA.}

Alltsedan Tor hade dräpt Vingner och Lora, gingo bland jättarne rykten,
att han var omätligt stark och af obetvingligt mod. Det var för dem af
vigt att få veta, huru härmed förhöll sig, och om Tor kunde vara så
farlig för dem som det påstods.

Och så uppgjorde de med Loke, att han skulle egga Tor att göra ett besök
hos eldjätten Fjalar, höfding för Suttungs söner och mer än någon annan
förfaren i trollkonster.

Fjalar och hans stam bo icke i det egentliga Jotunheim, det nordliga,
utan söder ut i ett underjordiskt land med djupa och skumma dalar.

Tor, väpnad med sin hammare och följd af Loke, gaf sig på färd. Till en
början var vägen den vanliga, öfver Bifrost ned till dess södra broände,
samt öfver några underjordsfloder, som genomvadades. Därefter hade man
att klättra ned i Fjalars »djupdalar» och fortsätta vandringen där. Den
var icke af det upplifvande slaget: ingenting vackert var att se,
snarare mycket, som kunde göra en kuslig till mods. Tor och Loke
vandrade länge genom halfmörker, därefter genom mörker, och slutligen,
när de blifvit rätt trötta, var det nästan som natt omkring dem. Dock
kunde man skymta
\protect\hypertarget{lb1625905.xhtmlux5cux23start41}{}{}\protect\hypertarget{lb1625905.xhtmlux5cux23start41-a}{}{}\protect\hypertarget{lb1625905.xhtmlux5cux23start41-b}{}{}\protect\hypertarget{lb1625905.xhtmlux5cux23start41-c}{}{}\protect\hypertarget{lb1625905.xhtmlux5cux23start41-d}{}{}
nära föremål. En skog syntes ligga tätt framför dem, och öfver den
skönjde de en människoliknande skepnad, alldeles ofantligt hög. Skogen
syntes som busksnår i jämförelse med honom. I detsamma hördes brak och
buller, som om en tung kropp fallit öfver en mängd träd och buskar och
brutit dem, och jätteskepnaden var försvunnen. Loke blef rädd eller
låtsade vara det. Bredvid sig märkte han en klippa eller hvad det kunde
vara, med en bred öppning, nästan så stor som klippväggen, hvari hon
var. Loke tog sin tillflykt dit in, och Tor kom efter. Inne i grottan
var åt ena sidan en tämligen trång och kort gång. Dit kröp Loke in och
gömde sig. Han ställde sig som vore han alldeles förskrämd. Tor satte
sig i sidogångens öppning med handen om hammarskaftet och afvaktade hvad
komma kunde. Kort därefter uppstod ett besynnerligt, långdraget,
snöflande och rullande dån, som vardt allt förfärligare och liknade en
underjordisk åska. Marken gungade, och grottan, hvari Tor och Loke gömt
sig, skälfde i sina grundvalar. Det var jordbäfning, och den blef
långvarig. Med korta uppehåll förnyade sig de rullande dånen och de dem
åtföljande jordskalfven hela natten. Tor själf tyckte, att detta var
hemskt, hälst mörkret var så tjockt, att man ej såg hand framför sig.
Ändtligen var det, som om mörkret förtunnats så pass, att det var
möjligt att nyttja sina ögon. Då gick Tor ut och späjade, så godt det
lät sig göra i dunklet på den bäfvande skogsmarken, där träden vacklade
på sina rötter eller redan lågo kullslagna. Loke följde honom efter. Då
upptäckte han ett par fötter, större än en storvuxen man, och vid dem
ett par ben, långa i förhållande till fötterna, och en bål, hvars
flåsande bröst höjde och sänkte sig, och slutligen ett hufvud, ur hvars
mun och näsborrar det jordskakande dånet kom. Helt visst var det jätten,
som Tor hade sett, innan han gick in i grottan. Jätten sof och snarkade.
Tor vill annars aldrig slå sofvande man; men nu förekom det honom nödigt
att göra ett undantag. Han lyfte hammaren och lät den med asakraft slå
ned på jättens panna. Denne grinade litet i sömnen, men pannan var hel,
och
\protect\hypertarget{lb1625905.xhtmlux5cux23start42}{}{}\protect\hypertarget{lb1625905.xhtmlux5cux23start42-a}{}{}\protect\hypertarget{lb1625905.xhtmlux5cux23start42-b}{}{}\protect\hypertarget{lb1625905.xhtmlux5cux23start42-c}{}{}\protect\hypertarget{lb1625905.xhtmlux5cux23start42-d}{}{}
han fortfor att snarka. Då spände Tor sitt starkhetsbälte hårdare kring
midjan och slog än ett slag. Men utan större värkan. Han slog ett
tredje. Då vaknade resen, förde handen till tinningen och mumlade, att
ett blad eller barr måste fallit ned från trädet ofvanför och väckt
honom. Tor kände sig häpen och visste ej annat än fråga, hvad han hette.
Han svarade, att han hette Skrymer; »men om ditt namn behöfver jag ej
spörja», sade han; »jag vet, att du heter Asator». Därpå satte han sig
upp, såg sig omkring och sade: »jag lade från mig en vante, när jag
skulle gå till hvila i går kväll,» och därmed räckte han ut armen och
tog sin vante, och Tor fick då se, att det var den som han och Loke om
natten vistats i, och att sidogången var tumfingern. »Om du ämnar dig
till Fjalars», tillade han; »så tag dig i akt! Där finner du män, som
äro svårare att ha att göra med än jag». Tor frågade närmaste vägen till
Fjalars gård, och Skrymer utpekade riktningen. Därefter reste han sig
med hela sin längd och var med några steg försvunnen ur Tors åsyn.

Om nu, såsom sannolikt är, Loke rådde Tor att vända om, så skämdes dock
Tor att göra det, utan fann han det bättre anstå sig att fortsätta
äfventyret, hvilken utgång det än finge. Mörkt eller halfmörkt var det
ännu alltjämt, ty någon riktig dager har Fjalars rike aldrig. När de
gått ett drygt stycke väg, skymtade framför dem en den största byggnad,
mer högrest, tycktes det, än Valhall. Snart stodo de framför porten till
omgärdningen. Den kunde Tor ej få upp; men han kröp in mellan spjälarne
i stängslet, och Loke följde honom. Nu sågo de framför sig en väl
upplyst sal, hvars dörr var uppslagen, och de stego in. Fjalar satt där
i högsätet, och på bänkar kring väggarna sutto hans fränder och husfolk.
Så långa som Skrymer voro de på långt när icke; men de fleste likväl så
stora, att Tor i jämförelse med dem torde liknat en halfvuxen pojke. De
båda färdemännen gingo fram och hälsade höfdingen. Denne bjöd dem
välkomne och sporde hvad piltarne hette. Tor och Loke sade sina namn.
Borden gjordes i ordning för en måltid, hvari de främmande bjödos att
deltaga. Medan man spisade,
\protect\hypertarget{lb1625905.xhtmlux5cux23start43}{}{}\protect\hypertarget{lb1625905.xhtmlux5cux23start43-a}{}{}\protect\hypertarget{lb1625905.xhtmlux5cux23start43-b}{}{}\protect\hypertarget{lb1625905.xhtmlux5cux23start43-c}{}{}\protect\hypertarget{lb1625905.xhtmlux5cux23start43-d}{}{}
föll talet på idrotter. Fjalar berömde sina män där i salen för
skicklighet i hvarjehanda sådana, och frågade i hvilka idrotter gästerna
voro bäst öfvade. Loke svarade, att han trodde sig innehafva konsten att
äta fort och mycket; i öfrigt vore han icke idrottsman. Fjalar skrattade
och sade: »midt emot dig vid bordet har du min tjänsteman Loge. Han
öfvar samma konst, och kunnen I nu kappas». Der frambars ett aflångt
fat, fyldt med kött, och ställdes mellan dem tvärs öfver bordet. Båda
åto nu allt hvad de förmådde och möttes midt i fatet. Då hade Loke ätit
allt köttet på sin hälft af detsamma; men Loge både köttet och benen och
fatet dessutom. Fjalar sporde nu Tor i hvilka idrotter han vore
förfaren. Tor svarade, att han där hemma i Asgard gälde för att vara en
god brottare och ansågs äfven kunna lyfta tung börda och dricka
försvarligt. Då fylldes mjödhornen öfver lag. Framför Tor stod ett, som
ej var större än de andres. Desse tömde sina i botten; men Tor, fast han
drack så mycket han orkade, kunde knappt se, att mjödet i hans horn hade
sjunkit något. Detta förtröt honom i hans själ, och han fann sig liten
och obetydlig i jämförelse med dessa jättar. Sedan man hvilat sig på
måltiden, sade Fjalar: »Tor talade om, att han kan lyfta tung börda.
Släpp in min katt i salen, och låt oss se, om Tor kan lyfta honom. Här
är detta en lek för barn.» I detsamma kom en stor grå katt fram på
golfvet. Tor tyckte nog, att Fjalar förnärmade honom med ett sådant
förslag; men nu förekom han sig själf så ringa, att han teg och gjorde
som Fjalar ville. Han lade sin hand under magen på katten och lyfte; men
allteftersom han lyfte, sköt katten rygg, och högre upp kunde han icke
få honom, än att katten stod med tre af sina fötter kvar på golfvet.
»Det kommer sig däraf», sade Fjalar, »att Tor är så lågvuxen. Annars
hade det väl lyckats. Nu lär Tor efter egen utsago vara en skicklig
brottare, och jag önskar se honom bevisa det. Med någon af mina män här
inne vill jag dock icke, att han försöker sina krafter; ty då skulle det
gå honom illa. Men ropa in min fostermor Elle!» I detsamma inträdde i
salen en käring, som
\protect\hypertarget{lb1625905.xhtmlux5cux23start44}{}{}\protect\hypertarget{lb1625905.xhtmlux5cux23start44-a}{}{}\protect\hypertarget{lb1625905.xhtmlux5cux23start44-b}{}{}\protect\hypertarget{lb1625905.xhtmlux5cux23start44-c}{}{}\protect\hypertarget{lb1625905.xhtmlux5cux23start44-d}{}{}
såg mycket ålderstigen och skröplig ut. Tor skulle vägrat att brottas
med henne, men hon grep tag i honom och han måste. Han fann, att hon var
vådligt stark; ju fastare han tog i, dess fastare stod hon. Knep mot
henne ville han dock icke nyttja, för att få henne omkull. Men Elle
kunde ej häller med ärliga medel besegra Tor; då satte hon »käringkrok»
för honom och fick honom ned på ett knä. Fjalar sade nu, att brottningen
skulle upphöra, och de båda motståndarne släppte hvarandra. Tor vardt
djupt nedslagen och väntade att bli beskrattad af Fjalars män. Men
underligt nog sutto desse och sågo helt häpne ut. Förödmjukad som han
var, ville Tor lämna Fjalars borg genast; men denne bad honom och Loke
stanna kvar till följande morgon, och de bemöttes allt intill sin affärd
med uppmärksamhet och aktning. Ingen skämtade med deras underlägsenhet.
När de följande morgon begåfvo sig på väg, följde dem Fjalar ett stycke,
och så uppfylld var han af förvåning och beundran öfver hvad Tor
uträttat, att han icke kunde tiga med sanningen. Han visade Tor en
klippa, som hade tre fyrkantiga djup. »Där ser du», sade han, »märken af
dina tre hammarslag. De träffade bärget, men icke mitt hufvud, ty jag är
mästare i synvillor. Synvillor var det mästa, som du hos mig erfarit.
Loge, som åt i kapp med Loke, är min tjenare eldslågan. Hornet, hvarur
du drack, stod i hafvet, men det såg du icke. Katten, som du lyfte, var
Midgards-ormen; det såg du ej häller. Den gamla, med hvilken du
brottades, och som icke utan knep kunde vinna en half seger öfver dig,
är ålderdomen, som besegrar den starkaste. Nu skiljas vi åt, och jag
önskar icke att se dig mer som gäst, ty du blir farlig för oss alla. Men
kommer du, så är jag mästare nog i trollska konster för att kunna med
sådana värna min borg.» Efter dessa ord försvann Fjalar. Tor och Loke
återvände till Asgard.

Från en senare tid förtäljes denna sägen så, att icke blott Loke, utan
äfven Tjalve och Raskva, Egils fosterbarn, följde Tor på färden till
Fjalar, som i den berättelsen kallas Utgardaloke, samt att Tjalve bar
Tors matsäck, tills de första
\protect\hypertarget{lb1625905.xhtmlux5cux23start45}{}{}\protect\hypertarget{lb1625905.xhtmlux5cux23start45-a}{}{}\protect\hypertarget{lb1625905.xhtmlux5cux23start45-b}{}{}\protect\hypertarget{lb1625905.xhtmlux5cux23start45-c}{}{}\protect\hypertarget{lb1625905.xhtmlux5cux23start45-d}{}{}
natten lade sig att hvila i Skrymers vante. Under natten inträffade den
af Skrymers snarkningar orsakade jordbäfningen. Då Tor fram mot morgonen
gick ut och fick se jätten, ville han slå ihjäl honom; men denne vaknade
i detsamma och frågade, om Tor ville hafva honom till reskamrat. Därtill
jakade Tor. Skrymer föreslog, att de skulle förena sin vägkost, och då
Tor därtill samtyckte, lade Skrymer all maten i en påse och bar den på
ryggen. Sedan vandrade de tillsammans hela dagen och lade sig om kvällen
att hvila under en stor ek. Skrymer somnade. Tor tog då matpåsen och
skulle öppna den, men kunde icke få upp knutarne. Då vredgades han, grep
sin hammare och slog Skrymer i hufvudet. Skrymer vaknade och frågade, om
ett blad fallit ned på honom. Därefter somnade han åter. När Tor vid
midnattstid hörde honom snarka, tog han hammaren och slog honom midt i
hjässan. Skrymer vaknade och frågade, om ett ollon fallit ned på honom.
Sedan han fram på morgonen åter börjat snarka, sprang Tor upp, svingade
hammaren, slog jätten i tinningen och märkte, att hammaren sjönk ned
ända till skaftet. Skrymer vaknade och förmodade, att något ris fallit
ned på honom från trädet. Därefter åtskildes de, sedan Skrymer utpekat
vägen till Utgardaloke och sökt skrämma Tor för de farlige kämparne
därstädes.

Vid de täflingar, som sedan egde rum i borgen, uppträdde äfven Tjalve
och täflade i kappspringning med en liten pilt, som hette Huge. Tjalve
besegrades, ty Huge var jättehöfdingens hug (tanke), som hann till målet
förr än han. Om Tor säges det, att han vid dryckestäflingen drack så
mycket, att hafvet märkbart sjönk, ehuru detta föga syntes i hornet.

\protect\hypertarget{lb1625905.xhtmlux5cux23start46}{}{}\protect\hypertarget{lb1625905.xhtmlux5cux23start46-a}{}{}\protect\hypertarget{lb1625905.xhtmlux5cux23start46-b}{}{}\protect\hypertarget{lb1625905.xhtmlux5cux23start46-c}{}{}\protect\hypertarget{lb1625905.xhtmlux5cux23start46-d}{}{}

\paragraph{XVI.}

\paragraph{TORS FÄRD TILL JÄTTEN HYMER.}

Asagudarne voro en dag samlade till ett fästligt samkväm, där man vid
måltiden pröfvade och fann behag i steken af vildt, som jagats och
fångats i skogen. Mycket vardt drucket, medan denna rätt förtärdes, och
under inflytandet af det rikliga drickandet kommo nya tankar och förslag
att omordas. Så framställdes den mening, att de makter, som upptagits i
fredsförbundet med asarne, voro underordnade och hade att visa dem sin
undersåtliga vördnad genom att, åtminstone en gång om året, mottaga dem
i sina boningar och där sörja för godt underhåll. Tanken slog an, och
innan man steg upp från bordet, hade man med lottning afgjort, hvem som
vore lämpligast att börja hos. Lotten hade gifvit tillkänna, att
hafsjätten Öger skulle bli gudarnes förste värd och att han hade råd att
bjuda dem riklig undfägnad. Det beslöts också, att Öger skulle därom
genast underrättas; Tor spände för sin char och for dit. Öger satt på
sin klippborg ute i hafvet, när asaguden kom. Tor frambar ärendet i
befallande ton och med en skarp blick in i jättens ögon. Öger svarade
icke nej till ärendet; men han var förnärmad af Tors uppträdande och
uppgjorde genast sin hämdeplan. Han ville, sade han, gärna emottaga
asarne i sin borg och bjuda dem det mjöd han förstod att brygga; men för
att mjödet skulle varda så rikligt som en sådan fäst kräfde och han
själf önskade, behöfde han en mycket större bryggkittel än han nu egde.
Han angaf måtten, som voro mycket stora, och torde sagt sig vara
öfvertygad, att Asator skulle kunna anskaffa den. När det skett, vore
asarne välkomne till honom vid den tid på året, då linet bärgas. Med det
beskedet återvände Tor till Asgard och meddelade det till sina fränder.
Men en kittel, så stor som den Öger önskade, fanns icke i Asgard, och
man undrade, om en sådan fanns i hela världen. Kitteln kunde visserligen
beställas hos underjordskonstnärerna;
\protect\hypertarget{lb1625905.xhtmlux5cux23start47}{}{}\protect\hypertarget{lb1625905.xhtmlux5cux23start47-a}{}{}\protect\hypertarget{lb1625905.xhtmlux5cux23start47-b}{}{}\protect\hypertarget{lb1625905.xhtmlux5cux23start47-c}{}{}\protect\hypertarget{lb1625905.xhtmlux5cux23start47-d}{}{}
men Tyr tog sin vän Tor afsides och sade sig kunna ge råd. »Du vet»,
sade han, »att jag är styfson och fosterson till jätten Hymer, som bor å
andra sidan Elivågor i yttersta Norden. En kittel så stor som den du
lofvat Öger finns hos honom.» -- »Men hur kan man få den?» -- Endast med
list, min vän», sade Tyr; »vi måste begifva oss dit och försöka». Tor,
som älskade äfventyr, fann det behagligare att följa detta råd än att
bedja konstnärerna om hjälp.

Detta var just hvad Öger uträknat. Han visste, att Hymer, och han
allena, hade en kittel af de angifna måtten, och att Tor, för att
anskaffa den, måste göra en färd till denne ytterst farlige frostjätte,
känd för att vara »storm-ögd.» Den, som är stormögd, kan, när han riktar
vredgad blick på någonting, urladda ur ögonen en förstörande kraft; men
är urladdningen gjord, dröjer det en tid, innan hans ögon återfå
tillräckligt ämne därtill. Tyr kände till den saken och bad Tor vara på
sin vakt och lyda de råd, som de nog skulle få af husmodern på stället,
Tyrs moder.

Från Asgard till Elivågor, där Egils borg, Ysäter, är belägen, är det
för Tor en dagsresa, när han åker efter sina bockar. Han tog Tyr med sig
på sin char. Människorna i Midgard hörde åskan dåna, och deras fält
fingo uppfriskande skurar, där asarne foro fram. Mot kvällen anlände de
till Ysäter och välkomnades af Egil och Groa. Bockarne afselades, och
charen ställdes in mällan de väggar där den vid Tors besök alltid har
sin plats. Tor far aldrig efter sitt bockspann öfver Elivågor, utan
vadar öfver detta vattendrag, och han färdas alltid till fots i
Jotunheim. Medan han är där, vakta alf-krigarne i Ysäter hans dragare
och resedon. Tor är en ypperlig vadare, och intet vattendrag, hur djupt
det må vara, hindrar honom, ty han har den egenskapen, att i mån af
vattnets djup växer hans egen höjd, så att vattnet aldrig når honom
högre än till starkhetsbältet, fast det händer, att böljorna, om
sjögången är stark, kunna slå honom upp mot skuldrorna. När han kommit i
land, återfår hans kropp sin vanliga längd.

\protect\hypertarget{lb1625905.xhtmlux5cux23start48}{}{}\protect\hypertarget{lb1625905.xhtmlux5cux23start48-a}{}{}\protect\hypertarget{lb1625905.xhtmlux5cux23start48-b}{}{}\protect\hypertarget{lb1625905.xhtmlux5cux23start48-c}{}{}\protect\hypertarget{lb1625905.xhtmlux5cux23start48-d}{}{}

Tor gaf sig denna gången icke tid att stanna och förpläga sig hos Egil.
Han begaf sig med Tyr öfver Elivågor, sedan han, som vanligt, uppryckt
en rönn vid stranden, för att hafva den till staf under vadandet.

Sedan de gått, kom Loke till Egils borg, i hvilket föregifvet ärende är
okändt, och emottogs gästvänligt. Kvällsmat skulle nu beredas, och som
Tors bockar funnos att tillgå, beslöt man, att den ene af dem skulle
slagtas. Detta kan utan fara ske, om man blott iakttager, att den
slagtade bockens ben blifva i oskadadt skick nedlagda på det afflådda
skinnet, ty när Tor svänger sin hammare öfver lämningarna och signar
dem, står bocken upp igen, lika stark och kry som någonsin. Vid
kvällsmaten pratade Loke med fostersonen på stället, den unge Tjalve,
och narrade honom att med knifven klyfva ett af bockens lårben för den
läckra märgens skull. I god tid, och innan Tor och Tyr återkommit från
Jotunheim, begaf sig Loke därifrån.

Från Elivågor är det ett tämligen långt stycke väg genom skogvuxna
klyftor och förbi bärg, i hvilkas hålor de jättar, som höra till Hymers
slägt, ha sina bostäder. Bland dessa finnas åtskillige, som ännu till
utseendet likna de flerhöfdade rimtursarne. Så gjorde äfven Hymers
moder.

När de båda asarne hunnit fram till gården och stigit in i salen,
emottogos de vänligt med välkomsthornet af Tyrs vackra och guldprydda
moder. Hymer själf var ute på jagt. Salen var stor och uppbars af
väldiga stenpelare. På en pelare vid gafveln hängde åtta kittlar. En af
dem var den eftersökte.

Emot aftonen, när Hymer väntades hem, rådde hans hustru Tor att med Tyr
gömma sig bakom gafvelpelaren, där kittlarne hängde, ty, sade hon, »min
man är ofta icke just vänlig mot gäster, och han är snar till vrede.»
Där sutto de, när Hymer, sent om kvällen, inträdde i salen. Ispiggar
skramlade kring honom, och hans kindskägg var fruset. Husfrun hälsade
honom välkommen från jagten och meddelade den glada nyheten, att sonen
Tyr, den länge väntade,
\protect\hypertarget{lb1625905.xhtmlux5cux23start49}{}{}\protect\hypertarget{lb1625905.xhtmlux5cux23start49-a}{}{}\protect\hypertarget{lb1625905.xhtmlux5cux23start49-b}{}{}\protect\hypertarget{lb1625905.xhtmlux5cux23start49-c}{}{}\protect\hypertarget{lb1625905.xhtmlux5cux23start49-d}{}{}
kommit på besök. Så till vida var allt godt, men då hon tillade: »med
honom har Tor, människornas vän, kommit hit», fingo Hymers stygga ögon
det utseende, som bådade urladdning. Då sade den kloka kvinnan: »se bort
till gafvelpelaren! där bakom gömma de sig.» Hymer såg dit och kraften i
hans ögon bröt ut med sådan våldsamhet, att pelaren, så tjock han var,
brast itu och kittlarne gingo sönder och föllo i golfvet. Blott en, den
största och hårdast smidde, förblef hel. Nu trädde asarne fram och
hälsade, och den gamle jätten mätte med blicken sin ätts ryktbare
fiende, Midgards värnare, från hjässan till fotabjället. Och utföll den
mönstringen så, att Hymer tyckte det vara bäst för sig själf att visa
sig höflig och uppfylla en värds skyldigheter. Tre oxar slaglades och
kokades till kvällsmaten. Af dem åt Tor ensam upp de två. Det förtröt
Hymer, och han kunde icke låta bli att säga: »i morgon få vi lefva af
den föda, som vi kunna fånga oss ute.» Följande morgon föreslog Tor, att
de skulle fara ut på fiske, och frågade hvad slags agn hans värd
nytjade. Hymer, som hade stora och vildsinta jättetjurar i sin hjord
sade: »gå till min hjord, om du törs, och tag dig agn där af en oxe.»
Tor gick till skogen och kom snart tillbaka med hufvudet af en svart
tjur; ej häller nu kunde Hymer undertrycka sin förargelse. Så gingo de
ner till Elivågor, med sina fiskedon. Hymer lösgjorde en båt och satte
sig till årorna, Tor tog plats på aktertoften. När de rott ett stycke ut
på fjärden, lade Hymer upp årorna. Tor ville att de skulle ro längre ut;
men det vägrade jätten bestämdt, troligen därför att å motsatta kusten
bodde jättefienden Egil, som plägade på skidor göra färder å vattnet,
väpnad med en båge, hvars pilar icke förfelade sitt mål. Det var därför
rådligast att icke våga sig för långt från jättestranden. Hymer och Tor
firade ut sina fiskelinor. Hymer hade god fiskelycka och drog på en gång
upp två hvalar. Tor hade fästat tjurhufvudet på sin krok och fick ett
väldigt napp. Det var Midgards-ormen, som slukat agnet. Han var redan då
ett vidunder af förfärande storlek och styrka, ehuru
\protect\hypertarget{lb1625905.xhtmlux5cux23start50}{}{}\protect\hypertarget{lb1625905.xhtmlux5cux23start50-a}{}{}\protect\hypertarget{lb1625905.xhtmlux5cux23start50-b}{}{}\protect\hypertarget{lb1625905.xhtmlux5cux23start50-c}{}{}\protect\hypertarget{lb1625905.xhtmlux5cux23start50-d}{}{}
på långt när icke så stor som han till slut skall bli. Tor halade in
linan, fast ormen stretade emot, och han fick det fula hufvudet upp mot
relingen. Hans ena hand höll linan; den andra ryckte hammaren ur
starkhetsbältet och gaf ormen ett slag i skallen. Det var ett hårdt
slag, och ormen tjöt och vältrade sig, så att det genljöd ur djupet och
jorden skalf. Men lös kom han och sjönk åter ned i hafvet. Nästa slag af
Tors hammar får han icke förr än i Ragnarökstriden.

När efter slutadt fiske Hymer rodde tillbaka med sin gäst, satt han
dolsk vid årorna och sade icke ett ord. Några vilja veta, att när Tor
med sin hammar slog Midgardsormen i hufvudet, afskar Hymer hans
fiskelina, så att Midgardsormen på det sättet kom lös, samt att Tor i
förargelse häröfver gaf Hymer en örfil, så att han föll omkull i båten.
Huru än härmed förhöll sig, visst är, att Hymer var vid dåligt lynne.
När de kommit i land, frågade han, om Tor ville bära hvalarne upp till
gården eller göra fast båten. Tor tog båten, men band den icke vid
stranden, utan bar den med ösvatten och åror hela vägen upp till jättens
sal. Deri gjorde Tor klokt, ty på det sättet betryggade han sig
obehindrad återfärd öfver Elivågor. När de kommit hem, ville jätten se
andra prof på Tors styrka. Det sista profvet han begärde var att se, om
Tor kunde lyfta hans store bryggkittel och bära den ut ur salen. Det var
nog Tyr, som lockade Hymer att föreslå det profvet. Själf gjorde Tyr två
försök att lyfta den, men fick den ej ur fläcken. Tor grep tag i
kittelns rand och spände fötterna därvid så hårdt mot golfvet, att han
trampade igenom det. Kitteln lyfte han upp på sitt hufvud och bar den så
ut ur salen. Tyr följde honom och de kommo icke igen. När de läto vänta
på sig, anade Hymer oråd och skyndade ut. De båda asarne hade så brådt
om, att de sprungit ett godt stycke, innan de sågo sig om. Då märkte de,
att de voro förföljda. En hel jätteskara kom rusande ut ur sina hålor
och följde Hymer. Då måste Tor lyfta ned kitteln och taga sin hammare.
Tyr bistod honom manligt, och lyktade striden så, att angriparne
\protect\hypertarget{lb1625905.xhtmlux5cux23start51}{}{}\protect\hypertarget{lb1625905.xhtmlux5cux23start51-a}{}{}\protect\hypertarget{lb1625905.xhtmlux5cux23start51-b}{}{}\protect\hypertarget{lb1625905.xhtmlux5cux23start51-c}{}{}\protect\hypertarget{lb1625905.xhtmlux5cux23start51-d}{}{}
stupade. Därefter kommo asarne oantastade öfver Elivågor till Egils
borg. Här mottogos de med glädje. Tor signade sin slagtade bock med
hammaren och han stod upp, bockspannet spändes för charen, och med den
lyckligt förvärfvade kitteln skulle nu färden fortsättas till Asgard.
Men långt hade de icke åkt, förrän den ene bocken blef lam i ena benet.
Tor for tillbaka till Egils borg och förhörde sig om hvad som kunnat
vållat detta. Tjalve bekände då, att han klufvit den ena lårpipan för
märgens skull. Tor var mycket vred och kräfde af Egil vederlag för
skadan. Dock gick vreden sin kos, när han såg, huru bedröfvad hans vän
var öfver Tjalves oförståndiga gärning. »Tag», sade Egil, »mina
fosterbarn i vederlag; något bättre kan jag väl icke ge dig.» Och Tor
tog Tjalve och Raskva i böter, så nämligen, att han vardt deras
fosterfader och lät Tjalve få del af sina egna bragders heder.

\paragraph{XVII.}

\paragraph{TORS FÄLTTÅG MOT JÄTTEN GEIRRAUD.}

För den, som iakttog tidens tecken, var det tydligt, att jättarne vordo
med hvarje år, som gick, djärfvare och farligare. Egil med sina alfer
hade mycket att göra, för att hålla dem inom deras gränser. Allt som
oftast for han på sina skidor omkring å de stormiga, töckenhöljda och
trollska Elivågor, för att utspäja deras företag. Två svåra envigen hade
han haft att bestå, det ena med jätten Koll, det andra med hans vilda
och starka syster Sela, och lyckats fälla dem båda. Han och hans alfer
voro hjärtligt hatade af jättarne för den trägna och käcka vakttjänst de
höllo. Ej minst var Egil en nagel i ögat på jätten Geirraud, höfding för
en talrik och stridslysten jättestam och fader till de ryktbara
jättinnorna Greip och Gjalp. Loke, som, med och utan gudarnes vetskap,
ej sällan uppsökte sina fränder i Jotunheim och rådgjorde med dem, blef
ense med Gjalp och Greip om
\protect\hypertarget{lb1625905.xhtmlux5cux23start52}{}{}\protect\hypertarget{lb1625905.xhtmlux5cux23start52-a}{}{}\protect\hypertarget{lb1625905.xhtmlux5cux23start52-b}{}{}\protect\hypertarget{lb1625905.xhtmlux5cux23start52-c}{}{}\protect\hypertarget{lb1625905.xhtmlux5cux23start52-d}{}{}
en plan till Egils och hans alfers fördärf och, om möjligt, äfven till
Tors.

I Asgard talade Loke med Tor om Geirraud och hans folk, som om mycket
farliga jättar, hvilka det vore alldeles nödvändigt att näpsa. Tor
borde, menade Loke, göra ett ordentligt fälttåg emot dem, följd af Egil
och hans stridsmän. »Men akta dig att göra det», sade han, »om du icke
tror dig god nog till företaget, ty farligt är det.»

Tor lät Loke icke länge egga sig att uppsöka »de brådbranta
fjäll-altarnes tempelprest» Geirraud. Han bad Loke skaffa noggranna
underrättelser om vägen till Geirrauds gård, och denne gaf honom
beskrifning på en väg, om hvilken han försäkrade, att en truppstyrka
kunde på den framtränga utan att möta trånga pass eller svåra
vattendrag. »Gröna stigar», sade han, »ligga till Geirrauds väggar.» Tor
körde då till Egils borg, och när han mot aftonen kommit dit och omtalat
sitt uppsåt att i spetsen för borgens stridsmän uppsöka Geirraud och
hans slägt, för att tukta dem, vardt det mycken fröjd bland Egils
kämpar, och de rustade sig att följande morgon vara färdiga till
uttåget.

På utsatt tid bröto de upp och satte öfver Elivågor, Tor vadande, Egil
och Tjalve på skidor, de öfriga i båtar, och ryckte till fots in i
Jotunheim på den väg, som Loke hade anvisat. Loke var icke med; det
aktade han sig visligen för. Ej häller saknades han af någon, ty han
stred på långt när icke så käckt med vapen, som med sin skarpa,
mångordiga och lögnaktiga tunga. Men hvad Tor icke anade var, att Loke
låtit Geirraud veta, att asaguden och hans stridsmän vore att vänta.
Jättarne beredde sig på att mottaga dem, och ett försåt var planlagdt på
den af Loke förordade vägen. Jättekvinnor höllo utkik från bärgen och
hastade till sina män för att underrätta dem om fiendernas framryckande.

Vägen såg till en början lofvande ut, och äfven vädret var drägligt.
Efter hand ändrade sig detta. Man kom in i en fjälltrakt, som tedde sig
allt vildare och hemskare, och i samma mån slog vädret om och vardt
ohyggligt.
Stormvindar\protect\hypertarget{lb1625905.xhtmlux5cux23start53}{}{}\protect\hypertarget{lb1625905.xhtmlux5cux23start53-a}{}{}\protect\hypertarget{lb1625905.xhtmlux5cux23start53-b}{}{}\protect\hypertarget{lb1625905.xhtmlux5cux23start53-c}{}{}\protect\hypertarget{lb1625905.xhtmlux5cux23start53-d}{}{}
frambrusade mellan bärgen, svarta molnmassor seglade utefter
fjällsidorna, och skurar af hagel störtade ned. Och när dalen gjorde en
krök, sågo Tor och hans kämpar framför sig en älf, i hvilken digra
strömmar, svällande af hagelskurarne, nedvältrade från bärgen med
fräsande isvatten och stängde deras väg.

För Tor, om han varit ensam, hade detta ej utgjort hinder, ehuru det
äfven om honom gäller, att han vid vadande måste anstränga sig i mån af
forsens våldsamhet. Men för hans kämpar såg det värre ut. Här i de
hvirflande vattenmassorna kunde Egil och Tjalve icke bruka sina skidor,
och deras alfkämpar icke sin simkonst. Tor fick röna, att Loke var en
bedräglig vägvisare. Han hade sagt, att ingen älf stängde den af honom
angifna leden. Det gällde nu att finna ett vadställe för stridsmännen,
och det var ingen lätt sak. Tor gick ut i älfven och undersökte bottnen
med tillhjälp af sin staf, »väghjälpens träd.» Han fann ett vadställe
och anvisade det. Själf beslöt han gå fram i det djupa vattnet där
nedanför. Han såg med nöje, att Egil och Tjalve och deras kämpar icke
visade tveksamhet. Med god förtröstan stego de i vattnet, och vadandet
började. Tor gick lugn och trygg, men måste ha blicken riktad ej endast
på forsen och på sina ledsagare, utan ock på motsatta stranden och
landet där bortom, ty det tycktes honom, att det myllrade någonting
mörkt där borta, som kunde vara en annalkande jätteskara. Ofvanför
fjällbranterna, hvarifrån skummande strömmar vräkte sig ned i älfven och
ökade dess vattenöfverflöd, såg han jättinnor stå, som nog bidrogo
därtill. Det var Greip och Gjalp och deras leksystrar.

Alfkämparne satte med kraft fötterna mot älfbottnen, och som stafvar
nyttjade de sina spjut, som de strömledes sköto ned mot den steniga
bottnen. De flockade sig tillsammans till ömsesidig hjälp, så godt som
vadställets smalhet tillät det. Främst gick Egil. I midten af de andre
den hurtige Tjalve. Klangen af spjuten, när deras metallbroddade
\protect\hypertarget{lb1625905.xhtmlux5cux23start54}{}{}\protect\hypertarget{lb1625905.xhtmlux5cux23start54-a}{}{}\protect\hypertarget{lb1625905.xhtmlux5cux23start54-b}{}{}\protect\hypertarget{lb1625905.xhtmlux5cux23start54-c}{}{}\protect\hypertarget{lb1625905.xhtmlux5cux23start54-d}{}{}
ändar stötte mot bottenstenarne, blandade sig med dånet af hvirflarne,
som brusade kring älfvens klippblock.

Men floden steg oupphörligt i höjd och våldsamhet. Tor själf, den
storvuxne Odenssonen, märkte, att han måste använda sin inneboende
förmåga att skjuta i höjd i mån af vattnets ökade djup. Mycket värre var
det för Egil och hans stridsmän. Strömgången vardt för strid för
spjuten, som de stödde sig mot. Hvirflarne ville beröfva dem fotfäste
och lyfta dem med sig, vattnet höll på att växa dem öfver hufvudet, och
det yrande skummet slog hvita dok öfver deras hjälmar. Så kom en
hafsflod-svällande bölja, som de icke kunde motstå; hon nådde själfve
Tor till skuldrorna. Egil, »vinddrifven i flodmarkens drifvors (de
hvitskummiga vågornas) storm», störtade mot asagudens axlar och slog
armarne kring hans hals. Tjalve och de andre stridsmännen lyftes på en
och samma gång af den sjudande böljan; de grepo tag i hvarandra och
drefvos i en knippa ned emot Tor. Tjalve fick ett duktigt grepp i hans
starkhetsbälte och räddade därmed sig och de andra. Så många af dem, som
fingo rum, hängde sig i starkhetsbältet rundt omkring guden; de andre
höllo fast i kamraterna. Med Egil på sina skuldror och med en mängd af
kämpar kring sin midja och i släptåg vadade Midgards värnare, den
väldige Asator, genom vilda flöden. Geirrauds jätteskara hade nu hunnit
ned mot stranden. Jättarne hade fördröjt sig något, för att njuta det
skådespel, som alfkämparnes kamp med flodhvirflarne beredde dem och som
de väntade skola slutas med dessas undergång. Men nu, när de sågo dem
räddade ur »flodmarkens drifvor», rusade de med vilda skrän fram till
älfkanten och svängde i luften sina stenbeväpnade slungor. Men nu hade
också Tor hunnit i smult vatten. Hans kämpar voro på fötter igen; Egils
båge klingade; slungstenarne möttes af susande spjut och hvinande pilar.
Blott några steg, och skarorna drabbade samman bröst mot bröst, och
öfver vimlet af de stridande lyfte sig och föll i täta slag Tors
hammare. Det var en hård strid, hvari jättarne
ådagalade\protect\hypertarget{lb1625905.xhtmlux5cux23start55}{}{}\protect\hypertarget{lb1625905.xhtmlux5cux23start55-a}{}{}\protect\hypertarget{lb1625905.xhtmlux5cux23start55-b}{}{}\protect\hypertarget{lb1625905.xhtmlux5cux23start55-c}{}{}\protect\hypertarget{lb1625905.xhtmlux5cux23start55-d}{}{}
trotsigt mod, och mången alfkämpe fick minnesbetor. Men den slutade med
Midgardsvännernas seger. Asahammaren krossade hvad den träffade, och
hvad som efter striden återstod af Geirrauds utsända skara flydde uppåt
dalens klyftor tillbaka till sin höfdings gård. Tor och hans vänner,
Egils »med idrottsinne borna skräckskara», drog djupare in i Jotunheim,
ledsagade af jättekvinnors vilda tjut uppe ifrån fjällbranterna.

Så kommo de ändtligen till Geirrauds gård. Inne i hans bärgsalar var
allt hans husfolk, alla de från striden undkomne jättarne, samt Greip
och Gjalp och andra starka och vildsinnade jättinnor samlade. När man
hunnit fram, räckte Tor sin staf, den ur skogen uppryckta rönnen, till
Egil, för att han skulle använda den som vapen, ty här förestode ett
handgänge, hvari »stridsladans rågs stänglar» (kogrets pilar) ej kunde
vara till gagn. Så ryckte Tor upp den tunga stendörren, och han och Egil
inträdde främst och jämsides i jättens dryckessal. Geirraud, som stod i
sitt högsäte vid salens motsatta gafvel, slungade mot Tor en glödgad
järnbult med sådan fart, att Egil raglade vid luftdraget; men Tor mötte
med sin hammare järnet i flykten och sände det tillbaka så, att det
genomborrade Geirraud och slog in i bärgväggen bakom honom. Efter Tor
och Egil stormade alferna in i salen. Det blef blodig kamp och svår
trängsel. Egil svängde »väghjälpens träd», Tor hammaren. Gjalp och Greip
störtade sig mot Odenssonen och brottades med honom; för att varda dem
lös, måste han göra ett grepp, som bräckte deras ryggben. Alla
bärgsalens inneboar lågo slagna, innan segervinnarne lemnade Geirrauds
gård, och utan vidare äfventyr tågade de tillbaka genom Jotunheim och
kommo till Egils borg, där de höllo ett muntert gille, innan Tor på sin
char for genom luften till Asgard.

Loke, bedragaren, borde råkat illa ut. Men nu, likasom då han ställdes
till rätta för att han hade narrat Tjalve, visste han att med hal tunga
och skenfagra ursäkter fria sig från straff.

\protect\hypertarget{lb1625905.xhtmlux5cux23start56}{}{}\protect\hypertarget{lb1625905.xhtmlux5cux23start56-a}{}{}\protect\hypertarget{lb1625905.xhtmlux5cux23start56-b}{}{}\protect\hypertarget{lb1625905.xhtmlux5cux23start56-c}{}{}\protect\hypertarget{lb1625905.xhtmlux5cux23start56-d}{}{}

\paragraph{XVIII.}

\paragraph{TORS HAMMARE STULEN.}

Tor har haft två hammare. Den som hittills blifvit omtalad är ett godt
konstvärk, gjordt af hårdaste sten, blankfejadt och skinande och laddadt
med vafereld i gryet. Men som det sedan visade sig, hade den icke alla
upptänkliga goda egenskaper. Denne äldre hammare har, likasom den yngre,
varit kallad Mjölner (»krossaren»).

Tor hade under någon af sina utflykter lagt sig att sofva. När han
vaknade, var hans hammare försvunnen. Det var en obehaglig upptäckt.
Himmelens och jordens säkerhet berodde på detta vapen. Tor kände sig så
förfärad att han skälfde, hvilket annars icke plär hända honom. Han
letade på marken rundt omkring, men gagnlöst.

Farligt var det också att tala om förlusten. Ty om jättarna finge veta
om den, skulle den öka deras mod och mana dem till angrepp på Midgard.
Men tiga och intet göra var också farligt. Så beslöt Tor att meddela den
fyndige och rådsnabbe Loke hemligheten. Icke därför att Tor just trodde
på honom. Men Loke hade skäl att ställa sig väl med Tor och skingra det
misstroende denne hade anledningar att hysa. »Märk, Loke, hvad jag nu
mäler», sade Tor, »och hvad ingen i himmelen eller på jorden vet: från
asaguden är hammaren stulen.» »Det torde hjälpas», sade Loke, »om du kan
ställa det så, att jag får låna Fröjas fjäderham.» De gingo då till
Folkvang och trädde in i Fröjas sal, och Tor bad henne om lånet. Kär,
som Tor är, för alla gudar och människor, svarade Fröja, att hon gärna
lånade honom den, vore den aldrig så dyrbar. Loke iklädde sig
fjäderskruden och flög med susande vingslag ur Asgard till Jotunheim.
Det är sannolikt, att han på förhand visste hvem som stulit hammaren, ty
han tog raka vägen till jätten Tryms gård. Trym, turshöfdingen, satt på
sin utkikskulle utanför gården, snodde guldband åt sina hundar och
kammade sina hästars
\protect\hypertarget{lb1625905.xhtmlux5cux23start57}{}{}\protect\hypertarget{lb1625905.xhtmlux5cux23start57-a}{}{}\protect\hypertarget{lb1625905.xhtmlux5cux23start57-b}{}{}\protect\hypertarget{lb1625905.xhtmlux5cux23start57-c}{}{}\protect\hypertarget{lb1625905.xhtmlux5cux23start57-d}{}{}
manar. Loke och han voro gamla bekanta, och Trym begrep i hvilket ärende
han kom. »Hur står det till hos asar och alfer, och hvarför kommer du nu
till Jotunheim?» sporde han. »Det står illa till hos asar och alfer»,
sade Loke, »och det är väl du, som gömt Tors hammare?» Det erkände Trym
genast, ehuru han förtegat det för alla andra. Tors hammare duger icke i
hvilken hand som helst, och den hade för tjufven intet annat värde än
det vederlag han kunde få för dess återlämnande. »Jag har», sade Trym,
»gömt hammaren åtta raster nere i jorden, och ingen får den, som icke
hitförer Fröja som min brud.» Med detta besked flög Loke till Asgard.
Tor stod på borgens gård och väntade otålig och lät icke Loke sätta sig,
innan han omtalat ärendets utgång. »Säg det i luften», sade han, »ty
ofta är den sittandes tal föga pålitligt, och den hvilandes ord fulla af
lögn.» Loke frambar sitt ärende. Tor gick då till Fröja, omtalade för
henne hur det var, och bad henne för gudars och människors bästa att
kläda sig i bruddrägt och sitta upp i hans char och följa honom till
Trym. Men vid blotta tanken på en så ovärdig brudgum häfde sig Fröjas
barm så häftigt, att länkarne i hennes bröstsmycke Brisingamen brusto.
Hon ville icke höra ett ord mer därom. Då måste gudarådet sammankallas:
asarne gingo till tings och asynjorna till rådstämma för att besluta om
hvad som skulle göras. Heimdall stod upp och föreslog, att Tor skulle
klädas i brudlin och smyckas med Brisingamen och föras till Trym, som om
han vore Fröja. Tor stod därefter upp och sade, att det ville han visst
icke göra; han skulle bli utskämd och kallas en karlkvinna, om han läte
kläda sig i bruddrägt. Loke stod då upp och sade till Tor: »hämtar du
icke din hammare, då varda jättarne herrar i Asgard.» Tor lät då bekväma
sig därtill, och asynjorna klädde honom till brud. Hans blonda skäggfjun
klipptes bort, hans hår sattes upp efter kvinnobruk och pryddes med band
och spetsar, man påtog honom en veckrik klädning, vid bältet fästes en
nyckelknippa, och bröstet pryddes med Fröjas
Brisingamen\protect\hypertarget{lb1625905.xhtmlux5cux23start58}{}{}\protect\hypertarget{lb1625905.xhtmlux5cux23start58-a}{}{}\protect\hypertarget{lb1625905.xhtmlux5cux23start58-b}{}{}\protect\hypertarget{lb1625905.xhtmlux5cux23start58-c}{}{}\protect\hypertarget{lb1625905.xhtmlux5cux23start58-d}{}{}.
Tor var ung och vacker och smärt om midjan. En ståtligare brud kunde man
knappt få se; men väl mycket bred öfver härdarne var den väldiga ungmön.
Detta var en lek i Lokes smak, och han bad att få följa med, klädd som
brudtärna. Tors bockar drefvos hem från betet på de saftiga
Asgardsängarna och spändes i sina skaklar. Den, som fått sitt lårben
skadadt, var längesedan läkt genom helande konst och galdersång. Färden
gick snabbt, men med blixt och brak, ty Tor var i sin bruddrägt icke vid
bästa lynne. När Trym såg dem komma, befallde han sina jättar att bona
sal och bänkar och duka bord. »Mycket har jag, som fröjdar mitt hjärta,
rika skatter och många smycken, kor med guldhorn och svarta oxar; Fröja
allena saknade jag. Fören nu in i salen Fröja min brud, dottern af Njord
från Noatun!» Så skedde med högtidlighet. Tidigt på aftonen satt man
kring bordet vid fyllda fat och dryckeshorn. Bruden satt vid
jättehöfdingens ena sida, och Trym såg förälskad på henne; å andra sidan
hade han brudtärnan. Men förvånad vardt Trym öfver Fröjas matlust. Tor
förstod sig icke på att vara behagsjuk. Han åt ensam upp en hel oxe,
åtta laxar och alla de sötsaker, som uppdukats för fruntimren, och han
sköljde ned det med tre mål mjöd. Trym hviskade till brudtärnan, att han
aldrig i sin lefnad sett en ungmö äta och dricka så mycket på en gång.
Loke sade det vara väl förklarligt, ty Fröja hade så längtat till
Jotunheim, att hon hungrat och törstat i åtta dygn. Längre fram på
aftonen ville Trym kyssa sin utkårade. Han lutade sig emot henne under
det fina brudfloret, men flög häpen tillbaka och sade till brudtärnan:
»Hvad Fröjas ögon kunna ge skarpa blickar! Det var som brann det eld i
dem!» »Det är varmaste kärlekseld», försäkrade Loke, »hon har trånat
till dig så, att hon icke sofvit på åtta dygn.»

Så kom kvällen, och vigseln skulle ega rum. Då befallde Trym,
jättehöfdingen: »Bären in hammaren Mjölner till brudens vigning och lägg
den i hennes knä! Vig oss sedan samman med löfte och ed!» Hammaren bars
in. Då
\protect\hypertarget{lb1625905.xhtmlux5cux23start59}{}{}\protect\hypertarget{lb1625905.xhtmlux5cux23start59-a}{}{}\protect\hypertarget{lb1625905.xhtmlux5cux23start59-b}{}{}\protect\hypertarget{lb1625905.xhtmlux5cux23start59-c}{}{}\protect\hypertarget{lb1625905.xhtmlux5cux23start59-d}{}{}
log Tor i sitt hjärta. Den lades i hans knä. Då grep han om den, och
hvad som sedan skedde behöfver icke sägas. Han drap hammartjufven och
hela hans slägt. Med Mjölner i brudbältet återvände han till Asgard.

\paragraph{XIX.}

\paragraph{HURU ASGÅRD FICK SIN VALLGÖRDEL OCH ODEN SIN HÄST.}

Trym hade velat få Fröja till sig i Jotunheim. Det var många jättar som
ville det. Men olycklige hade gudar och människor varit, om de urtida
frostmakternas afkomma fått kärlekens och fruktsamhetens gudinna i sitt
våld.

En dag fingo gudarna veta, att en jätte sagt, att om han finge Fröja och
sol och måne, så skulle han i gengäld uppföra kring Asgard en vall, som
frostmakter och jättar aldrig kunde öfverstiga, äfven om de eröfrat hela
den öfriga världen. Han hade tillsagt, att han kunde göra det på tre
halfår. Det var Loke som omtalade detta för asarne och han tyckte att
förslaget var värdt att tänka på; man kunde ju låta jätten bygga vallen;
sen vore det tid att se till, huru man skulle ställa med den äskade
arbetslönen. Asarne öfvervägde förslaget, dock hvarken i afsigt att
bedraga jätten eller utlämna Fröja; och de läto genom Loke jätten veta,
att de gingo in på hans anbud, om han kunde göra vallen färdig på en
vinter utan någon annans hjälp; men om det efter sista vinterdagen
fattades något på vallen, skulle han ingen lön ha. Jätten antog
förslaget på det vilkor, att han finge begagna sin häst Svadelfare till
arbetet. Loke tillstyrkte det, och det medgafs honom. Gudarne voro
öfvertygade, att han icke skulle hinna med arbetet på utsatt tid. Då
hade de utan kostnad fått en del af vallen uppförd, och den kunde ju
fortsättas af andra händer efteråt.

Öfverenskommelsen bekräftades med ed, och jätten försäkrades om trygg
vistelse i Asgard.

Första vinterdag började arbetet. Det hvilade hvarken
\protect\hypertarget{lb1625905.xhtmlux5cux23start60}{}{}\protect\hypertarget{lb1625905.xhtmlux5cux23start60-a}{}{}\protect\hypertarget{lb1625905.xhtmlux5cux23start60-b}{}{}\protect\hypertarget{lb1625905.xhtmlux5cux23start60-c}{}{}\protect\hypertarget{lb1625905.xhtmlux5cux23start60-d}{}{}
natt eller dag. Om natten drog Svadelfare grus och sten, och asarne
häpnade öfver de lass han orkade med. Om dagen byggde jätten på vallen,
och det gick med otrolig raskhet. Asarne började varda ängsliga, att
arbetet skulle medhinnas, och Loke gick och fröjdade sig i sitt hjärta
åt deras bekymmer. Då endast tre dagar voro kvar af vintern, var vallen
så nära färdig, att knappt mer på densamma fattades än hvad som borde
fattas till plats för porten. Då kommo asarne att tänka på att Loke,
ehuru han upptagits i Asgard, var en jätte och mer än en gång gifvit
dåliga råd och gjort sig skyldig till tvetydiga gärningar, ja till rent
skadliga, ehuru man skyllt dem mer på okynne och kitslighet än på
elakhet. Loke märkte, att hans trygghet i Asgard stod på spel och hade
snart grubblat ut, huru han skulle ställa det så, att asarne, i stället
för att vredgas på honom, skulle få skäl att tacka honom. Samma afton,
då byggmästaren for ut efter sten med sin häst Svadelfare, kom ett sto
springande från skogen och vrenskade åt hästen. Hästen blef vild, slet
sig lös och sprang efter. Hela den natten måste arbetet hvila, och nästa
dag var det ingen sten att bygga med. Stort bättre gick det ej de andra
dagarne. Då jätten såg, att hans sak var förlorad, blef han ursinnig och
rusade till strids mot gudarne. Därmed hade han själf förspillt den
trygghet, som blifvit honom lofvad; Tor ställde sig i hans väg och
krossade hans hufvud. Hvem stoet var vet Loke bäst. Han födde någon tid
därefter ett gråhvitt föl med åtta fötter, som vardt den bäste häst i
världen. Oden rider honom och han heter Sleipner. Sedan dess var Loke
djupt föraktad af gudarne; men han gick fritt omkring som förut i
Asgard, och ingen kunde vederlägga honom, då han sade, att han gjort
gudarne de största tjänster genom att skaffa Asgard en för jättarne
oöfverstiglig vallgördel och Oden den snabbaste och bäste af alla
hästar. Sleipner kan hoppa öfver höga murar, och han är den ende, som
kan spränga öfver vaferlågor.

Det återstod att i vallen insätta en lämplig port. Den
\protect\hypertarget{lb1625905.xhtmlux5cux23start61}{}{}\protect\hypertarget{lb1625905.xhtmlux5cux23start61-a}{}{}\protect\hypertarget{lb1625905.xhtmlux5cux23start61-b}{}{}\protect\hypertarget{lb1625905.xhtmlux5cux23start61-c}{}{}\protect\hypertarget{lb1625905.xhtmlux5cux23start61-d}{}{}
borde varda ett konstvärk af första ordningen, och gudarne ämnade därom
vidtala underjordssmederna eller Ivaldes söner. Porten var icke färdig,
när jätten Rungner, såsom sedan skall berättas, kom till Asgard. Men den
kom snart därefter till stånd och är ett underbart värk.

\paragraph{XX.}

\paragraph{TÄFLINGEN MELLAN URTIDSKONSTNÄRERNA.}

Loke utfunderade en plan, som såg mycket förmånlig ut för asarne, men
skulle draga de största olyckor öfver dem och världen. Valand, Ivaldes
son, hade nyligen smidt utomordentliga konstvärk och skänkt dem till
asarne: det ena var spjutet Gungner, som han förärade Oden; det andra
var skeppet Skidbladner, som han gaf åt sin fosterson Fröj, Njords son
och alffurstarnes höfding. Skeppet Skidbladner har alltid medvind; det
kan hopvecklas som en duk och hållas i handen, ehuru det annars är så
stort, att det kan bära samtliga asar, med allt hvad de behöfva för ett
ledungståg, genom lufthafvet. Valand hade gjort än ett konstvärk som
beundrades. Loke hade af kitslighet klippt af håret på Ivaldedottern
Sif, Valands halfsyster, som sedan blef gift med Tor. Valand gjorde
henne af guld lockar, som fastnade och växte som annat hår.

Loke träffade någon tid därefter underjordssmeden Brock och ville slå
vad om sitt hufvud med honom, att hans broder Sindre, Mimers
konstskickligaste son, icke kunde göra tre kostbarheter lika goda som
dessa. Sindre antog vadet och gick med Brock till sin smedja. Han lade
ett vildsvinsskinn i ässjan och smidde däraf galten Gullenborste.
Därefter lade han guld i ässjan och smidde däraf ringen Draupner.
Slutligen lade han järn i ässjan och smidde däraf den yngre Mjölner,
järnhammaren. Brock skötte blåsbälgen, medan dessa saker tillvärkades,
och han oroades då hela tiden af en geting, som surrade kring honom och
stack honom
\protect\hypertarget{lb1625905.xhtmlux5cux23start62}{}{}\protect\hypertarget{lb1625905.xhtmlux5cux23start62-a}{}{}\protect\hypertarget{lb1625905.xhtmlux5cux23start62-b}{}{}\protect\hypertarget{lb1625905.xhtmlux5cux23start62-c}{}{}\protect\hypertarget{lb1625905.xhtmlux5cux23start62-d}{}{}
på hand och hals. Värst blef getingen, då järnet var lagdt i ässjan. Då
stack han Brock mellan ögonen, så att blod rann och han släppte bälgen.
Däraf fick hammaren det fel, att skaftet vardt något för kort. Getingen
var den okynnige Loke. I öfverenskommelsen ingick, att
underjordssmederna skulle, i likhet med Valand, förära asarne de smidda
klenoderna. Oden fick ringen Draupner, som har den egenskapen, att han
hvar nionde natt framföder åtta ringar lika tunga som han själf. Fröj
fick Gullenborste, som är af största gagn för växtlighet och åkerbruk.
Tor fick järnhammaren, som försäkrades hafva den äldre hammarens goda
egenskaper och dessutom den ovärderliga, att den icke kunde förloras, ty
den återkomme i sin egares hand, när hälst han ville det, antingen han
kastat till måls med den eller tappat den eller bestulits på den.
Äfventyret med jätten Trym hade nyligen bevisat, huru vigtig den
egenskapen var.

Sindres gåfvor öfverlämnades till gudarne genom Brock, som begärde deras
dom i det med Loke föreliggande vadet. Loke skröt med, att han, för att
skaffa gudarne dessa dyrbarheter, satt sitt hufvud på spel. Vore det nu
så, att Sindres smiden förklarades vara bättre än Valands, så hade han,
Loke, förlorat vad och hufvud.

Eftersom gudarne voro klenodernas egare och de ende, som kunde pröfva
deras värde, måste de åtaga sig domen och låta veta, att de efter
fullgjord pröfning skulle afkunna den.

Loke var glad åt sitt påhitt. För sitt hufvud hyste han ingen rädsla.
Men hvad han var säker på var, att huru än domen utfölle, skulle den
väcka hätskhet mellan de täflande konstnärerna inbördes, samt fiendskap
åtminstone hos den tappande sidan mot gudarne. Ditintills hade asarne
och de store smederna varit vänner och de senare uppbjudit hela sin
skicklighet för att gifva Asgard världsskyddande konstvärk och sköna
smycken. Hädanefter skulle förhållandet efter Lokes uträkning varda ett
helt annat. Alldeles viss var han om att ifall Valand tappade, skulle
det stå gudarne dyrt. Han
\protect\hypertarget{lb1625905.xhtmlux5cux23start63}{}{}\protect\hypertarget{lb1625905.xhtmlux5cux23start63-a}{}{}\protect\hypertarget{lb1625905.xhtmlux5cux23start63-b}{}{}\protect\hypertarget{lb1625905.xhtmlux5cux23start63-c}{}{}\protect\hypertarget{lb1625905.xhtmlux5cux23start63-d}{}{}
hade noga iakttagit denne Ivaldesons skaplynne. Valand var trofast och
hängifven, så länge hans rättskänsla eller stolthet icke sårats, men
annars oblidkelig och hänsynslöst hämdgirig.

\paragraph{XXI.}

\paragraph{JÄRNHAMMAREN PRÖFVAS.}

Oden, iklädd guldhjälm, for ut att profrida Sleipner, den åttafotade
hästen. Sleipner simmade med ditintills icke sedd snabbhet i lufthafvet.
Han var uppenbarligen den bäste hästen i Asgard. Oden tog vägen öfver
Jotunheim och såg ned på jättarnes bygder. Då hörde han där nedifrån en
röst, som ropade: »Hvem är du med guldhjälmen? Det är en god häst du
har.» Den som ropade så var Rungner, som ansågs vara Jotunheims
väldigaste kämpe, en jättarnes Tor. Rungner var Jotunheims-åskans herre
vid denna tid -- Farbaute, Lokes fader, hade varit det före honom -- och
hade en skinande häst, Gullfaxe, med hvilken han var i stånd att rida
bland stormmolnen, och en till vigg formad hen (brynsten), som var ett
farligt kastvapen. Egils träffande pilar fruktade han icke, och vakten
vid Elivågor hejdade honom svårligen, ty han red så högt han ville öfver
Egils borg. Dock hade Egil, likasom äfven Valand, i någon mån förmågan
att med galdersång tillbakadrifva stormmolnen, som kommo från Jotunheim;
men att städse passa på, när Rungner kom, var icke lätt. Hitintills hade
dock Rungner aktat sig för att sammandrabba med Asator. Men i Jotunheim
gick det rykte, att han var Tor vuxen.

Oden ropade ned till Rungner, som stod bredvid sin häst: »Så god häst
som min finns icke i Jotunheim; det håller jag mitt hufvud på.» »Det
skola vi fresta», svarade Rungner och hoppade upp på Gullfaxe. Oden
svängde om åt Asgard till, och det vardt nu en kapplöpning mellan
Sleipner och Gullfaxe, och så ifrig var Rungner, att han knappast
märkte, hvar han var, när han, strax bakom Oden, sprängde in i
\protect\hypertarget{lb1625905.xhtmlux5cux23start64}{}{}\protect\hypertarget{lb1625905.xhtmlux5cux23start64-a}{}{}\protect\hypertarget{lb1625905.xhtmlux5cux23start64-b}{}{}\protect\hypertarget{lb1625905.xhtmlux5cux23start64-c}{}{}\protect\hypertarget{lb1625905.xhtmlux5cux23start64-d}{}{}
Asgard genom den öppning i borgvallen, som var lämnad för den ännu icke
insatta porten. Ryttarne häjdade sig utanför Vallhallsdörrarna, och
Rungner kände nog sin ställning betänklig; men när asarne gingo fram och
bjödo den oväntade gästen in att dricka, och när han ej såg Tor ibland
dem, blef han käck till lynnet, som vanligt, och steg in och satte sig
på anvisad plats vid dryckesbordet. Här ställdes Tors stora dryckeshorn
framför honom, och han tömde det i ett andedrag gång på gång. Ju mer han
drack, dess stormodigare och skrytsammare vardt han. Han kunde dricka ut
så mycket mjöd som fanns i Asgard, sade han; och man borde tacka honom,
om han icke bröte ned Asgardssalarne och sloge ihjäl asarne och toge
Fröja med sig till Jotunheim. Asarne hade länge roligt af hans skryt,
och Fröja sjelf ifyllde hornet åt honom; men när de ledsnat vid det,
kallade de på Tor, som varit på en utfärd och nu hemkom. Tor steg in och
frågade, huru det kom sig, att en jätte satt inne i Valhall och hvem som
gifvit Rungner säkerhet att vara där. Rungner antog då en annan ton och
sade, att Oden själf inbjudit honom, och att han stod under hans
beskydd; men, tillade han, är det så, att du vill strida i ärlig kamp
med mig, så svarar jag icke nej. Här saknar jag min sköld och min hen,
och du vill nog icke dräpa vapenlös man. Men möt mig på min egen mark, i
Griotunagard, om du har lust. Tor antog utmaningen, dag för tvekampen
bestämdes, och det vardt öfverenskommet, att hvardera fick hafva en man
med sig. Rungner red därefter hem på sin Gullfaxe. Hans kappridt med
Oden och äfventyr i Valhall och det beslutade mötet omtalades mycket i
Jotunheim, och alla jättar voro ense om, att det var dem af stor vigt
hvem som blefve segraren.

Det borde nu afgöras hvem Rungner skulle ha till stridsbroder i det
beramade mötet. Man väntade, att Tor skulle utvälja någon af de allra
starkaste asagudarne eller också Egil till sin. Jättarne kommo på den
tanken, att de mot denne skulle uppställa en gestalt, som genom sin
oerhörda storlek och sitt förskräckande utseende kunde redan på
afstånd\protect\hypertarget{lb1625905.xhtmlux5cux23start65}{}{}\protect\hypertarget{lb1625905.xhtmlux5cux23start65-a}{}{}\protect\hypertarget{lb1625905.xhtmlux5cux23start65-b}{}{}\protect\hypertarget{lb1625905.xhtmlux5cux23start65-c}{}{}\protect\hypertarget{lb1625905.xhtmlux5cux23start65-d}{}{}
ådraga sig asarnes akt och sänka deras mod. Fördenskull gjorde de af ler
ett otäckt beläte, som reste sig till en otrolig höjd, och för att gifva
det själ och lif, insatte de hjärtat af ett sto i dess bröst och sjöngo
trollsånger öfver det, tills det fick sinnen, medvetande och makt öfver
sina lemmar. Lerjätten kallade de Mockerkalve och väntade stort gagn af
honom.

Tor for till Egils borg vid Elivågor och satte in sitt bockspann där,
såsom hans vana var. Tor var ensam, och det är sannolikt, att han räknat
på att få Egil med till stridsmötet. Men Egil var ute på sin vakttjenst
å Elivågor och kom icke hem den kvällen. Groa hade han skickat till
Asgard. Tjalve erbjöd sig att i Egils ställe följa Tor, som tyckte om
hans anbud och antog det, ehuru Tjalve ännu var väl ung för en tvekamp,
sådan som här kunde väntas. Följande morgon satte de öfver Elivågor. Där
stormade hårdt och rådde hisklig köld. Regn och hagel piskade ned genom
vägförvillande töcken, som drefvo öfver vattnet, och Tor tyckte, medan
han vadade och hade den å skidor gående Tjalve vid sin sida, att Egil, i
det väder som rådde, hade en hård tjänst att förrätta. Han sågs icke
till, och långt kunde man ej häller se framför sig.

Tor och Tjalve stego i land å Jotunheimsstranden och gingo fram mot det
öfverenskomna stället för mötet. Innan de hunnit fram till Griotunagard,
skönjde de Mockerkalves ofantliga skepnad resa sig mot synranden. Slik
jätte hade Tor dittills hvarken sett eller hört omtalas, och han var
bekymrad för huru det skulle gå med fostersonen. Hunne Tor i god tid att
fälla Rungner, kunde han komma Tjalve till hjälp. Annars såge det illa
ut. De gingo vidare och funno nu Rungner med sköld och hen stå vid
Mockerkalves sida. Rungners sköld var af sten och omtalad som mycket
hård. Hård som sten var äfven Rungner själf. Så snart de fått godt sigte
på hvarandra, lyfte Rungner henen och Tor järnhammaren. Båda sigtade
väl, den ene mot den andres panna. Hammaren och henen möttes med blixt
och brak, och henen brast i stycken.
\protect\hypertarget{lb1625905.xhtmlux5cux23start66}{}{}\protect\hypertarget{lb1625905.xhtmlux5cux23start66-a}{}{}\protect\hypertarget{lb1625905.xhtmlux5cux23start66-b}{}{}\protect\hypertarget{lb1625905.xhtmlux5cux23start66-c}{}{}\protect\hypertarget{lb1625905.xhtmlux5cux23start66-d}{}{}
Ett af styckena for med sådan kraft i hufvudet på Tor, att han föll till
jorden med hensplinten fastsittande i pannan. Järnhammaren träffade
Rungner i hufvudskålen, krossade hans hjärna och återvände mellan den å
marken utsträckte Tors fingrar. Mockerkalve och Tjalve stodo ensamme
kvar å stridsplatsen. Detta var icke så farligt för Tjalve som det såg
ut, ty blotta åsynen af Tor hade försatt lerjätten i sådan skrämsel, att
han icke mäktade lyfta arm och vapen, utan skedde det med honom något,
som ej gärna kan omtalas, men stundom sker med skrämda barn. En pil från
Tjalves båge genomborrade hans stohjärta, och lerbelätet störtade
tillsammans. Mockerkalve var fallen, och det kan sägas: med föga heder.
Tor var icke värre sårad, än att han snart stod på fötter igen, men
henen var kvar i såret. Så ändade tvekampen å Griotunagard.

På återvägen funno de Egil, men i ett bedröfligt tillstånd. Han var så
medtagen af sin långa färd i köld och oväder, att Tor kom just lagom för
att bispringa den vanmäktige. Tor satte honom i den korgsäck, hvari han
på ryggen plägade bära sin vägkost, och på det sättet bar han sin vän
öfver Elivågor genom töcken och snöstorm till Egils borg, där han
omvårdade honom, så att han åter kom till krafter. Men en tå hade Egil
ohjälpligt förfrusit. Tor afbröt den och kastade den upp mot
himlahvalfvet med önskan, att den skulle bli en vacker stjärna till
heder för bågskytten. Och tån vardt till den stjärna, som sedan kallats
»Örvandils tå» och lyser som vore den en stråle af solen.

När Tor återkommit till Trudvang, fann han Groa där. Denna
växtlighetsdis var mer än de flesta kunnig i de läkande runornas konst
och sjöng nu goda galdersånger öfver Tors panna. När denne märkte, att
henen började lossna, blef han glad och berättade hvad som händt Egil:
att han var hemkommen i godt behåll, fast han mist en tå, som nu lyste
bland himmelens klaraste stjärnor, samt att han snart vore att vänta i
Trudvang, för att hämta Groa. Då vardt Groa så glad, att hon glömde
galdersångens fortsättning. Henen
\protect\hypertarget{lb1625905.xhtmlux5cux23start67}{}{}\protect\hypertarget{lb1625905.xhtmlux5cux23start67-a}{}{}\protect\hypertarget{lb1625905.xhtmlux5cux23start67-b}{}{}\protect\hypertarget{lb1625905.xhtmlux5cux23start67-c}{}{}\protect\hypertarget{lb1625905.xhtmlux5cux23start67-d}{}{}
är därför kvar i Tors panna, men märkes ej och är ej till vanprydnad.

Att stjärnan kallas Örvandils tå beror därpå, att Egil, såsom redan är
nämdt, bär binamnet Örvandil (»den med pilen skickligt sysslande»).

Den af Sindre smidde järnhammaren hade nu visat hvad den dugde till.
Efterhand pröfvades också de andra klenoderna och jämfördes med
hvarandra. Asarne ville hafva rundlig tid att göra flere rön, innan de
fällde den vanskliga domen i det af Loke gjorda vadet. De började väl
också tycka, att domen, huru den utfölle, vore en betänklig gärning.
Emellertid inträffade något, som här nedan skall förtäljas och gjorde
saken än vanskligare.

\paragraph{XXII.}

\paragraph{MJÖDET I BYRGERS KÄLLA. ODEN HOS FJALAR.}

Det är förut sagdt, att det rena oblandade mjödet i skaparkraftens och
visdomens källa ursprungligen innehades af Mimer allena. Det är den
dyrbaraste saften och den mäst efterträdda i världen, och, som omtaladt
är, var det endast med själfuppoffring, böner och tårar som Oden i sin
ungdom fick en dryck däraf.

Men det inträffade en dag, att uppe i Svitiod det kalla, som var Ivaldes
rike, upptäcktes i skogen, ej långt från hans egen borg, en källa,
kallad Byrger, hvars källsprång vid hennes uppkomst och under hennes
första dagar måtte haft någon gemenskap med Mimers, ty ehuru hon ej
förlänade visdom, skänkte hon dock diktkonst och glädje. Ivalde
hemlighöll upptäckten och skickade, när natten inbrutit, två af sina
hemmavarande yngsta barn, flickan Bil och gossen Hjuke, med en så till
Byrger, för att ösa och hemföra dess mjöd.

Men barnen återkommo icke. Månen hade gått upp, när de voro vid källan,
och månguden såg dem, när de öste mjödet. Han och Ivalde voro icke
vänner. Ivalde hade i
\protect\hypertarget{lb1625905.xhtmlux5cux23start68}{}{}\protect\hypertarget{lb1625905.xhtmlux5cux23start68-a}{}{}\protect\hypertarget{lb1625905.xhtmlux5cux23start68-b}{}{}\protect\hypertarget{lb1625905.xhtmlux5cux23start68-c}{}{}\protect\hypertarget{lb1625905.xhtmlux5cux23start68-d}{}{}
urtiden bortfört en af mångudens döttrar, Hildegun, och utan hennes
faders samtycke gift sig med henne.

Månguden straffade nu dotterrofvet med att taga till sig Bil och Hjuke,
när de, bärande den mjödfylda sån, voro på väg till hemmet. Mjödet tog
han äfven. Barnen behandlade han med ömhet; de voro hans dotterbarn. Bil
fick en asynjas värdighet.

Det tagna mjödet förvarades i månens skeppsliknande silfverchar. Asarne
underrättades om fyndet. Det måtte haft den egenskapen, att det föga
minskades, när man drack däraf, det tyckes hafva räckt genom långa
tider. Oden inbjöds att njuta af månsnäckans mjöd och kom ofta dit,
efter förrättadt dagsvärf, när charen sakta sjönk ned mot västerns rand.
Charen kallas då Söckvabäck, »det sig sänkande skeppet». Sittande där,
mottog Oden ur Bils hand i gyllene bägare den härliga drycken, medan
lufthafvets svala böljor susade öfver och under honom. Äfven Brage
inbjöds dit och fick där dricka den saft, som gjort honom till skald,
vältalare och visdomsman.

Ivalde vardt högligen förbittrad öfver den förlust han gjort. Att
månguden tagit de båda barnen till sig kunde betraktas som en rättvis
vedergällning; men att han beröfvat honom mjödet, denna ypperliga skatt,
det fyllde honom med hämdbegär.

När måncharen sjunkit djupt under jordens västra rand, ligger dess väg
tvärs igenom underjorden hän emot öster till de hästdörrar, genom hvilka
den sedan far upp igän på himmelen.

Ivalde, som hade sin af gudarne anförtrodda vaktpost vid den
underjordiska delen af Elivågor mot Nifelheims rimtursar, kunde från sin
högtbelägna borg se måncharen hvarje dygn färdas denna väg. Han planlade
ett bakhåll för den, öfverföll den och röfvade dess mjödförråd. Sin
vaktpost öfvergaf han och svek därmed den ed han svurit gudarne. Och för
att mjödet icke återigän skulle falla i deras händer, skyndade han att
föra det ned i eldjätten Fjalars djupa och mörka
\protect\hypertarget{lb1625905.xhtmlux5cux23start69}{}{}\protect\hypertarget{lb1625905.xhtmlux5cux23start69-a}{}{}\protect\hypertarget{lb1625905.xhtmlux5cux23start69-b}{}{}\protect\hypertarget{lb1625905.xhtmlux5cux23start69-c}{}{}\protect\hypertarget{lb1625905.xhtmlux5cux23start69-d}{}{}
dalar och anförtro det åt honom att förvaras i det innersta af hans
bärgsalar. Det blef öfverenskommet, att Ivalde skulle gifta sig med
Fjalars dotter Gunnlöd, och att de skulle ega mjödet tillsammans. Ivalde
hade därmed för alltid gjort sig till gudafiende. Han lämnade Fjalar,
för att sluta förbund med Jotunheims jättar, men skulle på fastställd
dag återkomma och fira brölloppet med Gunnlöd.

Dock, Oden är icke obekant med hvad som föregår i mörkret där nere i
Fjalars rike. Hans korpar Hugin och Munin flyga dagligen öfver
Jormungrund (underjorden), och de se ej endast hvad som händer i dess
sköna och ljusa ängder; de utspana äfven hvad som sker i det töckniga
Nifelheim och i de dunkla djupdalar, som beherrskas af Suttung, såsom
Fjalar också kallas. Där och i Nifelheim äro korparne utsatta för faror,
och Oden fruktar, att det kan gå dem illa; men hittills har deras
klokhet skyddat dem, och hvarje afton ha de återkommit till Valhall,
satt sig på asafaderns axlar och mält i hans öra hvad de utforskat. Det
var väl genom dem som Oden fick veta, hvar Ivalde dolt mjödet, och när
hans bröllopp med Gunnlöd skulle stå. Det bor dagskygga dvärgar i
Fjalars land, som förrätta trältjenst åt honom och hans slägt. En af dem
var Fjalars dörrvaktare. Han lofvade att vara Oden till hjälp i det
äfventyr, som han nu gick att fresta.

Dagen kom, då Ivalde skulle fira sitt bröllopp med Gunnlöd. Fjalars
fränder voro samlade i hans upplysta salar, och gäster, hörande till
rimtursarnes slägt, hade kommit dit från Jotunheim. En gyllene stol var
för den väntade brudgummen framsatt midt emot Fjalars högsäte vid
dryckesbordet. Brölloppsfesten skulle också vara en förbundsfest mellan
Ivalde och de gudafiendtliga makterna. Det rådde stor fröjd bland dessa,
och de tyckte sig hafva ljusa utsigter nu till att kunna störta asarne
och ödelägga Midgard. Vakten, som gudarne uppställt vid Hvergelmer, var
af Ivalde öfvergifven, och Ivalde själf var en väldig kämpe, väl egnad
att föra Jotunheimsskarorna till kamp. Han var känd som den ypperste af
alla spjutkämpar, lika ryktbar för den idrotten som
\protect\hypertarget{lb1625905.xhtmlux5cux23start70}{}{}\protect\hypertarget{lb1625905.xhtmlux5cux23start70-a}{}{}\protect\hypertarget{lb1625905.xhtmlux5cux23start70-b}{}{}\protect\hypertarget{lb1625905.xhtmlux5cux23start70-c}{}{}\protect\hypertarget{lb1625905.xhtmlux5cux23start70-d}{}{}
hans son Valand för sin smideskonst och hans andre son Egil för sin
skicklighet som bågskytt och skidlöpare.

Brudgummen kom i god tid. Upp slogos dörrarna, som skilde den starka
belysningen inne i eldjättens salar från mörkret, som rufvar öfver hans
dalars djup, och in trädde den ståtlige Ivalde och hälsades och fördes
till sin gyllene stol.

Men hedersgästen var icke den han såg ut att vara. Han var Oden, som
iklädt sig Ivaldes skepnad. Oden hade stigit ned i Suttungsrikets dystra
afgrunder och vandrat fram öfver samma villsamma marker, där Tor en gång
hade sina äfventyr med Skrymer. Oden hade icke svårt att finna vägen
genom mörkret, ty han hade till ledsagare Heimdall, som ser hundra
raster framför sig genom den svartaste natt. Heimdall medförde sin
eldborr, som har blixtens borrande och klyfvande kraft, när dess egare
sätter den mot bärgets grund. När de hunnit fram emot det fjäll, som är
Fjalars borg, skildes gudarne åt. Heimdall steg upp på fjällborgens tak.
Hans öra, som kan höra gräset växa, kunde också höra allt, som föregick
där nere. Den dagskygge salväktaren stod utanför fjälldörrarna, såg
genom mörkret Oden komma och öppnade för honom.

Där inne firades sedan en munter fäst. Brudgummen var gladlynt och
ordrik, och aldrig hade gästerna hört en man, som lade sina ord så väl
och hade att förtälja så mycket som var värdt att lyssna till. Men det
gällde för Oden på samma gång att yttra sig med mycken varsamhet, ty ett
oförsigtigt ord var farligt och hans hufvud stod på spel. Varsamhet var
till en början icke så svår att iakttaga; men det blef svårare sedan.

Under fästens fortgång vardt brudgummen till ära hans dryckeshorn
iskänkt med den dyrbara saften ur Byrgers källa. Hornet räcktes honom af
bruden, Gunnlöd, söm var en vacker och älsklig jättemö.

Sedan skreds till vigseln, och på den heliga ringen svuro Oden och
Gunnlöd hvarandra trohetens ed.

Den glada fästen fortsattes, och hornen fylldes flitigt,
\protect\hypertarget{lb1625905.xhtmlux5cux23start71}{}{}\protect\hypertarget{lb1625905.xhtmlux5cux23start71-a}{}{}\protect\hypertarget{lb1625905.xhtmlux5cux23start71-b}{}{}\protect\hypertarget{lb1625905.xhtmlux5cux23start71-c}{}{}\protect\hypertarget{lb1625905.xhtmlux5cux23start71-d}{}{}
isynnerhet brudgummens. Ivalde var känd ej allena som den ypperste
spjutkämpen, utan ock som en dryckeskämpe, jämngod med Rungner och i den
idrotten kommande närmast Tor. Däraf hade han ock fått till binamn
»Stordrickaren» (Svigder, Svegder). För att i allo uppföra sig som
Ivalde måste Oden fördenskull dricka mycket, mer än han ville.
Besinningstjufvens häger, som stjäl förstånd och sans, sväfvade öfver
hans dryck, och han vardt, efter hvad han själf omtalat, »drucken,
mycket drucken hos Fjalars».

Då var det icke godt att väga sina ord, och öfver Odens läppar kommo nu
sådana, som de mindre druckne bland gästerna funno besynnerliga, för att
vara yttrade af Ivalde, och som fram på natten, när de efter fästens
slut öfvervägde dem, ingåfvo dem misstankar.

Gillet afslutades ändtligen, och Oden och Gunnlöd begåfvo sig till
brudgemaket. Därifrån gick en gång genom fjället till det dyrbara
Byrgermjödets förvaringsrum. Gunnlöd visade Oden denne skattkammare, och
Heimdall, som lyssnade därofvanför och hörde hvad de sade, satte
eldborren till dess tak. Gunnlöd hade gifvit sin make hela sitt hjärta
och tagit på heligt allvar den trohet hon honom svurit. Oden yppade sig
för henne och slöt den hängifna i sin famn. Med hennes hjälp stod han nu
vid sitt mål och i besittning af mjödet. Heimdalls eldborr öppnade honom
en hålväg upp igenom bärget. Men utan strid kom han ej därifrån. Hans
ovarsamma ord hade burit frukt; en broder till Gunnlöd hade hållits
vaken af de tankar, som de orden väckt hos honom och kom, då allt var
färdigt till flykt, in i bärgkammaren, där Oden och Gunnlöd voro. Oden
måste kämpa och fälla honom. Allt det öfriga, som skedde därinne, är
numera höljdt i dunkel; men själf har Oden sagt, att utan Gunnlöds
bistånd hade han aldrig kommit ut ur jättegården. Nu kom han ut, och i
örnham flög han med Byrgers mjöd upp genom hålgången, som Heimdalls
eldborr öppnat, ur Fjalars skumma värld genom ljusa rymder till det
skimrande Asgard.

Men Gunnlöd, den goda kvinnan, satt där nere och grät
\protect\hypertarget{lb1625905.xhtmlux5cux23start72}{}{}\protect\hypertarget{lb1625905.xhtmlux5cux23start72-a}{}{}\protect\hypertarget{lb1625905.xhtmlux5cux23start72-b}{}{}\protect\hypertarget{lb1625905.xhtmlux5cux23start72-c}{}{}\protect\hypertarget{lb1625905.xhtmlux5cux23start72-d}{}{}
öfver sin fallne broder och öfver förlusten af den make hon gifvit sitt
trogna hjärta.

Aldrig talade Oden sedan om Gunnlöd utan i tillgifna, tacksamma och
själfförebrående ord.

Men öfver den lyckligt utförda bragden gladde sig alltid hans håg. Han
skänkte af det vunna mjödet till gudar och människor. De barn, som komma
till världen, för att varda visdomsmän och skalder, få, innan de inträda
i jordelifvet, smaka det, och de bära det sedan inom sig som i en källa,
hvarur ingifvelsen till ädla, hugstärkande, andelifvande sånger kommer.
Från Byrgers mjöd stammar i Midgard den skaldekonst, som lifvar till
hjältegärningar och ger tröst åt sorgen.

Ivalde, den rätte brudgummen, kom icke långt efter Oden till Fjalars
borg. Men in kom han aldrig. Då den salvaktande dvärgen såg den
anländande mannen, sprang han emot honom och sade, att Oden var därinne.
Ivalde tyckte sig se, att ingången till bärget stod öppen, ty ljus
strömmade där emot honom. Dit skyndade han nu, men föll i ett af dvärgen
lagdt försåt och vardt aldrig synlig mer. Några säga, att han krossades
under stenblock, som nedvräktes från bärget.

\paragraph{XXIII.}

\paragraph{DOMEN ÖFVER VALANDS KONSTVÄRK.}

Dagen var nu inne, då gudarne skulle afkunna dom i målet mellan Sindre
och Loke. Den egentliga frågan var den, om Loke tappat vadet och
förbrutit sitt hufvud till Sindre eller icke. Också hade Sindres broder
Brock infunnit sig i god tid på gudarnes tingsplats, för att där på
stället taga Lokes hufvud, ifall domen tillstadde det. Loke var icke
älskad af Mimer och hans underjordskonstnärer. De visste, att han
åsyftade gudarnes fall, världsträdets förhärjande och världens
undergång.

Domen måste stödja sig på en oväldig och sakkunnig jämförelse mellan
Sindres konstvärk och Valands. Vore
Valands\protect\hypertarget{lb1625905.xhtmlux5cux23start73}{}{}\protect\hypertarget{lb1625905.xhtmlux5cux23start73-a}{}{}\protect\hypertarget{lb1625905.xhtmlux5cux23start73-b}{}{}\protect\hypertarget{lb1625905.xhtmlux5cux23start73-c}{}{}\protect\hypertarget{lb1625905.xhtmlux5cux23start73-d}{}{}
bättre än Sindres, så hade Loke vunnit vadet; i motsatt fall hade han
tappat det.

Brock var en vältalig sakförare för sin broders värk. Men hvarken Valand
eller någon af de andre Ivaldesönerna hade infunnit sig på tinget. De
för sin del hade ju icke ingått något vad. De hade aldrig tänkt på att
täfla med Sindre eller att låta en dom afkunnas öfver de klenoder, som
de af vördnad och vänskap skänkt till asarne.

Gudarnes dom vardt den, att ehuru Valands smiden voro förträffliga, vore
dock Sindres än mer värderika, särdeles för järnhammarens skull, som
blifvit pröfvad i Tors strid med Rungner och befunnits så ypperlig.

Brock ville nu taga Lokes hufvud. Loke erkände, att han förbrutit det,
men han fäste domstolens akt därpå, att vadet gälde hufvudet allena, men
att han aldrig lofvat, att Sindre eller hans ombud finge skada halsen.
Halsen var Lokes egendom, och öfver den hade ingen annan än han själf
rätt. Ej häller hade Brock rättighet att fördärfva hufvudet, utan blott
att »taga» det. Brock invände, att han icke kunde taga hufvudet, utan
att halsen skadades. Loke genmälde, att det skulle han ha tänkt på, när
vadet ingicks; nu var det för sent. Gudarne hade att döma äfven i denna
sak, och de dömde så, att eftersom Loke icke satt sin hals i vad, så
hade Brock ingen rätt öfver den. »Med hans hufvud står det mig likväl
fritt att göra hvad jag vill, om jag blott icke fördärfvar det», sade
Brock förbittrad; »och det bästa jag med det kan göra är att täppa det
gap, hvarifrån lögn och smädelse flyta». Brock tog upp en knif och en
tråd och ville sticka hål på Lokes läppar för att sy ihop hans mun. Men
Loke förstod att göra sig så pass hård, att knifven icke bet. Då ropade
Brock på Sindres syl. Och är det ett bevis på hvilka utomordentliga
konstvärk Sindres värktyg voro, att sylen på kallelse kom genast från
hans underjordssmedja till Asgard och stannade i Brocks hand. Brock
hopsydde Lokes läppar och gick bort, föga glad åt ärendets utgång. Lokes
läppar vordo snart åter fria; men ärren efter Sindres
\protect\hypertarget{lb1625905.xhtmlux5cux23start74}{}{}\protect\hypertarget{lb1625905.xhtmlux5cux23start74-a}{}{}\protect\hypertarget{lb1625905.xhtmlux5cux23start74-b}{}{}\protect\hypertarget{lb1625905.xhtmlux5cux23start74-c}{}{}\protect\hypertarget{lb1625905.xhtmlux5cux23start74-d}{}{}
syl gingo aldrig bort. Munnen fick det fula utseende, som anstår en
försmädares, och Lokes fagra utseende var från den dagen skämdt.

Domen öfver Ivaldesönernas smiden vardt inom kort bekant i alla världar
och uppfyllde Ivaldesönernas fiender med skadeglädje. Men Sindre själf
kände väl ingen glädje öfver sin seger. Han var bedragen på segerpriset,
och han ansåg, att domen skulle hafva farliga följder, ty Valands lynne
och Valands krafter kände han från den tid, när denne gjorde sina
lärospån i Sindres smedja.

\paragraph{XXIV.}

\paragraph{FÖRSÖKEN ATT FÖRSONA VALAND. IVALDESÖNERNAS FLYKT.}

Fröj vistades hos sin fosterfader Valand, när de båda underrättelserna
kommo till Ivaldesönerna, att Oden vållat deras faders död och att
Valands smiden blifvit jämförda med Sindres och underkända.

Valand och Egil möttes och rådgjorde, men sade till andra ingenting. Ej
häller begärde de af Oden böter för sin faders död. Valand var, som
vanligt, vänlig mot sin fosterson och dolde de onda nyheterna för honom.
Men de guldsydde praktfulla bonaderna i brödernas salar nedtogos, och
bland guldsmycken och vapen, som glänst på deras väggar, saknades efter
hand de bästa. Deras förut fyllda klenodkamrar tömdes. Efteråt fick man
veta, hvart de blifvit flyttade. De, som icke äro återfunna af ryktbara
hjältar i en senare tid, ligga ännu gömda i jordhålor och bärgsalar, där
de rufvas af drakar, eller nedsänkta i djupa flodbäddar, där de vaktas
af vättar, som hålla till i närbelägna åklippor eller strandåsar.

Gudarne började öfvertänka, att Ivaldes söner hade skäl till missnöje
med dem. Njord, hvars son var i Valands vård och våld, blef orolig. Han
samrådde med Oden om hvad borde göras, och de fattade det beslut, att
Valand skulle hedras
\protect\hypertarget{lb1625905.xhtmlux5cux23start75}{}{}\protect\hypertarget{lb1625905.xhtmlux5cux23start75-a}{}{}\protect\hypertarget{lb1625905.xhtmlux5cux23start75-b}{}{}\protect\hypertarget{lb1625905.xhtmlux5cux23start75-c}{}{}\protect\hypertarget{lb1625905.xhtmlux5cux23start75-d}{}{}
med ett till gudarne knutet slägtskapsband. Njord skulle begära hans
dotter Skade till äkta, och Skade upphöjas till asynja. På detta sätt
ville gudarne godtgöra den dom som blifvit fälld öfver Valands smiden,
och visa, huru högt de värderade honom. För Ivalde ville gudarne
förmodligen gifva böter, ehuru han brutit sin ed till dem och själf
vållat sin förtjenta undergång.

Njord skickade ett sändeskap, valdt bland vaner, som lydde under honom,
till Valand. Dennes borg är belägen på hans odalmark, Trymheim, som är
ett bärglandskap i Svitiod det kalla. Sändemännen skulle frambära Njords
friareärende; men de återkommo aldrig, och troligt är, att Valand dödade
dem. Medan man väntade på dem, beslöt Oden att göra en utfärd till
Trymheim, för att se huru landet låg. Han åtföljdes af Höner och Loke.

Besöket skulle göras i all enkelhet och utan att väcka uppseende i
världen. Oden satte sig icke i gyllene rustning på Sleipners rygg, utan
han och hans ledsagare iklädde sig vanliga färdemäns skick och
underkastade sig sådanas vilkor. Så kommo de till Trymheim. Här vandrade
de länge i villsamma dalar mällan snöhöljda bärgåsar och kunde icke
finna vägen till Valands borg. Denne, som hade ett skarpare späjareöga
på gudarne än de på honom, var lika stor trollkarl som han var konstnär
och lika hemma i Gullveigs runor som i Heimdalls. Han ställde det så,
att de tre gudarne färdades åt många håll, blott icke åt det rätta. På
ett ställe hade han ett försåt planlagdt, och dit styrde han ändtligen
färdemännens gång. Det var vid en källa i en ekbevuxen dal, där det såg
inbjudande ut att hvila. Där hade han nedlagt ett trolskt värktyg, som
han smidt och som liknade en vanlig stör eller stång.

Färdemännen voro trötte och Loke alldeles uppgifven af hunger. Valand,
för hvars skull han måste vandra så här och svälta, önskade han i alla
onda vättars våld. Så kom man till en dalöppning. Därifrån hördes
klangen af en pingla, en sådan som skällkor och lockrenar bära. När
vandrarne
\protect\hypertarget{lb1625905.xhtmlux5cux23start76}{}{}\protect\hypertarget{lb1625905.xhtmlux5cux23start76-a}{}{}\protect\hypertarget{lb1625905.xhtmlux5cux23start76-b}{}{}\protect\hypertarget{lb1625905.xhtmlux5cux23start76-c}{}{}\protect\hypertarget{lb1625905.xhtmlux5cux23start76-d}{}{}
kommo in i dalen, sågo de det behagliga stället vid källan och ej långt
därifrån en betande renhjord. Pinglan, som klingat angenämt i deras
öron, bars af en fet renoxe, som gick där strax bredvid. Det var en af
Valands hjordar, och hans dotter Skade, den skidlöpande och pilslungande
disen, torde nyss varit där och drifvit hjorden ned i dalen till
stället, där den nu betade.

Loke föreslog, att man skulle hvila sig här och äta sig mätt. Renoxen
fångade han utan svårighet och slagtade. En eld antändes för köttets
tillredande. Detta vardt buret öfver elden, men när det borde varit
färdigt, hade det icke blifvit mört alls, och det förekom som om elden
ej hade någon värkan på det. Medan gudarne undrande talade härom, kom en
örn och slog ned i ett träd därbredvid. Örnen hade ett ovanligt stort
och majestätiskt utseende, och ögonen, som framlyste ur hammen, hade
icke djurets blick. Gudarne gissade genast hvem han var. Det var kändt,
att Valand, som gjort svaneskrudar åt växtlighetsdiserna, hade gjort en
örnham åt sig själf. Oden hälsade honom med ord, som angåfvo, att
gudarne erkände sig stå i tacksamhetsförbindelse till honom. »Du, som
gömmer dig i fjäderskruden», sade han, »är konstnären, som gudarne hafva
att tacka för många smiden». Han berättade därefter om den underliga
tilldragelse de nu bevittnade, att elden, hvarmed maten tillreddes,
likasom förlorat sin kraft, och sporde den mångvise Valand, huru detta
skulle förklaras. Valand gaf sin faders baneman icke ett enda ord till
svar. Han vände sig i stället till Höner och sade: »blås i elden, och
han återfår sin kraft, om I tillerkännen mig min fulla andel i den helga
måltiden». Höner blåste, och tillredningen var nu genast färdig. Oden
bjöd Valand att taga plats ibland dem, och tillsade Loke att dela köttet
i fyra delar och framlägga en del åt dem hvar. Valand flög ned, men lade
icke bort fjäderskruden, såsom väl Oden väntat. Loke gjorde delningen,
men torde varit harmsen öfver att han, som fått värdigheten af asagud,
skulle betjena en, som endast hade alfbörd. Då Loke delat, tog Valand
alla fyra delarna och
\protect\hypertarget{lb1625905.xhtmlux5cux23start77}{}{}\protect\hypertarget{lb1625905.xhtmlux5cux23start77-a}{}{}\protect\hypertarget{lb1625905.xhtmlux5cux23start77-b}{}{}\protect\hypertarget{lb1625905.xhtmlux5cux23start77-c}{}{}\protect\hypertarget{lb1625905.xhtmlux5cux23start77-d}{}{}
flög upp i trädets nedersta gren, där han i de hungrande asarnes åsyn åt
med örnens stolta later och glupska matlust. Han hade i själfva värket
ingenting förtärt alltsedan han erfor sin faders död och den smädliga
domen öfver sina värk. Den fulla andel, som tillkom honom i måltiden,
var hele renoxen, ty denne var tagen ur hans hjord. Han hade icke
fordrat mer än som tillkom honom, då han ej gjort sig till värd på
stället. Nu vardt Loke ursinnig, och vreden gaf honom ett mod, som han
annars icke egde. Han tog stören, som låg bredvid honom, och gaf örnen
ett slag öfver ryggen. Stören stannade med sin öfre ände hos örnen, och
Loke kunde icke få den lös. Då ville han släppa den, men kunde det icke.
Hans händer voro som fastlödda, och i nästa ögonblick häfde örnen sina
vingar och lyfte sig i rymden, medförande stören och den sprattlande
Loke. Örnen flög hän öfver skogen och upp mot fjällsidan och försvann
med sitt byte ur de häpna gudarnes åsyn.

Örnen valde sin kosa så, att Lokes kropp slängde mot trädtoppar och
klipputsprång. Loke ville bedja om förskoning, men kunde endast skrika;
han gjorde sig så tung som möjligt, men Valand flög likväl med honom en
lång sträcka, innan han tröttnade och sänkte sig till marken med honom.
Där låg nu Loke och kröp vid hans fötter och bad om nåd. Valand sade,
att han skulle skona honom på ett enda vilkor: att Loke eftersade den
ed, som förestafvades honom, att han skulle föra Idun ut ur Asgard och
bringa henne till Valand. Loke var en edsbrytare till naturen och brydde
sig ej oftare om sina löften och bedyranden än det föll honom själf i
smaken; men den förfärliga ed, som Valand aftvang honom, var sådan, att
han aldrig skulle vågat bryta den. Idun var i Asgard strängt bevakad,
emedan asarnes läkedom mot tidens inverkan var endast i hennes händer
kraftig; det kunde därför dröja länge, innan tillfälle erbjöd sig; men
Loke försäkrade att vid första läglighet skulle eden uppfyllas.

Det förmäles icke hvad han efter sin återkomst till Oden och Höner hade
att förtälja. Men var han sig den gången
\protect\hypertarget{lb1625905.xhtmlux5cux23start78}{}{}\protect\hypertarget{lb1625905.xhtmlux5cux23start78-a}{}{}\protect\hypertarget{lb1625905.xhtmlux5cux23start78-b}{}{}\protect\hypertarget{lb1625905.xhtmlux5cux23start78-c}{}{}\protect\hypertarget{lb1625905.xhtmlux5cux23start78-d}{}{}
lik, sökte han nog intala dem, att han frivilligt höll i stören och hade
haft en svår dust med Valand och slutligen lyckats drifva honom på
flygten.

Så aflopp Odens besök i Trymheim. Han hade i Asgard intet godt att
omtala från färden. Njord beslöt då att själf begifva sig dit, Balder
och Had erbjödo sig att följa honom. De väpnade sig, satte sig i sadel
och redo dit. Resans ändamål var dock icke strid, utan försoning.
Fördenskull ville Njord icke ledsagas af Tor eller Tyr eller någon annan
af de asar, som ha den vanan att slå hastigt till. Had var visserligen
till lynnet häftig i sina första ungdomsdagar, men hade nu länge under
Balders ledning visat sig fridsam och räknades till de gudar, som
kallades fredsdomare, bland hvilka Balder var främst.

De tre gudarne kommo till Valands borg, men funno den tom. De redo då
hän emot Egils vid Elivågor. Med förvåning märkte de, att här och där på
afstånd Jotunheimsvarelser framskymtade. Hitintills hade jättarne aldrig
kommit öfver Elivågor fram till kusten af Midgard. Men Ivaldesönernas
vakt vid dessa vatten hade nu upphört, och de tilläto jättarne komma, så
många de ville, inpå den mark, som asarne skapat åt människoslägtet.
Njord och hans ledsagare påskyndade sin ridt och fingo syn på den de
sökte: Valand. Han var åtföljd af Egil och den tredje brodern, Slagfinn,
och de voro på väg till Jotunheim. De tre bröderna stannade, när de sågo
gudarne komma, och Valand ropade: »Hvad viljen I!» Njord svarade, att
han ville försoning med sin sons vårdare och fosterfader, med gudarnes
vän och smyckeskänkare. »Ingen försoning» hördes Valands stämma; »utan
hämd!» -- »Hvar är Fröj, din fosterson?» frågade Njord. -- »Utlämnad
till jättarne». -- »Ve dig, som sviker heliga löften!» -- »Ve Eder
själfve, I orättfärdige domare, I oduglige gudar!» svarade Valand. --
»Hvarje ord är här spildt», ropade Had; »vapnen måste tala. I förrädiske
söner af Ivalde, vågen I hålla stånd och strida?» Då trädde Egil fram
och utmanade Njord. »Du lågättade,
\protect\hypertarget{lb1625905.xhtmlux5cux23start79}{}{}\protect\hypertarget{lb1625905.xhtmlux5cux23start79-a}{}{}\protect\hypertarget{lb1625905.xhtmlux5cux23start79-b}{}{}\protect\hypertarget{lb1625905.xhtmlux5cux23start79-c}{}{}\protect\hypertarget{lb1625905.xhtmlux5cux23start79-d}{}{}
du träl!» ropade Had; »vågar du utmana en af Asgards yppersta?» Had grep
sin båge, lade pil på strängen och sigtade på Egil. Men denne hade
kommit honom i förväg. Ryktbara äro Egils tre pilar, smidda af honom med
Valands hjälp: de återkomma till sitt koger. Egils förste pil susade mot
Hads bågsträng och sönderskar den vid öfre fästet. Sin andre pil riktade
han så, att den, medan Had fastknöt sin bågsträng, gick mellan hans
krökta fingrar och handlofven. Hans tredje bortsköt Hads till strängen
lagda pil. Nu visade sig bakom Ivaldesönerna en mängd af deras nye
förbundsvänner, en skara af de ohyggliga jättar, som höra till Beles
(»skällarens») stam och hvilkas hufvuden likna hundars. Det var till dem
Fröj utlämnats af Valand. De kommo i dimmor och töcken, som förmörkade
näjden. De flyktande Ivaldesönerna försvunno i töcknet. Gudarne insågo,
att deras ärende var gagnlöst. De redo sorgsne tillbaka till Asgard.

Men innan Ivaldesönerna fortsatte sin väg, hade de lagt sina händer på
en af Egils pilar, och Valand svor: »Det är min ed, att när jag ödelagt
asarnes värld och skapat en bättre, skall ingen heta träl och ingen
hånas som lågättad.»

De begåfvo sig in i Jotunheim.

\paragraph{XXV.}

\paragraph{DEN FÖRSTE FIMBULVINTERN. BARNEN I ODÖDLIGHETSÄNGDEN. VALANDS
HÄMDESVÄRD.}

Efter Ivaldesönernas ankomst till Jotunheim började den förste
fimbulvintern. Den andre kommer kort före världsförstörelsen. Långt
bakom den yttersta bygden i Jotunheim, i den mäst aflägsna Norden, nära
jordskifvans rand och invid Amsvartners haf, öfver hvilket evigt mörker
rufvar, ligger ett land, till hvilket Ivaldesönerna ställde sin kosa.
Mellan isjökeltäckta fjäll draga sig dalar, bevuxna med träd af det
slag, som trifves i den ryktbara Järnskogens mörker och köld.
\protect\hypertarget{lb1625905.xhtmlux5cux23start80}{}{}\protect\hypertarget{lb1625905.xhtmlux5cux23start80-a}{}{}\protect\hypertarget{lb1625905.xhtmlux5cux23start80-b}{}{}\protect\hypertarget{lb1625905.xhtmlux5cux23start80-c}{}{}\protect\hypertarget{lb1625905.xhtmlux5cux23start80-d}{}{}
Svarta afgrunder gapa med okändt djup. I en af dem är en hålväg, som
leder ned till Nifelheim i underjorden.

Här i ett dalstråk, som kallas Ulfdalarne, byggde sig Valand och hans
bröder hus och smedja. Här kände de sig trygge mot hvarje förföljelse,
och hvad de förehade kunde icke ses från Asgards utsigtstorn Lidskjalf.

Valand hade två ändamål att vinna med sin vistelse i detta ogästvänliga
land. Han var kunnig som Gullveig, och mer än hon, i trolldomens alla
hemligheter och var herre öfver alla dess krafter, fastän han ditintills
ej velat använda dem. Nu skulle han bruka dem för att sända frost och
stormar öfver världen och göra Midgard obeboeligt. Det var hans ena
uppsåt. Hans andra var att smida ett svärd, hvari han ville nedlägga
hela sin konst, ett oemotståndligt svärd, som skulle bringa den af
Sindre smidde hammaren på skam, fälla Tor och tillintetgöra Asgards
makt.

Egil och Slagfinn gingo på skidor och jagade de vilda djur, som mäkta
lefva i denna världstrakt. Valand sysslade hela dagen med sina
trolldomsredskap och sitt smide. När icke hammarslagen från smedjan
genljödo mot det dystra dalstråkets fjäll, hördes Valands trollsång
ljuda, en underlig, dämpad hemsk sång, som fyllde själfva vilddjurens
hjärtan med ångest och mer än den värsta frostnatt kom dem att skälfva
af köld under sina pälsar. Då stod Valand på en klippa med ansigtet
riktadt mot söder och skakade ett skynke, liknande ett segel, eller han
spridde aska i luften eller förehade andra konster, och då mörknade
luften framför honom af frostdimmor, som förtätade sig till jättemoln
och drefvo fram öfver Jotunheim och Elivågor till Midgard, öfver hvars
fält de urladdade sig i snöstormar, hagelskurar och ödeläggande
hvirfvelvindar.

Hans trollsång trängde ned till Nifelheim i underjorden och grep med
yrsel de nio jättekvinnorna, som kringvrida världskvarnen och
stjärnehimmelen. Två starka tursmöar, Fenja och Menja, sällade sig till
dem och satte kvarnen i en rasande fart, så att jordens alla djup
bäfvade. Ur
kvarnstenen\protect\hypertarget{lb1625905.xhtmlux5cux23start81}{}{}\protect\hypertarget{lb1625905.xhtmlux5cux23start81-a}{}{}\protect\hypertarget{lb1625905.xhtmlux5cux23start81-b}{}{}\protect\hypertarget{lb1625905.xhtmlux5cux23start81-c}{}{}\protect\hypertarget{lb1625905.xhtmlux5cux23start81-d}{}{}
sprungo klippstycken, som kastades högt upp ur hafvet, jordens bärg
sprutade eld och rök, kvarnens resvärk rubbades och stjärnehimmelen fick
den sneda ställning han allt sedan innehar.

Valands trollsång trängde, likasom med tunga vingslag, ända hän till
Asgard och fyllde rymderna med skräck. Mödosamt banade sig solvagnens
och den tidmätande månens strålar väg genom de töckenfyllda djupen ned
till jorden. Oden lyssnade från Lidskjalf och förnam, att sången kom
från den yttersta Norden, där himlens rand och jordens närma sig
hvarandra. Han sände sina kloka korpar dit att späja. Men deras vingars
senor slappnade, deras blod stelnade, deras iakttagelseförmåga mattades,
och de måste vända om och hade intet att mäla. Oden sände Heimdall till
Urd, för att spörja hvad som förestod världen och gudarne. Hon grät och
nekade att svara.

Knappt en dag förgick under många år, då icke Valand, till omväxling med
arbetet i sin smedja, sjöng någon stund sin förfärliga galdersång. Den
trängde genom jorden och tärde på dess safter. År efter år sköto åkrarne
i Midgard kortare strå och tunnare ax, och hvad som återstod att skörda
härjades allt oftare af järnnätterna. Människorna blotade förgäfves till
gudarne och började tvifla på deras makt.

Till och med i Urds och Mimers heliga källor minskades safterna under
dessa år, världsträdets nordliga grenar vordo allt tommare på knopp och
blad.

Men så var ju också Fröj, årsväxtens gud, i jättevåld. Snart därefter
äfven Fröja, kärlekens och fruktsamhetens gudinna. Snart därefter äfven
Idun med sina ungdomsbevarande äpplen.

Gudarne slogos med förskräckelse, när de saknade Fröja i Asgard. De
sammanträdde på sin tingstad och öfverlade om hvem som förrådt henne åt
jättarne. De upptäckte den brottsliga. Det var en jättemö, som Fröja
upptagit i sin hofstat, och de gjorde än en upptäckt: att den förrädiska
jättemön var den pånyttfödda Gullveig. Tor slog henne då med
\protect\hypertarget{lb1625905.xhtmlux5cux23start82}{}{}\protect\hypertarget{lb1625905.xhtmlux5cux23start82-a}{}{}\protect\hypertarget{lb1625905.xhtmlux5cux23start82-b}{}{}\protect\hypertarget{lb1625905.xhtmlux5cux23start82-c}{}{}\protect\hypertarget{lb1625905.xhtmlux5cux23start82-d}{}{}
sin hammare till döds; häxans lik hölls än en gång öfver lågorna. Men än
en gång hände, att hennes hjärta endast blef halfsvedt, och att Loke,
letande efter det i askan, fann och slukade det. Han födde därefter till
världen den ulf, som heter Fenrer och är årsbarn med den förste
fimbulvintern.

Ej mindre grepos gudarne af förfäran, när de funno, att Idun var
försvunnen ur Asgard. Loke hade hållit den ed han svor Valand och
bortlockat henne därifrån. »Du är Ivaldes dotter», torde han ha sagt,
»och har blodshämd att utkräfva af Oden. Vill du tjäna din faders
baneman? Flyg till dina bröder och var hos dem! Det är din plikt.» Och
Idun flög i svaneham att uppsöka sina bröder, till hvilka hennes längtan
stod, mäst bland dem till Valand, som hon älskat från sin barndom. Tre
svanar kommo en morgon till Ulfdalarne och sågos af bröderna simma i ett
vatten i närheten af deras boning. En af dem var Idun. Den andra var
också en Ivaldedotter och växtlighetsdis, Auda. Den tredje var Sif,
deras fränka, hon med de gyllene lockarne, en växtlighetsdis, äfven hon.
Sif kom till Egil, för att vara hans hjälp och tröst, ty hans maka Groa,
hennes syster, kunde icke komma. Hon var fånge hos en Midgardshjälte vid
namn Halfdan. Auda lade sitt hufvud till Slagfinns bröst, och Idun slog
sina armar om Valands hvita hals.

Sålunda voro nu äfven dessa växtlighetsdiser, de som smyckat Midgards
ängar med blomster, försvunna för gudarne. Disernas håg var förvandlad:
de önskade Ivaldesönernas seger, Midgards förstörelse och Asgards
undergång.

Valands smedja var nu en värkstad för trollredskap. Där hängde ett långt
rep af lindbast, i hvilket knutar voro på jämna afstånd gjorda. I hvar
och en af dem var en stormvind bunden. För hvarje vecka, som gick,
upplöste han en knut och gjorde den bundna vinden lös och sände den med
sin galdersång söder-ut, mättad med snömoln och hagel.

Valand hade af guld smidt en armring, liknande den af Sindre smidde
ringen Draupner däruti, att andra ringar, ehuru
\protect\hypertarget{lb1625905.xhtmlux5cux23start83}{}{}\protect\hypertarget{lb1625905.xhtmlux5cux23start83-a}{}{}\protect\hypertarget{lb1625905.xhtmlux5cux23start83-b}{}{}\protect\hypertarget{lb1625905.xhtmlux5cux23start83-c}{}{}\protect\hypertarget{lb1625905.xhtmlux5cux23start83-d}{}{}
ej så många, droppade ut ur honom. Ur Valands kom en ring hvarje vecka.
Äfven de ringarne voro trollredskap och hade formen af ormar. För hvarje
på repet upplöst knut knöt Valand dit en ormring. På detta sätt talde
han de veckor han tillbringade i Ulfdalarne, och han hade räknat ut, att
när alla knutarne på repet voro upplösta, skulle Midgard vara
förvandladt till en istäckt människotom öken.

Dagligen smidde Valand på hämdesvärdet. Gambantein (»hämdetenen») är det
svärdets namn. Han härdade dess klinga i vatten från alla älfvar, som
flyta genom Nifelheim med etterfyllda böljor. Han hamrade hämd och hat,
ve och ofärd in i hennes gry, han slipade den giftiga eggen hvass som
vaferlågan, fäjade hennes yta med solens glans och ristade i henne
ondskans runor. Svärdets guldfäste var ett undervärk af skönhet. Smidet
höljde sin fördärflighet i ögontjusande fägring. Och likväl syntes det
aldrig blifva färdigt. Hvarje dag fann Valand något nytt att göra på
det. År efter år förled; men han fortfor med att fullkomna det.

Mimer, väktaren vid underjordens vishetsbrunn, såg, att allt
förskräckligare lidanden skulle hemsöka människoslägtet. Det försämrades
också ständigt till tänkesätt och seder genom de af Gullveig spridda
runorna. Mimer ville icke, att Asks och Emblas ättlingar skulle alla
varda fördärfvade genom nöd och synd. Visdomskällans mjöd hade gifvit
honom syner i en aflägsen framtid, efter denna tidsålders slut, och åt
den framtiden ville han frälsa ett ofördärfvadt människopar. Två
oskyldiga och väna barn, Leiftraser och Lif, uppsökte han i Midgard och
tog dem med sig till sitt underjordsrike. Där hade han i morgonrodnadens
land planterat en härlig lund och i den lunden låtit uppbygga en
praktfull sal. Hans sju älste söner -- Sindre och de sex andre --, som
med honom vårda världsträdets mällersta rot, smyckade salen och omgåfvo
den med en mur och satte in i muren en port, lika konstrik som
Asgardsporten. Inom den muren komma aldrig sorg och lidanden, aldrig
lyte och sjukdom, aldrig ålderdom och död. Salen heter Breidablik.
Delling,
morgonrodnadsalfen\protect\hypertarget{lb1625905.xhtmlux5cux23start84}{}{}\protect\hypertarget{lb1625905.xhtmlux5cux23start84-a}{}{}\protect\hypertarget{lb1625905.xhtmlux5cux23start84-b}{}{}\protect\hypertarget{lb1625905.xhtmlux5cux23start84-c}{}{}\protect\hypertarget{lb1625905.xhtmlux5cux23start84-d}{}{},
är dess väktare. Barnen därinne näras med den kraftförlänande
honungsdagg, som faller från Yggdrasils nedersta löfhvalf i Mimers och
Urds dalar.

\paragraph{XXVI.}

\paragraph{BALDERS DÖD.}

Stora voro de förluster Asgard lidit och stora de faror, som hotade
lifvet i Midgard, sedan Fröj och Fröja, Idun och de andra
växtlighetsdiserna kommit till Jotunheim, och sedan naturvärkmästarne
och naturförsvararne, Ivaldes söner, blifvit gudarnes fiender. Men ännu
hyste Oden hopp om räddning. Balder, solens gud och beskärmare,
godhetens och rättfärdighetens främjare, var dock ännu i Asgard och höll
sin välsignelserika hand öfver hvad godt som fanns bland människorna och
sin strålande sköld till skydd mot skapelsens förhärjare.

Balder hade nästan alltid Had i sin närhet. De voro länge oskiljaktige
vänner, och Balders mildhet värkade lugnande på Hads svallande lynne. Om
Had förtäljes, att han som gosse var så obetänksam, själfsvåldig och
okynnig och så gärna höll sig i Lokes sällskap, att Oden fann det bäst
att skicka honom från Asgard för att uppfostras af Mimer. Denne åtog sig
fosterfaderns plikter mot den vackre och med rika anlag utrustade
asasonen. Gossen var läraktig och vardt skicklig i skaldskap, sång,
smideskonst och många idrotter, och han blomstrade upp och fick, innan
han hunnit ynglingaåren, ovanliga kroppskrafter. Men hans obetänksamhet
och häftighet voro icke lätta att bota. Okynnig visade han sig
isynnerhet när han var i smedjan. En gång när Sindre hade gifvit honom
en tillrättavisning, tog pojken Draupners frejdade smed i håret och drog
honom på det sättet ut öfver tröskeln; vid ett annat tillfälle, när han
tillsades att smida och icke ville det, slog han med släggan ett sådant
slag i städet, att det flög i flisor. Men Mimers visdom lyckades
\protect\hypertarget{lb1625905.xhtmlux5cux23start85}{}{}\protect\hypertarget{lb1625905.xhtmlux5cux23start85-a}{}{}\protect\hypertarget{lb1625905.xhtmlux5cux23start85-b}{}{}\protect\hypertarget{lb1625905.xhtmlux5cux23start85-c}{}{}\protect\hypertarget{lb1625905.xhtmlux5cux23start85-d}{}{}
efter hand tämja hans uppbrusande lynne, om än icke läka hans
obetänksamhet, och när fosterfadern återsändt honom till Asgard som en
fager, höfvisk och i många idrotter skicklig yngling, fick Oden honom
mycket kär, och var han af alla sina fränder i Asgard väl upptagen.

Någon tid därefter blef Balder gift med mångudens dotter Nanna. Had
skulle önskat att ega en maka, henne lik, skön och älsklig och modig som
hon. Men af sig själf skulle Had aldrig kommit på den onde tanken att
tillegna sig Nanna.

Han var ifrig jägare och uppsökte gärna de skogar, som från Lidskjalf
kunde skönjas i långt fjärran, för att jaga där och nedlägga vilddjur
och drakar. Så hade han en morgon anländt till den i det östliga
Jotunheim belägna Järnskogen, i hvars hemska inre ingen dödlig vågat
intränga. Ensam jagade han där hela dagen och förirrade sig, så att han
ej fann en återväg. Dimmor föllo, och det liknade sig till en obehaglig
natt. Då såg han ett ljussken, gick åt det hållet och fann en
bärgkammare, där en högvuxen och vacker kvinna satt. Hon bjöd honom
härbärge öfver natten, räckte honom en trolldryck, som förvillade hans
sans och återuppväckte hans obetänksamhet och häftiga lidelser, och hon
talade om Nanna så, att hon tände upp hans omedvetna böjelse för henne i
brännande låga. Han gaf det löfte att taga henne från Balder och göra
henne till sin egen hustru. Trollkvinnan gaf honom en brynja som pant på
öfverenskommelsen dem emällan och såsom välbehöflig i den strid, hvari
hans beslut komme att inveckla honom. Om morgonen, då han vaknade, voro
trollkvinnan och bärgkammaren försvunna, och Had såg sig ligga under
öppen himmel. Men brynjan, som han fått, bevisade, att nattens samtal
var värklighet och ej en elak dröm. Han red ur skogen, så ångerfull
öfver det skamliga löfte han gifvit, att han icke kunde återse Balder
och Valhall, utan beslöt han att begifva sig till Jotunheim och sälla
sig till gudarnes fiender där, för att, kämpande i deras led, söka
infria sitt löfte.

Så bar han då till någon tid vapen mot sina egna fränder,
\protect\hypertarget{lb1625905.xhtmlux5cux23start86}{}{}\protect\hypertarget{lb1625905.xhtmlux5cux23start86-a}{}{}\protect\hypertarget{lb1625905.xhtmlux5cux23start86-b}{}{}\protect\hypertarget{lb1625905.xhtmlux5cux23start86-c}{}{}\protect\hypertarget{lb1625905.xhtmlux5cux23start86-d}{}{}
emedan han gjort löftet att taga Nanna med våld från Balder. Men Balder
besegrade honom med vapen först, men med godhet sist och fullkomligast.
Han återförde sin broder till Asgard, förlät, urskuldade och tröstade
honom, ja lät honom förstå, att han, Balder, snart skulle dö. Med hvem
ville han då se Nanna hällre förenad än med Had?

I själfva värket anade Balder, att han icke skulle lefva länge, och det
vardt för asarne bekant, att dödsbud visat sig för honom i drömmar.
Äfven järtecken bådade, att ondt hotade honom. Den trogne gångare, i
sadeln på hvilken han gjorde färder öfver himmelen och skyddade solen
och månen på deras vandringar i rymden, fick sin fot vrickad och kunde
icke genom gudinnornas läkesånger, utan blott genom Oden botas. Själf
led Balder af afmattning i sina fötter. De bekymrade gudarne vände sig
till världsträdets väktare, Urd och Mimer, för att få råd. Urd såg
Balders fördolda öde, men yppade det icke; Mimer ej häller. Den
ångestfulla Frigg kom då på den tanken att med ed förpligta alla
varelser och alla ting att icke skada Balder. Så allmänt älskad var han,
att gudarne trodde, att ett försök därmed kunde göras.

Det finnes lif i allt, äfven i det som för oss synes liflöst. Det finnes
en ordning, hvari hvarje varelse och hvarje ting har sitt rum. Elden,
vattnet och luften lyda sina herrskare, stenarne, malmerna, örterna och
djuren bilda samhällen och slägter. Och genom hela skapelsen spänner Urd
trådarne af de lagar, som bjuda vördnad för det heliga. De oskyldiga
tingen rysa för edsbrott. Endast bland de själar, som fått ett
personligt öde, finnas trotsare af nornans stadgar; men för dem finnas
också straff.

Så långt som gudarnes välde var erkändt, ja än längre, aflade alla
skapelsens varelser och ting eden. Aldrig hafva godheten och
rättfärdigheten haft en segerrikare dag än den, då den ädlaste af asar
tillförsäkrades trygghet genom en ed, som uttalades af alla himmelens
stjärnor och upprepades af luftens, hafvets och jordens innebyggare, af
alla bärg och dalar, af alla skogar och öknar, en ed, som genljöd från
Urds och Mimers
\protect\hypertarget{lb1625905.xhtmlux5cux23start87}{}{}\protect\hypertarget{lb1625905.xhtmlux5cux23start87-a}{}{}\protect\hypertarget{lb1625905.xhtmlux5cux23start87-b}{}{}\protect\hypertarget{lb1625905.xhtmlux5cux23start87-c}{}{}\protect\hypertarget{lb1625905.xhtmlux5cux23start87-d}{}{}
riken, ja från Jotunheim, där jättarne sade sig vara alla gudars
fiender, blott icke Balders; ja från själfva Nifelheim, där äfven
farsoternas och sjukdomarnes andar med bleka läppar uttalade
förpliktelsen.

Bland de varelser, som aflagt eden, fanns blott en, som icke ämnade
hålla den: Loke. Han betraktade hela edsförpliktelsen med löje. Det
fanns också varelser, som icke aflagt eden: Ivaldesönerna i Ulfdalarne.
Men bland dem var Valand den ende, som skulle haft hjärta att skada
Balder, ty hans hämdlystnad var omättlig. Ting funnos också, till hvilka
edsförpliktelsen ej kommit. Dock kunde den forskande Loke ej få reda på
mer än ett enda sådant: det var en späd mistel, en helig ört, som råkat
att växa på ett träd i den oheliga Järnskogen österut och, ensam som den
var och oskadlig till utseendet och fjärran från sin slägt blifvit vid
edsförpliktelsen förbigången.

Loke afskar misteln och begaf sig med den på en långresa: till de
aflägsna Ulfdalarne. Han skulle knappast vågat en sådan färd, om icke
skadeglädjen gifvit honom mod därtill. Med förvåning sågo Ivaldesönerna
Loke inträda i Valands smedja. Han sade sitt ärende och bad Valand göra
en pil af misteln, en sådan som Had plägade slöjda åt sig, och med den
egenskapen att ofelbart döda, där den träffade. Valand gjorde det.
Misteltein, såsom pilen kallas, vardt Valands andra Gambantein
(»hämdeten»).

Sedan den stora edsförpliktelsen var gjord, var det en lek bland asarne
att ute på idrottsvallen skjuta och kasta på Balder, emedan det icke
skadade honom. Vid ett sådant tillfälle sköt Had med båge, utan att
veta, att Loke lagt en pil i hans koger. Så kom Had att afskjuta
Misteltein på Balder. Pilen genomborrade honom och han föll död till
marken. Den djupaste bestörtning grepo alla gudarne. När de försökte
tala, brusto de i gråt. Asafadern hade mist sin älskling,
världsordningen underpanten för sitt bestånd, godheten sin förespråkare.

Så djupt nu asafadern grämdes af sin egen förlust, sörjde
\protect\hypertarget{lb1625905.xhtmlux5cux23start88}{}{}\protect\hypertarget{lb1625905.xhtmlux5cux23start88-a}{}{}\protect\hypertarget{lb1625905.xhtmlux5cux23start88-b}{}{}\protect\hypertarget{lb1625905.xhtmlux5cux23start88-c}{}{}\protect\hypertarget{lb1625905.xhtmlux5cux23start88-d}{}{}
han dock, såsom världens styresman anstod, än mer öfver världens. Han
sadlade Sleipner och red ned i underjorden till Mimer och besvor honom
att säga hvad han visste. Skulle med Balder världen gå sin eviga
undergång till mötes? Mimer sade blott, att själfuppoffring kräfves för
den, som vill skåda ned i så djupa gåtor. Då ref Oden ut sitt ena öga
och kastade det i visdomskällan. Det sjönk djupt och såg allt klarare,
ju djupare det sjönk, tills det såg Balder som den kommande
världsålderns konung i ett högsäte, omgifvet af lycksaliga
människoskaror.

Då vardt han lugn, men red från Mimer till Urd för att få lära något om
denna världsålders slut. Urd svarade, att hon redan vet, att han gömt
sitt ena öga i Mimers brunn och därmed sett hvad som var honom nödigt
att veta. Hvarför frestade han henne då? Men Oden lade för hennes fötter
sköna Valhallsklenoder och bad henne om de förutsägelser hon förnimmer
ur sin källas brus och ur suset af Yggdrasils krona: Då sjöng hon en
sång om denna världsålders sista tider, om Odens undergång och död på
slagfältet, och om Ragnaröks lågor. Asafadern lyssnade och förfärades
icke af ödets dom öfver honom. Att glad gå sin bane till mötes, det hade
han anbefallt människorna, sina skyddslingar, och borde göra det själf,
och att dö på en valplats anstod honom. Hans orolige ande fann någon
hvila, sedan han fått veta, att Balder, som var det bästa af honom,
skulle återvända och styra en fullkomnad värld.

Gudarne togo den döde Balder och buro honom till Asgards västra strand,
där hans skepp Ringhorne låg förtöjdt i lufthafvet. På skeppet byggde de
ett bål af stammar från Glasers guldlund. Oden bar i sin famn Balders
lik och lade det på bålet. När Nanna såg detta, sjönk hon död till
jorden med brustet hjärta. Oden lade henne bredvid hennes make. När allt
var färdigt och den heliga elden gniden till vedens antändning och bålet
signadt med Tors hammare, drog Oden från sin arm ringen Draupner och
lade den på Balders bröst. Man såg, att han lutade sig ned och hviskade
i den dödes öra.
\protect\hypertarget{lb1625905.xhtmlux5cux23start89}{}{}\protect\hypertarget{lb1625905.xhtmlux5cux23start89-a}{}{}\protect\hypertarget{lb1625905.xhtmlux5cux23start89-b}{}{}\protect\hypertarget{lb1625905.xhtmlux5cux23start89-c}{}{}\protect\hypertarget{lb1625905.xhtmlux5cux23start89-d}{}{}
Hvad han då sade har han aldrig omtalat för gudar eller människor. Det
blåste en stark nordanvind från Järnskogen: en af dessa vindar, som
Valand lösgjort och som i Midgard kom med köld och hagel. Bålet tändes,
seglen hissades, Ringhorne dref ut i lufthafvet, och asarne bidade på
stranden, tills det af lågor omhvärfda skeppet sjunkit ned bakom
synranden.

Det förde Balder och Nanna till underjordens västra port. Här emottogos
de döda af underjordsmakterna och ledsagades öfver glänsande fält till
Mimers odödlighetslund och borgen Breidablik i morgonrodnadens land. Där
väntades Balder och Nanna af Leiftraser och Lif, barnen som skola vistas
i Mimers lund intill världsförnyelsen. Salarne voro skrudade med bonader
och guldklenoder; på bordet framför högsätena stod, betäckt med en
sköld, en dryckesskål, hvari de kraftgifvande safterna från de tre
underjordskällorna voro blandade. Det är den dryck, som räckes de
saligen döde, för att utplåna märkena af den jordiska döden och skänka
dem förmåga att njuta salighetslifvets rena glädje. Skölden aftogs, och
de anländande välkomnades med »de skira krafternas dryck».

Balder och Nanna stanna i underjordens Breidablik intill den nya
världsålderns ankomst. Lif och Leiftraser äro hos dem och uppfostras af
dem till att varda fläckfria stamföräldrar till ett fläckfritt
människoslägte.

Frigg grät bittra tårar öfver älsklingssonen, och då Oden vandrade
mällan Asgardsborgarne, var det honom svårt att nalkas Balders sal, där
ej harpans toner klingade mer, där mjödhallen stod tom och genomsusad af
vindarne. Snart reste sig ett förskräckligt spörsmål genom Balders död
för Hads pil. Att frändedråp borde hämnas af frände var en lag, som
nornorna gifvit och asarne godkänt. Men huru skulle Balders död hämnas?
I Asgard voro alla öfvertygade om att Had icke hade velat sin broders
död, och ingen sörjde djupare än han det skedda. Skulle den i sitt
uppsåt oskyldige Had dräpas, blod af asafaderns blod än en gång utgjutas
och den ene sonens död medföra den andres? Ingen kunde råda till
\protect\hypertarget{lb1625905.xhtmlux5cux23start90}{}{}\protect\hypertarget{lb1625905.xhtmlux5cux23start90-a}{}{}\protect\hypertarget{lb1625905.xhtmlux5cux23start90-b}{}{}\protect\hypertarget{lb1625905.xhtmlux5cux23start90-c}{}{}\protect\hypertarget{lb1625905.xhtmlux5cux23start90-d}{}{}
detta, ingen ville lyfta sin hand mot Had. Men ej en gång gudarne äro i
stånd att genomskåda hvarandras innersta bevekelsegrunder. I denna,
såsom i alla liknande saker, var det deras domareplikt att icke blunda
för gärningsmannens tidigare lif, utan spörja, om ej däri fanns något,
som kunde varit en driffjäder till gärningen. Då Loke utsåg Had till
Balders baneman, hade han just haft detta för ögat. Asarne måste lägga
vigt vid, att det fanns yttre grunder för att Had handlat med ond
öfverläggning: Had hade ju en gång varit Balders fiende och efterfikat
hans maka.

Rätten i sin stränghet kräfde fördenskull, att broderdråparen skulle
straffas med döden, eftersom böter i ett sådant fall som detta ej kunde
gifvas. Ty hvilka böter voro tillräckliga för dråpet på solens gud och
rättfärdighetens herre? Nornans runor, som hälga blodshämden, äro icke
ristade på vattnets yta eller i hedens flygsand, och låta icke gäcka
sig. Men Oden kunde icke besluta sig att kräfva en sons död i vederlag
för en annans. Då omslingrade honom Urd med sina osynliga band, drog
honom bort från Asgard, förde honom till Rind, Billings solljusa dotter,
och ingaf honom en oemotståndlig kärlekslidelse till henne, för att hon
med Oden skulle föda en son, som blefve Balders hämnare. Bland de ofödda
själarna var en redan af henne utkårad att varda son af Oden och
fullgöra detta kall, och af nåd hade hon lagt den sonens lefnadslotter
så, att han skulle fullgöra hämden, innan »han kammat sitt hufvud och
tvått sina händer», innan han ännu visste, att han var Odens son och
Hads broder, eller kunde göras tveksam af den sorg han måste förorsaka
med sin gärning.

Men trollska makter hade bemäktigat sig Rinds sinne och gjort henne hård
mot Odens böner och sin faders råd. Oden kunde upphäfva den på Rind
hvilande trolldomen, men endast så, att han begagnade sig af trolldom
själf, af Gullveigs säjdkonst och förbjudna runor. Detta gjorde
asafadern -- nyttjande, när framgång ej annorlunda kunde vinnas, ett
tadelvärdt medel att komma till ett af rättvisan och ödet fordradt mål.

\protect\hypertarget{lb1625905.xhtmlux5cux23start91}{}{}\protect\hypertarget{lb1625905.xhtmlux5cux23start91-a}{}{}\protect\hypertarget{lb1625905.xhtmlux5cux23start91-b}{}{}\protect\hypertarget{lb1625905.xhtmlux5cux23start91-c}{}{}\protect\hypertarget{lb1625905.xhtmlux5cux23start91-d}{}{}

Rind födde med Oden en son, som kallas Vale. Hon födde honom tidigt, ty
han slet otåligt i de band, som fängslade honom under modershjärtat, och
han var endast dygngammal, då han uppvuxit till en kämpe, som i tvekamp
fällde Had. Den döde Had steg ned till underjorden. Balder emottog honom
där. De lefva där nere tillsammans, och då världsförnyelsens dag kommer,
skola de tillsammans återvända och tillsammans bo på Valhalls tomter.

\paragraph{XXVII.}

\paragraph{FIMBULVINTERNS HÖJDPUNKT. FOLKUTVANDRINGEN FRÅN NORDEN.
SKÖLD-BORGAR OCH HALFDAN.}

Efter Balders död fanns det i Asgard ingen makt, som kunde häjda de af
Valand från Jotunheim utsända hvirfvelvindarne, snöstormarne och
hagelskurarne. Skilnaden mällan sommar och vinter försvann allt mer, och
det såg ut som om vintern skulle tillegna sig årets alla månader. Landet
närmast söder om Elivågor, där Egils borg stod, vardt öfverdraget af
jöklar och isfält, som sommarsolens strålar icke smälte. De alfer, som
bott där och varit Valands smidesbiträden och Egils och Tjalves
stridsmän, utvandrade därifrån och begåfvo sig till svearne. Jättar
satte öfver Elivågor i båtar, sådana som Hymer egde, och nybyggde de
öfvergifna ängderna. De togo sina bostäder under de jökelhöljda bärgens
tak; deras svarta oxar och kor med guldhorn trifvas godt, där Midgards
hjordar dö af hunger, ty de slicka, som Audumla, föda ur rimfrosten och
uppkrafsa under dalgångarnes drifvor närande mossa.

På den stora ö i det nordiska hafvet, hvars sydligaste del är
Aurvangalandet, bodde nu många folkstammar, komna från Ask och Embla,
alla talande samma tungomål och följande samma sedvänjor. Söder om
svearne och nedemot Järavallarne, som utgöra Aurvangalandets södra kust,
bodde goter, daner, heruler, gepider, viniler (longobarder), angler,
\protect\hypertarget{lb1625905.xhtmlux5cux23start92}{}{}\protect\hypertarget{lb1625905.xhtmlux5cux23start92-a}{}{}\protect\hypertarget{lb1625905.xhtmlux5cux23start92-b}{}{}\protect\hypertarget{lb1625905.xhtmlux5cux23start92-c}{}{}\protect\hypertarget{lb1625905.xhtmlux5cux23start92-d}{}{}
saksare, tyringar, vandaler och andra stammar. Dit intills hade de alla
lefvat i endrägt och ansett sig för hvad de voro: grenar på ett och
samma folkträd. Så långt som de utbredt sig öfver Norden, hade landet
varit godt, vackert och skördegifvande.

Men med fimbulvintern kom nöd öfver dem, främst öfver svearne, som bodde
nordligare än de andra. Fröj och Fröja främjade icke längre fruktbarhet
och fruktsamhet; växtlighetsdiserna skänkte icke längre af Yggdrasils
befruktande honungsdagg åt åkrar och ängar. Hjordarne aftogo, men
björnarnes och ulfvarnes skaror växte, och den värste ulfven,
hungersnöden, öppnade sitt gap mot folket.

Då samlade sig alfer och sveahöfdingar till möte vid Svarins hög, och
allt sveafolket deltog i deras rådslag, och vardt då beslutadt att
öfvergifva fädernas land och draga mot söder. Kunde man icke vinna
bättre ängder med vänlig öfverenskommelse, skulle man taga dem med
svärdet. Och till strid kom det ofta, då nu svearne trängde ned mot
goterna och desse mot danerna, och desse mot herulerna och de andra
stammarne. Påträngandet från norr till söder vardt för hvarje år
starkare, ty för hvarje år eröfrade fimbulvintern och de osmältbara
isfälten större område från det beboeliga landet.

Vid denna tid lefde ännu och herrskade i Aurvangalandet som höfding och
domare Sköld-Borgar, sonen af den vanagud (Heimdall), hvilken som barn
kom till nämda land med den heliga elden, de goda runorna, sädeskärfven,
värktygen och vapnen. Sköld-Borgar var nu gammal vorden. Under en lång
och lofvärd styrelse hade han fått upplefva mycket. Han erinrade sig
från sin barndom och ungdom guldålderns lyckliga tider, fria från
laster, strider, brott och nöd. Hans mannaår hade sett mänsklighetens
kopparålder och slägtenas försämring. På gamla dagar fick han upplefva
fimbulvinterns inbrott och begynnelsen till en järnålder, hvarom han
hört en förutsägelse vid sin sons födelse. Sonen hette Halfdan och var
nu en yngling; Sköld-Borgars maka, Halfdans moder, hette Drott.

\protect\hypertarget{lb1625905.xhtmlux5cux23start93}{}{}\protect\hypertarget{lb1625905.xhtmlux5cux23start93-a}{}{}\protect\hypertarget{lb1625905.xhtmlux5cux23start93-b}{}{}\protect\hypertarget{lb1625905.xhtmlux5cux23start93-c}{}{}\protect\hypertarget{lb1625905.xhtmlux5cux23start93-d}{}{}

Om Halfdans födelse förtäljes, att den natt, då han kom till världen,
rådde stark storm. »Heliga vatten nedstörtade från himmelsfjällen»
(stormmolnen), som med väldigt dån utslungade korsande blixtar. Det var
Tor, åskans gud, som öfverskyggde huset och gaf sin gudomliga närvaro
tillkänna. Allmänt förtäljes, att Tor delade med Sköld-Borgar
fadersrätten till den nyfödde på samma sätt som Heimdall i de hem, som
han i forntiden gästade, hade delat med husbonden på stället
fadersrätten till de söner som där sågo dagen. Därför gällde Halfdan för
att vara Tors son och på samma gång Sköld-Borgars.

Natt låg ännu öfver Sköld-Borgars gård, när de tre ödesdiserna, Urd och
hennes systrar, kommo dit och betraktade det nyfödda ädla barnet och
igenkände i dess drag -- hvad redan Drott igenkänt -- att det hade ej
allenast en mänsklig fader, utan ock en gudomlig. Och de bestämde, att
han skulle varda den ypperste i sin ätt, den förnämste af höfdingar och
bland de nordiska folken den förste med konunganamn. De tvinnade hans
ödes trådar starka, varpen i hans ödes väf redde de af guld och fäste
dess ändar i öster och väster under »månens sal» (rymden). I öster och
väster skulle Halfdan utan svåra hinder utbreda sitt välde. Men åt norr
kastade Urd en enda tråd. Hon visste att från norr skulle komma
fruktansvärda makter, i början oemotståndliga, för hvilka Halfdan måste
rygga; men Urd bad till den okända makt, hvars ombud hon är, att den
tråd hon åt norr kastat måtte evigt hålla. Därefter aflägsnade sig
nornorna. Men Sköld-Borgar gick ut i den åskdigra natten, plockade på
ängen en härlig blomma och satte den som en spira i barnets hand.

Följande dag sutto två korpar i trädet, som skuggade öfver Sköld-Borgars
tak. Genom vindögat (rököppningen i taket) tittade de in i salen och
sågo den nyfödde. Framsynta som korpar äro, förekom det dem som om
gossen redan stod vapenrustad. »Ser du», sade den ene korpen till den
andre, »han är blott en dag gammal, denne gudaättling, och ändå ser jag
honom skrudad i brynja. Nu är fredens ålder slut;
\protect\hypertarget{lb1625905.xhtmlux5cux23start94}{}{}\protect\hypertarget{lb1625905.xhtmlux5cux23start94-a}{}{}\protect\hypertarget{lb1625905.xhtmlux5cux23start94-b}{}{}\protect\hypertarget{lb1625905.xhtmlux5cux23start94-c}{}{}\protect\hypertarget{lb1625905.xhtmlux5cux23start94-d}{}{}
nu är vår tid inne. Vi och ulfvarne ha länge hungrat, men ser du, hur
hvassa ögon den gossen har! Ja, vår tid har kommit. Vi skola trifvas.»

Sköld-Borgar hade af sin fader Heimdall fått lära fågelspråket. Drott
kände det äfven. De förstodo hvad korpen sade, och det fyllde dem med
ängslan och bekymmer för de kommande slägtena. De tolkade det så, att
stridsåldern hade med detta barn kommit i världen.

Sköld-Borgar hade en frände, som hette Hagal, hans närmaste vän. Till
honom skickade han Halfdan att uppfostras. Hagal hade en son, som hette
Hamal, jämnårig med Halfdan. Halfdan och Hamal lekte som gossar
tillsammans, vordo tillsammans ynglingar och svuro hvarandra evig
vänskap. De voro de skönaste människosöner i Midgard; i valkyriedrägt
skulle de sett ut som valkyrior. Och de voro hvarandra till utseendet så
lika, att det var svårt att säga hvem som var Halfdan eller Hamal. Men
däruti skilde de sig, att Halfdan var ordrik och Hamal tystlåten,
Halfdan snar att fatta beslut och utföra dem, Hamal eftertänksam och
besinningsfull, men djärf, äfven han, i beslutets värkställande. Halfdan
hade stora snillegåfvor. I de af Heimdall lärda heliga runorna vardt han
kunnigare än sin fader, och han blef forntidens störste skald, den
ypperste i Midgard. Han var frikostig och älskade att strö guld omkring
sig. Och när nu därtill kom, att han vardt den starkaste hjälten bland
alla sina samtida, så är ej att förvånas öfver, att han beundrades mer
än de fleste och vardt besjungen från slägte till slägte. Den, hos
hvilken snille och skönhet, styrka och frikostighet äro förenade, han
synes mången vara fullkomlig. Sköld-Borgar, Halfdans fader, var dock i
somligas ögon bättre än han; ty fadern älskade friden och sträfvade, mer
än till något annat, till befästande af endrägtens band. Halfdan älskade
krig och äfventyr. Men han var rätte mannen för den tid, som kom. Friden
var för alltid flyktad från Midgard. Den världsålder, som kallas
»stormtid och ulftid, yxtid och kniftid», hade inbrutit, och den skall
fortgå till Bagnarök.

\protect\hypertarget{lb1625905.xhtmlux5cux23start95}{}{}\protect\hypertarget{lb1625905.xhtmlux5cux23start95-a}{}{}\protect\hypertarget{lb1625905.xhtmlux5cux23start95-b}{}{}\protect\hypertarget{lb1625905.xhtmlux5cux23start95-c}{}{}\protect\hypertarget{lb1625905.xhtmlux5cux23start95-d}{}{}

Halfdan förde med skicklighet alla slags vapen. Men hans älsklingsvapen
var klubban. Svärdet hade under fridsåldern varit obekant. De första
svärden smiddes i Valands smedja, och de började nu komma i bruk. Äfven
Halfdan hade sådana. Dock misstänktes och skyddes svärdet af de fleste
stridsmännen, alltsedan Valand vardt gudafiende och förvandlad till den
värste af Jotunheims bebyggare. Man trodde fördenskull att förbannelse
hvilade på det.

När Egil flydde med sina bröder till Ulfdalarne, efterlämnade han sin
hustru Groa och bad henne begifva sig till hennes fader Sigtrygg, som
var en höfding bland alferna och vän till Ivaldes söner. Sigtrygg hade
sin borg i Svitiod det stora i närheten af Svarinshög och ej långt från
svearnes landamären. Groa väntade att få en son; hon hade annars följt
sin man till det frostkalla land, dit han gjort sig landsflyktig. Det är
redan omtaladt, att sedan Ivaldesönerna upphört att vakta Elivågor,
begåfvo sig många jättar öfver till det nordligaste Midgard och bosatte
sig där. Tor gjorde en färd dit upp för att bortdrifva eller döda dessa
farliga nybyggare, och Halfdan fick följa sin Asgardsfader på färden.
Det hände då, att medan Halfdan och Hamal redo genom en skog, mötte de
en trupp af fagra kvinnor, likaledes till häst. Det var Groa och de
ungmör, som tjenade henne. De ämnade sig till en skogssjö, för att bada.
Halfdan vardt betagen af växtlighetsdisens skönhet och bad henne följa
honom till Aurvangalandet och stanna där. Väl tyckte Groa, att Halfdan
var den vackraste bland ynglingar, men hon hade gifvit sin tro till en
annan och svarade nej. Då nödgade han henne att följa med honom. Tor
gillade denna gärning, eftersom Egil blifvit gudarnes fiende, och enär
det vore till gagn för Midgard, att en växtlighetsdis där var fästad.
Men Sigtrygg, Groas fader, gillade den icke, och det kom mällan honom
och Halfdan till en strid, som lyktades så, att Halfdan fällde honom med
en klubba, på hvars skaft en guldkula var fästad. Växtlighetsdisernas
fader kunde ej falla för annat vapen än ett, som liknade solen. Halfdan
förde Groa till Sköld-Borgars gård.
\protect\hypertarget{lb1625905.xhtmlux5cux23start96}{}{}\protect\hypertarget{lb1625905.xhtmlux5cux23start96-a}{}{}\protect\hypertarget{lb1625905.xhtmlux5cux23start96-b}{}{}\protect\hypertarget{lb1625905.xhtmlux5cux23start96-c}{}{}\protect\hypertarget{lb1625905.xhtmlux5cux23start96-d}{}{}
Där födde hon en son, som ej var Halfdans, utan Egils. Sonen kallades
Od-Svipdag. Ett år därefter födde hon med Halfdan en son, som fick
namnet Gudhorm. Hon kände sig dock alltid som en främling i Halfdans
hem. Hon led af tanken att vara maka till den man, som blifvit hennes
faders bane, och hennes hjärta trängtade efter Egil. Efter några års
förlopp blef hon fördenskull bortsänd af Halfdan, och hon begaf sig med
sin unge son Od-Svipdag tillbaka till Svitiod det stora, där hon väntade
på Egils återkomst. Frostnätter, snöstormar och hagelskurar kommo; men
icke Egil. Då trånade växtlighetsdisen bort och dog. På dödsbädden sade
hon till sin son Od-Svipdag, att om han behöfde hennes hjälp, skulle han
gå till hennes graf och kalla på henne. Hennes lik lades i en grift,
byggd på en bärghäll med väggar, tak och dörr af tunga stenblock.

Tor och Halfdan gjorde flere utfärder mot de jättar, som tagit land i
det nordliga Midgard. Många jättar och krigiska jättekvinnor blefvo då
dödade; men detta tjenade i det hela till föga, ty andra inflyttade i
stället, och så god Tors järnhammare var, kunde han med den icke hindra,
att allt vidsträcktare isfält lade sig öfver bärgslätterna, och allt
flere dalar lågo, äfven sommartiden, höljda i snö.

Och nu, när den stora folkvandringen inträdt, sedan svearne börjat rycka
mot söder, fick Halfdan annat att skaffa än att följa Tor på utfärder
mot jättarne. Den första mot söder trängande folkböljan satte en andra i
rörelse, och den andra en tredje och så vidare, så att bölja efter bölja
svallade mot Aurvangalandet, människoslägtets urhem, där Sköld- Borgar
så länge och lyckligt varit folkets lagstiftare och domare. På
Aurvangalandets gräns stodo nu hårda strider, i hvilka Halfdan och Hamal
utförde många bragder. Men den gamle Sköld-Borgar såg, att motstånd i
längden vore gagnlöst, så länge fimbulvinterns makter rasade bakom de
nordligare stammarne och nödgade dem att tränga på. De hade ju att välja
detta eller hungersdöden. Och enär alla dessa stammar voro fränder och
ledde sin härkomst från
Aurvangalandet\protect\hypertarget{lb1625905.xhtmlux5cux23start97}{}{}\protect\hypertarget{lb1625905.xhtmlux5cux23start97-a}{}{}\protect\hypertarget{lb1625905.xhtmlux5cux23start97-b}{}{}\protect\hypertarget{lb1625905.xhtmlux5cux23start97-c}{}{}\protect\hypertarget{lb1625905.xhtmlux5cux23start97-d}{}{},
ville Sköld-Borgar icke se dem förstöra hvarandra i brödrafäjd.
Fördenskull beslöt han med sitt folk tåga söderut, äfven han. Så skedde,
och många stammar förenade sig under honom, för att å andra sidan det
nordiska hafvet vinna land och bosätta sig. Skepp byggdes och förde dem
öfver hafvet. Halfdan vardt alla dessa stammars anförare; deras
höfdingar lyfte honom på sköld och kårade honom till konung. Ingenting
mäktade motstå dem. De lade under sig söder om hafvet vidsträckta land,
där segelbara floder söka sig väg mellan djupa löfskogar och saftiga
betesfält. De vunno och delade mellan sig ett rike, som i väster hade
till gräns den väldiga flod, som heter Ren, i söder det skogshöljda
högland, som ligger i skuggan af de högsta bärgen i Midgard, men i öster
sträckte sig långt in i ett omätligt slättland med många älfvar, som
bilda vägar ned mot ett sydligt haf (Svarta Hafvet). Så uppfylldes
nornornas spådom, att Halfdan skulle hafva ett rike, vidsträckt i väster
och öster, så långt som de i dessa väderstreck hade knutit de gyllne
trådarne i varpen af hans väf; men åt norr låg blott en enda tråd, och
det gällde, om den skulle hålla, eller om människoslägtets urhem och
fädernas heliga grafvar skulle för alltid stanna i frostmakters och
gudafienders våld.

\paragraph{XXVIII.}

\paragraph{VALANDS SENARE ÖDEN OCH DÖD.}

Sju vintrar hade svanemöarna Idun, Sif och Auda tillbragt i Ulfdalarne.
I den åttonde vintern blefvo de sjuka af längtan efter solglans,
vårfläktar och blommor; ty sitt innersta skaplynne kunde de icke
besegra. Sorgsna och tynande uthärdade de ännu ett år hos Ivaldes söner.
Men därefter togo de sina svaneskrudar och flögo bort öfver Jotunheim
och Elivågor.

Då vardt det tungt äfven för bröderna att stanna. Men Valand hade för
stora mål i sigte, för att längtan och sorg skulle kunnat afvända honom
från dem; han var ej allenast
\protect\hypertarget{lb1625905.xhtmlux5cux23start98}{}{}\protect\hypertarget{lb1625905.xhtmlux5cux23start98-a}{}{}\protect\hypertarget{lb1625905.xhtmlux5cux23start98-b}{}{}\protect\hypertarget{lb1625905.xhtmlux5cux23start98-c}{}{}\protect\hypertarget{lb1625905.xhtmlux5cux23start98-d}{}{}
oläkligt hämdgirig, utan också ihärdig och tålmodig utan like. Till
bröderna sade han: »Då I längten härifrån, så är det bättre för eder att
gå.» De togo sina skidor och bågar och gingo på olika vägar åt söder.
Egil följdes af en gosse, som var hans och Sifs son, född i Ulfdalarne
och af fadern uppfostrad till bågskytt och skidlöpare. Hans namn var
Ull, och ödet hade med honom stora uppsåt.

Valand var nu ensam. Han fortfor med sina trollkonster och sitt smide på
hämdesvärdet. Dessemällan jagade han och skaffade sig så sin föda. År
gingo, galdersången fortfor, och stormby efter stormby med frost och
hagel lösgjordes i Ulfdalarne, för att härja skapelsen. Isfälten sköto
ned emot Aurvangalandet, och den ofantliga Nordhafs-ön, som förut varit
grönskande och bebodd af så många folk, var en människotom öken af is
och snö. Endast själfva Aurvangalandet var ännu beboeligt; där byggde nu
svearne och med dem var en skara alfer.

Då det uppenbart visat sig, att asagudarne icke voro i stånd att häjda
fimbulvintern, lät Urd Mimer veta, att nu hade han rätt och plikt att
gripa in. Mimer och hans sju älste söner, Sindre bland dem, stego
välbeväpnade till häst. Mimers drottning och döttrar, som äro nattens
diser, tolf till antalet, satte sig äfven i sadel och följde dem.
Nattdisernas hjälp var behöflig vid denna färd. Så redo de öfver
Nidafjällens branta, fuktiga bärgstigar, förbi världskvarnen och
Hvergelmerkällan ned i det töckniga Nifelheim, och därifrån upp genom
den hålväg, som leder från underjorden till Ulfdalarne. Komna dit,
ordnade de sig så, att nattdiserna bildade en ring kring männen.
Därigenom kom hela truppen att för en utanför varandes ögon likna ett
mörkt töcken, som dref utefter dalgången. Deras afsigt var att
öfverrumpla Valand, hvars hämdesvärd gjorde honom oöfvervinnelig. Månens
skära stod på himmelen och lekte med sitt bleka sken på ryttarnes
hjälmar, brynjor och vapen; men det var blott som om månen lekt med en
dimma. De kommo sent om kvällen till Valands smedja och stannade vid
dess gafvel. Töcknet lägrade sig där.

\protect\hypertarget{lb1625905.xhtmlux5cux23start99}{}{}\protect\hypertarget{lb1625905.xhtmlux5cux23start99-a}{}{}\protect\hypertarget{lb1625905.xhtmlux5cux23start99-b}{}{}\protect\hypertarget{lb1625905.xhtmlux5cux23start99-c}{}{}\protect\hypertarget{lb1625905.xhtmlux5cux23start99-d}{}{}

Dörren stod öppen, och i smedjan sågs ingen. Mimer och hans söner stego
ur sadeln och gingo in. Med kännares ögon betraktade de Valands
smidesredskap. Så gingo de till lindbastrepet och besågo detta
trollredskap och undersökte de hundratals guld-armringar, som hängde
där. Det blef dem klart, att Valand gjort en guldring, liknande Sindres
Draupner, och att från den hade de andra droppat. Mimer och Sindre
undersökte alla ringarne och funno den, som var de andras moder. Mimer
tog den och bar den ut och gaf den till sin dotter Baduhild, en af
nattdiserna. Därefter lämnades smedjan, och männen gingo in i töcknet
vid hennes gafvel.

Ändtligen kom Valand. Man såg honom i det svaga månljuset glida på sina
skidor ned för en fjäll-lid, väpnad med jagtspjut och båge. Hämdesvärdet
lyste vid hans sida. På sina skuldror bar han en björn, bytet af dagens
jagt. Han gick in i smedjan nu, lade vindtorr ved och ris på äriln,
tillagade af björnkött sin måltid och satte sig vid bordet på den med en
björnhud bonade bänken. Eldskenet från äriln speglade sig i ringarne på
lindbastet. Valand gick dit och räknade, såsom han ofta gjorde, sina
ringar. Där funnos nu 840, men en, de andras moder, saknades. Tjufvar
hade aldrig sett Ulfdalarne, och ingen, med undantag för Ivaldes barn,
eller Mimers söner, skulle kunnat skilja mällan ringarne och finna den
yppersta. Genom Valand ilade en rysning af sällhet. »Idun har
återkommit», tänkte han; »jag trodde alltid, att hon skulle återkomma.
Nu är hon här. Hon har tagit ringen till ett tecken, att hon är här. Hon
söker mig därute på jagtens stigar, och mina skidors spår skola leda
henne tillbaka hit».

Han satt länge och inväntade henne. Hans hjärta, som brunnit så länge af
hämdens vällust och kval, svalkades ljuft af tankar på trogen kärlek.
Men tröttheten öfverväldigade honom. Hufvudet, som hvälfde de
världsfördärfliga planerna, sjönk ned emot sitt bröst. Valand sof.

Han vaknade, men ej till glädje. Lindbastrepet, hvari
\protect\hypertarget{lb1625905.xhtmlux5cux23start100}{}{}\protect\hypertarget{lb1625905.xhtmlux5cux23start100-a}{}{}\protect\hypertarget{lb1625905.xhtmlux5cux23start100-b}{}{}\protect\hypertarget{lb1625905.xhtmlux5cux23start100-c}{}{}\protect\hypertarget{lb1625905.xhtmlux5cux23start100-d}{}{}
han bundit stormarne, var nu bundet kring stormutsändaren själf.
Trollredskapet slingrade sig kring hans armar och ben och var en fastare
boja än någon af järn. Framför honom stod Mimer och hade hämdesvärdet i
sitt bälte, och bredvid honom Sindre, hvars hammarsmide hämdesvärdet
skulle bringa på skam. Valand skar tänder, och hans ögon gnistrade. Hans
hämdevärk, nu nästan färdigt, var med ens tillintetgjordt. Så stark var
dock den väldiges själ, att han i nästa stund var lugn igen, och i hans
dystra drag sågs hvarken trots eller förtviflan, endast beslutsamhet och
tålamod.

Valand medfördes bunden till underjorden. I en sjö under Nidafjällens
branter, ej långt från Mimers söners och döttrars borg, är en holme och
på den en af underjordssmedjorna. Dit fördes Valand. Mimers gemål, som
märkt den förfärliga blicken i Valands ögon, när han såg hämdesvärdet i
hennes makes våld, var gripen af hemska aningar om stundande olyckor för
henne och hennes närmaste och gaf sig ingen ro, förrän man afskurit
Valands knäsenor. Hon trodde, att han därmed var oskadliggjord och dömd
till evig fångenskap på holmen.

Snart hördes hammarslag därifrån. Mimer besökte sin fånge och fann honom
lugn och arbetslysten. Guld och andra malmer, ädelstenar och hvad hälst
Valand önskade att arbeta på sändes honom. Ett förråd af sorgdöfvande
och hjärtstyrkande mjöd öfverfördes till hans smedja. Valand arbetade
raskt, och Mimer, klenodsamlaren, såg med glädje huru genom Valands flit
de underbara skatter ökades, som Mimer vill bevara åt en kommande
världsålder.

Men Valand smidde också på något, som hans väktare icke hade en aning
om. Han gjorde sig en flyginrättning, liknande den örnham han förut egt.

Vid stranden af sjön låg en båt, hvari Mimer och hans söner,
underjordssmederna, stundom foro öfver till holmen, samspråkade med
Valand och beskådade hans händers värk. De dolde icke sin beundran öfver
dem. Sindre, den store konstnären, hade aldrig hyst afund till Valand,
ehuru han
\protect\hypertarget{lb1625905.xhtmlux5cux23start101}{}{}\protect\hypertarget{lb1625905.xhtmlux5cux23start101-a}{}{}\protect\hypertarget{lb1625905.xhtmlux5cux23start101-b}{}{}\protect\hypertarget{lb1625905.xhtmlux5cux23start101-c}{}{}\protect\hypertarget{lb1625905.xhtmlux5cux23start101-d}{}{}
icke ville anses för sämre smed. Han hade undersökt hämdesvärdet och
förklarat, att det var det yppersta smide i hela världen, märkvärdigare
än hammaren Mjölner, och att han själf skulle behöfva lika lång tid som
Ivaldesonen för att göra ett svärd med lika världsgagnande egenskaper
som detta hade världsfarliga i onda händer. Efter den dom, hvarmed
gudarne kränkte Valands stolthet och frånkände Sindre hans segerpris,
hade också Mimers söner upphört att smida klenoder åt Asgardsgudarne,
ehuru fadern städse förblef gudarnes vän, emedan han var
världsordningens.

För att hämdesvärdet icke skulle komma i obehörig hand, beslöt Mimer,
att det skulle förvaras inne i själfva världsträdet. En gåta säger, att
det gömdes »i det vattenbegärliga karet med nio omspännande lås». Det
vattenbegärliga karet är asken Yggdrasil, som suger sina safter ur de
tre underjordskällorna, och karets nio omspännande lås äro de nio
årsringarna kring kärnan af trädet, som lefvat genom nio världsåldrar:
Audumlas, Ymers, Trudgelmers, Bergelmers, världsdaningens och
människoskapelsens, tidsåldern före Heimdalls ankomst till Midgard,
guldåldern, kopparåldern och järnåldern. Här syntes det farliga vapnet
vara i säkert förvar. Nattdiserna voro förvaringsrummets väktarinnor.

Mimer hade, utom sina sju äldre söner, två, som ännu endast voro i
gossåldern och ofta lekte vid stranden af den sjö, hvari Valands holme
var belägen. De voro förbjudne att besöka holmen; men deras nyfikenhet
var stor, och en gång, när de visste sig obemärkte, togo de båten och
rodde dit. Valands örnham var då nästan färdig. Gossarne stannade på
tröskeln och tittade in i smedjan. Valand tog vänligt emot dem. De
frågade, om han ville visa dem klenoderna, som förvarades i den kista,
som stod i smedjan. Han gaf dem nyckeln; de öppnade och sågo där de
vackraste saker af guld. Valands ögon lyste styggt, medan piltarne
fröjdade sig med åsynen af skatterna. Han sade till dem: »Kommen hit i
morgon bittida, när ingen ser er; då skall jag skänka er af klenoderna.
Men låt ingen veta att I varit hos mig.»
\protect\hypertarget{lb1625905.xhtmlux5cux23start102}{}{}\protect\hypertarget{lb1625905.xhtmlux5cux23start102-a}{}{}\protect\hypertarget{lb1625905.xhtmlux5cux23start102-b}{}{}\protect\hypertarget{lb1625905.xhtmlux5cux23start102-c}{}{}\protect\hypertarget{lb1625905.xhtmlux5cux23start102-d}{}{}
Följande morgon kommo de åter till smedjan; men rodde icke tillbaka.
Valand kröp till holmens strand och gaf den tomma båten en stöt, så att
den for ut i sjön. De saknade barnen efterletades, men anträffades
ingenstädes; man trodde väl, att de farit på sjön och drunknat där. Den
vise fadern kände icke deras öde. Visdomskällans mjöd gaf honom inblick
i de stora världsgåtorna, men ej i sådana ting. Ej långt därefter sände
Valand till Mimer två konstrikt arbetade dryckeskärl af silfver, till
hans drottning fyra de klaraste guld infattade ädelstenar och till
Baduhild ett hvitglänsande bröstsmycke. Silfret i dryckeskärlen var
smidt kring de små brödernas hufvudskålar; ädelstenarne voro deras ögon;
bröstsmycket var förfärdigadt af deras tänder. Deras kroppar hade Valand
gräft ned under bälgen.

Någon dag därefter rodde Baduhild hemligen öfver till holmen. Något fel
hade kommit på den dyrbara armringen hon fått af sin fader. Hon ville
icke omtala det för någon, utan bad Valand, som var ringens smed, att
bota skadan. Han lofvade göra det genast, medan hon var där. Hon satte
sig på bänken, och han räckte henne en bägare mjöd. Hon drack och
somnade. När hon vaknade, var Valand försvunnen. Han hade iklädt sig sin
örnham och flugit bort.

Valand begaf sig till sitt odalland Trymheim i Svitiod det stora. I
dalarne mällan dess ishöljda bärg väntade han att återfinna Idun och
Skade och de andra fränderna. Där fann han dem äfven. I ett af bärgen
därstädes inrättade han vidsträckta salar, som han prydde med sin
oförlikneliga smideskonst. Dörren till dem öppnade sig mot en liten
fjällsjö, som hans galdersång befriade från is. Här inne i bärget ämnade
han lefva med Idun genom århundraden. Asarne, beröfvade den läkedom mot
åren, hvilken Idun allena egde, skulle åldras, och verldstyglarne falla
ur deras slappnade händer. Det var nu den plan, som Valand uppgjort.

Från underjorden kom bud till asarne, att Valand flytt och att nya faror
förestode. Asarne rådgjorde. De sade hvarandra, att de själfva vållat
det hat Valand till dem hyste,
\protect\hypertarget{lb1625905.xhtmlux5cux23start103}{}{}\protect\hypertarget{lb1625905.xhtmlux5cux23start103-a}{}{}\protect\hypertarget{lb1625905.xhtmlux5cux23start103-b}{}{}\protect\hypertarget{lb1625905.xhtmlux5cux23start103-c}{}{}\protect\hypertarget{lb1625905.xhtmlux5cux23start103-d}{}{}
och att det vore bättre att försona sig med honom och Ivaldeslägten än
att utrota dem, ifall valet stode gudarne fritt. Det var dem framför
allt af vigt, att Idun återkomme till Asgard. Måhända kunde Valand, om
han försonades, återbringa dem äfven Fröj och Fröja. Loke åhörde dessa
rådslag. Han visste väl, att de vore till ingen nytta, ty Valand var
oförsonlig. Men han hade sina skäl att hindra, att ett försök till
underhandling gjordes. Han erbjöd sig fördenskull att återbringa Idun
till Asgard, och han höll sitt löfte. Ingen i Asgard visste hvar Valand
dolt sig; men Loke gissade på Trymheim, och som han kunde förvandla sig
till flere slags djur, såsom häst, ulf, säl, lax och insekt, dock ej
till fågel, var det honom mer tålamodspröfvande än svårt att utransaka
Valands tillhåll. Han fann den lille fjällsjön, där Ivaldesonen plägade
fiska, och dörren till hans salar. När det skett, begärde han af gudarne
att få låna Fröjas i Folkvang kvarlämnade falkham. Det fick han och flög
till Trymheim. En dag, då Valand och Skade voro på fiske, hörde de
vingslag tätt bakom sig och sågo en falk. Han flög ut från bärget, hvars
dörr stod öppen, och han bar en frukt i sina klor. Valand misstänkte
hvad som skett och det visade sig, att Idun var försvunnen ur bärget.
Skade hämtade örnhammen, och Valand iklädde sig den. Det som falken flög
bort med, var den till en frukt förvandlade växtlighetsdisen. Valand
svingade sig upp i rymden och såg röfvaren, som hade godt försprång och
var säker om att hinna fram. Loke saktade farten, för att locka Valand
att förfölja. Valand sköt en flygt, som i hast blott kunde mätas med
hans stormande känslors. Asagudarne, som väntade Loke, upptäckte från
Lidskjalf örnen, som förföljde honom. Då grep Tor sin hammare och de
andre sina kastspjut och ställde sig vid vallen. Man gjorde sig i
ordning att tända vaferlågorna. Falken slog ned innanför Asgardsmuren,
vaferlågorna tändes, gudaspjuten susade i luften, Valand kom i ohäjdlig
fart, stormade, spjutsårad, genom vaferlågornas hvirflar och föll med
brinnande vingar ned vid vallen. Örnhammen sjönk tillsammans i rök och
gnistor, och
\protect\hypertarget{lb1625905.xhtmlux5cux23start104}{}{}\protect\hypertarget{lb1625905.xhtmlux5cux23start104-a}{}{}\protect\hypertarget{lb1625905.xhtmlux5cux23start104-b}{}{}\protect\hypertarget{lb1625905.xhtmlux5cux23start104-c}{}{}\protect\hypertarget{lb1625905.xhtmlux5cux23start104-d}{}{}
gudarne sågo Valand själf. Han sökte resa sig på sina förlamade ben, för
att kämpa intill döden. Tor slungade Mjölner mot hans panna, och död låg
den väldige med krossadt hufvud.

Sindres hammare, som han skulle öfverträffa, hade blifvit hans bane.
Fimbulvinterns upphofsman var icke mer, och Idun hade återkommit till
Asgard.

\paragraph{XXIX.}

\paragraph{OD-SVIPDAG OCH FRÖJA.}

Egil och hans son Ull hade haft åtskilliga äfventyr på sin färd från
Ulfdalarne, men kommo dock oskadde fram till Svitiod det stora och
återfunno Sif, som bodde där med Egils och Groas son Od-Svipdag. Några
år förflöto, under hvilka Egil stannade i hemmet hos de sina. Han, den
raske bågskytten och stridsmannen, kände under de åren större behof af
hvila än äfventyr, och hans tankar voro icke alltid glada. Han ihågkom
den tid, då han var gudarnes edsvurne väktare vid Elivågor och Tors
förtrogne vän. Han tänkte på den skymf, som genom Halfdan blifvit
tillfogad Od-Svipdags moder, som nu sof i sin kummelgraf på bärget. Han
tänkte på den ensamme Valand, hvilken han, för att vara en trogen
broder, följt, men hvars uppsåt med världen alltid synts honom
förfärliga. De skatter, som Egil före sin flykt till Ulfdalarne hade
gräft ned i jorden eller sänkt i forsar, fingo ligga där de voro. Blott
ur det närmaste förvaringsstället upptog han några klenoder till sitt
hems och till Sifs prydande.

Under dessa år växte Od-Svipdag och Ull upp till vackra och hurtiga
ynglingar. Båda hade goda själsgåfvor; särskildt utmärkte sig Od-Svipdag
för snarfyndighet och kvickhet i tal och svar.

Sif hvälfde många tankar. Växtlighetsdisen sörjde öfver den förstörelse,
som genom Valand öfvergick världen, och hon märkte, huru Egil rufvade på
sin brutna ed och på det
\protect\hypertarget{lb1625905.xhtmlux5cux23start105}{}{}\protect\hypertarget{lb1625905.xhtmlux5cux23start105-a}{}{}\protect\hypertarget{lb1625905.xhtmlux5cux23start105-b}{}{}\protect\hypertarget{lb1625905.xhtmlux5cux23start105-c}{}{}\protect\hypertarget{lb1625905.xhtmlux5cux23start105-d}{}{}
med gudarne slitna vänskapsbandet. Sif var framsynt och sierska till
skaplynnet, och det är väl också troligt, att Urd gaf henne ingifvelser
och ledde hennes beslut.

En dag kallade Sif Od-Svipdag inför sig och sade honom, att som han nu
var fullvuxen, borde han gå och uträtta något prisvärdt. Svipdag
svarade, att han tänkt gå söderut, för att kämpa mot Halfdan, därför att
denne röfvade hans moder Groa från Egil; dock funne han det betänkligt,
emedan han lefvat sina första år under Halfdans tak och alltid var af
honom väl sedd. Och under Halfdans tak hade han en halfbroder Gudhorm,
som också var Groas son. Sif sade, att hon delade Svipdags
betänkligheter och ville råda honom till annat. Han skulle begifva sig
in i Jotunheim och därifrån återföra Fröj och Fröja.

Svipdag svarade icke nej; det blygdes han för. Men företaget syntes
honom vida öfverstiga hans krafter, och han misstänkte, att Sif, som var
hans styfmoder, icke frågade mycket efter hans väl. Han tyckte sig förut
ha märkt, att Sif gynnade sin son Ull mer än honom. Dock hade detta
aldrig rubbat vänskapen mellan Svipdag och halfbrodern, ty Ull hade
alltid underordnat sig Svipdag och sett upp till honom.

Efter nattens inbrott gick han till Groas kummelgrift på bärget. Han
hade hört, att de döda äro lättare att återkalla och lyssna mer till
jordelifvet, när natten är inne. Han ställde sig vid griftens dörr och
sade högt: »vakna, Groa! vakna, moder! Det är din son som väcker dig ur
dödens sömn. Minns du hvad du sade, att jag skulle gå till din graf, om
svåra öden hotade mig? Vakna, goda kvinna och hjälp din son!»

Groas till underjorden gångne ande hörde sonens röst och kom och lifvade
stoftet i grafkammaren. Där inifrån ljöd hennes röst: »hvilket öde har
drabbat mitt barn? Till hvilken olycka är du född, då du kallar på din
moder, som farit ur de lefvandes värld?» Svipdag svarade: »Hon, som
efter dig famnade min fader, har illslugt ålagt mig att söka ett
oupphinneligt mål.» Men rösten från grafven uppmanade
\protect\hypertarget{lb1625905.xhtmlux5cux23start106}{}{}\protect\hypertarget{lb1625905.xhtmlux5cux23start106-a}{}{}\protect\hypertarget{lb1625905.xhtmlux5cux23start106-b}{}{}\protect\hypertarget{lb1625905.xhtmlux5cux23start106-c}{}{}\protect\hypertarget{lb1625905.xhtmlux5cux23start106-d}{}{}
honom att icke tveka. »Bidar du själf en gynnsam utgång, skall nog
nornan leda dig på rätt stråt.» »Sjung då», bad Svipdag, »goda galder
öfver din son!» Och Groa sjöng, att Urd skall upprätthålla honom, då han
glädjelös vandrar sina stigar; att om underjordsälfvar svälla och hota
hans lif med sina flöden, skola de sjunka och strömma ned till Hel; att
om han råkar ut för fiender och slås i bojor, skall fiendtligt sinne
vända sig till förlikning och bojorna lossna för hans andedrägt; att om
han hotas af orkan på hafvet, skola i världskvarnen vind och böljor
endrägtigt gånga samman och gifva honom trygg färd; att om hans väg går
genom Nifelheim, han där må vandra trygg i nattligt mörker, samt om han
kommer till Mimers rike och har att samtala med den vapenrike jätten,
mannavett och vältalighet må vara honom rikligt gifna. Groa gaf Svipdag
sin välsignelse till afsked, och han gick tröstad därifrån.

Följande morgon fick han veta, att Ull ville följa honom, och att
föräldrarne tillåtit det. Egil gick med sina söner till en af sina
skattegömmor och lät dem välja åt sig hjälmar, brynjor och svärd.
Dagligen förde han dem till lekvallen, lät dem där pröfva alla idrotter,
hvari han själf var mästare, och lärde dem mången fint, hvarmed en
snarrådig och spänstig kämpe kan öfvervinna starkare motståndare. När de
hunnit den fullkomlighet han önskat, bestämdes dagen för deras affärd.
Sköldar, sådana att de kunna göras till skidor och båtar, gaf han dem
äfven.

Vid afskedsmåltiden hade Sif tillredt en visdomsrätt. I Midgard tro
många, att den rätten beredes genom kokning på ett visst slags ormar.
Hon kunde icke låta bli att ställa fatet så, att den starkare delen af
visdomsfödan kom framför Ull. Svipdag svängde på fatet och tog hvad hon
ämnat den egne sonen. Efter måltiden vinkade Sif Svipdag afsides och bad
honom icke se ned på Ull. Svipdag svarade: »Den som ser ned på en trogen
vän blir mindre än han.»

Bröderna sade hemmet farväl och vandrade norrut till Elivågor. Där sågo
de sin faders öfvergifna borg, där Tor
\protect\hypertarget{lb1625905.xhtmlux5cux23start107}{}{}\protect\hypertarget{lb1625905.xhtmlux5cux23start107-a}{}{}\protect\hypertarget{lb1625905.xhtmlux5cux23start107-b}{}{}\protect\hypertarget{lb1625905.xhtmlux5cux23start107-c}{}{}\protect\hypertarget{lb1625905.xhtmlux5cux23start107-d}{}{}
så mången gång gästat honom. De gingo öfver Elivågor och följde utefter
Jotunheimskusten. Där de funno jättegårdar, sökte de natthärbärge och
emottogos väl, emedan de voro Valands brorsöner. Bland alla jättar
gällde Valand den tiden som höfding.

Med ingen vågade de dock tala om Fröj och Fröja. Det skulle väckt
misstankar. Men en ledtråd hade de för färden. De visste, att Fröj och
Fröja vistades bland jättarne af Beles stam och de letade ut, att denna
stam, en af de styggaste i Jotunheim, bodde högt upp vid det nordliga
hafsbandet i en labyrint af förvillande töckenhöljda skär.

Färden dit upp var lång och svår. Bröderna hade att uthärda månget dygn
sådant som det då Tor fann deras fader hjälplös i Elivågor. Men Groa
hade sjungit, att Urd skall upprätthålla Svipdag på glädjelösa stigar
och föra honom på rätt stråt. Så skedde ock.

De hunno den skärgård, där Beles stam bygger. Många gånger under dessa
år hade Njord med andra gudar å Skidbladner kringirrat i dessa
farvatten, gagnlöst sökande son och dotter, kämpande med hagelstormar,
som förmörkade luften, med hafsvidunder, som klängde sig kring skeppets
köl och ville draga det i djupet, med resar, som från strandklipporna
kastade stenblock med sina slungor.

Valands brorsöner kunde här taga sig bättre fram än asagudarne.
Stormutsändarens fränder hade hitintills blifvit väl mottagne af
Jotunheimsbebyggarne. De blefvo det äfven här hos Beles stam, fast på
sitt eget vis. Jättarne däruppe hade ryktesvis hört, att bröderna voro
att vänta, och afvaktade dem med ett sändeskap på stranden. Sändeskapet,
till hvilket hörde tre bröder Grepp, som bland de sina gällde som
skalder, bar, för högtidlighetens skull, en nidstång med sig, hvarpå ett
hästhufvud hängde. Sedan gästerna hälsats, föreslog den älste Grepp
Svipdag att förkorta vägen till borgen med en täflan i nidvisor och
hemställde, om ej segraren borde taga den öfvervunnes hufvud som
segerpris. Den kvicke Svipdag gjorde jätten snart svarslös, och denne
rusade
förbittrad\protect\hypertarget{lb1625905.xhtmlux5cux23start108}{}{}\protect\hypertarget{lb1625905.xhtmlux5cux23start108-a}{}{}\protect\hypertarget{lb1625905.xhtmlux5cux23start108-b}{}{}\protect\hypertarget{lb1625905.xhtmlux5cux23start108-c}{}{}\protect\hypertarget{lb1625905.xhtmlux5cux23start108-d}{}{}
före dem till jättehöfdingens sal och skrek, att detta var gäster, som
borde mista lifvet. När Svipdag och Ull inträdde i salen -- Svipdag
främst, Ull bakom honom -- hälsades de af de innevarande välkomna med
hundskall och tjut, och mattan, hvarpå Svipdag trampade, drogs undan
hans fötter, så att han skulle fallit baklänges, om icke Ull hindrat
det. Då sade Svipdag: »Bar är broderlös rygg, äfven om den är
brynjeklädd».

Bland dessa tursar ansågs ett sådant mottagande som älskvärdt skämt,
nidvisan som det rätta skaldskapet, fräckheten som kvickhet och sveket
som bevis på utveckladt förstånd.

Det var så mycket mer öfverraskande att se höfdingen och hans syster,
som suto i högsätena. De voro båda unga, till utseendet så sköna och
till later så ädla, att intet tvifvel kunde vara om att ynglingen var
Fröj och systern Fröja. Men det var också tydligt, att båda voro under
trolldomsinflytande. Fröj såg dyster och misstänksam ut. Fröja var som
frånvarande och försjunken i drömmar. De behandlades af omgifningen som
de förnämsta på stället, och om Fröj befallde något, åtlyddes det.
Orsaken härtill var den, att Valand på detta vilkor utlämnat
gudasyskonen och hotat med värsta hämd, om icke villkoret uppfylldes. En
enda gång hade det händt, att en af jättarne uppträdde fräckt mot den
unge vanaguden. Jätten var Bele (»skällaren»), stammens egentlige
höfding. Då ryckte Fröj från väggen ett hjorthorn och gaf honom ett
dråpslag. Fröja skulle alla tursarne velat ega; men ingen dristade att
närma sig henne. Den älste af Grepparne sökte göra sig behaglig för
hennes ögon, men hon tycktes icke veta om hans närvaro. Det smeknamn,
hvarmed tursarne sins emellan benämde Fröja, var Syr (»grisen»), namnet
på ett djur, som i deras tycke var det behagligaste af alla och mönstret
af skönhet och prydlighet.

Om de dagar alfbröderna tillbragte i detta lag kan i korthet sägas, att
de önskat dem vara färre. Från morgon till afton vilda dryckeslag,
ohöfviska sånger uppstämda af skällande och tjutande röster, öfverfall,
slagsmål och dråp.
\protect\hypertarget{lb1625905.xhtmlux5cux23start109}{}{}\protect\hypertarget{lb1625905.xhtmlux5cux23start109-a}{}{}\protect\hypertarget{lb1625905.xhtmlux5cux23start109-b}{}{}\protect\hypertarget{lb1625905.xhtmlux5cux23start109-c}{}{}\protect\hypertarget{lb1625905.xhtmlux5cux23start109-d}{}{}
Jättarne hade tydligen föresatt sig, att Svipdag och Ull icke skulle
komma lefvande därifrån. Den älste Grepp, som tyckt sig finna, att den
tysta Fröjas blick ofta hvilade på Svipdag, blef svartsjuk och sökte
lönnmörda denne, men vardt i försöket nedhuggen af Ull, som sörjde mer
för broderns än sin egen säkerhet. Grepps bröder ropade på hämd; Fröj
sade då, att Grepp var fallen på sina gärningar, och att de, som ville
taga hans gästers lif, skulle göra det i ärlig strid. Nu följde utmaning
på utmaning och tvekamp på tvekamp. Svipdag var uppfinningsrik i att
välja gynnsamma vilkor för striden, och bröderna förstodo att ömsesidigt
skydda hvarandra och på samma gång gemensamt angripa. Det var som om
Egil i dem tvefaldigats, och då härtill kom deras i Ivaldesönernas
smedja hamrade vapens ofantliga öfverlägsenhet, så är det ej att undra,
att de vunno ständig seger. De trolöse jättarne kunde ej häller
öfverraska dem i sömnen, ty den ene brodern höll vapenvakt öfver den
andres slummer.

Under allt detta försummade Svipdag intet för sin plans bästa. Att få
ett hemligt samtal med Fröja var svårt och skulle tjenat till intet,
emedan hon knappt syntes fatta hvad man sade till henne. Fröj åter
undvek samtal eller afklippte dem, innan Svipdag kommit till sak. Men en
gång, när de blifvit ensamme, påminde denne om att han var son af Egil,
som med Valand varit Fröjs fosterfader, och att de två fördenskull vore
förenade med fosterbrödralagets band. Fröj anmärkte, att Valand och Egil
illa uppfyllt vårdares plikter. Svipdag sade, att han och Ull kommit att
försona detta och få den skuld häfd, som för brutna eder tyngde på
Ivaldes slägt. De hade också kommit för att frälsa världen och
människoslägtet, som skulle tillintetgöras, om icke årsväxtens gud och
fruktsamhetens gudinna återvände till Asgard. Fröj genmälde, att hans
långa fångenskap hos de uslaste af Ymers afkomma och hans oförmåga under
denna tid att hjälpa människorna, som trott på hans makt och blotat till
honom, hade höljt honom med så mycken skymf och vanära, att han fann det
bättre att stanna där han var. Svipdag frågade, om han
\protect\hypertarget{lb1625905.xhtmlux5cux23start110}{}{}\protect\hypertarget{lb1625905.xhtmlux5cux23start110-a}{}{}\protect\hypertarget{lb1625905.xhtmlux5cux23start110-b}{}{}\protect\hypertarget{lb1625905.xhtmlux5cux23start110-c}{}{}\protect\hypertarget{lb1625905.xhtmlux5cux23start110-d}{}{}
glömt, att han hade en fader, som sörjde hans förlust, och oaflåtligt,
under svåra faror, efterletade honom. Då brast Fröj i gråt, men
upprepade, att han icke skulle följa Svipdag. Men han ville gifva
bröderna ett tillfälle att fly med Fröja, och han skulle, om möjligt,
fördröja och missleda förföljelse.

Härmed måste alfbröderna åtnöja sig. En natt då jättarne hållit ett
vildt dryckeslag och voro omtöcknade till sina sinnen eller försjunkna i
tung sömn, lämnade bröderna jättegården och hade Fröja med sig. De
styrde kosan öfver skärgårdens upprörda böljor till kustlandet. Där
utbredde sig vida skarsnöbetäckta slätter, öfver hvilka deras skidor
struko fram nästan med pilens fart, och längre fram mötte djupa skogar,
som i nödfall kunde gömma mot förföljelse. De stannade icke, förrän de
hunnit in i en skog. Här beslöto de hvila och tillreda åt Fröja ett
läger så godt det läte göra sig. Medan Ull flätade ett skydd af
barrträdets kvistar och gjorde därunder en bädd af mossa, satt Svipdag
bredvid vanadisen. Månen stod öfver trädtopparne och sken på hennes
anlete. Svipdag såg betagen in i det, och han talade glada, uppmuntrande
ord. Men Fröja stirrade på marken; hon lyfte icke en enda gång sin
blick. Han bad henne göra det och sade, att hans kärlek ej hoppades mer
än det enda ögonkast, hvarom han bad. Men ögonlockens fransar förblefvo
orörliga. Han förde henne då till hennes bädd, och bröderna bjödo henne
god natt.

När de om morgonen sökte henne där, var hon försvunnen, och de
efterletade henne länge och förgäfves. De ropade, men fingo intet svar.
Flere dagar förledo med ängsligt sökande. Det var den tid på året, då
nätterna äro som längst. Så kom aftonen af årets kortaste dag. Pinglor
hördes i skogen, en jättegård måtte finnas i grannskapet, och bröderna
gingo, för att söka härbärge, åt det håll, hvarifrån pinglandet ljöd. De
kommo till ett glad (öppen plats i skogen), begränsadt på en sida af ett
lodrätt bärg. Mot en dörr i detta gick en skara getter med klingande
skällor, och bakom dem två kvinnor, den ena en jättinna, den andra klädd
som getpiga.
\protect\hypertarget{lb1625905.xhtmlux5cux23start111}{}{}\protect\hypertarget{lb1625905.xhtmlux5cux23start111-a}{}{}\protect\hypertarget{lb1625905.xhtmlux5cux23start111-b}{}{}\protect\hypertarget{lb1625905.xhtmlux5cux23start111-c}{}{}\protect\hypertarget{lb1625905.xhtmlux5cux23start111-d}{}{}
När hjorden drifvits in, öppnade jättinnan en annan dörr, och
vinterkvällsolens röda sken föll in därigenom och belyste hennes sal.
Bröderna gingo fram och bådo om natthärbärge. Hon mottog dem vänligt,
när hon hört, att de voro Valands brorsöner. »Han har nu länge varit den
bäste af alla jättar», sade hon. Bröderna sågo, att getpigan var Fröja.
Jättinnan hade funnit henne i skogen och gifvit henne tjänst. Svipdag
och Ull låtsade icke känna henne, och hon syntes ej häller känna dem.
Jättinnan var litet misstänksam i den punkten, men lugnade sig snart.
Vid aftonmåltiden betraktade hon sina gäster, och de syntes henne vara
vackra ynglingar. Mäst tyckte hon om Svipdag, som vid bordet var munter
och infallsrik, och det föreföll henne, att hon ej kunde önska bättre än
att han stannade och blefve hennes man. Hon berömde sin gård, sina
hjordar och egodelar och sporde, om det icke vore tröttsamt i längden
att drifva omkring i världen på äfventyr. Svipdag genmälde, att han
redan, så ung han var, fått nog däraf och gärna ville bli bofast man med
hustru och barn; och eftersom hans slägt hade brutit både med gudar och
människor, före han nu med sin broder de villsamma stigarne i Jotunheim,
för att där leta sig hustru. Dock väntade han sig föga lycka på
giljareväg därstädes, ty han hade icke en jättes växt och art. Jättinnan
svarade, att en liten karl vore också en karl, och ville Svipdag ha en
kortare hustru, så förstod hon konsten att göra sig så kort som han
behagade. Svipdag sade, att hans ärende i Jotunheim vore lyckligt
uträttadt, om han funne nåd för hennes ögon. Hon lät veta, att det hade
han funnit, och att det var henne en heder att befryndas med den ärorika
och fräjdade Ivaldeslägten. Dryckesskålarne bräddades med jättegårdens
öl; det glammades lustigt under kvällen -- endast getpigan teg --, och
det vardt öfverenskommet, att brölloppet skulle hållas, så snart som
brudens närmaste slägt hunnit underrättas och infinna sig till fästen.
Hon skulle följande morgon begifva sig ut och bjuda dem. Vigseln skulle
ske i all enkelhet, och endast de aderton närmaste jättefamiljerna
inbjudas.

\protect\hypertarget{lb1625905.xhtmlux5cux23start112}{}{}\protect\hypertarget{lb1625905.xhtmlux5cux23start112-a}{}{}\protect\hypertarget{lb1625905.xhtmlux5cux23start112-b}{}{}\protect\hypertarget{lb1625905.xhtmlux5cux23start112-c}{}{}\protect\hypertarget{lb1625905.xhtmlux5cux23start112-d}{}{}

Långt innan följande morgons sol uppgått, hade gårdens egarinna färdats
ut i bröllopsbjudnings-ärende. Svipdag och Ull hade knappt sett henne
fara in i skogen, förrän de påspände sina skidor och med Fröja fortsatte
sin flykt. Mot aftonen hade de hunnit ned till kusten och satte öfver
Elivågor. Då kom bakom dem en rasande orkan, som öfver störtande tallar
bröt sig väg genom skogen ned till stranden och öfver vattnet, så att
Elivågors böljor reste sig skyhöga. Bröderna gissade, att det var bruden
och de aderton inbjudna jättefamiljerna, som beredde dem denna
afskedshälsning. Galdern, som Groa sjungit, lugnade hafvet. Svipdag och
Ull anlände lyckligt med Fröja till Egils och Sifs hem.

Svipdag var likväl mer sorgsen än glad. Han omtalade för Sif, att han
icke fått en enda blick, icke ett enda ord af henne, som han frälsat ur
jättevåld. Sif, som såg, att han var förälskad, bad honom icke tala till
Fröja ett ord om kärlek. Vore det så, att vanadisen gömde en känsla för
honom, skulle Sif nog upptäcka det. Annars måste han försöka glömma
henne; men i hvarje fall vore det hans plikt och hans ära att sända
henne ren och oskyldig tillbaka till Asgard.

Valands dotter Skade vistades vid denna tid hos Egil. Sif tillställde
ett låtsadt bröllop mällan Svipdag och henne. Fröja kläddes till
brudtärna. Fram mot natten följde hon Svipdag och Skade till
brudgemaket. Där stod hon tyst med fälld och tårad blick och såg icke,
att facklans låga närmades hennes hand, och kände icke dess sveda.
Svipdag tog facklan och sade, att brölloppet var ett skämt. Då klarnade
hennes anlete, och hon såg på Svipdag. Den kloka Sif anordnade då genast
en värklig bröllopsfäst mällan Fröja och honom. Men när hvilans stund
kom, lade Svipdag, såsom Sif tillsagt honom, ett bart svärd mällan sig
och bruden. Sif tog svaneham och flög mot söder, och när hon återkom,
visste man i Asgard, att Fröja var genom Svipdag frälsad ur jättevåld.
Hon hämtades till den glänsande gudaborgen däruppe, och stor var
asafaderns och alla gudars och gudinnors glädje öfver hennes återkomst.

\protect\hypertarget{lb1625905.xhtmlux5cux23start113}{}{}\protect\hypertarget{lb1625905.xhtmlux5cux23start113-a}{}{}\protect\hypertarget{lb1625905.xhtmlux5cux23start113-b}{}{}\protect\hypertarget{lb1625905.xhtmlux5cux23start113-c}{}{}\protect\hypertarget{lb1625905.xhtmlux5cux23start113-d}{}{}

Sif och Skade togo åter svaneham och flögo mot norr. De kände nu genom
Svipdag, hvar borgen låg, där Fröj var på samma gång höfding och fånge.
De flögo öfver de svarta holmarne i Bele-stammens skärgård och sågo, att
jättarne där voro stadda på flyttning. Sedan Fröja blifvit bortförd och
deras tillhåll upptäckt, kände de sig icke säkra; de hade satt båtar i
sjön och voro på väg med Fröj till än nordligare ängder. Svanemöarna
flögo vidare och späjade. De funno Skidbladner, bemannadt med vaner,
ligga förtöjdt i en vik. På en klippa där bredvid satt Njord sorgsen och
grubblande. Trolldom och töcken gäckade städse hans letningar. En
gyllene ring föll nu i hans knä; han skådade upp och såg två svanar
glänsa i rymden mot de mörka drifvande molnen. På ringen såg han runor
ristade, som sade honom, hvar Fröj vore att finna. Skidbladner, städse
gynnadt af medvind, styrdes dit; de flyttande jättarne öfverrumplades.
Njord nedlade med egen hand det vidunder, som under färden vaktade Fröj,
och han återvände med sin son till Asgard.

Idun, Fröja och Fröj voro återvunna åt gudarne. Växtlighetsdiserna
visade sig åter som deras vänner. Valand, den förfärlige, var dräpt och
fimbulvinterns makt därmed bruten.

\paragraph{XXX.}

\paragraph{HALFDANS TÅG MOT NORDEN. EGILS DÖD.}

Nu kommo öfver Midgard år som voro löftesrika för lif och blomstring.
Isfälten, som betäckte den stora nordiska ön, smälte under solens varma
strålar. Så tyst det förut varit på de ofantliga jökelslätterna, så
lifligt var där nu, ty oräkneliga rännilar sorlade där och förenade sig
till bäckar, som gräfde sig allt djupare bäddar och med samlad kraft
vordo till åar och älfvar, hvilka mällan grönskimrande isväggar brusade
hän till hafvet. Mer och mer drogo sig isfälten tillbaka. För hvarje år
visade sig grönskan något tidigare i dalarne och räckte något längre.
Vintern började närma sig
\protect\hypertarget{lb1625905.xhtmlux5cux23start114}{}{}\protect\hypertarget{lb1625905.xhtmlux5cux23start114-a}{}{}\protect\hypertarget{lb1625905.xhtmlux5cux23start114-b}{}{}\protect\hypertarget{lb1625905.xhtmlux5cux23start114-c}{}{}\protect\hypertarget{lb1625905.xhtmlux5cux23start114-d}{}{}
lagstadgade gränser och nöja sig med de månskiften, som ursprungligen
blifvit tilldelade den. Trädslag, som under fimbulvintern blifvit
förkrympta till buskar eller till refvor, som kröpo under snön utefter
marken, började resa sig med stam och krona, och de många, som dött ut,
fingo efterträdare genom frön, som vind och bölja förde öfver till
norden.

Denna förvandling gick först sakta, men vardt underbart påskyndad genom
en tilldragelse i människovärlden. Folkstammarne söder om hafvet i det
stora rike Halfdan behärrskade, sammankallades af honom till tings, och
han frågade dem, om de vore af hans tanke, att det heliga land, som hade
i sitt sköte fädernas grafvar, borde återvinnas och nybyggas. Folkets
samlade stridsmän gåfvo sitt ja med rop och vapenklang. Alla de
utvandrade stammarne uppställde manstarka krigareskaror, anförda af män,
som tillhörde deras yppersta ätter. Budlungar, hildingar och lofdungar
sällade sig med sina stridsmannaföljen under Halfdans fanor. Dessa ätter
och flere af dem, som äro fräjdade i Midgard, voro befryndade med
Sköld-Borgars och härstammade från de män och kvinnor, som voro ypperst
i Aurvangalandet, när Heimdall kom dit med sädeskärfven. En flotta
utrustades, och hären landsteg i nämda land. Svearne och de med dem
sammanboende alferna, som byggde där, ville icke ansluta sig till
härfärden och draga norrut. De bördiga fält de nu en tid innehaft, dem
ville de behålla. De vägrade utbyta dem mot sitt gamla land omkring den
örika sjö, hvars vatten blandar sig med saltsjöns, ty där uppe härrskade
ju ännu vintermakterna. Svearnes höfdingar samlade sina stridskrafter,
för att tillbakavisa Halfdans. Främst bland desse höfdingar voro de af
skilfingarnes (ynglingarnes) ätt, som, äfven den, var befryndad med
Sköld-Borgars; och på svearnes sida, likasom på den andra, stredo
hildingar och budlungar. Kampen vardt hård, och vikande för öfvermakten,
måste svearne öfvergifva Aurvangalandet och draga nordligare. Med
förvåning märkte de, att deras återtåg följdes af vår och blomster och
flyttfågelskaror. Isfälten veko allt efter som de tågade mot norr.
\protect\hypertarget{lb1625905.xhtmlux5cux23start115}{}{}\protect\hypertarget{lb1625905.xhtmlux5cux23start115-a}{}{}\protect\hypertarget{lb1625905.xhtmlux5cux23start115-b}{}{}\protect\hypertarget{lb1625905.xhtmlux5cux23start115-c}{}{}\protect\hypertarget{lb1625905.xhtmlux5cux23start115-d}{}{}
Framför dem i fjärran drogo moln, som sände ljungeldar ned mot de
smältande isjöklarne. Det var Tor, som åskade där och gjorde det
obehagligt för därvarande invandrare från Jotunheim att kvarstanna i det
längsta. Man såg jättarne lämna sina nybyggen och med svarta
boskapshjordar tåga bort öfver snöfälten, och allt efter som dessa
smälte, ryckte svearne efter, och bakom svearne Halfdans härskaror, som
trängde dem framför sig. På Moins hedar gjorde svearne halt och sökte än
en gång häjda sina fränder från södern. Det kom till ett blodigt
fältslag, hvari svearnes bragder väckte Halfdans beundran; men de
kämpade ej blott mot människor, utan mot gudarne och ödet och måste
vika. Öfver Halfdans fylkingar redo i luften valkyrior, rustade med
hjälmar, brynjor och gyllene spjut. Från deras hästars manar och betsel
droppade äringsgifvande dagg öfver fält, som i åratal ej burit annat än
frostblommor.

Bland svearnes kämpar var hildingen Hildeger en af de ypperste. Han var
Halfdans halfbroder och son af Drott, som varit gift med en sveahöfding,
hildingarnes stamfader, innan hon blef Sköld-Borgars maka. Halfdan hade
på svearnes sida iakttagit en ståtlig kämpe, som stred bland de främste
och åstadkom stort manfall. Denne var Hildeger. När härarne en gång
stodo uppställde mot hvarandra, gick Halfdan fram och utmanade honom
utan att veta, att det var hans halfbroder. Hildeger hade för
brodersbandets skull länge undvikit att sammanträffa med Halfdan i
drabbningarna. Nu efter den offentliga utmaningen kunde han icke undvika
det längre, såvida han icke ville uppenbara, att han var Halfdans
broder. Men det ville han icke för Halfdans skull; ty, tänkte han,
omtalar jag, att jag är hans broder och Halfdan likväl påyrkar tvekamp,
så gör han sig skyldig till nidingsdåd; men afstår han från tvekampen,
kunna onda tungor få ett skenskäl att säga, att han en gång varit rädd.
Fördenskull gick nu Hildeger fram ur svearnes led. Tvekampen mällan dem
fördes med svärd, och hafva våra fäder sagt, att Valands uppfinning,
svärdet, åtminstone i dess första tider, sällan drogs
\protect\hypertarget{lb1625905.xhtmlux5cux23start116}{}{}\protect\hypertarget{lb1625905.xhtmlux5cux23start116-a}{}{}\protect\hypertarget{lb1625905.xhtmlux5cux23start116-b}{}{}\protect\hypertarget{lb1625905.xhtmlux5cux23start116-c}{}{}\protect\hypertarget{lb1625905.xhtmlux5cux23start116-d}{}{}
utan att vålla frändedråp. Hildeger föll, efter skiftade hugg, dödligt
sårad. Då lät han Halfdan veta hvem han var. »Mig födde Drott i Svitiod;
dig bland danerna. Förlåt, att jag icke sade dig, hvem jag är. Vi ha
hvilat vid samma modersbarm. Låt nu brodern varda svept i sin broders
mantel!» Halfdan grät och lade sin mantel kring hans lik.

Svearne hade slutligen måst rygga så långt upp som till Svarins hög,
samma ställe, där vid fimbulvinterns början deras höfdingar och alferna
hade samlat sig och beslutat den utvandring, till följd af hvilken
Sköld-Borgar med sitt folk och så många andra drog söderut och hans son
Halfdan grundlade sitt stora rike på andra sidan hafvet. Det fanns för
svearne intet skäl mer att fortsätta kriget. De vackra ängder de förut
bebyggt kring den örika och fjärdrika sjön, stodo åter gröna med
speglande skogar utefter stränderna, och med andakt hade de återsett
sina fäders kummelgrifter och ättehögar. Här var deras odalland, och här
bjöd dem Halfdan att stanna. Det ämnade de också göra, men ansågo det
lända dem till mindre heder att emottaga fred från hans hand än nödga
honom med vapen att vika tillbaka. Egil och Svipdag kommo norrifrån,
slöto sig till dem och eggade dem att tillbakavisa fredsanbudet. Egils
hopp var att vid Svarins hög nedlägga Groas röfvare. Till Halfdan och
hans här kom bud, att Egil med de alltid träffande pilarne anländt till
fiendens läger. Detta var ingen god nyhet, och ansåg sig Halfdan själf
kårad att dö för ett skott från hans båge, om Egil uppträdde på
valplatsen. Kvällen före slaget kommo Halfdans späjare och sade, att de
visste, hvar Egil tagit natthärbärge. Följd af Hamal och andra valda
kämpar, bland hvilka hildingen Hildebrand, smög Halfdan mällan svearnes
utposter till det hus, där Egil och Svipdag suto i dryckesgille med
sveakämpar. Huset stormades så oförmodadt och häftigt, att Egil ej
hunnit lägga pil till bågsträng, innan han föll under Halfdans klubba.
Svipdag och andra där inne räddade sig med flykten.

Följande dag stod slaget vid Svarins hög. Här föllo
\protect\hypertarget{lb1625905.xhtmlux5cux23start117}{}{}\protect\hypertarget{lb1625905.xhtmlux5cux23start117-a}{}{}\protect\hypertarget{lb1625905.xhtmlux5cux23start117-b}{}{}\protect\hypertarget{lb1625905.xhtmlux5cux23start117-c}{}{}\protect\hypertarget{lb1625905.xhtmlux5cux23start117-d}{}{}
många å ömse sidor, och bland de fallne voro alfkämpar, som en gång
hängde med Tjalve vid Tors starkhetsbälte, då han vadade genom Elivågor.
Slaget ändade så, att svearne ändtligen bekvämde sig att emottaga fred.
Äfven de hyllade nu Halfdan som konung öfver alla de folk, som talade
urfolkets språk och hade fått Heimdalls runor (germanfolken). Och enär
årsväxten följt hans segertåg och fimbulvintern vikit för honom, gåfvo
de, såsom andra folk, Halfdan, sonen af åskans gud, gudomlig ära efter
hans död.

Bland de fångar Halfdan gjorde vid Svarins hög var hans styfson Svipdag,
Egils och Groas son. Svipdag hade i striden kämpat hjältemodigt och sökt
hämna sin faders död. Halfdan talade vänligt till honom, bjöd honom att
vara hans son och emottaga ett konungarike under honom. Svipdag svarade,
att han icke läte muta sig af sin faders baneman. Han sade, att dödade
icke Halfdan honom nu, skulle han döda Halfdan längre fram. Halfdan lät
då binda honom vid ett träd i skogen och öfverlämnade honom åt hans öde.

Sedan Halfdan afslutat detta lyckosamma fälttåg mot Norden, förrättade
han åt gudarne ett stort offer och bad om en styrelsetid, lång som hans
faders, öfver de riken han ägde. Det svar han fick från gudarne innebar,
att hans regering och lifstid skulle varda mycket kortare än
Sköld-Borgars, men att i hans ätt skulle under tre hundra år ingen
kvinna och ingen obetydande man födas.

Medan Halfdan framträngde mot Svarins hög, for Tjalve, ofta i Tors
sällskap, mällan öarna i det nordiska hafvet, ränsade dem från de troll
och jättar, som där under fimbulvintern tagit bostad, gjorde dem
beboeliga för människor och ditförde nybyggare. På Lässö var det nära
att han blifvit dödad af rasande jättinnor, men Tor kom i god tid och
räddade honom. Med ön Gotland var det före Tjalves ditkomst så stäldt,
att hon vid soluppgången sjönk i hafvet, men vid solnedgången dök upp
igen. Tjalve bar gnideld kring ön och gjorde henne därmed fast. De
nybyggare, som följde honom dit, voro af den gotiska stammen. Goter var
det
\protect\hypertarget{lb1625905.xhtmlux5cux23start118}{}{}\protect\hypertarget{lb1625905.xhtmlux5cux23start118-a}{}{}\protect\hypertarget{lb1625905.xhtmlux5cux23start118-b}{}{}\protect\hypertarget{lb1625905.xhtmlux5cux23start118-c}{}{}\protect\hypertarget{lb1625905.xhtmlux5cux23start118-d}{}{}
också, som bosatte sig i Götaland, söder om svearne, och på den halfö,
som sedan kallats Jutland. Hela detta rike fick namnet Reidgotaland. På
de bördiga öarna utanför Aurvangaland byggde danerna och söder om dem
anglerna, saksarne och många andra befryndade folk.

\paragraph{XXXI.}

\paragraph{SVIPDAG HÄMTAR HÄMDESVÄRDET. HALFDANS DÖD.}

De band, med hvilka Halfdan låtit binda Svipdag, höllo icke mot den
galdersång, som Groa hade sjungit öfver sin son. Svipdag andades på dem,
och de föllo bort.

Men friheten, som han därmed vann, var honom föga kär, och han tyckte,
att lifvet hade för honom ringa värde, sedan hans fader fallit och han
själf öfvervunnits af dennes oöfvervinnelige baneman. Det ålåg Svipdag
att hämna sin faders död. Men huru vore det honom möjligt att utkräfva
hämden på den väldige Halfdan, som var Tors son och stod under hans
beskydd?

Tyngd af dessa tankar gick Svipdag nattetid på månlyst stig emot sitt
hem. Han såg till månen, som stod vid synranden, och månguden talade
till honom och sade: »Jag känner dina tankar. Du tilltror dig ej att
kunna hämna din fader. Det kan du ej häller af egen kraft. Men dig, som
räddat Fröja ur Jotunheims våld, höfves det dig att misströsta?»

»Hvad skall jag då göra?»

»Hämta din farbroder Valands oemotståndliga svärd. Det ligger i
underjorden i »det vattenbegärliga karet inom nio ombindande lås»
(världsträdet). Det vaktas af den mörka Sinmara (»senskärarinnan»,
Mimers drottning, nattdisernas moder).

»Du talar i gåtor», sade Svipdag, »men jag förstår dina ord. Kan jag
hoppas, att Sinmara utlämnar svärdet?»

»Hon utlämnar det åt ingen, som ej medför och lägger i hennes hand den
blanka skära, hvarmed hon kan afskära
\protect\hypertarget{lb1625905.xhtmlux5cux23start119}{}{}\protect\hypertarget{lb1625905.xhtmlux5cux23start119-a}{}{}\protect\hypertarget{lb1625905.xhtmlux5cux23start119-b}{}{}\protect\hypertarget{lb1625905.xhtmlux5cux23start119-c}{}{}\protect\hypertarget{lb1625905.xhtmlux5cux23start119-d}{}{}
en af Urd tvinnad tråd. Skärans egg biter, när Urd tillåter det.»

»Hur skall jag finna den skäran?»

»Bland hanen Vidofners
nystkotor»{\protect\hyperlink{lb1625905.xhtmlux5cux23n2}{2}}. Just som
månguden sagt detta, blänkte det till i luften, och en silfverhvit skära
nedföll vid Svipdags fötter. Svipdag stack den i sitt bälte och tackade
månguden.

Månguden sade vidare:

»Nedgången till underjorden finner du i bärgen norrut från Valands
smedja i Ulfdalarne. Tämj okhjortar (renar), som ditföra dig snabbt
genom den mordiska kölden! Nedgången har en väktare. Akta dig, att han
icke får se din skugga, innan du sett hans!»

Efter dessa ord gick månen ned bakom bärgen.

{\_\_\_\_\_\_\_\_}

Någon tid därefter åkte Svipdag efter ett spann af okhjortar ned i
Ulfdalarne förbi Valands öfvergifna smedja och norr upp i bärgen. Där
fann han mällan klipporna hålvägen, som går ned till underjorden.
Tältslädan, hvari han åkt, och som nu skulle vara hans boning, ställde
han invid hålvägens öppning, men så att hvarken den eller dess skugga
kunde ses därinifrån. Dagarne tillbragte Svipdag med att jaga, sköta
sina renar och hvila sig. Om nätterna höll han oaflåtlig vakt. Så kom en
månljus natt, då han såg en skugga falla från hålvägsöppningen och röra
sig framåt på snön. Han gissade, att det var en af Mimers väktare, som,
klädd i det slags kappa, hvilken gör sin bärare osynlig för vanliga
ögon, hade gått ut ur hålvägen, för att se sig omkring. Kappan kan dölja
bäraren, men icke hans skugga. Svipdag kastade sitt spjut mot den punkt
på snön, där han dömde, att väktarens ena fot måste vara, och spjutet
fastnade också i den osynliges häl.
\protect\hypertarget{lb1625905.xhtmlux5cux23start120}{}{}\protect\hypertarget{lb1625905.xhtmlux5cux23start120-a}{}{}\protect\hypertarget{lb1625905.xhtmlux5cux23start120-b}{}{}\protect\hypertarget{lb1625905.xhtmlux5cux23start120-c}{}{}\protect\hypertarget{lb1625905.xhtmlux5cux23start120-d}{}{}
Då sprang Svipdag fram och grep honom, afryckte kappan och såg honom nu
för sina ögon. De brottades. Svipdag segrade och aftvang den öfvervunne
en ed, att han finge ostörd gå ned i underjorden. Väktaren svor eden och
sade, att komme han blott oskadd genom Nifelheim, så vore faran ingen;
ty Mimer och hans söner skola visserligen icke förgripa sig på en gäst,
äfven om han kommer objuden. Väktaren gaf honom sin kappa till skydd
under vandringen i Nifelheim och sade, att den som trotsar farorna där,
för att komma till Mimers land, han måste ha ett vigtigt ärende dit.

Svipdag kom ned genom hålgången och tillryggalade stigar genom
Nifelheims stinkande träsk. Hvad där är att se skall längre fram
förtäljas. Det fordras en modig hug för att uthärda de synerna. Groas
galder, genom hvilka Urd själf talat, värnade färdemannen och afvände
farorna, när han mötte de rimturs-spöken och de hemska sjukdomsandar,
som kringirra bland Nifelheimsträsken. Svipdag klättrade öfver
Nidafjället, såg med förvåning den ofantliga världskvarnen, dånet af
hvars gång han hört på långt afstånd, såg den brusande Hvergelmerkällans
vidsträckta rund och steg ned i Mimers grönskande rike.

Dunkel hvilar öfver mycket, som han där rönte. Visst är, att Mimer och
hans fränder mottogo honom gästvänligt och läto honom se de många undren
i hans ängder. Själfva Breidablik, där Balder bor med Nanna och den
kommande världsålderns stamföräldrar, fick han se, ehuru icke beträda.
Med Mimer hade han samtal, hvari han till dennes fägnad ådagalade stor
skarpsinnighet, och det är möjligt, att Mimer med uppsåt icke utfrågade
af honom hans ärende. Visdomskällans väktare visste, att Svipdag leddes
på sina vägar af Urd, och han visste väl än mer därom. Men till honom
vände sig Svipdag icke med sitt ärende.

Den unge alfen fick tillåtelse att se Mimers drottnings och nattdisernas
underbara boning. Där vid drottningens högsäte satt Baduhild, och en
pilt lekte vid hennes knän. »Är han din son?» frågade Svipdag Baduhild.
»Ja», svarade
\protect\hypertarget{lb1625905.xhtmlux5cux23start121}{}{}\protect\hypertarget{lb1625905.xhtmlux5cux23start121-a}{}{}\protect\hypertarget{lb1625905.xhtmlux5cux23start121-b}{}{}\protect\hypertarget{lb1625905.xhtmlux5cux23start121-c}{}{}\protect\hypertarget{lb1625905.xhtmlux5cux23start121-d}{}{}
hon. »Gossen har jag kallat Vidga. Du bör veta hans namn, ty du och han
äro syskonbarn. Din fader var Egil; denne gosses fader var Valand.»

»För mig är detta en stor nyhet», sade Svipdag. Han lyfte upp gossen,
kysste honom, bar honom till salens andre ände och tillsade honom att
leka där. Han återvände till högsätet och kvinnorna och sade till
Baduhild: »Jag har icke hört, att du var gift med Valand.»

»Det var jag ej häller. Mot min vilja och mitt vetande vardt Valand mitt
barns fader.»

»Då hatar du hans minne?»

»Nej, han var ju så olycklig. Alla kalla honom ond. Jag vet likväl, att
godhet låg gömd i hans hjärta. Han talade med min fader, innan hans
örnham bar honom härifrån. Han drog täckelset bort från sin grymma hämd
och sade min fader allt, men först sedan han med ed bundit honom att
icke vredgas på sin dotter eller det barn hon skulle föda. Sådan var
Valand.»

»Väl mig», sade Svipdag, »att du födt Valand en son! En börda har därmed
fallit, som ville trycka mig till jorden.»

»Hvad menar du?» frågade Baduhild och hennes moder, Mimers maka.

»Jag vill nu säga er mitt ärende», fortfor Svipdag. »Jag kom hit för att
få Valands hämdesvärd till mig utlämnadt. Jag visste icke, att Valand
hade en son; jag trodde mig vara hans närmaste arfving -- arfving till
hans svärd och skyldig att kräfva blodshämd på Asgards gudar för hans
död. Nu hör jag, att det finnes en närmare arfving till svärdet och till
blodshämdens plikt. Vidga skall bära det till kamp mot Asgardsmakterna.»

Mimers drottning sade: »Här ger du ord åt tankar, som kommit mig att
rysa i sömnlösa nätter. Skall en gudafiende uppväxa vid gudavännen
Mimers härd invid Yggdrasils vördnadsbjudande stam, vid randen af den
heliga visdomskällan? Skall Vidga växa upp för att fullgöra ett ogörligt
värk, dö förbannad under gudarnes vapen och försändas till de plågor,
\protect\hypertarget{lb1625905.xhtmlux5cux23start122}{}{}\protect\hypertarget{lb1625905.xhtmlux5cux23start122-a}{}{}\protect\hypertarget{lb1625905.xhtmlux5cux23start122-b}{}{}\protect\hypertarget{lb1625905.xhtmlux5cux23start122-c}{}{}\protect\hypertarget{lb1625905.xhtmlux5cux23start122-d}{}{}
som äro åt de förbannade bestämda? Förfärliga tankar! Egde jag
silfverskäran, hvarmed jag kunde afskära den bloddränkta tråd, som Urd
tvinnat åt Baduhilds son! Men skäran kommer aldrig i den hand, som Urd
ej därtill kårat, och hennes domar äro orubbliga, ty de äro ej hennes,
utan en okänd makts.»

Baduhild sänkte sitt hufvud och grät.

»Skäran, hvarom du talar, är här», sade Svipdag och tog den ur sin
gördel. Och i det han förde den skinande lille lien genom luften, sågos
två blodröda trådar. »Här är den blodshämdens tråd, som Urd tvinnat åt
mig; där är den hon tvinnat åt Vidga. Jag kan genomskära min egen, men
icke båda, ty endera af oss, Vidga eller jag, måste söka tillfredsställa
blodshämdens kraf och falla i försöket under gudarnes vrede. Genomskär
jag min, så är för Vidga intet hopp. Lämnar jag skäran till dig, och du
genomskär hans, så är för mig hoppet ute. Hvilketdera skall jag göra?»

Mimers maka sade: »Jag kan icke begära skäran ur din hand. Ditt offer
vore för stort. Vidga har att bära sitt öde.»

Svipdag sade: »Bär hit hämdesvärdet och fäst det i min gördel! Jag
öfvertager Valands arf och arfvingens plikter. Silfverskäran skall varda
din. Evig blygd öfver den man, som själf kan bära en förbannelse, men
låter den falla på ett barns hufvud!»

Mimers drottning gick, kom tillbaka med hämdesvärdet och fäste det vid
Svipdags gördel. Han gaf henne silfverskäran, och hon afskar den för
Vidga tvinnade blodshämdstråden.

Så fick Svipdag Valands hämdesvärd, och med det återkom han till
Midgard. Här uppsökte han de jättehöfdingar, som ännu bodde med sina
stammar i Svitiod det stora, och sade dem, att om de icke rusta sig till
motvärn, är deras tid snart ute. Halfdan, understödd af asagudarne,
skall komma med sina härskaror och drifva dem bort öfver Elivågor till
deras urland Jotunheim. Äfven till Jotunheim begaf sig Svipdag, till den
onde, mycket fruktade jotunfursten Gymer, och kallade därvarande
jättestammar till strid. Han lofvade dem seger och sade sig ega det
svärd, som deras förre
öfverhöfding\protect\hypertarget{lb1625905.xhtmlux5cux23start123}{}{}\protect\hypertarget{lb1625905.xhtmlux5cux23start123-a}{}{}\protect\hypertarget{lb1625905.xhtmlux5cux23start123-b}{}{}\protect\hypertarget{lb1625905.xhtmlux5cux23start123-c}{}{}\protect\hypertarget{lb1625905.xhtmlux5cux23start123-d}{}{},
hans farbroder Valand smidt, men aldrig hann begagna, ett svärd, som
segern helt visst skulle följa.

En af Gymer uppbådad talrik jättehär samlades och ryckte med Svipdag som
anförare mot söder. Faran var stor -- större än asagudar och människor
förutsågo, ehuru gudarne dock aktade den vara så allvarlig, att de måste
stiga ned och leda Midgards fylkingar. Oden, Tyr, Vidar, Njord, Fröj,
Heimdall och andre hjältegudar kommo och visade sig i sin vapenskrud och
till häst för de dödliges ögon. Tor kom med sin järnhammare och ställde
sig ej långt från Halfdan. Härarne ryckte under sköldsång mot hvarandra.
Gymers och hans kämpars sång liknade lösgjorda stormars vilda tjut;
Midgardskämparnes liknade bruset af hafvets bränningar. Oden red i
spetsen med Gungner i sin hand å sin åttafotade häst. Fylkingarna
drabbade samman, och jättarne, framför dem alla Gymer, kämpade med vildt
mod. Oden sprängde med Gungner genom deras slagordning och tryckte många
Jotunheimssöner under Gungners udd och Sleipners hofvar. Tyr och Vidar
bröto väg med svärdet. Störst var manfallet där Tor for fram med
hammaren. Men jättefylkingarne slöto sig ständigt på nytt och splittrade
på sina ställen Halfdans slagordning. Där Svipdag svängde hämdesvärdet,
föllo icke färre Midgardskämpar än det föll jättar för Tors hammare.
Svipdag sökte sig fram mot Halfdan, som slog med blodig klubba, men
ditintills undvikit att sammandrabba med Svipdag, hvars styffader han ju
var. Men då tvekampen ej mer kunde underlåtas, red han fram mot
styfsonen. Klubban och hämdesvärdet möttes; Halfdans älsklingsvapen,
pröfvadt i många strider, sprang sönder som träffadt af blixten. Och
likt ljungelden flammade Valands svärd. För hvarje hugg, hvartill det
lyftes, glänste ett sken öfver stridsvimlet och hela Midgard. Än ett
hugg och Halfdans brynja var klufven och han själf sårad. Då kastade Tor
sin hammare mot Svipdag. Han mötte den i flykten med hämdevapnet.
Sindres bästa smide sammanstötte med Valands bästa. Nu stod den
egentliga och afgörande domen öfver dem, och den lyktade så, att hammare
\protect\hypertarget{lb1625905.xhtmlux5cux23start124}{}{}\protect\hypertarget{lb1625905.xhtmlux5cux23start124-a}{}{}\protect\hypertarget{lb1625905.xhtmlux5cux23start124-b}{}{}\protect\hypertarget{lb1625905.xhtmlux5cux23start124-c}{}{}\protect\hypertarget{lb1625905.xhtmlux5cux23start124-d}{}{}
och hammarskaft splittrades och kom i obrukbara stycken tillbaka i
asagudens hand, men Valandssvärdets egg hade icke fått en skåra. Tor var
vapenlös och måste rygga. Han tog den sårade Halfdan på sin skuldra och
drog sig undan striden upp emot ett bärg, hvarifrån han och den blödande
hjälten kastade stenblock ned mot fienden. Svipdag och hans jättehär
drogo sig då tillbaka. Deras seger var ju redan vunnen. Oden och de
andre gudar, som deltagit i striden, kvarstannade på slagfältet och hade
åtminstone äran att vara de siste där. Mot asamajestätet skydde Svipdag
att gå till tvekamp; angripa Njord eller Fröj ville han icke. Den ene
var ju Fröjas fader, den andre hennes broder.

Svipdag fick snart veta, att Halfdan dött af det honom tillfogade såret.
Mot det hjälpte icke läkerunor och galdersång. Hatets gift låg i dess
egg, Nifelheimsälfvarnas etter i dess klinga. Svipdag hade således
hämnat sin faders död och sin moder Groas skymf.

Valands död var visserligen icke hämnad. För att detta skulle ske, måste
Svipdag bekriga Asgard och nedlägga alla i hans död delaktige gudar. Han
rufvade öfver denne tanke, förkastade den och tog upp den igän. Stundom
sade han sig, att detta var en åtagen plikt; stundom invände han, att
han redan gjort mer än hämnat Valands död: han hade återställt Valands
kränkta ära och bragt gudarnes dom öfver Ivaldesonens konstvärk på skam.
Och kunde Valand begära större upprättelse än den, att Tor, hans
dråpare, måst rygga för hans svärd?

Svipdag hemförlofvade sin jättehär, som återvände segerstolt till sina
gårdar. Den hade lofvat att samla sig på nytt, så snart budkafvel eller
vårdkas gåfve bud därom.

I Asgard rådde förfäran. Det var tydligt, att den som egde hämdesvärdet
kunde göra sig till världens herre. Asgard bäfvade de dagarne på udden
af Svipdags svärd. Dock -- han hade återskänkt Fröja till gudarne och
trolofvat sig med henne. Kunde icke hans kärlek besegra hans
maktlystnad? I detta fall skulle han varda välkommen till Asgard och
erkännas\protect\hypertarget{lb1625905.xhtmlux5cux23start125}{}{}\protect\hypertarget{lb1625905.xhtmlux5cux23start125-a}{}{}\protect\hypertarget{lb1625905.xhtmlux5cux23start125-b}{}{}\protect\hypertarget{lb1625905.xhtmlux5cux23start125-c}{}{}\protect\hypertarget{lb1625905.xhtmlux5cux23start125-d}{}{}
af gudarne som Fröjas make. Han var hvarken af asa- eller vanablod. Men
en försoning med Ivaldes slägt och alferna var för gudarne önskvärd, och
kunde den köpas med Svipdags upptagande i asarnes och vanernas slägt och
med hämdesvärdets upptagande bland Asgardsklenoderna, så beredde det
köpet vinning åt Oden och åt världen.

\paragraph{XXXII.}

\paragraph{SVIPDAG KOMMER TILL ASGÅRD. HANS FÄRD TILL BALDER.}

Det var vid den tid på dygnet, då dvärgen, som står utanför
morgonrodnadsalfen Dellings dörrar, sjunger väckelsesången öfver världen
och välsignelse öfver alfer, asar och allfader. Det var vid den stund,
då Yggdrasil droppar honungsdagg, och solhästarne, vädrande
morgonluften, längta till fimmelstång och tyglar.

Det var vid den tid på året, då i Midgards lundar knopparne svälla och
markernas mattor ha den friska grönska, som visar, att de nyss blifvit
väfda af växtlighetsdiserna; då det blåa lufthafvet är så rent och
genomskinligt, att människans längtan stiger högre däri än fågeln bäres
af sina vingar. Det var den tid, som väcker kärleksträngtan i all
naturen.

Bifrosts väktare, Heimdall, såg en yngling i vapenskrud, med sollysande
svärd i bältet, gå uppför den bro, som ingen beträder utan i kraft af
Urds beslut. Hans annalkande bebådades för asarne, och det vardt glädje
i gudaborgarne, när budet sade, att Svipdag kom, helt visst i godt
ärende, ty han liknade själf en glad vårdag.

Asgards fallbrygga var nedslagen öfver älfven, som flöt med ilande fart,
men klar och glittrande kring den ofantlige vallmuren. När Svipdag
hunnit upp för Bifrost, såg han Glasers guldlund och gudarnes vida
lekslätter och Valaskjalfs silfvertak och Lidskjalfs torn glänsa i
morgonbelysning. Han steg upp på den skans, som ligger midt för
fallbryggan. Därifrån är utsigt öfver den konstrike porten till
gudaborgarne
\protect\hypertarget{lb1625905.xhtmlux5cux23start126}{}{}\protect\hypertarget{lb1625905.xhtmlux5cux23start126-a}{}{}\protect\hypertarget{lb1625905.xhtmlux5cux23start126-b}{}{}\protect\hypertarget{lb1625905.xhtmlux5cux23start126-c}{}{}\protect\hypertarget{lb1625905.xhtmlux5cux23start126-d}{}{}
bakom vallmuren. Framför porten stod en väktare; strax innanför dess
galler låg Odens ene ulfhund och sof; den andre var vaken och upphof ett
skall. Midt emot Svipdag reste sig en guldsirad hall högt bakom en
lustgård, hvars träd skuggade en blomsterhöljd kulle. På kullen sutto
diser. En var den förnämsta; de andra syntes i en krets kring hennes
knän. Svipdag igenkände Fröja: hon liknade en bildstod, orörlig och som
försjunken i drömmar. De diser, som sutto henne närmast, voro Glans,
Vän, Frid, Blid, Hjelp och Eir läkedoms-asynjan.

Med häpnad såg Svipdag världsträdet, som i Midgard är osynligt, utbreda
öfver Asgard och hela himmelen sin krona med luftiga bladvärk, i hvilka
frukter skimrade. Högt uppe i grenarne glänste den gyllene hanen
Vidofner-Gullenkamme.

Mannen vid porten gick fram till Svipdag. Han låtsade icke veta hvem
denne var och emottog honom på det sätt, som bland väktare är brukligt,
när de finna en främling på förbjuden mark. »Hvem söker du här? Ingen
kringstrykare kommer inom denna port. Drag du hädan på fuktiga vägar!»

»Drag du själf din kos med dina ovänliga ord! Hvem, tror du, vänder sig
bort från det, som tjusar hans blick? Här ser jag gårdplanerna gifva
återsken af gyllne salar. Här vill jag stanna och njuta sällhet.»

»Hvad heter du?» sporde väktaren.

»Vindkall. Min fader hette Vårkall, min farfader Hårdkall. Hundarne där
bakom portgallret synas mig vara bistrare väktare än du. Men nog torde
man kunna komma dem förbi.»

»Omöjligt. Den ene vakar, då den andre sofver, och trogen vakttjänst
skola de bestrida, så länge världen står. Olycklig är den objudne, som
råkar ut för dem. Och bakom dem finnas älfva väktare inne i borgen.»

»Du menar, att i världen är intet, som kan afvända dessa hundars
vaksamhet och fresta deras matlust, så att de glömma sin tjänst?»

»Intet utom de läckra stekarne under Vidofners vingar.
\protect\hypertarget{lb1625905.xhtmlux5cux23start127}{}{}\protect\hypertarget{lb1625905.xhtmlux5cux23start127-a}{}{}\protect\hypertarget{lb1625905.xhtmlux5cux23start127-b}{}{}\protect\hypertarget{lb1625905.xhtmlux5cux23start127-c}{}{}\protect\hypertarget{lb1625905.xhtmlux5cux23start127-d}{}{}
Kan du kasta dem till Gere och Freke, så glömma de sin tjänst, och du
kan smyga dem förbi.»

»Hvad heter den guldglänsande hanen högt där i trädet?»

»Det är just Vidofner.»

»Finns det ett vapen, hvarmed jag kan få honom ned? Annars finge jag ej
stekarne och komme ju icke här in.»

»Det finns ett sådant vapen. Det smiddes af en ond smed längst borta i
Jotunheim. Men vapnet vardt honom fråntaget.»

»Hvart kom det sedan?»

»Till den mörka disen Sinmara, som gömde det i det vattenbegärliga karet
med nio bandlås.»

»Antag nu, att någon begifvit sig till Sinmara, för att få det vapnet!
Tror du då, att han kommer tillbaka med det?»

»På ett vilkor är det möjligt. Sinmara tynges af en sorg. Hon rufvar på
en enda tanke. Vidofner också känner den tanken tung. Mällan hans runda
kotor ligger en skinande skära. Den som får den skäran och bär den till
Sinmara, han erhåller af henne det dolda vapnet. Och den, som fått det
vapnet, han kommer in genom denna port.»

»Hvem är den sköna mö, som, omgifven af diser, sitter, försjunken i
tankar, på den vackra kullen?»

»Hon heter Smyckeglad. Sorgsna ungmör och lidande kvinnor få tröst, om
de vandra upp till henne. Äfven åt hennes diser äro altaren resta; de
bringa hjälp och hugnad åt bedjande.»

»Vet du, om någon är bestämd att hvila på Smyckeglads hvita arm?»

»En är af ödet därtill kårad. Hans namn är Svipdag.»

»Öppna då porten, ty Svipdag ser du här! Men vänta! In vill jag icke,
förrän jag vet, att jag har den älskligas hjärta.»

Just som Svipdag sade sitt namn, sprang Asgardsporten, den underbare,
upp af sig själf. Båda hundarne voro nu vakna, gingo fram till Svipdag,
hälsade honom med muntra språng och slickade hans hand. Väktaren gick
och sade Fröja, att en främling kommit, som helt visst är Svipdag. Då
steg hon upp och skyndade ut till ynglingen. »Säg ditt
\protect\hypertarget{lb1625905.xhtmlux5cux23start128}{}{}\protect\hypertarget{lb1625905.xhtmlux5cux23start128-a}{}{}\protect\hypertarget{lb1625905.xhtmlux5cux23start128-b}{}{}\protect\hypertarget{lb1625905.xhtmlux5cux23start128-c}{}{}\protect\hypertarget{lb1625905.xhtmlux5cux23start128-d}{}{}
namn och din ätt», sade hon; »mina ögon vilja ha dina läppars
vittnesbörd!»

»Jag heter Svipdag. Min faders namn är en »solglänsandes» namn
(Egil-Örvandels solglänsande stjärnas). Jag vräktes från hans hem af
vindarne ut på kalla vägar. Men Urds dom, hur än den faller, jäfvas af
ingen makt.»

Fröja hann knappt säga välkommen, innan hon kysste honom.

»Min hälsning flög dig till mötes, men hanns upp af min kyss. Hur länge
jag satt där på kullen, dag efter dag, och längtade till dig! Nu,
älskade, ser jag dig åter och har dig i mina salar. Där skola vi lefva
evigt tillsammans.»

{\_\_\_\_\_\_\_\_}

Svipdag och Fröja firade sitt laggilla bröllopp i Asgard. I brudköp och
som bot för Valands svikna fosterfadersplikt skänkte Svipdag hans svärd
till Fröj.

Svipdag önskade, att Ull, hans halfbroder, måtte få dela hans ära och
upptagas i Asgard, likasom han förut delat hans faror i Jotunheim. Denna
önskan villfors af Oden så mycket hällre som Tor tillkännagaf, att Sif
och han beslutat att vara makar. Sif, den guldlockiga, kom till Asgard
och medförde Ull och Valands dotter Skade. Tor och Sif firade sitt
bröllopp kort efter Svipdags och Fröjas.

Ett tredje och fjärde bröllopp firades därefter. Idun vardt gift med
Brage. Njord hade före Valands flykt till Ulfdalarne friat till Skade.
Denna kraftiga och stolta ungmö kom vapenklädd till Asgard och kräfde
böter för slagen fader. Hon fick i böter asynjans värdighet och rang
bredvid Frigg och Fröja. Tor visade henne två glänsande stjärnor på
himmelen och berättade, att strax efter det hennes fader Valand stupat,
samlade sig gudarne kring den fallne och beklagade hans död och prisade
hans forna gärningar, då han var gudarnes vän och smyckegifvare. Och då
hade Tor med sin faders bifall tagit Valands ögon och kastat dem till
himmelen, för att de skulle varda stjärnor. Detta bevisade för Skade,
\protect\hypertarget{lb1625905.xhtmlux5cux23start129}{}{}\protect\hypertarget{lb1625905.xhtmlux5cux23start129-a}{}{}\protect\hypertarget{lb1625905.xhtmlux5cux23start129-b}{}{}\protect\hypertarget{lb1625905.xhtmlux5cux23start129-c}{}{}\protect\hypertarget{lb1625905.xhtmlux5cux23start129-d}{}{}
att Valands minne var i Asgard äradt, och då Njord förnyade sitt
friareanbud till henne, vardt hon rikedomsgudens maka.

Sålunda var nu försoning stiftad mällan asar och vaner å ena sidan och
Ivaldeättlingarne, alfernas yppersta slägt, å den andra. Försoningen var
bekräftad med ett fyrfaldigt äktenskapsband.

Om Slagfinn, Valands och Egils broder, och de öden han rönte skall i en
annan saga än denna förtäljas.

{\_\_\_\_\_\_\_\_}

Under alla de fäster, som vid denna tid firades i Asgard, bar Frigg på
en sorg, som vardt tyngre genom glädjen hon hade gemensam med de andra.
Hon saknade Balder, sin älsklingsson.

Sa kom på henne den tanke, att när ödet visade sig så gynnsamt mot
Asgard som nu, skulle det måhända kunna bevekas under något vilkor att
skänka Balder tillbaka till hans tomma salar.

Svipdag åtog sig att bära hennes önskan till nornorna och framföra en
hälsning från gudarne till Balder och Nanna i Breidablik, i fall det
vore möjligt att träffa dem. Men Delling, morgonrodnadsalfen,
Breidabliks väktare, har nyckeln till dess port och sätter den ej i
låset, förrän Balder och Nanna skola med Leiftraser och Lif återvända i
världsförnyelsen. Breidabliks mur är gjord att vara oöfverstiglig. Det
gällde fördenskull att se, om ej Sleipner kunde hoppa öfver den, likasom
öfver den mycket höge Asgardsmuren. Gudarne renade och helgade Svipdag,
för att intet ondt skulle med honom komma in i Breidablik, och Oden gaf
honom Sleipner att rida på färden.

Svipdag sprängde bort, följd af allas önskningar om lycklig utgång. Den
väg han tog var icke den, som han förra gången tillryggalade, när han
steg ned i underjorden. Det var icke den, som från näjden af Ulfdalarne
leder ned till Nifelheim och genom detta land öfver Nidafjället till
Mimers rike. Nu red han nedför Bifrost och genom en östlig port -- den
för de döde bestämde -- in i underjorden. Detta var närmaste
\protect\hypertarget{lb1625905.xhtmlux5cux23start130}{}{}\protect\hypertarget{lb1625905.xhtmlux5cux23start130-a}{}{}\protect\hypertarget{lb1625905.xhtmlux5cux23start130-b}{}{}\protect\hypertarget{lb1625905.xhtmlux5cux23start130-c}{}{}\protect\hypertarget{lb1625905.xhtmlux5cux23start130-d}{}{}
vägen till Mimers lund. Efter att hafva porten bakom sig, färdades
Svipdag nio dygn genom djupa dalar och kom till en guldbelagd bro. Där
satt Madgun, en af Urds diser, som brovaktare. Dånet af Sleipners hofvar
mot broläggningen sade tydligt, att ryttaren och hästen voro lefvande
väsen, icke skuggor. Därom vittnade också deras utseende. Madgun bjöd
honom stanna och redogöra för namn, ätt och ärende. Sedan Svipdag det
gjort, sade hon, att han kunde fortsätta resan och finge förblifva ett
dygn i underjorden. Ville han uppsöka Breidablik, låg vägen åt norr; men
till Urds källa låg vägen åt söder. Svipdag red åt norr, där Mimers
lunds kronor tecknade sig mot synranden. Så kom han till den höga mur,
som omgifver Breidablik. Då steg han af hästen, spände bukgjorden
fastare, satte sig åter upp och sporrade Sleipner, som i ett väldigt
hopp kom öfver muren. Bland de blomsterhöljda träden därinne såg han en
sal, hvars dörr stod öppen. Väggarna voro beklädda med dyrbara bonader,
bänkarna beströdda med smycken från Mimers skattkammare. I högsätet
sutto Balder och Nanna. De bjödo Svipdag välkommen. Heligt mjöd fanns i
ädelstensgnistrande dryckesskålar, och sedan färdemannen förfriskat sig,
framförde han hälsningar och ärende och omtalade allt, som var värdt att
höra. Under märkliga samtal försvunno timmarne till inpå natten. Om
morgonen bjöd han farväl. Balder gaf honom ringen Draupner att återlämna
till Oden; Nanna sände en slöja och några andra gåfvor till Frigg och en
guldfingerring till Friggs syster Fulla. Dessa gåfvor voro vartecken,
att ingen fimbulvinter mer skall hemsöka jorden förr än kort före
världsförstörelsen. Därefter red Svipdag till Urd. När gudarne begära
något af henne, sker det vanligen genom sändebud. Vördnadsfull steg
Svipdag inför Urd och hennes systrar, som sutto i runotecknade stolar
under susande löfvärk vid sin källas silfverklara vatten. Han framförde
gudarnes bön. Nornorna rådgjorde. Därefter sade Urd: Balder får med
Nanna återvända till Asgard, om ingen varelse är, som icke begråtit
eller vill begråta, att Balder dött och ej skall återkomma.

\protect\hypertarget{lb1625905.xhtmlux5cux23start131}{}{}\protect\hypertarget{lb1625905.xhtmlux5cux23start131-a}{}{}\protect\hypertarget{lb1625905.xhtmlux5cux23start131-b}{}{}\protect\hypertarget{lb1625905.xhtmlux5cux23start131-c}{}{}\protect\hypertarget{lb1625905.xhtmlux5cux23start131-d}{}{}

Med denna utsaga red Svipdag tillbaka. Han tänkte öfver Urds dom och
tyckte väl stundom, att den var hård och gaf föga hopp, stundom att man
likväl hade skäl att hoppas. Ty hvem kunde vara så förhärdad, att han
önskade, att det goda skall ej blott i denna världsålder hafva att kämpa
med det onda, utan att det alltid skall vara så, att det alltid skall
råda öfvervåld, svek, lögn och nöd? Vid guldbron hälsade han Madgun och
kom utan äfventyr ur underjorden. Han hade icke ridit långt därifrån, då
han i en bärghåla såg en kvinna af styggt utseende. Han frågade hennes
namn, och hon svarade Töck. »Jag kommer från Urds domaresäte», sade
Svipdag, »och har till världen ett vigtigt bud att framföra. Har du
begråtit eller vill du begråta, att Balder dött och aldrig återkommer?»
Töck svarade: »Töck begråter med torra tårar Balders bålfärd. Lefvande
eller död är mig Odens son till ingen gamman. Behålle dödsriket hvad det
hafver!»

Hvem Töck var är ovisst. Säkert lär dock vara, att hon antingen var
Gullveig eller Loke. Svipdag hade intet trösteligt svar att medföra till
Asgard, men likväl kära hälsningar och minnen från Balder och Nanna.

\paragraph{XXXIII.}

\paragraph{FRÖJ FRIAR TILL GERD.}

Fröj satt en dag i Lidskjalf och skådade ut öfver Midgard och Jotunheim.
Då såg han i jättehöfdingen Gymers gård en mö, medan hon gick från salen
till frustugan. Det spred sig ett sken öfver himmel och haf, och skenet
tycktes komma från hennes hvita armar. Den unge vanaguden såg åter och
åter i sina drömmar denna mö. Hennes fader Gymer var den af gudarne mäst
hatade jätten i Jotunheim. Ett rykte gick, att den pånyttfödda Gullveig
blifvit hans hustru och att det var med henne han födt sin undersköna
dotter, densamma som Fröj från Lidskjalf sett. Hennes namn var Gerd.
Fröj
\protect\hypertarget{lb1625905.xhtmlux5cux23start132}{}{}\protect\hypertarget{lb1625905.xhtmlux5cux23start132-a}{}{}\protect\hypertarget{lb1625905.xhtmlux5cux23start132-b}{}{}\protect\hypertarget{lb1625905.xhtmlux5cux23start132-c}{}{}\protect\hypertarget{lb1625905.xhtmlux5cux23start132-d}{}{}
blygdes öfver den lidelse, som gripit honom till dottern af ett sådant
par. Han förteg och bekämpade den. Men striden var gagnlös. Han var
älskogssjuk och trånade bort. Njord och Skade bådo honom säga hvad det
var som grämde honom, men fingo intet veta. Då vände de sig till
Svipdag, som hade Fröjs förtroende, och bådo honom utforska orsaken till
Fröjs dystra lynne. Svipdag sade, att han ville försöka, ehuru han
väntade afvisande ord. Han gick till Fröj och sade:

»Hvarför, min drott, sitter du ensam alla dagar i dina vida salar?»

Fröj svarade: »Hvarför skulle jag förtälja min tunga sorg? Alfrödul
(solen) strålar hvarje dag, men ej på mina önskningar.»

»Så stora kunna dina önskningar ej vara, att du ej vågade förtro dem
till mig. Hafva vi ej gemensamma ungdomsminnen? Och vittna de ej, att du
kan lita på mig?»

Då sade Fröj: »Jag är kär i en mö, och aldrig var en flicka mer älskad
af en yngling. Men hon, som vunnit hela min hug, bor i den hatade Gymers
gårdar. Där såg jag henne gå, och hennes armar spredo ett sken genom
rymden och öfver hafvet. Aldrig skall asagud eller alf tåla, att Gymers
dotter och jag mötas.»

»Din sorg kan botas», sade Svipdag. »Låna mig Valandssvärdet, skaffa mig
Sleipner och gif mig att medföra dyra klenoder till jättemön! Då rider
jag till Gymers gårdar och friar å dina vägnar.»

Gudarne, som voro mycket bekymrade öfver Fröjs tvinsot, kände sig föga
hugnade, när de erforo dess orsak. Gerds fader Gymer var icke endast en
våldsam, för Midgard farlig fiende, som i Svipdags krig med gudarne och
Halfdan visat sin kraft och sitt mod; han var också mer än de fleste
trollkunnig och förrädisk. Gudarne torde icke utan skäl hållit före, att
Fröjs trånsjuka var honom af Gymer eller hans hustru -- om denna var
Gullveig -- påhäxad, och de väntade intet godt för årsväxten i Midgard
af en förbindelse mällan skördeguden och den höfdingeslägt i Jotunheim,
som
\protect\hypertarget{lb1625905.xhtmlux5cux23start133}{}{}\protect\hypertarget{lb1625905.xhtmlux5cux23start133-a}{}{}\protect\hypertarget{lb1625905.xhtmlux5cux23start133-b}{}{}\protect\hypertarget{lb1625905.xhtmlux5cux23start133-c}{}{}\protect\hypertarget{lb1625905.xhtmlux5cux23start133-d}{}{}
efter Geirrauds och hans slägts utrotande var frostens och stormarnes
egentlige härskare. Men här fanns intet val: Fröj skulle täras bort och
dö, om gudarne ej bragte de offer, som kräfdes för en lycklig utgång af
Svipdags ärende. De skickade med Svipdag älfva guldäpplen, ett från hvar
asagud, och ringen Draupner i friaregåfvor. Dock borde Draupner, som
hvilat på Balders bröst, icke öfverlämnas till jättemön utan i nödfall.

Det var skymning, då Svipdag satte sig till häst. Till Sleipner sade han
och klappade hans manke: »Mörkt är det ute; nu ha vi att rida öfver
frostiga fjäll och tursafolkets bygder; antingen komma vi båda tillbaka,
eller tager oss den förfärlige jätten.»

I morgongryningen var Svipdag framme. Han hade beräknat att komma, medan
Gymer och hans salkämpar ännu sofvo. Väktaren på utkiksklippan invid
gårdvallens stängda grind var dock vaken och ropade ryttaren an. Svipdag
såg från Sleipners sadel in öfver gårdvallen och fann, att å ömse sidor
frustugans dörr var en ulfhund af styggt utseende bunden.

»Du, som kommer där högt till häst, hvad är ditt ärende?» ropade Gymers
väktare.

»Jag har ett ärende till Gerd, din herres dotter. Huru slipper man in
för de glupske hundarne?»

»Antingen är du en död, som spökar, eller ock är du nu till död kårad,
du som djärfves att vilja tala till Gerd. Till henne får du icke ett ord
att mäla.»

»Det gäller att fresta», sade Svipdag. »Den som är på äfventyr väljer
dålig lott, om han brys af de faror han uppsökt. Min lefnads trådar
tvinnades det dygn, då jag såg lifvet.» Därmed sprängde han öfver
omgärdningen in på den gräsvuxna gården framför frustugan. Det
jordskakande dånet af Sleipners hofvar, ledsagadt af ulfhundarnes ilskna
skall, väckte Gerd och hennes tjenstemö. Denna såg ut och sade: »En
yngling har kommit; han har redan stigit ur sadeln och släppt sin
gångare att beta på gårdsplanen.» »Bjud honom in att dricka en bägare
mjöd», sade Gerd; »hans ärende vill jag
\protect\hypertarget{lb1625905.xhtmlux5cux23start134}{}{}\protect\hypertarget{lb1625905.xhtmlux5cux23start134-a}{}{}\protect\hypertarget{lb1625905.xhtmlux5cux23start134-b}{}{}\protect\hypertarget{lb1625905.xhtmlux5cux23start134-c}{}{}\protect\hypertarget{lb1625905.xhtmlux5cux23start134-d}{}{}
veta, ehuru jag anar, att han är min broders bane.» Svipdag steg in.
»Hvem är du», frågade Gerd, »alf eller asason eller en af de vise
vaner?» Svipdag sade sig komma från Asgard, frambar friareärendet från
Fröj, vanaguden, och bjöd henne de älfva guldäpplena.

Gerd sköt guldäpplena ifrån sig. »Dem tager jag icke i friaregåfva.
Aldrig i detta lifvet skall jag bo samman med Fröj.»

»Här gifver jag dig ringen, som följde Odens unge son å bålet. Åtta
jämntunga ringar drypa ur honom hvar nionde natt.»

»I friaregåfva tager jag ej häller den. Jag saknar icke guld i Gymers
gårdar. Min fader har däraf nog.»

Då drog Svipdag Valandssvärdet. »Ungmö, ser du detta spänstiga
bildristade svärd? Jag gick till Mimers lund och till det saftsvällande
trädet (Yggdrasil), för att hämta hämdetenen, och hämdetenen fick jag.
Med dess egg skall ditt hufvud skiljas från sin hals, ger du mig ej ditt
ja att föra till Fröj!»

»Hot gör mig ej till en mans hustru. Min fader är icke långt borta; han
är stridslysten som du. Hård varder striden, som mällan er skall
stånda.»

»Ungmö, ser du detta spänstiga bildristade svärd? Din fader skall segna
till jorden under dess egg. Tvinga skall jag dig, mö, efter min vilja.
Jag flyttar dig med ett slag af hämdetenen ned om dödsportarne till den
ort, där mannasöner aldrig få se dig. Du dör under gudarnes hämnande
vrede. Du sändes till rimtursarnes, din ätts fäders, rysliga värld, till
sjukdomsandarnes rike. Där väntar dig ett uselt bo mällan jättespökens
gårdar, och ser du ut från dess grind, skola tigande flerhöfdade
vidunders hemska ögon möta dig. Där varder dig din föda förhatligare än
giftormen är hatad af människan, och din dryck den äckligaste.
Sjukdomsandarne skola varda gäster i din boning och ditt följe, när du
krälar i träskvattnet mällan dina likars gårdar. Tramar (onda vättar)
skola kröka dig ned i gyttjan. Tope (»Vanvett»), Ope (»Skakande Gråt»),
Otåle (»Rastlös Oro») skola aldrig lämna dig i ro. En make skall du få,
men aldrig en makes kärlek. Hustru skall du varda åt den trehöfdade
Trudgelmer, åt honom,
\protect\hypertarget{lb1625905.xhtmlux5cux23start135}{}{}\protect\hypertarget{lb1625905.xhtmlux5cux23start135-a}{}{}\protect\hypertarget{lb1625905.xhtmlux5cux23start135-b}{}{}\protect\hypertarget{lb1625905.xhtmlux5cux23start135-c}{}{}\protect\hypertarget{lb1625905.xhtmlux5cux23start135-d}{}{}
hvars fötter, såsom Ymers, föda barn. Hören mig, I alle Nifelheims
bebyggare, hören mig, spöken af tursar och jättar, hören mig, Suttungs
söner! Ja, hören mig äfven, I sälle, som bon i Mimers rike, nu då jag
besvärjer och bortbannar all kärlek och glädje från denna mö! Drifven af
osläckt trånad skall du nattetid kräla upp till det bärg, där Are (»Örn»
i Nifelheim) har sin tufva, för att sitta där uppe i tidig morgonstund
och genom Nifelheimstöcknen stirra hän till sällhetsrikets klara rymder,
som du försakat för evigt. Vred är dig Oden, vred är dig Tor, vred vare
dig Fröj, du afskyvärda mö! Gråt får du till gamman, och du skall nära
din sorg med tårar.»

Gerd hörde förfärad dessa hotelser. Hon sade: »Välkommen hälsade jag dig
icke, när du steg in i salen; men jag gör det nu. Af mitt bästa mjöd
bjöd jag dig icke; men nu räcker jag dig en bägare af det tidsbepröfvade
och säger: »Hel dig!» ehuru jag hittills icke trodde, att jag någonsin
skulle egna vaner och vaners fränder en välönskan.»

Svipdag sade: »Jag nöjer mig icke med blida ord. Afgjordt skall mitt
ärende vara, innan jag lämnar dig. När vill du unna Njords son ett
möte?»

Gerd svarade: »Barre heter den tysta lunden. Vi känna den båda. Njords
son må där möta mig, sedan nio nätter förlidit.»

Svipdag red till Asgard. Fröj stod ute och inväntade honom, och hans
svåger fick icke taga sadeln af Sleipner, innan han omtalat ärendets
utgång. Fröj syntes nio dygns väntan lång. »Oftare förekom mig i min
längtan månaden kortare än halfva natten.»

Var Gymers och hans salkämpars sömn så tung, att de icke vaknade, när
marken skalf under Sleipners hofvar och ulfhundarne upphäfde sitt tjut?
Nej, Gymer såg Svipdag komma och hade länge väntat en böneman från Fröj.
Han och Gullveig hade uppgjort sins emellan, huru Valandssvärdet, som
gjorde Asgard oangripligt, skulle komma i jättevåld. Gerd var den
skönaste mön i Jotunheim; men hennes hvita armar
\protect\hypertarget{lb1625905.xhtmlux5cux23start136}{}{}\protect\hypertarget{lb1625905.xhtmlux5cux23start136-a}{}{}\protect\hypertarget{lb1625905.xhtmlux5cux23start136-b}{}{}\protect\hypertarget{lb1625905.xhtmlux5cux23start136-c}{}{}\protect\hypertarget{lb1625905.xhtmlux5cux23start136-d}{}{}
hade icke utan Gullveigs trollkonst kastat ett sken öfver haf och
himmel.

Efter de nio dygnens förlopp möttes Fröj och Gerd i Barres lund. Hon
lofvade att varda hans, men på dessa, af föräldrarna fastställda vilkor:
Valandssvärdet skulle i brudköp öfverlämnas till Gymer; Svipdag och
Fröja å utsatt dag infinna sig hos honom och å Fröjs vägnar högtidligen
anhålla om Gerds hand; Gerd upptagas i Asgard och hafva asynjas
värdighet.

Den bedårade Fröj hade lofvat mer, om Gerd fordrat det. Valandssvärdet
lade han genast i hennes hand.

För gudarne var detta en vida större förlust än det var för Jotunheim en
vinst. Valand hade icke velat, att andra gudafiender än Ivaldes
ättlingar skulle hafva omedelbart gagn af svärdet, och han hade vidtagit
sina varsamhetsmått därefter. På hennes klinga hade han med stor konst
inristat en skildring af den tilldragelse i urtiden, då rimtursarne
drunknade i Ymers blod. Det har fördenskull blifvit sagdt om
Valandssvärdet, att »det kämpar af sig själf mot jätteätten», och det
var på dessa bilder Svipdag pekade, när han visade klingan för Gerd. De
åskådliggöra en egenskap, som Valand inhamrade i klingans gry. Drages
svärdet af en jätte, så fäller han visserligen sin motståndare, men han
omkommer också själf och med honom Jotunheims hela makt. Denna svärdets
egenskap var för jättarne bekant, och de voro lika rädda för att nyttja
det som de varit ifriga att få det bort från Asgard, där det var en
underpant på gudarnes trygghet. När fördenskull Gerd öfverlämnade
svärdet till sin fader, var för denne ingenting mer trängande än att väl
förvara svärdet och gifva det en pålitlig väktare, som var hans frände
och kallas Eggter (»Svärdvakt»).

Svipdag misstänkte, att Gymer rufvade på en förrädisk plan, och han
vidtog sina åtgärder därefter. Han kom på den aftalade dagen till Gymer
och medförde Fröja. Men redan före dem hade Tor och Ull brutit upp till
Jotunheim och på lönliga vägar kommit till grannskapet af Gymers
fjällgård och
\protect\hypertarget{lb1625905.xhtmlux5cux23start137}{}{}\protect\hypertarget{lb1625905.xhtmlux5cux23start137-a}{}{}\protect\hypertarget{lb1625905.xhtmlux5cux23start137-b}{}{}\protect\hypertarget{lb1625905.xhtmlux5cux23start137-c}{}{}\protect\hypertarget{lb1625905.xhtmlux5cux23start137-d}{}{}
gömt sig där. Gymer emottog sändeskapet från Asgard väl och hedrade det
med ett gille. Följande morgon skulle de återvända och medföra Gerd till
Asgard. Då gillet var i full gång, tog Gymer Svipdag afsides och sade:

»Hör nu på något märkvärdigt! Jag har sett på dina ögon, att du fägnas
af Gerds åsyn, likasom mina fägnas af Fröjas. Det är mitt beslut, att
du, icke Fröj, skall vara min svärson. Är det så, att du har lust till
Gerd, så afstår du Fröja åt mig i bot för min son, som du råkat att
dräpa. Då öppna sig andra utsigter för dig än att gå däruppe i Asgard
som ett slags tjenare åt Oden och Fröj. Valandssvärdet är nu i min ego.
Du var dum, när du skänkte bort det. Med det i hand kunde du kastat
Asgard öfver ände, tagit Fröja med våld och gjort dig själf till gud.
Därmed hade du också hämnat din farbroder Valand, hvilket du nu
försummat. Denna dumhet har jag godtgjort. Du kan få Valandssvärdet
åter. Du är rätte mannen att föra det, och det skall än en gång ske i
spetsen för Jotunheims härar. Kom i håg, att du står i förbindelse till
mig för den seger du vann och som öppnade dig Asgards port! Du vann den
icke ensam; jag och mitt folk hade vår andel däri, och det är din plikt
att handla som vår bundsförvandt.»

Svipdag medgaf riktigheten af Gymers ord och sade sig ingå på hans
förslag. »Det är väl», sade Gymer, »ty annars hade du icke med lifvet
kommit från min gård. Vi förvandla nu detta gille till ett
dubbelbröllop, mällan mig och Fröja samt dig och Gerd.

Svipdag ingick äfven härpå, men frågade, om icke Gymer redan hade
hustru. »Jo visst», sade han, »men ingen hindrar mig att ha så många
hustrur jag vill. Har du icke hört ryktesvis, att jag är gift med
Gullveig?» -- »Jag ser henne icke här», sade Svipdag; »hvar finns hon?»
-- »Jo, det är en lustig sak», sade Gymer; »Fröja har låtit narra sig än
en gång af henne. Hon har nu en tid varit i Folkvang som Fröjas
tjenarinna. Du känner ju Aurboda? Det är hon.» Svipdag vardt högligen
förvånad öfver denna upptäckt.

\protect\hypertarget{lb1625905.xhtmlux5cux23start138}{}{}\protect\hypertarget{lb1625905.xhtmlux5cux23start138-a}{}{}\protect\hypertarget{lb1625905.xhtmlux5cux23start138-b}{}{}\protect\hypertarget{lb1625905.xhtmlux5cux23start138-c}{}{}\protect\hypertarget{lb1625905.xhtmlux5cux23start138-d}{}{}

Gymer hade icke få salkämpar och äfven några gäster hos sig. Hans mening
var, att när det druckits in på natten och glädjen stigit som högst mot
tak, skulle äktenskapen tillkännagifvas och bekräftas med öfliga bruk.
Men innan det skett, stego två objudna gäster in i salen. Den ene hade
svärd vid sidan och båge på axeln och bar i handen »väghjälpens träd»,
en rönn, som Tor ryckt upp, när han vadade genom Elivågor, fullt så god
att utdela slag med som den, hvilken Ulls fader Egil en gång fått af Tor
att svänga mot Geirrauds jättar. Bakom Ull kom Tor, och han såg bister
ut. Svipdag reste sig från sin stol och ropade: »Välkommen, Midgards
värnare! Välkommen, Egils son, min broder! Det vardt som jag sade er.
Här firas två bröllopp: Gymers med Fröja och mitt med Gerd. Vig du nu
Gymer med din hammare!»

Sedan järnhammaren blifvit splittrad, bar Tor sin gamle hammare af sten.
Den var ej så flygskicklig, men den var säker i slaget. Gymer, hans
salkämpar och gäster rusade upp. De fleste af desse flydde genom
bärgsalarnes långa räcka ut. Men några, som ej kunde undkomma, stannade
kring Gymer, och var denne en så väldig kämpe, att det fordrades Tors
asakraft och hammare för att fälla honom, hvilket icke skedde förr än
efter hård strid. De andre stupade under Ulls och Svipdags hugg.

Gerd måste åse denna kamp. Den var för henne en föga glad inledning till
hennes förmälningsfäst i Asgard.

Asgardshjältarne letade i Gymers många gömmor efter Valandssvärdet. Det
fanns där icke. Eggter hade i god tid begifvit sig bort med vapnet och
flytt österut. Han åtföljdes på flykten af en skara ulfvar, liknande
dem, som voro bundna utanför Gerds frustuga. Fader till denna hos Gymer
fostrade ulfhjord var Loke; modern var Gullveig. Med desse »Fenrers
fränder» begaf sig Eggter till den ogenomtränglige Järnskogen. Där äro
de ännu. På en kulle, under hvilken svärdet är nedgräfdt, djupt inne i
denna obygd håller Eggter vakt intill Ragnarök.

\protect\hypertarget{lb1625905.xhtmlux5cux23start139}{}{}\protect\hypertarget{lb1625905.xhtmlux5cux23start139-a}{}{}\protect\hypertarget{lb1625905.xhtmlux5cux23start139-b}{}{}\protect\hypertarget{lb1625905.xhtmlux5cux23start139-c}{}{}\protect\hypertarget{lb1625905.xhtmlux5cux23start139-d}{}{}

\paragraph{XXXIV.}

\paragraph{BRYTNING MÄLLAN ASAR OCH VANER.}

Gerd fördes af Asgardshjältarne till Folkvang. Där måste hon upplefva
ett nytt ve. Det lät sig icke göra att dölja för gudarne, att Gerds
moder var, såsom ryktet förmält, Gullveig, och att Gullveig var Fröjas
tärna, Aurboda.

Asarne samlades till ett förberedande rådslag i Valhall och rådgjorde om
hvad som skulle göras med henne. Gullveig ställdes inför dem. Oden
påvisade, att Gullveig längesedan var dömd till döden, att domen var
bestående och fullgiltig och ej kunde upphäfvas däraf, att den onda
trollkvinnan blifvit pånyttfödd till världen. Det vore gudarnes plikt
att afrätta henne, när hälst och under hvilka skepnader och namn hon
ånyo anträffades. Ny ransakning och dom behöfdes således icke. När Tor
hörde Oden säga detta, stod han upp och gaf för tredje gången
trollkvinnan dödsslaget med sin hammare. Vanerna voro icke närvarande,
och när Njord erfor hvad som skett, sade han, att saken blifvit för
hastigt afgjord, och att det häftade betänkligheter vid den, hvilka han
vid annat tillfälle, efter samråd med vaner och alfer, ville framställa.
Men asarne stötte sina spjut i häxans kropp och förbrände den i bålets
lågor. Nu som förut visade det sig omöjligt att förvandla hennes hjärta
till aska. Loke gick ännu fritt omkring, hvar han behagade och ej minst
i Asgard, emedan Oden icke ville bryta den ed han svurit honom i
tidernas morgon. Att alla skydde och undveko honom, syntes han icke
mycket bry sig om. Han efterletade Gullveigs hjärta, fann och slukade
det och födde någon tid därefter en ohygglig dotter, pästvarelsen Leikin
och många bröder till henne, de så kallade baningarne (förpästarne).

Tor kastade Leikin ned i Nifelheim. Hon stötte sin ena sida svårt och
hennes benbyggnad bröts i fallet. Däraf kommer sig väl, att hon är till
hälften svartblå, till hälften likblek, och att hon har en stupande
gång. I Nifelheim gjorde henne
\protect\hypertarget{lb1625905.xhtmlux5cux23start140}{}{}\protect\hypertarget{lb1625905.xhtmlux5cux23start140-a}{}{}\protect\hypertarget{lb1625905.xhtmlux5cux23start140-b}{}{}\protect\hypertarget{lb1625905.xhtmlux5cux23start140-c}{}{}\protect\hypertarget{lb1625905.xhtmlux5cux23start140-d}{}{}
sjukdomsandarne till drottning. Lokes son, ulfven Fenrer, hade fått
stanna i Asgard, emedan han i början var lekfull och syntes oskadlig;
men hans fader såg med nöje, huru Fenrer växte och lofvade att med tiden
varda så stor i sitt slag som Midgardsormen i sitt. Asarne förvånades
öfver, att den mat, som lades för honom, var tillräcklig att gifva honom
så hastig tillväxt i storlek och styrka, fast han ej fick mer än Gere
och Freke att äta; men den föda, som gjorde honom så stor och farlig,
var gudarnes felsteg och människornas synder.

Fröjs giftermål med Gerd firades med föga glädje. Hennes fader och moder
hade ju blifvit dräpta af dem, som voro de förnämste gästerna vid
fästen.

Kort därefter begärde Njord gudarådets sammankallande. Asar och vaner
sammanträdde på sin heliga tingsstad, alla med en förkänsla af att något
vigtigt och ödesdigert förestod.

Våra fäder älskade väl mycket strider med vapen, men lika mycket strider
med skäl och grunder, med påståenden och invändningar. Den
öfverläggning, som nu uppstod mällan asar och vaner, har varit
vidlyftigt och noggrannt omförmäld i forntidssånger.

Njord talade å vanernas vägnar. Han, som är den fredliga samfärdselns
gud, lade nog sina ord väl och varsamt, ty ärendet var grannlaga och
kunde icke få god utgång, utan att sinnena stämdes till förlikning. Det
var tydligt, att han samrådt med vaner och alfer, och att de för
endrägtens skull icke ville bestrida riktigheten af Odens mening, att
den för långa tider tillbaka fällda domen öfver Gullveig ännu vore
bestående, ehuru tvifvel därom med skäl kunde hysas, ty ovisst kan det
ju synas, om en tre gånger af olika föräldrapar till världen född
varelse skall i ett rättsmål gälla för en och samma eller för tre
särskilda. Med afseende på Gullveigs onda gärningar kan man säga, att de
falla tre onda tursakvinnor eller en enda till last. När fördenskull
tvifvelsmål i denna punkt låta sig yppa, hade det varit bäst, att
ordentlig ransakning med vittnesförhör och dom äfven denna gång
företagits\protect\hypertarget{lb1625905.xhtmlux5cux23start141}{}{}\protect\hypertarget{lb1625905.xhtmlux5cux23start141-a}{}{}\protect\hypertarget{lb1625905.xhtmlux5cux23start141-b}{}{}\protect\hypertarget{lb1625905.xhtmlux5cux23start141-c}{}{}\protect\hypertarget{lb1625905.xhtmlux5cux23start141-d}{}{}
och att man därvid hållit sig ensamt till de gärningar, som den
anklagade efter sin tredje pånyttfödelse troddes hafva begått.

En betänklig sak var äfven, att Gullveig blifvit dräpt och bränd i Odens
egen högheliga sal, som ej borde med blod fläckas, och där inför
asamajestätet enhvar borde känna sig under lagens skydd.

Men det egentliga klagomålet från vanernas sida var, att asarne, när de
denna gång dräpte Gullveig, icke beaktat, att hon hade blifvit med
frändskapsband knuten till gudarne, främst till vanerna. Det var Fröjs
svärmoder, som blifvit af asarne dödad. Däri låge det betänkliga, och
det var detta mål, som vaner och alfer nu framlade till behandling och
afgörande.

Från asarnes sida kunde nu invändas, att det frändskapsband, som knutits
mällan vanerna och Gullveig icke bör gälla som ett skydd för henne, utan
tillräknas henne som följden af ett bland hennes allra värsta brott. Man
kunde ju icke betvifla, att hon för onda ändamål eftersträfvat det
frändskapsbandet och vunnit det genom sin trolldom, denna gång riktad
mot Njords egen son, den blide och gagnelige, af gudar och människor
älskade Fröj. Hans sinnen slog hon i sådana trolldomsband, att för
Asgard var intet annat val än att se Fröj tvina bort och dö eller att
gifva samtycke till ett äktenskap, som måste varit Njord själf
ovälkommet, och där bruden måste köpas med Asgards yppersta skyddsmedel:
Valandssvärdet. Mot gudarne och människorna var detta anslag riktadt,
men främst mot vanerna. Oden väntade fördenskull, att desse ej skulle
göra sak af bestraffningen, som Gullveig senast undergått, ehuru han
kunde medgifva, att asarne därvid gått för hastigt till väga.

För Fröj måste det varit hårdt att öfvervara denna rådplägning, eftersom
den kärlek han hyste till Gerd här var föremål för undersökningar, som
ej kunde annat än gräma honom. Också hade öfverläggningen icke räckt
länge, innan han framlade sitt och de andre vanernas käromål i denna
\protect\hypertarget{lb1625905.xhtmlux5cux23start142}{}{}\protect\hypertarget{lb1625905.xhtmlux5cux23start142-a}{}{}\protect\hypertarget{lb1625905.xhtmlux5cux23start142-b}{}{}\protect\hypertarget{lb1625905.xhtmlux5cux23start142-c}{}{}\protect\hypertarget{lb1625905.xhtmlux5cux23start142-d}{}{}
sak. De hade beslutat fordra böter af asarne för det på hans svärmoder
gångna dråpet.

Oden skulle väl utan tvekan medgifvit, att han och hans söner vore till
böter förfallne, och att han, oaktadt den höga ställning han fått i
världen, borde erkänna sig skyldig och gifva billigt vederlag, om icke
den tyngsta betänklighet vidlådde ett sådant medgifvande och gjorde det
farligt. Bland slägterna i Midgard kunde blott allt för många uppfatta
sakens utgång så, att asarne ej skulle gifvit böter, om de icke erkänt
som en förbrytelse, att de dödade henne, som spred de onda runorna,
uppfann den onda säjden och är första upphofvet till alla nidingsdåd i
Midgard. Huru skulle människorna sedan skilja mällan godt och ondt?
Oden, såsom världens styresman, lagarnes handhafvare och människornas
fader och konung, måste ur denna synpunkt se målet. Han med sina söner
ville fördenskull icke gifva de af vanerna begärda böterna.

Men likväl var det en helig, af Urd gifven, af Oden stadfäst lag, att
frändedråp skall af närmaste fränder hämnas eller ock af gärningsmannen
böter gifvas. Vanerna hade begärt böter för fridens skull, emedan
utkräfning af hämd skulle medföra de största olyckor för alla. Oden
borde helt visst betänka, att det för människorna vore ett farligt
föredöme, om gudarne själfve lämnade en så helig lag bruten.

Vanerna förklarade, att de icke kunde afstå från sin fordran på böter.
Denna fordran var en ovilkorlig plikt, omöjlig att åsidosätta.

Oden genmälde, att denna plikt också var af vanerna fullgjord. De hade
fordrat böter. Nu ålade dem en annan plikt att gifva de bättre grunderna
rätt och af hänsyn till världens bästa icke lämna tingssätena förr än
öfverenskommet blifvit, att allt emellan asar, vaner och alfer härmed
var utjämnadt.

Här syntes nu asar och vaner stå omedgörlige mot hvarandra. Men
medgörlighet höfdes Oden i denna sak. Hvad som lades Gullveig med rätta
till last var ond säjd och
trolldom\protect\hypertarget{lb1625905.xhtmlux5cux23start143}{}{}\protect\hypertarget{lb1625905.xhtmlux5cux23start143-a}{}{}\protect\hypertarget{lb1625905.xhtmlux5cux23start143-b}{}{}\protect\hypertarget{lb1625905.xhtmlux5cux23start143-c}{}{}\protect\hypertarget{lb1625905.xhtmlux5cux23start143-d}{}{};
men den onda säjden hade ju af Oden själf pröfvats, då han sökte vinna
Rinds gunst. Vanerna måste påminna härom, icke för att förbittra
asafadern, som de vördade, men för att beveka hans sinne till billighet.

Då Oden hörde sig den onda säjden lagd till last, förvandlades hans
anlete. Han vardt påmind om en förnedring, som han icke kunnat afvärja
från sig och som förmodligen varit Gullveigs värk, äfven den, en
förnedring, som han försonat med en sons död.

Han stod upp från sitt högsäte i tingskretsen, tog spjutet Gungner, som
stod bredvid honom, och kastade det öfver vanernas hufvud, till tecken
att bandet mällan honom och dem var slitet, och att målet skulle afgöras
med andra vapen än grunder och skäl. Han lämnade tingsstaden utan att
säga ett ord, och asagudarne följde honom.

Vanerna och alferna stannade och öfverlade. De lämnade icke tingsstaden,
innan de fattat ett beslut. Detta tillställdes Oden och var af den
lydelse, att enär han och Gullveig gjort sig skyldiga till samma
förbrytelse, föröfvandet af ond säjd, och Gullveig rättvisligen
straffats härför med döden, bör Oden rättvisligen afsättas från sin med
samma brott fläckade värdighet af gudars och människors fader. Efter
detta beslut utrymde vanerna och alferna sina borgar i Asgard, och
Asgardsporten stängdes efter dem.

\paragraph{XXXV.}

\paragraph{VÄRLDSKRIGET. VANERNA INTAGA ASGARD.}

Oden sände bud till Mimer, lät honom veta hvad som skett och bad om hans
råd. Det var tydligt, att genom brytningen mällan asar och vaner var
världsordningens och världsträdets bestånd än en gång hotadt. Komme det
till krig emellan gudarne och en eller flere bland dem fölle, skulle
detta lända till största men, ty hvar och en af dem är under denna
världsålder nödvändig och har sitt kall att
\protect\hypertarget{lb1625905.xhtmlux5cux23start144}{}{}\protect\hypertarget{lb1625905.xhtmlux5cux23start144-a}{}{}\protect\hypertarget{lb1625905.xhtmlux5cux23start144-b}{}{}\protect\hypertarget{lb1625905.xhtmlux5cux23start144-c}{}{}\protect\hypertarget{lb1625905.xhtmlux5cux23start144-d}{}{}
uppfylla i det helas ordning. Mimer lofvade att försöka en medling; men
misslyckades den och fiendtligheter utbröto, borde Oden och hans söner
innestänga sig i Asgard och vaka öfver dess försvar, men icke göra
onödiga utfall eller i vapenskifte eftersträfva sina förre medstyresmäns
och vänners lif. Om vanerna lyckades komma inom Asgardsvallen, borde
Oden utrymma Asgard hällre än att fläcka den heliga borgens gård med
gudadråp. En strid man mot man inom vallen kunde medföra många gudars
fall och världens fördärf. Man kunde förutse, att om Oden lydde dessa
råd, kunde den dag komma, då vaner och alfer åter hyllade honom som sin
fader och konung, för att aldrig mer ställa sin vilja mot hans.

Odens stridslystne söner funno dessa råd hårda att följa, men kunde ej
förneka deras visdom. Men Mimer lärer ha gifvit ytterligare ett råd, som
de utan missmod kunde gilla. Rådet var, att Loke aldrig mer borde få
sätta sin fot inom Asgardsporten. Allt för länge hade Oden fördragit
Loke, detta för att icke bryta sitt i tidernas morgon gifna löfte till
honom. Men med löftena vore så, att ett löfte, ensidigt gifvet och utan
förbehåll, vore alltid bindande; men ett löfte, som innebar ett fördrag
mellan två eller flere, vore icke bindande för den, som velat hålla det,
sedan den andre eller de andre löftesmännen uppsåtligt svikit det. I
annat fall vore det en boja på den trofaste och ett fribref för den
trolöse att få göra med den trofaste hvad han ville. Löftet, som bundit
Oden vid Loke, var ömsesidigt, och, alltsedan det aflades, så godt som
dagligen sviket af Farbautes son, som städse syftat till sin
fosterbroders och hans ätts undergång. Mimer torde hafva upplyst, hvad
man i Asgard icke vetat, att Loke var Balders egentlige baneman.

Till vanerna lär Mimer vändt sig med liknande råd, sedan de begärt att
höra hans mening.

Med Oden kvarstannade i Asgard Tor, Tyr, Brage, Vidar, Vale och Forsete.
Några asynjor förblefvo där äfven, men bland dem voro icke de förnämsta.
Frigg ansåg sin systerplikt och sin härkomst från vanerna fordra, att
hon slöte sig
\protect\hypertarget{lb1625905.xhtmlux5cux23start145}{}{}\protect\hypertarget{lb1625905.xhtmlux5cux23start145-a}{}{}\protect\hypertarget{lb1625905.xhtmlux5cux23start145-b}{}{}\protect\hypertarget{lb1625905.xhtmlux5cux23start145-c}{}{}\protect\hypertarget{lb1625905.xhtmlux5cux23start145-d}{}{}
till dem. Asgards drottning lämnade fördenskull sin make och följde sin
broder Njord. Fröja följde också sin broder Fröj. De båda alferna
Svipdag och Ull hade ställt sig på vanernas sida, såsom ju också var att
vänta, då Svipdag var Fröjas make och Fröjs förtrognaste vän, och då Ull
var Svipdags halfbroder och honom mycket tillgifven. Hela den skara af
högre och lägre makter, som bebo Vanaheim och Alfheim, omfattade samma
sak som de och afföllo från asarne. Endast Höner och Mimer förblefvo
Oden trogne.

Men Skade följde icke sin make Njord, utan kvarstannade i Asgard.
Vanaguden och hon voro af olika skaplynne, och de hade ej kunnat trifvas
väl tillsammans. Äfven i afseende på bostad och omgifning hade de olika
tycken. Skade älskade sin fader Valands bärgiga hemland Trymheim, på
hvars snöslätter det var hennes nöje att gå på skidor och fälla
villebråd med sina pilar. När Njord, för att vara henne till viljes,
vistats i nio dygn bland fjällen, vardt han led vid bärgväggar och
ulftjut och längtade till svanesången vid hafvets strand; men när Skade
följt honom dit, kunde hon icke uthärda att hvarje morgon väckas af
sjöfåglarnes skrik. Hon hade fordom drömt om att varda Balders maka. Den
drömmen vardt aldrig uppfylld. Nu trifdes hon bäst i närheten af Balders
fader och lyssnade hällre till hans ord än till någon annans.

När asarne gifvit sin önskan till känna att höra Mimers tanke, torde
han, som aldrig påtrugar någon sina råd och tiger, tills han blifvit
tillfrågad, hafva sagt att han gillade vanernas fordran på böter för
dråpet på Gullveig. Med denna fordran efterkommo de en helig lag. Men
han måste hafva ogillat, att de intet förslag framställt, som kunde
häfva Odens välgrundade betänkligheter mot böters gifvande. Det var ju
sannolikt, att Gullveig skulle än en gång pånyttfödas och än en gång
stämpla till gudarnes och världens fördärf. Skulle hon få göra detta i
skydd af sin slägtförbindelse med vanerna, eller skulle desse kräfva
böter för henne på nytt, ifall hon för sina onda gärningar blefve genom
asarne än en gång
\protect\hypertarget{lb1625905.xhtmlux5cux23start146}{}{}\protect\hypertarget{lb1625905.xhtmlux5cux23start146-a}{}{}\protect\hypertarget{lb1625905.xhtmlux5cux23start146-b}{}{}\protect\hypertarget{lb1625905.xhtmlux5cux23start146-c}{}{}\protect\hypertarget{lb1625905.xhtmlux5cux23start146-d}{}{}
skild från lifvet? Hade icke vanerna gjort väl uti att afgifva en
försäkran, att deras afsigt ej var sådan? Och då i öfrigt dessa
Gullveigsbrännor visat sig tjena till intet, kunde vanerna ju
föreslagit, att om hon åter framträdde, de ville förena sig med asarne
om att förvisa henne för evigt från himmel och jord. Medel att
värkställa en sådan dom låge icke utanför gudarnes maktområde. Mimer
tillrådde vanerna att upphäfva den förhastade afsättningsdomen öfver
Oden och vända sig till denne med det medlingsförslag, som han nu
påvisat.

Vaner och alfer sammanträdde till rådslag. Å detta infann sig en ond
rådgifvare, sannolikt Loke, ehuru han ej var kallad. Man hade gjort bäst
uti att taga ordet från honom; men när detta icke skedde, talade han,
och ehuru man väl i början med ovilja hörde hans röst, lade han sina ord
så skickligt och med så stor vältalighet, att de efter hand gjorde
intryck på åhörarne.

Rådstämman började därmed att Höner, som var höfdingen i Vanaheim,
uttalade sin mening, och gillade han i allo Mimers förslag och förordade
det till antagande. Det tal, hvarmed den onde rådgifvaren följde, sade i
slugt och genomskinligt bemantlade ord, att Höner var en aktningsvärd
beskedlighetsmakare utan eget omdöme och ett värktyg i Mimers och Odens
hand. Mimer skildrades af talaren som en beslöjad gudafiende, hvilken
afundsamt vakade öfver och undanhöll för andra visdomskällans mjöd och
såsom innehafvare af denna egde en makt, som icke tillkom honom, utan
borde vara i gudarnes våld. Han förutsade vanernas saks fall och deras
egen undergång, om de längre kunde tåla, att en gudarnes förrädare, som
undangömde skatter för en kommande världsålder, då han själf väntade att
varda erkänd för världens herre -- om en sådan förrädare skulle hafva
sitt säte i skapelsens midt och genom Höner bestämma vanernas beslut.

Rådstämman ändade med en ny afsättningsdom. Höner beröfvades sin
värdighet af Vanaheims höfding, och vardt denna värdighet öfverlåten åt
den stolte och hetsige Lodur. Åt Njord öfverläts att leda vanernas och
alfernas fylkingar
\protect\hypertarget{lb1625905.xhtmlux5cux23start147}{}{}\protect\hypertarget{lb1625905.xhtmlux5cux23start147-a}{}{}\protect\hypertarget{lb1625905.xhtmlux5cux23start147-b}{}{}\protect\hypertarget{lb1625905.xhtmlux5cux23start147-c}{}{}\protect\hypertarget{lb1625905.xhtmlux5cux23start147-d}{}{}
i kriget, och skulle, om vanerna segrade, Njord, Fröj, Ull och Svipdag
vara härrskare i Asgard.

Kort därefter vardt Mimer dödad och hans afhuggna hufvud sändt till
Oden. Hvem som var hans handbane är höljdt i dunkel; ingen vet numera,
på hvilken den missgärningen hvilar. Visst är allena, att Loke var
Mimers rådbane.

Då Mimer dödades, lär den af honom vaktade källan sjunkit så djupt under
sin rand, att dess dyrbara safter länge voro oåtkomliga. Alltsedan
Mimerträdets källa förlorade sin vårdare, har det börjat åldras och
skall vid denna tidsålders ände förete ett af åren medtaget utseende.

Mimers sju älste söner, de store urtidskonstnärerna, hade med sin fader
delat omsorgen om Yggdrasil. Enhvar hade sin sjundedel af året, då han
ur hornet, som är asafaderns pant till underjordsmakterna, sköljde den
store askens stam med hvitglänsande fors ur källan. Men sorgsne öfver
världshändelsernas gång och trötte vid att skåda det fortgående
förfallet, drogo sig Mimers söner efter sin faders död undan till den
gyllne borg, som de åt sig uppfört i norra delen af Mimers rike, på
Natts odalmarker, under Nidafjället. Stuckne med sömntörnetagg lade de
sig ned till oräknade århundradens hvila. Det finns i borgen många
salar, hvilkas väggar och bänkar lysa af de vapen och andra konstvärk,
som de skapat. I den innersta salen sofva Sindre och hans bröder, höljde
i praktfulla mantlar. De dödlige, som ödet någon gång medgifver att få
träda in i Mimersönernas borg och se dess hemligheter, må akta sig att
vidröra de sofvande. Vidrörandet straffas med obotlig tvinsot. Allt
synes också slumra omkring dem. I närmaste sal stå sju hästar: Sindres
häst, Moden, och de andre brödernas. De äro sadlade, som om de hvarje
ögonblick borde vara färdige för sina ryttare. Natt höljer borgen i
skymning. Dånet af världskvarnen och Hvergelmers brusande flöden
förnimmas därinne som ett vattenfalls entonigt vyssande sång. Stundom
prassla genom salarnes räcka nattdisernas steg, när de komma att
betrakta sina fränder och bortfläkta dammet från deras mantlar och
vapen.
\protect\hypertarget{lb1625905.xhtmlux5cux23start148}{}{}\protect\hypertarget{lb1625905.xhtmlux5cux23start148-a}{}{}\protect\hypertarget{lb1625905.xhtmlux5cux23start148-b}{}{}\protect\hypertarget{lb1625905.xhtmlux5cux23start148-c}{}{}\protect\hypertarget{lb1625905.xhtmlux5cux23start148-d}{}{}
Så skola de sofva, tills Yggdrasil skälfver och Heimdallslurens
världsgenomträngande klang väcker dem till den sista striden mällan det
goda och det onda.

Heimdall räknas till vanernas stam. Han ville ej kämpa mot sina fränder,
men ej häller svika sin trohetsplikt mot Oden. Båda de fiendtliga
gudaflockarna vordo enige därom, att Heimdall skulle stå utanför deras
fäjd och vara, såsom ditintills, Bifrosts väktare och höfding öfver de
alfkrigare, som utgjorde den vid Bifrosts norra brohufvud uppförda
borgen Himmelsvärns besättning. Öfverenskommelsen påbjöds af
nödvändigheten, ty det var för vanerna lika viktigt som för asarne, att
jättarne ej genom en öfverrumpling komme i besittning af Asgard. Nu, då
gudarne voro splittrade i olika läger, var det sannolikare än någonsin,
att jättarne skulle rufva på en sådan plan och tro på dess framgång.

Men genom denna anordning miste själfva Asgard den väktare, som kräfver
mindre sömn än en fågel och hvars blick tränger genom nattdjupen och
hvars öra ej de svagaste ljud på långa håll undgå. Dock trodde asarne
själfve, att detta icke minskade deras säkerhet. Den höge Asgardsvallen
är oöfverkomlig; Sleipner allena kan sätta öfver den. Vallen kringbrusas
af den breda älf, som störtar ned från Eiktyrner med vafer-utdunstande
vattenmassor. Vaferdimmorna antändes, och älfven liknade hela den tiden
en hvirflande eldfors, som stänkte blixtar högt i luften. I världen
fanns ingen annan häst än Sleipner, som kunde springa genom och öfver
vaferlågor. Asgardsporten var ett utomordentligt konstvärk: den liksom
vaktade sig själf och fångade den främling, som lade hand på honom. Dock
voro hans egenskaper för de vaner, som bott i Asgard, väl bekanta. Vordo
hans på insidan anbragta lås söndersprängda, så kunde han öppnas utan
fara.

Asarne ansågo sig säkre inom sin vallgördel.

Från Bifrosts södra broände ryckte Njord med en stor här af Vanaheims
och Alfheims stridsmän upp på Asgards vidsträckta utmärker. Oden och
hans söner sågo från Lidskjalf och vallmuren en glänsande vapengördel
bilda sig kring
\protect\hypertarget{lb1625905.xhtmlux5cux23start149}{}{}\protect\hypertarget{lb1625905.xhtmlux5cux23start149-a}{}{}\protect\hypertarget{lb1625905.xhtmlux5cux23start149-b}{}{}\protect\hypertarget{lb1625905.xhtmlux5cux23start149-c}{}{}\protect\hypertarget{lb1625905.xhtmlux5cux23start149-d}{}{}
dess borg, dock ännu på betydligt afstånd från densamma. Där stannade
den länge och kom icke närmare. Men nattetid smögo Njords
späjaretrupper, anförda af Fröj, Svipdag eller Ull, så nära som skenet
från vaferlågorna medgaf det. De märkte, att hvarje natt red någon af
asarne på Sleipner ett stycke fram mot deras utposter och utefter dem,
och de iakttogo sedan några lysande föremål, som rörde sig på den skans,
hvilken utgör den med Asgardsporten förenade fallbryggans yttre
brohufvud. Efter många nätters späjande kommo vanerna till den slutsats,
att det var asarnes hästar, bland hvilka några voro skinande, som
nattetid utsläpptes att beta på de gräsbevuxna skanssluttningarna, sedan
Sleipners ryttare återvändt från sin kringfärd, och att sannolikt äfven
Sleipner då släpptes lös och betade bland kamraterna under den
vakthafvande asagudens tillsyn. Denna slutsats var riktig. De planlade
då en öfverrumpling och kommo en natt så plötsligt öfver väktaren, att
denne ej hann fram till den betande Sleipner, innan en af vanerna svängt
sig upp på hans rygg. Asaguden, som hörde vanernas spjut och pilar susa
omkring sig, betäckte sig med sin sköld och vek ned emot fallbryggan.
När han kommit öfver henne, öppnade sig Asgardsporten på glänt och drog
på samma gång upp bryggan, samt slöt sig igen bakom den öfverrumplade
väktaren. På detta sätt kom Sleipner i vanernas våld, och de egde den
häst, som kunde sätta öfver vaferlågorna och Asgardsvallen. De andre
asahästarne infångade vanerna icke. De skyndade därifrån i all hast
tillbaka till lägret.

Denna händelse var egnad att nedslå asarnes mod. Dock syntes förlusten
för dem vara större än vinsten för vanerna. Ty hvad kunde en, eller, låt
vara, några vaner uträtta, ifall de på Sleipners rygg kommit inom
Asgardsmuren? Förmådde de hålla stånd mot asafadern och hans käcke
söner? Säkerligen icke.

När asarne en mörk och stormig natt stigit upp från dryckesbordet och
gjorde sin vanliga rund mällan borgarne och Asgardsvallen, upptäckte de
till sin öfverraskning
Sleipner\protect\hypertarget{lb1625905.xhtmlux5cux23start150}{}{}\protect\hypertarget{lb1625905.xhtmlux5cux23start150-a}{}{}\protect\hypertarget{lb1625905.xhtmlux5cux23start150-b}{}{}\protect\hypertarget{lb1625905.xhtmlux5cux23start150-c}{}{}\protect\hypertarget{lb1625905.xhtmlux5cux23start150-d}{}{}
gå inne på gårdsplanen. De drogo däraf den slutsats, att han slitit sig
lös, sprungit ur vanalägret och satt öfver Asgardsmuren, för att
återkomma till sin spilta och sina kamrater. Samtalande härom återvände
asagudarne till dryckessalen. Deras ord hördes af Njord från loftgången
öfver Valhalls gafvelport, där han stod i nattdunklet under de bilder af
ulfven och örnen, som pryda gafveln. Det var han, som på Sleipner
sprängt öfver vaferelden och vallen. Dånet af springarens åtta hofvar,
när han nådde marken, hade blandat sig med stormens tjut och älfvens
brus och icke beaktats af Valhallskämparne. När desse åter sutto vid
dryckesbordet, gick Njord till Asgardsporten. Han hade tillsagt sina
fylkingar att med största möjliga tysthet framrycka mot skansen. De hade
mörkfärgat hjälmar och brynjor, och icke ett vapen klirrade under deras
tysta framryckande. De liknade en spökhär. Njord hade medfört sin
stridsyxa, ett af urtidskonstnärernas mästervärk. Med den sprängde han
Asgardsportens lås. Den öppnade sig på vid gafvel, och öfver älfven
nedföll fallbryggan, hemsk att gå, när färden bar emällan vaferlågor,
men trygghetsgifvande tillika. Vanernas och alfernas skaror med sina
höfdingar i spetsen stormade in och trampade markerna, som Asgardsvallen
skulle skyddat. De utbredde sig utefter denna; men ryckte icke ända fram
mot Valhall. Njord och Fröj, Svipdag och Ull samt öfrige höfdingar för
Odensfienderna bidade till häst på asarnes ankomst.

Några i kappor höljde ryttare -- asar och asynjor -- kommo emot dem.
Främst red Oden på Sleipner, och bredvid honom gick Tor med hammaren.

»Ödet», torde Oden hafva sagt, »har till eder öfverlåtit mina
odalmarker.»

»Ja», torde Njord ha svarat. »Det gäller nu, om du fogar dig i dess
rådslag. Vi lyfta icke gärna våra vapen emot asarnes heliga slägt. Valet
mällan strid och fritt aftåg ligger i din hand, ej i min. Jag sörjer
för, att ditt namns ära ej skall minskas. Folken i Midgard skola i alla
tider vörda det namnet.»

\protect\hypertarget{lb1625905.xhtmlux5cux23start151}{}{}\protect\hypertarget{lb1625905.xhtmlux5cux23start151-a}{}{}\protect\hypertarget{lb1625905.xhtmlux5cux23start151-b}{}{}\protect\hypertarget{lb1625905.xhtmlux5cux23start151-c}{}{}\protect\hypertarget{lb1625905.xhtmlux5cux23start151-d}{}{}

»Öppnen då edra fylkingar!» De öppnade sig, och med aktning sågo
vanakrigarne den lilla asaflocken rida bort. Tor ensam vände om. Utanför
hans borg väntade honom hans spann. Han satte sig i sin char, och snart
därefter hörde vanerna den rulla med dånande hjul vid ljuset af
ljungeldar österut.

Mot östern begåfvo sig asarne. De medförde, jämte sina vapen, inga andra
klenoder än Draupner och en bild af guld, ett hufvud med ädla
anletsdrag. Det var Mimers hufvud, genom Odens runesång förvandladt.
Bilden talade, när den tillfrågades, och Oden hörde då Mimers röst och
Mimers tankar. Rösten hade sagt, att asarne borde begifva sig österut
till Manheim, till näjder, kända bättre af Mimer än af andra.

Det försäkras, att sedan Loke blifvit aflägsnad från Oden, och sedan
denne fått Mimers hufvud, begick han aldrig mer ett felsteg, och voro
hans rådslag alltid visa, hans gärningar alltid prisvärda. Två trappsteg
hade han att stiga, innan han nådde upp till det tronsäte, i hvilket han
sedan varit vördad af folken. Han gjorde det första steget, när han fick
drycken ur Mimers källa. Han gjorde det andra, när motgången kom och
bragte hans ande till fullmognad. Motgången kom med den bästa gåfvan:
Mimers ord och tankar.

Njord höll sitt ord, att Odensnamnets ära icke skulle minskas. När vaner
och alfer som herrar i Asgard sammanträdde å dess tingsplats, beslöts,
att en af dem hade att bära Odens namn och mottaga de Oden af folken
egnade offren och bönerna. Denne nye Oden borde, som den förste och
värklige, vara en stridsgud. Njord och Fröj äro visserligen hjältar och
föra sina vapen väl, när så kräfves, men de äro till sitt väsen
fredsgudar och kunde icke öfvertaga Odens namn och kall. Någon föreslog,
att Ull skulle göra det; och detta vardt rådstämmans beslut.

På jorden hade händelser timat, som påkallade Svipdags värksamhet och
närvaro där.

\protect\hypertarget{lb1625905.xhtmlux5cux23start152}{}{}\protect\hypertarget{lb1625905.xhtmlux5cux23start152-a}{}{}\protect\hypertarget{lb1625905.xhtmlux5cux23start152-b}{}{}\protect\hypertarget{lb1625905.xhtmlux5cux23start152-c}{}{}\protect\hypertarget{lb1625905.xhtmlux5cux23start152-d}{}{}

\paragraph{XXXVI.}

\paragraph{VÄRLDSKRIGET I MIDGARD. HADDINGS UNGDOMSÄFVENTYR.}

Halfdan hade efterlämnat två söner: Gudhorm, son af Groa, således
halfbroder till Svipdag, samt Hadding, son af växtlighetsdisen Alvig,
med hvilken Halfdan hade gift sig, sedan han bortsändt Groa och Svipdag.
Hadding var blott en liten gosse, när Halfdan dödligt sårades af
hämdesvärdet i Svipdags hand. Sitt namn (Hadding betyder »den lockige»,
»den hårfagre») fick han därför, att han hade ett ovanligt rikt och
vackert hår, som han beslutit att icke klippa, innan han återvunnit sin
andel i fadrens rike. Skägg fick han aldrig, och ehuru han växte upp
till stora krafter och vardt en fräjdad hjälte, liknade han i sina
ynglingaår en ungmö.

Tor, som gällde för att vara Halfdans medfader och alltid varit hans
beskyddare, åtog sig efter hans död hans båda söner. Han förde Gudhorm
och Hadding i hemlighet till Svitiod det stora. Där bodde två kämpar,
Hafle och Vagnhöfde, som hade jätteblod i sina ådror, men voro hederlige
och pålitlige och fördenskull fått Tors ynnest. Tor anförtrodde Gudhorm
åt Hafle och utsåg Vagnhöfde till Haddings fosterfader.

Orsaken hvarför Tor undanskaffade de båda Halfdanssönerna var den, att
han fruktade, att de skulle råka illa ut för Svipdag, som efter deras
faders nederlag och död tillegnat sig hela hans rike och styrde det
medels jarlar. Svipdag måste förutse, att endera sonen, om han komme
till manbara år, skulle kräfva blodshämd på sin faders baneman. Dessutom
lär väl Tor, efter närmare bekantskap med Svipdag, hafva märkt, att
under ytan af hans glada, raska och behagliga väsen låg förborgadt något
af hans farbroder Valands lynne. Svipdag tålde icke att höra Halfdan
omtalas, och det var tydligt nog, att han på Hadding öfverflyttat det
hat han
\protect\hypertarget{lb1625905.xhtmlux5cux23start153}{}{}\protect\hypertarget{lb1625905.xhtmlux5cux23start153-a}{}{}\protect\hypertarget{lb1625905.xhtmlux5cux23start153-b}{}{}\protect\hypertarget{lb1625905.xhtmlux5cux23start153-c}{}{}\protect\hypertarget{lb1625905.xhtmlux5cux23start153-d}{}{}
hyst till fadern. Alfernas lynne tros allmänt vara sådant, att de äro
älsklige och välvillige mot dem, som ej förtörnat dem, men mot hvarje
annan hämdgirige och svårblidkelige.

Ehuru det nu så var, insåg Svipdag, att försoning vore bättre än hat,
och erbjöd sig att gifva böter för Halfdans död. Blodshämdens tråd var
redan långt spunnen. Ivalde hade omkommit i ett af Oden stäldt försåt;
Ivaldes söner, Valand och Egil, hade sökt hämnas hans död och fallit,
Valand för Tors hammare, Egil för Halfdans klubba. Därefter hade Halfdan
blifvit dräpt af Egils son. Skulle nu Egils son i sin ordning falla för
en af Halfdans söner och den bloddränkta tråden fortspinnas från slägte
till slägte? Svipdag ville afklippa den. Han erbjöd Gudhorm och Hadding
fred och vänskap och lofvade dem konungamakt bland germanstammarne. De
voro ju alla tre förenade med brodersband: Svipdag var Gudhorms
halfbroder, Gudhorm var Haddings. Gudhorm antog tillbudet och fick ett
stort rike i det västra Germanien utefter Renströmmen. Men Hadding, när
han kommit något till åren, lät svara, att han framför att emottaga
välgärningar af en fiende, föredroge att hämna sin faders död. Detta
svar misshagade de fredsälskande vanagudarne, som förutsågo, att mycken
tvedrägt och örlog skulle härja Midgard, om de tre halfbröderna ej kunde
förlikas.

Loke, som ville ställa sig väl med vanagudarne och fullfölja bland dem
samma onda uppsåt som bland asarne, beslöt nu efterspana Hadding och
utlämna honom åt Svipdag. På samma gång torde han icke tröttnat att
påminna Svipdag om huru mycket ondt Ivaldes slägt hade lidit af asarne
och af Halfdan, och huru visst det vore, att Hadding, om han finge
lefva, skulle sträfva att varda Svipdags bane. Lokes mening härmed var,
att Svipdag, om han finge gossen i sitt våld, skulle dräpa honom och
därmed sätta en evig skamfläck på sin ära. Hadding var icke längre trygg
i Vagnhöfdes bärggård. Han var utsatt för Lokes snaror, ty denne hade
utspanat hvar han var gömd. Så omsorgsfullt Vagnhöfde och hans dotter
Hardgreip vakade öfver honom, kunde likväl deras
\protect\hypertarget{lb1625905.xhtmlux5cux23start154}{}{}\protect\hypertarget{lb1625905.xhtmlux5cux23start154-a}{}{}\protect\hypertarget{lb1625905.xhtmlux5cux23start154-b}{}{}\protect\hypertarget{lb1625905.xhtmlux5cux23start154-c}{}{}\protect\hypertarget{lb1625905.xhtmlux5cux23start154-d}{}{}
ögon icke alltid följa pilten, som längtade ut från bärggården, för att
leka på ängarna därutanför och titta vid skogsbrynet in i hvad som för
honom var en främmande värld. Då han, vaktad af Hardgreip, satt vid
fjällsalens vägg och såg ut, kom kanske ibland en ulf, som hade så
vänliga ögon och frågade, om han icke ville följa honom och se skogens
alla hemligheter, ibland en häst och sporde, om han icke ville rida
omkring världen och se allt märkvärdigt där. Sådant gjorde gossen
trånsjuk och missnöjd med sin fångenskap, och det var nog hans afsigt
att rymma, när han kunde. Då kom en afton till Vagnhöfdes gård en
ryttare på åttafotad häst. Han var en äldre, långskäggig man, hvars ena
öga var tillslutet. Han talade vänligt med Vagnhöfde och Hardgreip och
tackade dem för den vård de egnat pilten. Vagnhöfde lyfte Hadding upp i
sadeln framför ryttaren. Gossen var förtjust öfver att få komma ut, men
Hardgreip grät. Ryttaren svepte sin vida mantel öfver och omkring
Hadding och red bort. När de ridit en stund, vardt Hadding nyfiken och
ville se, huru det var omkring honom. Han öppnade på manteln och såg då
till sin häpnad och förskräckelse, att land och vatten lågo djupt under
springarens hofvar. Oden slog åter mantelfliken öfver hans hufvud och
tryckte honom till sitt bröst. Fram mot natten satte Sleipner sina
hofvar på marken. Hadding var då i Manheim, i det land, där Mimer
anvisat de landsflyktige asarne borgar och trygg vistelseort. Här fick
han leka på ängarna så mycket han ville. Här lär han på idrottsvallen
öfvats af Tyr, och af Brage i runor och skaldskap. Stundom kom äfven
Heimdall, hans stamfader, och såg honom och bevittnade hans förkofran i
vett och styrka. När Hadding kommit till vapenför ålder, sjöng Oden
öfver honom siande skyddsgalder och gaf honom en dryck, ljuf att smaka,
som kallas »Leifners eldar» -- hvilken förlänade Hadding den förmåga,
som Svipdag fått genom Groas galdersång, att med sin andedrägt upplösa
band och bojor.

Nu var det tid för Hadding att träda in på den lefnadsbana, som låg
framför honom. De mål, som han hade att
\protect\hypertarget{lb1625905.xhtmlux5cux23start155}{}{}\protect\hypertarget{lb1625905.xhtmlux5cux23start155-a}{}{}\protect\hypertarget{lb1625905.xhtmlux5cux23start155-b}{}{}\protect\hypertarget{lb1625905.xhtmlux5cux23start155-c}{}{}\protect\hypertarget{lb1625905.xhtmlux5cux23start155-d}{}{}
sträfva till, var att hämna sin faders död och återbörda sin andel i
väldet öfver de germaniska stammarne.

Bland dessa lefde Halfdans namn i äradt minne, och mången undrade
hvilket öde hans yngste son, den lille Hadding, rönt. Ett rykte gick,
att han lefde och en dag skulle visa sig bland dem. Många voro missnöjda
öfver den hårdhet, hvarmed Svipdag efter sin seger öfver Halfdan
förfarit mot dennes vänner och förnämste stridsbröder. Han hade
landsförvisat dem, och de hade dragit bort österut i de okända ängderna
bortom germanfolkens gränser, och ingen visste nu rätt, hvar de
uppehöllo sig, om de lefde eller voro döde. Bland dem voro Hamal,
Halfdans fosterbroder, och Hamals söner och fränder, som äro kände under
slägtnamnet Amaler. Bland dem voro också någre af ylfingarnes och
hildingarnes slägter, främst att nämna Hildebrand. I Hildebrands hus
hade Hadding under sina spädaste år vistats och fått sin första
uppfostran, och ynglingen kom väl i håg sin trygge och gladlynte
fosterfader och hans lärdomar.

När Hadding rustade sig till bortfärd från Oden, sade honom denne, att
han skulle rida västerut genom skogen till ett ställe, kalladt
Märingaborg, där han väntades af vänner. Färden var farlig, och Oden
sporde, om Hadding ville ha följe. Därtill svarade han nej och begaf sig
ensam på väg.

Vägen var, som Oden sagt, farlig. När Hadding efter en tröttsam dagsresa
vaknade en morgon, låg han, bunden till händer och fötter, i en
bärghåla, nära intill ett vidunder, och Loke stod framför honom. Denne
ville aftvinga honom en ed, att han utan motstånd skulle följa honom;
vägrade han, skulle han kastas till föda åt vidundret. Hadding begärde
en stunds betänketid och ensamhet, för att samla sina tankar. Loke gick.
Då andades Hadding på sina bojor, som brusto och föllo af honom, grep
sitt svärd, gaf vidundret banehugg och åt, såsom Oden för ett sådant
fall anvisat honom, dess hjärta. Han vardt därigenom vis och kunde tolka
djurläten. När han utträdde ur hålan, såg han icke Loke, men en skara
stridsmän ligga sofvande på marken. De voro alla
\protect\hypertarget{lb1625905.xhtmlux5cux23start156}{}{}\protect\hypertarget{lb1625905.xhtmlux5cux23start156-a}{}{}\protect\hypertarget{lb1625905.xhtmlux5cux23start156-b}{}{}\protect\hypertarget{lb1625905.xhtmlux5cux23start156-c}{}{}\protect\hypertarget{lb1625905.xhtmlux5cux23start156-d}{}{}
vederstyggliga med likbleka ansigten. Han erfor sedan, att de kallades
baningar (»fördärfvare», »förpästare») och voro bröder till pästväsendet
Leikin och söner af Loke med Gullveig. Odens galdersång följde Hadding
under hela hans väg till Märingaborg, och det var den, som hade söft
baningarne. Hadding fann sin häst och kom oskadd till Märingaborg.

Vid denna tid, sedan Loke förlagt sin värksamhet till Midgard, uppträdde
han som höfding och stamfurste, äfven han. Baningarne voro den stam, som
han behärrskade.

I Märingaborg emottogs Hadding med glädje. Det var hit som Halfdans
fördrifne vänner och stridsbröder samlat sig. Här sattes han i högsäte
mällan Hamal och Hildebrand, och där nedanför vid dryckesbordet sutto
välpröfvade kämpar: amaler, hildingar och ylfingar.

Några bland dem begåfvo sig ut till de stammar, som bebyggde det östra
Germanien, för att meddela dem, att Hadding lefde och skulle komma, för
att höja stridsfana mot Svipdag. De borde göra sig beredda och samla
sina stridskrafter, om de ville följa Halfdans och Alvigs son.

Förberedelserna till fälttåget kräfde tid. Medan de pågingo, kom till
Märingaborg ridande en ung kämpe, som sade sig heta Vidga, son af Valand
och Baduhild, Mimers dotter. Han red en af de vackraste hästar man sett.
Hjälmen med sin gyllne ormprydnad, den lysande brynjan, det skinande
svärdet, skölden, hvarpå tång och hammare voro målade till tecken af
hans härkomst, hela hans utrustning var den skönaste och yppersta --
värk af Valands och underjordssmedernas konst, klenoder ur Mimers
skattkamrar. Han sade genast, att han icke kom i vänligt uppsåt: som
ättling af Ivalde, son af Valand och syskonbarn med Svipdag var det hans
plikt att bekämpa Halfdans ätt. Han utmanade därför Hadding till tvekamp
på lif och död. Emellertid vardt han inbjuden i mjödhallen och där
fägnad. Hjältarne sågo, att han hade en ren och trofast blick och funno
mer behag i hans väsen än i hans ankomst dit, ty de fruktade för
Haddings lif. Deras fruktan var icke ogrundad. Tvekampen hölls,
\protect\hypertarget{lb1625905.xhtmlux5cux23start157}{}{}\protect\hypertarget{lb1625905.xhtmlux5cux23start157-a}{}{}\protect\hypertarget{lb1625905.xhtmlux5cux23start157-b}{}{}\protect\hypertarget{lb1625905.xhtmlux5cux23start157-c}{}{}\protect\hypertarget{lb1625905.xhtmlux5cux23start157-d}{}{}
och Hadding dukade under för Vidgas öfverlägsna vapen. Då Vidga höjde
svärdet till banehugg, ställde sig Hamal och Hildebrand framför honom
och talade bevekande ord. Vidga stack då sitt svärd i skidan och räckte
Hadding handen. Han stannade någon tid på Märingaborg, hade sin plats
vid bordet närmast Hadding, och de voro gode vänner. En gång sade
Hildebrand till Vidga, att han önskade, att detta vänskapsband aldrig
sletes. Vidga svarade, att ödet lagt deras lotter i motsatta vågskålar.
Den dag måste väl komma, då de strede hvar på sin sida; men ett ville
Vidga lofva: att om han kämpade mot Haddings härskaror, skulle han
likväl icke föra sitt svärd mot Hadding själf. Hildebrand tackade för
detta löfte, men omtalade det icke för Hadding.

Bud kom nu till Märingaborg, att de östre germanstammarne vore färdige
till härtåg, och att de väntade på Hadding, för att bryta upp. När Vidga
hörde det, sade han farväl. Själf ämnade han begifva sig till de
stammar, som hyllade Svipdags sak.

Hadding kom, och väldiga stridsmannaskaror samlade sig i det östra
Germanien under hans fälttecken. Svipdag visste säkerligen hvad som
föregick där, men torde ej velat gripa in, förrän Hadding själf höjt
stridsbanéret. Han, Svipdag, hade då ingen skuld i fäjdens utbrott, och
han gladde sig åt tanken att på valplatsen få nedlägga Halfdans son,
såsom han nedlagt fadern.

Svipdag steg ned från Asgard och uppenbarade sig i germanernas urland på
den skandiska ön och kallade dess stammar och danerna under vapen. Han
sände bud till sin halfbroder och underkonung Gudhorm, och denne samlade
sina stridskrafter, att förena dem med hans. Många och ofantligt stora
skepp byggdes, på hvilka svearnes och danernas skaror fördes öfver
hafvet.

Det nordiska urlandets, det västra och det östra Germaniens stammar voro
således på antåg mot hvarandra. Valkyrior, sköna diser i hjälm och
brynja, sågos rida genom luften. De kommo dels från Vanaheim och Asgard,
dels från
\protect\hypertarget{lb1625905.xhtmlux5cux23start158}{}{}\protect\hypertarget{lb1625905.xhtmlux5cux23start158-a}{}{}\protect\hypertarget{lb1625905.xhtmlux5cux23start158-b}{}{}\protect\hypertarget{lb1625905.xhtmlux5cux23start158-c}{}{}\protect\hypertarget{lb1625905.xhtmlux5cux23start158-d}{}{}
det i öster belägna Manheim. Svipdag lät, genom de skalder, som följde
hans och Gudhorms härar, förkunna för stridsmännen, att enligt
gudarådets beslut skall enhvar, som på deras sida faller i de
förestående drabbningarna, af valkyrior ledsagas till Fröjas härrliga
sal Sessrymner, för att lefva där i evig glädje. Hadding lät genom sina
skalder förkunna de östgermaniske krigarne, att enhvar, som å deras sida
faller, skall komma till Oden och njuta ovanskelig fröjd i hans salar.

Till Gudhorms hof hade vid denna tid anländt en främling, som sades vara
en höfding från något aflägset land och af Gudhorm emottogs med heder.
Han var så vältalig, erfaren och rådklok, att Gudhorm fick till honom
största förtroende och sällan gjorde något, som icke han tillrådt. Han
kallade sig Becke (»Vedersakare»). När han icke var på färder i Gudhorms
eller egna ärenden, hvilket ej sällan hände, var han alltid i Gudhorms
närhet.

Stundom kom på besök å Märingaborg en man, som, när man såg honom i
ögonen, var Becke mycket lik och, likasom han, vältalig, rådklok och med
ett insmygande väsen. Han sade sig heta Blind. De som efteråt lärde
närmare känna honom, kallade honom Blind bölvise (»illsluge»).

Becke och Blind var samme man. I gudavärlden hette han Loke.

Gudhorm rådgjorde med honom om det förestående fälttåget, och syntes
Becke ega så stora insigter i krigföring, att Gudhorm gaf honom att
anföra sin härs ene flygel och gjorde honom till förste mannen i sitt
krigsråd.

Gudhorms här förenade sig med Svipdags. Hadding tågade dem till mötes.
Bland de slägter, som slutit sig till Svipdag och Gudhorm, var
gjukungarnes. Gjukungarne voro söner af Valands och Egils broder
Slagfinn och således syskonbarn med Svipdag och Vidga, Valands son.

Från Svitiod det stora kom Gudhorms fosterfader Hafle med jättemöarna
Fenja och Menja till Gudhorms krigshär. Till Haddings kom från Svitiod
det stora hans fosterfader Vagnhöfde med sin dotter Hardgreip, som var
klädd i en
\protect\hypertarget{lb1625905.xhtmlux5cux23start159}{}{}\protect\hypertarget{lb1625905.xhtmlux5cux23start159-a}{}{}\protect\hypertarget{lb1625905.xhtmlux5cux23start159-b}{}{}\protect\hypertarget{lb1625905.xhtmlux5cux23start159-c}{}{}\protect\hypertarget{lb1625905.xhtmlux5cux23start159-d}{}{}
krigares drägt. Midgard hade ditintills aldrig och har sällan efteråt
sett så stora härar som de, hvilka nu ryckte emot hvarandra. De sträckte
sig öfver bärg och dalar. När de stodo i ordnade fylkingar till häst och
fot, liknade deras spjutmassor en oöfverskådlig sträcka af sädesfält,
mognade till skörd, och när det kom till slag, bröto de sig mot
hvarandra som bränning mot bränning utefter hafvets brädd.

Flere drabbningar egde rum, innan det för lång tid afgörande slaget
stod. Natten före hvarje drabbning sågo härarne öfverjordiska skepnader
likasom kämpa med hvarandra i stjärneljuset, och de igänkände i dem sina
gudomlige gynnare och motståndare. Öfver Nordens och Västgermaniens
fylkingar sväfvade ryttare, liknande Ull, Njord, Fröj och Vanaheims
makter; öfver Haddings tyckte man sig se Oden rida med Tyr, Vidar, Vale
och Brage, samt Tor bland molnen åka i sin char.

Man iakttog vidare, att när hagel- eller stormmoln förmörkade himmelen
och från den sida, där Nordens och Västerns fylkingar stodo, föllo öfver
Haddings skaror -- då kom i fladdrande mantel och lysande hjälm en
ryttare, följd af molnmassor från andra hållet, och i dessa svängde Tor
sin blixtrande hammare och dref västerns hagelbyar och stormskurar
tillbaka.

Så kom det stora afgörande slaget. Haddings fylkingar fördes af Hamal,
hans fosterfader och fältöfverste, och på deras sida var det han och
Hildebrand med amaliska hjältar, med hildingar och ylfingar, samt
Vagnhöfde och Hardgreip, som åstadkommo det största manfallet. Å andra
sidan var det Svipdag, Vidga Valandsson, gjukungarne Gunnar och Högne,
samt Hafle, Fenja och Menja. Dessa stridsmör från Jotunheim vadade genom
gråbrynjade fylkingars böljor bland brutna sköldar och genomhuggna
brynjor. Hardgreip kämpade bredvid Hadding. Den ena hären var den andra
jämlik i mod och dödsförakt. Becke höll sig bakom stridslinien och
undvek att råka in i kampens tummel. Under hela slaget drog han icke
sitt svärd ur skidan och skulle varit den förste
\protect\hypertarget{lb1625905.xhtmlux5cux23start160}{}{}\protect\hypertarget{lb1625905.xhtmlux5cux23start160-a}{}{}\protect\hypertarget{lb1625905.xhtmlux5cux23start160-b}{}{}\protect\hypertarget{lb1625905.xhtmlux5cux23start160-c}{}{}\protect\hypertarget{lb1625905.xhtmlux5cux23start160-d}{}{}
att taga till flykten, om han sett de sinas leder vackla; men icke dess
mindre bidrog han till den utgång slaget fick, ty han hade ordnat
Gudhorms fylkingar med stor skicklighet, så att de på hans flygel bragte
Haddings i oordning genom angrepp från sidan och i ryggen.

Det stora slaget ändade så, att östgermanernas slagordning vardt bruten
och sprängd. I spridda flockar och förföljde af segervinnarne hastade de
från den med hopar af döde betäckta valplatsen, och var deras nederlag
så grundligt, att Hadding efter slagets slut var ingenting annat än en
värnlös flykting. Följd af Hardgreip kom han undan i en skog och
flackade någon tid omkring i ödemarker, innan han påträffades af Hamal
och Hildebrand, som förde honom tillbaka till Märingaborg, där han
återfann de andre sina bordvänner, som kommit med lifvet från detta
vapenskifte.

Medan Hadding och Hardgreip irrade på villsamma stigar, kommo de sent en
kväll till en boning, där de fingo härbärge öfver natten. Husbonden på
stället var död, men ännu icke begrafven. För att utröna hvilka öden
väntade Hadding, ristade Hardgreip målrunor i ett trästycke och lät
Hadding lägga det under den dödes tunga. Denne skulle då återfå
talförmåga och sia om framtiden. Det skedde ock. Men hvad den döde sjöng
med förfärande röst, det var en förbannelse öfver Hardgreip, som nödgat
honom återvända från lifvet i underjorden till jordelifvet, och en
förutsägelse, att en straff-ande från Nifelheim skulle hemsöka henne för
hvad hon gjort. En följande natt, då Hadding och Hardgreip sökt skydd i
en af grenar och ris hopfogad koja, visade sig i denna en jättestor
hand, som trefvade under taket. Den förfärade Hadding väckte Hardgreip,
som reste sig i hela sin jättekraft, grep fast i den hemlighetsfulla
handen och tillsade Hadding att afhugga den med svärdet. Han försökte,
men ur de sår hans vapen tillfogade spökhanden utrann mer etter än blod,
och den grep med sina järnklor Hardgreip och sönderslet henne. När
Hadding på detta sätt förlorat sin ledsagarinna, trodde han sig
öfvergifven af alla. Men då kom
\protect\hypertarget{lb1625905.xhtmlux5cux23start161}{}{}\protect\hypertarget{lb1625905.xhtmlux5cux23start161-a}{}{}\protect\hypertarget{lb1625905.xhtmlux5cux23start161-b}{}{}\protect\hypertarget{lb1625905.xhtmlux5cux23start161-c}{}{}\protect\hypertarget{lb1625905.xhtmlux5cux23start161-d}{}{}
Heimdall, den skinande guden, och förde honom på en stig, där Hamal och
Hildebrand återfunno honom. Heimdall följde städse sin ättlings öden och
ingrep till hans bästa, när omständigheterna ovilkorligt kräfde det.

\paragraph{XXXVII.}

\paragraph{VÄRLDSKRIGETS SLUT.}

Nu inträdde en mångårig fredstid, hvarunder Germanien styrdes af
Svipdags underkonungar och jarlar. Själf hade han återvändt till Asgard,
där han lefde i lycklig sammanvaro med Fröja och med henne födde sköna
döttrar Noss (»Smycke») och Gersime (»Klenod») samt en son Asmund,
hvilken han utsåg till konung öfver Norden. Hadding vistades under dessa
många år -- det säges trättio -- i Märingaborg. Östgermanien var efter
det stora slaget så blottadt på stridsmän, att ett nytt slägte af
vapenföre behöfde växa upp och i folkförsamlingen förlänas med spjut och
sköld, innan Haddings fana åter kunde höjas.

Becke förblef under dessa år hos konung Gudhorm och var hans rådgifvare
i allt. Om Beckes -- eller för att nämna hans rätta namn: Lokes --
rådslag kan i korthet sägas, att de gingo ut på att omintetgöra hvarje
försoningsförsök, samt att medels förtal och lögner öka anledningarna
till fiendskap mällan Halfdans och Ivaldesönernas afkomlingar, för att
desse skulle utrota hvarandra inbördes. Han ville bereda germanfolkens
undergång likasom deras gudars.

Gudhorm hade blifvit enkling och egde en förhoppningsfull son Randver,
nyss uppvuxen till yngling. Gudhorm hade hört omtalas, att Svanhild,
dotterdotter af Slagfinn och drottning öfver en af de nordiska
stammarne, skulle vara en den skönaste kvinna, och enär hennes make
aflidit, sände han Becke och sin son Randver att å hans vägnar fria till
henne. Hon svarade ja och åtföljde sändemännen till Gudhorm. När de
framkommit, sade Becke i hemlighet till denne, att
Randver\protect\hypertarget{lb1625905.xhtmlux5cux23start162}{}{}\protect\hypertarget{lb1625905.xhtmlux5cux23start162-a}{}{}\protect\hypertarget{lb1625905.xhtmlux5cux23start162-b}{}{}\protect\hypertarget{lb1625905.xhtmlux5cux23start162-c}{}{}\protect\hypertarget{lb1625905.xhtmlux5cux23start162-d}{}{}
och Svanhild bedragit honom, samt att Svanhild bragt sin förre make om
lifvet. Intetdera var sannt, men Becke förstod att göra det sannolikt,
och Gudhorm lät då hänga sin ende son och kastade Svanhild att
söndertrampas under hästar. Vid åsynen af hennes sköna ögon ville
hästarne ej trampa på henne. En slöja kastades då öfver hennes ansikte.

Det fanns två unga bröder Imbrecke och Fridla, söner af Harlung och nära
befryndade med Halfdans slägt. Några vilja veta, att Halfdan utanför
sina laggilla äktenskap hade två söner, nämligen denne Harlung och Rolf
med binamnet Krake. Om detta är riktigt, så voro harlungarne Gudhorms
brorsöner. Med lögnaktiga beskyllningar uppretade Becke Gudhorm emot
dem, och denne lät afdagataga dem båda.

Svanhild hade två bröder, Sörle och Hamder. Deras moder befallde dem att
fara till Gudhorm och kräfva hämd på honom för deras systers grymma död,
och innan de begåfvo sig på väg, sjöng hon öfver dem en galdersång, som
gjorde dem hårda mot smidda vapen. De kommo till Gudhorm, där han satt
vid dryckeshornet i sin sal, omgifven af kämpar. De stego fram, drogo
sina svärd och höggo in på honom. Hans kämpar rusade upp och angrepo
dem. Äfventyret slutade så, att Gudhorm fick ett svärdstygn i sidan och
flera sår, hvaraf han alltsedan led, samt att Sörle och Hamder stenades
till döds, sedan man funnit, att udd och svärdsegg ej beto på dem.

{\_\_\_\_\_\_\_\_}

Det tilldrog sig nu stora händelser i gudavärlden. Den splittring som
rådde mällan asar och vaner, gaf jättarne hopp om att kunna göra en stor
eröfring. Alla deras stammar förenade sig för att angripa och ödelägga
Midgard, och de föreslogo Oden, att om han bistode dem mot de gemensamme
fienderna, vaner och alfer, skulle de förhjälpa honom att återtaga sitt
högsäte i Asgard. Men Oden, som tänkte mer på människornas väl än på sin
egen makt och härrlighet, skyndade att underrätta vanerna om det anfall,
som hotade, och
\protect\hypertarget{lb1625905.xhtmlux5cux23start163}{}{}\protect\hypertarget{lb1625905.xhtmlux5cux23start163-a}{}{}\protect\hypertarget{lb1625905.xhtmlux5cux23start163-b}{}{}\protect\hypertarget{lb1625905.xhtmlux5cux23start163-c}{}{}\protect\hypertarget{lb1625905.xhtmlux5cux23start163-d}{}{}
han lofvade dem sin hjälp, om de behöfde den. Den var högligen af nöden,
ty de härmassor, som samlat sig i Jotunheim och ryckte dels mot Bifrosts
norra brohufvud, dels öfver Elivågor in i Svitiod det stora, voro
ofantliga. Men asarnes, vanernas och alfernas förenade makt tillbakaslog
dem med oerhörd manspillan. Elivågor voro så uppfyllda af slagna jättars
kroppar, att man med svårighet rodde skeppen genom böljorna, och man
kunde å ömse kusterna göra en tre dagars ridt utan att se annat än
fallnes lik. Detta krig, som är kändt vid namnet hunkriget, bröt så
jättarnes makt och förminskade så deras antal, att de alldrig mer varda
farliga för världsträdet och Midgard förr än kort före Ragnarök, då en
ny fimbulvinter skall inträffa och jättefolken resa sig i sin gamla
styrka. Intill dess är Tors hammare tillräcklig att hålla deras tillväxt
inom vissa gränser.

Vanerna erkände, att Oden handlat högsinnadt, när han kom dem till
hjälp, ehuru de drifvit honom från Asgard. De insågo äfven, att de voro
i största behof af asafaderns och hans väldige söners bistånd i kampen
mot jotunvärlden. När hunkriget börjat, erbjödo de fördenskull asarne
försoning på följande vilkor: De må återvända till sina borgar i Asgard,
och Oden med en faders och herrskares hela rätt intaga sitt högsäte i
Valhall. De kämpar, som å Haddings sida fallit å valplatserna och åt
hvilka Oden under sin landsflykt anvisat boningar och lekfält i
underjorden, må följa honom till Valhall och som einheriar njuta sitt
sällhetslif där. Men de kämpar, som fallit på Svipdags sida, kvarstanna
i Fröjas sal Sessrymner, och hädanefter skola Oden och Fröja kåra
hälften hvar af det antal, som faller på slagfälten i Midgard. Vanerna
frikännas från hvarje ansvar för de af asafaderns gärningar, som de
ogillat, och till tecken på denna ansvarsfrihet må Njord, utan att vara
bunden af sin plikt som gisslan, återvända till Vanaheim i tidernas
fullbordan. Vanerna anse sig hafva fått bot för Gullveigs död; skulle
hon pånyttfödas och åter visa sig, må hon icke brännas, utan förvisas
till Järnskogen.

\protect\hypertarget{lb1625905.xhtmlux5cux23start164}{}{}\protect\hypertarget{lb1625905.xhtmlux5cux23start164-a}{}{}\protect\hypertarget{lb1625905.xhtmlux5cux23start164-b}{}{}\protect\hypertarget{lb1625905.xhtmlux5cux23start164-c}{}{}\protect\hypertarget{lb1625905.xhtmlux5cux23start164-d}{}{}

Dessa villkor antogos, och asarne tågade på Bifrost tillbaka in i
Asgard. Man såg därefter Tor och Ull, styffader och styfson, kämpa, som
förr, sida vid sida mot Jotunheims resar. De gamla banden voro
återknutna, och alla kände sig däröfver lyckliga.

Medan vanerna voro herrar i Asgard, hade bland människorna den tro
uppkommit, att gudarne nu fordrade större offer än i fädernas tid varit
brukligt, och att offer och böner, egnade på en gång åt flere eller alla
gudar, icke egde kraft att blidka och försona, utan borde enhvar af
gudarne hafva sin särskilda offertjänst. Många trodde också, att bönerna
i mån af sin längd och offren i mån af sin riklighet skulle af gudarne
anses vittna om större fromhet, samt påräkna villigare bönhörelse. Men
Oden lät förkunna för människorna, att detta var villfarelse. Den som af
egennytta frambär rikligare offer, för att af gudarne få vederlag i
rikligare belöning, är Oden mindre kär än den, som frambär en ringa
gärd, men gör det med fromt och oegennyttigt sinnelag.

Det återstod för gudarne en vigtig sak att ställa till rätta. Fred rådde
nu i gudavärlden; men freden i Midgard hotades på nytt af fäjd mällan
Halfdans söner. Det måste göras en ände på broderstriden, som nu igen
höll på att flamma upp, ty Östgermaniens stammar samlade sig åter under
Haddings fana. Gudarne läto Hadding veta, att \emph{de} icke skulle
ogilla, om han afstode från blodshämden, som ju vore omöjlig för en
människa att fullgöra mot en Asgardsinnevånare och medlem af gudarnes
krets. Då gick Hadding med sina rådgifvare Hamal och Hildebrand till
öfverläggning. Blind, som ofta kom till Märingaborg, deltog nog i
öfverläggningen och torde sagt det vara mäst ärofullt för Hadding att
gifva gudarne nej till svar; men Hamal och Hildebrand rådde honom svara,
att han ville afvakta Svipdags beslut, och detta svar afgaf han. Gudarne
uppmanade nu Svipdag att bjuda Hadding försoning, och den andel, som
tillkom honom af hans faders rike; men Svipdag genmälde, att han alldrig
förnyar ett afslaget anbud. Asar och vaner gingo då till sina
domaresäten och dömde
\protect\hypertarget{lb1625905.xhtmlux5cux23start165}{}{}\protect\hypertarget{lb1625905.xhtmlux5cux23start165-a}{}{}\protect\hypertarget{lb1625905.xhtmlux5cux23start165-b}{}{}\protect\hypertarget{lb1625905.xhtmlux5cux23start165-c}{}{}\protect\hypertarget{lb1625905.xhtmlux5cux23start165-d}{}{}
Svipdag att göra det. Men hvarken domen eller Fröjas tårar kunde beveka
honom. Han begaf sig från Asgard ned till skandiska halfön och befallde
sin son Asmund, som var svearnes konung, att samla sitt folk samt skicka
budkafvel till danerna och till Gudhorms stammar.

Inom kort voro de germaniska härarna åter i rörelse mot hvarandra.
Gudarne sände den trotsige Svipdag en yttersta hotande befallning att
ställa sin vilja under de laggilla världsstyrande makternas beslut. Han
svarade nej och seglade med den stora sveaflottan, starkt bemannad,
österut. Men innan flottan landat, var han försvunnen. Gudarnes vrede
hade drifvit honom att kasta sig i hafvet. Där märkte han med fasa, att
han var förvandlad i en besynnerlig djurskepnad. Förödmjukad och
förtviflad dök han ned i djupet. Gudarne dolde hans öde för Fröja.

Svearne måste undrat öfver sin store höfdings försvinnande och anat en
ond utgång på ledungsfärden. De landstego och tågade under konung
Asmunds ledning in i landet. Där förenade sig Gudhorms här med dem.
Becke följde hären äfven nu. En natt steg en högvuxen gammal man, enögd
och långskäggig, in i Asmunds tält, där denne satt i samtal med Becke.
Den gamle kallade sig Jalk. Han sade, att om Asmund icke eftersträfvade
brödrakrig och frändedråp, kunde strid ännu undvikas. Becke däremot
torde invändt, att om Asmund slöte fred och tågade hem, skulle han bryta
mot det åliggande han fått af sin fader, och det skulle tillräknas honom
som ett bevis på feghet. Man kan föreställa sig, att Jalk och Becke
betraktade hvarandra: att Odens skarpa öga såg in i Lokes med hotfull
glöd, och att denne svarade med ett fräckt och gäckande ögonkast. Jalk
aflägsnade sig. Mällan Asmund och Becke vardt det öfverenskommet, att
den senare skulle anordna härens uppställning och rörelser, ty däri hade
han visat sig vara mästare. Asmund skulle föra den i striden.

Oden red därifrån till Haddings läger. Man väntade fältslag nästa
morgon. Oden steg in i den gamle Hamals tält
\protect\hypertarget{lb1625905.xhtmlux5cux23start166}{}{}\protect\hypertarget{lb1625905.xhtmlux5cux23start166-a}{}{}\protect\hypertarget{lb1625905.xhtmlux5cux23start166-b}{}{}\protect\hypertarget{lb1625905.xhtmlux5cux23start166-c}{}{}\protect\hypertarget{lb1625905.xhtmlux5cux23start166-d}{}{}
och talade länge med honom. Han undervisade honom i ett ditintills icke
kändt sätt att uppställa fylkingarna, ett sätt, som gjorde Beckes
krigskunskap om intet. När han var förvissad om att Hamal fattat allt,
frågade han, om alla de till Haddings läger väntade kämparne anländt.
Hamal hörde sig för och erfor, att Vagnhöfde saknades.

Vagnhöfde hade sent blifvit träffad af Haddings bud och hade lång väg
till vapentinget. Men en enögd ryttare, som kallade sig Kjalar, mötte
honom på vägen, tog honom upp på sin häst och förde honom öfver vatten
och land, så att han, när slaget var som hetast och Hadding bäst behöfde
hans hjälp, stod i dennes sköldborg. Kjalar var Oden, och från denna
färd kommer det ordstäf, hvarmed hjälp i sista stund menas, att nämligen
»Kjalar drog kälke», ty äfven Vagn och Kälke har Vagnhöfde varit kallad
af skalderna.

Solen rann upp och härarna ryckte mot hvarandra. Sköldsången uppstämdes
å båda sidor; men denna gång sjöngo inga gudar med under nord- och
västgermanernas sköldar, medan valfaders röst samklingade med det
stigande bruset under östgermanernas. Västerns spjutskogar bildade
långsträckta fyrkanter; österns visade sig som kilar med spetsarne mot
fienden och med sköld- och svärdklädda sidor. Det var den uppställning,
som Oden under natten lärt Hamal. Att ordna trupper så har fördenskull
länge kallats att »fylka hamalt» (äfven »trynfylka»). Tecken antydde och
slagets utgång sannade, att samtlige gudar, vaner såväl som asar, nu
gynnade Haddings sak. Dock vägde stridens vågskålar länge lika, ty
Asmunds och Vidga Valandssons tapperhet öfvergick allt. Äfven Fenja och
Menja gingo hårdt fram, men fångades mällan sköldar och fördes bundna ur
striden. Asmund banade sig väg fram emot Hadding själf med skölden
kastad på rygg och med båda händerna kring fästet på ett slagsvärd, som
fällde allt omkring sig. Då ropade Hadding på asarnes bistånd, och
plötsligt stod Vagnhöfde, ditförd af Oden, vid hans sida, svängande mot
Asmund ett kroksvärd, medan Hadding kastade mot honom sitt spjut. Asmund
stupade under
\protect\hypertarget{lb1625905.xhtmlux5cux23start167}{}{}\protect\hypertarget{lb1625905.xhtmlux5cux23start167-a}{}{}\protect\hypertarget{lb1625905.xhtmlux5cux23start167-b}{}{}\protect\hypertarget{lb1625905.xhtmlux5cux23start167-c}{}{}\protect\hypertarget{lb1625905.xhtmlux5cux23start167-d}{}{}
dessa vapen. Hadding trängde därefter fram mot Vidga Valandsson, som
säges i detta fältslag ha med egen hand nedlagt flere hundra man. När
han såg Hadding komma, svängde han om sin häst och flydde, och då
Hadding upphann honom, lät han hällre döda sig af denne, än han lyfte
vapen mot honom. Detta för att hålla löftet, som han i Märingaborg gaf
Hamal och Hildebrand.

Då man fick veta, att Asmund fallit, uppstacks hvit sköld, och fienderna
räckte hvarandra händer till fred. Haddings härrskaremakt i östern
godkändes.

Han vardt en mild och lyckosamt styrande storkonung, känd af
eftervärlden äfven under namnet Tjodrik (Ditrik, »storkonungen»). På
världskriget i Midgard följde långvarig och lyckosam fred. Hadding och
en öfverlefvande sonson af Svipdag förenades af så innerlig
tillgifvenhet, att den senare vid en ogrundad underrättelse om den
förres död beröfvade sig själf lifvet. Och när Hadding erfor detta,
ville ej häller han lefva längre, utan gick genom frivillig död till
Valhall.

Då Svipdag icke afhördes, och ingen syntes veta något om hans öde, tog
den sörjande Fröja sin falkham och flög genom alla världar, letande
efter den älskade. Hon fann honom till slut vid ett skär i hafvet,
Singastein, äfven kalladt Vågaskär. Huru det blef henne bekant, att
djuret, i hvars ham Svipdag blifvit fängslad, omslöt denne, är numera
icke med visshet kändt. Man vet blott, att så afskyvärdt vidundret
föreföll henne, öfvervanns afskyn af hennes kärlek och medlidande, och
hon kvarstannade hos den olycklige och sökte med sin ömhet trösta honom.
Hon medförde Brisingamen, och antingen det nu var detta underbara
smyckes glans eller att själfva böljorna kände fröjd öfver att få hafva
det sköna anletet och det trofasta hjärtat hos sig -- från Singastein
spred sig vida öfver hafvets spegel ett härrligt skimmer, som efteråt
sällan varit sedt. Fröja bär alltsedan binamnet Mardöll
(»Hafs-skimmer»). När vidundret var vaket, försökte hon vara glad och
talade smeksamma ord. När det sof,
\protect\hypertarget{lb1625905.xhtmlux5cux23start168}{}{}\protect\hypertarget{lb1625905.xhtmlux5cux23start168-a}{}{}\protect\hypertarget{lb1625905.xhtmlux5cux23start168-b}{}{}\protect\hypertarget{lb1625905.xhtmlux5cux23start168-c}{}{}\protect\hypertarget{lb1625905.xhtmlux5cux23start168-d}{}{}
kunde hon obemärkt hängifva sig åt sin sorg och gråta. Hvarje hennes tår
vardt till en gulddroppe, som sjönk i hafvet, men ej gick förlorad, ty
det droppade en lika tyngd af medlidande med Svipdag och henne in i
gudarnes hjärtan, och som de med oro visste henne vara borta från
Asgard, vardt det deras beslut att förlåta Svipdag och återkalla honom.

Innan detta beslut hann värkställas, hände det, att Hadding en varm
sommardag kom ned till stranden, utanför hvilken Vågaskär ligger, och
gick i vattnet för att bada. Där råkade han i strid med ett besynnerligt
djur och dräpte det. Han drog det upp på land, för att de stridsmän
skulle få se det, hvilka åtföljde honom och slagit läger ett stycke
därifrån. Men då han återvände till lägret, ställde sig i hans väg en
kvinna, den skönaste han någonsin sett. Hon sade, att han dräpt ett i
djurham höljdt heligt väsen, hennes make, Fröjas make, och hon
nedkallade öfver honom gudarnes och alla elementers vrede, om han icke
försonade det begångna dråpet. Hadding förstod då, att han dödat Svipdag
och sålunda ändtligen tagit hämd på sin faders bane. Detta gladde honom,
och han vägrade gifva böter. Men då rönte han så många missöden och
motgångar och såg omkring sig så många järtecken, att han bekvämde sig
gifva böter åt Fröj för dråpet på hans svåger, och skedde detta genom en
stor offergärd åt honom, Fröjs-bloten, som alltsedan årligen egnas
honom.

Då Fröja såg Svipdag uppdragas död å stranden och skyndade sig att
ställa sig i vägen för Hadding, hade hon glömt Brisingamen kvar på
Vågaskär. Det låg där och upplyste näjden rundt omkring. Loke, som vid
denna tid än i synlig, än i osynlig måtto höll sig i Haddings närhet,
såg smycket, och kom genast på tanken att stjäla det. Det kunde vara
godt att hafva, för att ega en lösen för sitt lif, ty alla gudars vrede
hvilade på honom, och han visste, att Oden numera kände sig obunden af
den ed han svurit honom i tidernas morgon. Men äfven Heimdalls blick
följde Hadding och såg hvad som föregick omkring honom. Medan Fröja
\protect\hypertarget{lb1625905.xhtmlux5cux23start169}{}{}\protect\hypertarget{lb1625905.xhtmlux5cux23start169-a}{}{}\protect\hypertarget{lb1625905.xhtmlux5cux23start169-b}{}{}\protect\hypertarget{lb1625905.xhtmlux5cux23start169-c}{}{}\protect\hypertarget{lb1625905.xhtmlux5cux23start169-d}{}{}
talade till Hadding, kröp en säl upp på Vågaskär och närmade sig
Brisingamen. Det var Loke i sälham. Men från andra sidan kröp också en
säl upp på klippan fram till smycket. En ham kan ej förvandla blicken,
och Loke igenkände i den andra sälens ögon sin gamle fiende Heimdall.
Sälarne kämpade, och Farbautes onde son måste draga sig tillbaka med
oförrättadt ärende. Heimdall återförde Brisingamen till Asgard.

Fröja återvände dit, emottogs efter sin långa bortovaro med glädje och
fördes in i Valhall till Oden. Öfverst bland einheriarne och närmast
gudarne satt där Svipdag, så ung och vacker som i den stund, då han kom
med Valandssvärdet till Asgard. Tvisten mällan gudarne och honom var nu
bilagd och hans fel försonade, och Fröja och han kunde åter lefva
lyckliga tillsammans i Folkvangs salar.

Gudarne sammanträdde på sin tingsstad, för att rådgöra om vigtiga
ärenden. Fred var återställd i världen, men de onda makterna gingo ännu
lösa. Man beslöt, att de skulle bindas och aflägsnas ur gudars och
människors närhet. Mimers hufvud hade sagt, att Gullveig var pånyttfödd.
Loke hade efter tvekampen på Vågaskär försvunnit. I Asgard fanns
Lokesonen Fenrer ännu. Man visste nu, att han var född till gudarnes
skada och borde förskaffas därifrån. Detta vardt beslutadt. Döda Fenrer
kunde de icke, emedan han fått löfte om säkerhet till lifvet.

Fenrer hade Lokes lekfulla skick och falska ögon. Han var nu så stor och
hade emällanåt röjt så våldsamt lynne, att man icke vågade låta honom gå
fri omkring i Asgard. Han hade nog vid mer än ett tillfälle visat lust
att sönderslita Gere och Freke. Fördenskull hölls han nu inom ett högt
stängsel, och hade Tyr åtagit sig den vådliga sysslan att gå dit in och
gifva honom hans dagliga föda.

Redan förut hade gudarne låtit Fenrer pröfva sin styrka på en mycket
stark järnkedja, hvarmed han lät binda sig, väl vetande, att han kunde
spränga den, och det gjorde han i ett enda ryck. Nu gjordes en vida
starkare, och ulfven, stolt öfver sin styrka, lät lägga henne kring sig.
Med en
\protect\hypertarget{lb1625905.xhtmlux5cux23start170}{}{}\protect\hypertarget{lb1625905.xhtmlux5cux23start170-a}{}{}\protect\hypertarget{lb1625905.xhtmlux5cux23start170-b}{}{}\protect\hypertarget{lb1625905.xhtmlux5cux23start170-c}{}{}\protect\hypertarget{lb1625905.xhtmlux5cux23start170-d}{}{}
enda spänning kom han kedjan att brista, så att länkarne flögo åt alla
håll. Svipdag begaf sig då till underjorden, för att rådgöra med de
dvärgar, som plägat betjena Mimers söner i deras smedjor och ännu smidde
där. Dvärgarne lofvade försöka sin konst. De visste, att hvad det grofva
ej kan fjättra, kan måhända bindas af det lena. Efter någon tid hade de
smidt åt gudarne det band, som kallas Gleipner, fint, smalt, mjukt och
spänstigt utan like, gjordt af ämnen, som människans ögon knappt kunna
se och hennes hörsel knappt förnimma. Tyr visade ulfven detta band och
sporde, om han hade styrka att slita det. Fenrer svarade, att det såg så
ömkligt svagt ut, att man hällre borde ställa den frågan till ett barn
än till honom. Han kunde ej förvärfva någon heder med att spränga det.
Tyr sade, att bandet var vida starkare än det såg ut. Tor kunde nog
sönderslita det, men de andre gudarne svårligen. Fenrer anmärkte, att
bandet då måtte vara gjordt med list, och i detta fall vore det dumt af
honom att låta lägga det på sig. Tyr torde genmält, att han alldeles
icke ville dölja, att bandet var gjordt med list, ty frågan var nu den,
om dvärgarne med all sin konst förmådde uträtta något mot Fenrers
styrka. Man ville utransaka hvad som var starkast här i världen: våld
eller kunskap. Fenrer sade, att detta kunde vara skäl att pröfva. Han,
Fenrer, skulle fördenskull låta lägga bandet kring sig, men med det
villkor, att man gåfve honom pant för att det skulle tagas af honom, om
han icke fick makt med det. Tyr svarade, att pant skulle han få; men
emedan han icke ville bedraga Fenrer, borde denne veta, att mottagaren
af en pant har ingen rätt att säga sig sviken, om han får behålla
panten. Han frågade, om Fenrer ville nöja sig med hans, krigaregudens,
högra hand i säkerhet. Fenrer tillkännagaf, att den panten syntes honom
god, ty dåraktig vore Tyr, om han uppoffrade en hand, hälst högra
handen, och en krigaregud, saknande de fingrar, som gripa om
svärdfästet, vore just ej häller någon förmån för Asgard. Fenrer
tillade, att han trodde sig redan vara kommen till den ålder och styrka,
att han kunde mäta sig med Tor.
\protect\hypertarget{lb1625905.xhtmlux5cux23start171}{}{}\protect\hypertarget{lb1625905.xhtmlux5cux23start171-a}{}{}\protect\hypertarget{lb1625905.xhtmlux5cux23start171-b}{}{}\protect\hypertarget{lb1625905.xhtmlux5cux23start171-c}{}{}\protect\hypertarget{lb1625905.xhtmlux5cux23start171-d}{}{}
Om någon tid skulle han nog kunna mäta sig med alla gudarne
tillsammanstagna, och han ämnade då icke låta sig hållas inom ett
stängsel, utan skulle han fordra samma frihet att gå lös som han hade
som unge. Lekfull vore han ännu, och gudarne skulle få lustiga lekar.

Tyr underrättade gudarne om öfverenskommelsen mällan honom och Fenrer.
För gudarne syntes den hård; ett sådant offer som det, hvartill Tyr var
färdig, kunde de ej af honom begära. Tyr torde genmält, att det var
nödvändigt för Asgards och världens trygghet. De gingo nu in till
Fenrer. Tyr stack sin högra hand i ulfvens gap, och denne lät gudarne
kringlägga bandet. När fängslet var färdigt, spände ulfven alla lemmar,
men ju mer han spände, dess hårdare drog sig bandet kring honom. Han
arbetade af alla krafter, men gagnlöst. När han då öfvertygat sig om att
Gleipner var starkare än han, väntade han, att Tyr skulle draga sin hand
ur hans gap. Men Tyr lät handen förblifva där. Detta måste hafva synts
ulfven öfver måttan märkvärdigt. Han hade bedragit sig själf, då han
icke trodde, att Tyr kunde gifva sin hand förlorad. Då sprutade ilska ur
Fenrers ögon, och han afbet intill handleden Tyrs hand och sväljde den.
Gudarne begåfvo sig med fången ned till Nifelheim. Där finns i marken
ett gap, som leder ned till en rad af stora grottor. Den yttersta
grottan har en öppning, genom hvilken man kommer till stranden af
Amsvartners haf, hvaröfver evigt mörker rufvar. Ute i hafvet ligger
holmen Lyngve, som liknar ett klippfäste. Inne i fästet äro hålor,
somliga fulla af eldslågor, andra icke. Ofvan på fästet står en underlig
skog, hvars träd äro vattenstrålar, uppslungade ur heta källor. En af
dessa hålor bestämdes till fängelse åt Fenrer. Till bandet, som
omsnärjer honom, fogade gudarne länkar, som fästes djupt nere i jorden
under väldiga stenblock. De satte ett svärd i hans mun med udden i
öfverkäken. Där ligger nu Fenrer intill Ragnarök. Ur hans gap flyter
fradga, som bildar ån Von. Hålan närmast innanför bestämde gudarne åt
Loke.

Det gällde nu att fånga denne. Det dröjde länge, innan
\protect\hypertarget{lb1625905.xhtmlux5cux23start172}{}{}\protect\hypertarget{lb1625905.xhtmlux5cux23start172-a}{}{}\protect\hypertarget{lb1625905.xhtmlux5cux23start172-b}{}{}\protect\hypertarget{lb1625905.xhtmlux5cux23start172-c}{}{}\protect\hypertarget{lb1625905.xhtmlux5cux23start172-d}{}{}
så skedde, eftersom Loke kunde förvandla sig i flere slags djurskepnader
och vistas i vattnet som på land. Men med dem, som ega denna förmåga, är
det så stäldt, att de snart vantrifvas, om de icke allt emellanåt
återkomma och oftast äro i sin rätta skepnad.

Det var vid den tid på året, då linet bärgas, och gudarne hade sitt
vanliga gästabud hos Öger. Vid dryckeshornen samtalade de om mycket,
äfven om Loke och hans efterletande. Då visade sig oförmodadt Farbautes
son i mjödhallens dörr, och steg in. Han visste, att gästabudssalen var
fridlyst, och äfven vägfrid förkunnad för dem, som gingo dit och
därifrån. Tor hade ännu icke kommit; men de andre gudarne, äfvensom
asynjorna, suto kring bordet. När gudarne sågo Loke, vardt det tyst i
salen. Loke sade: »Lång väg har jag vandrat, och törstig kommer jag hit,
för att bedja asarne om en dryck mjöd. Men hvarför tigen I? Ettdera af
de två: gifven mig säte vid gillet eller visen mig på dörren!» Då sade
Brage: »Här vid gillet får du icke säte; men en vedergällningens dryck
är beredd åt en viss, som gudarne känna.» Brage anspelade härmed på ord,
som Skade sagt, innan Loke kom, att när de fångat honom och lagt honom i
bojor, skulle hon fästa en ettersprutande orm öfver hans mun. Loke vände
sig till Oden och påminde om, att de i ungdomen blandat blod samman, och
att han lofvat icke dricka mjöd, utan att det bures till dem båda. »Stig
upp, Vidar», sade då Oden, »och gif ulfvens fader säte vid bordet, så
att han icke förbittrar vårt gille med lastande ord!» Vidar steg upp och
skänkte i åt Loke, som sade, innan han drack: »Hel asar, hel asynjor och
alla heliga gudar, Brage dock undantagen!» Brage genmälde: »Vill du låta
bli att visa gudarne din ilska, så skänker jag dig af mitt eget gods
häst och svärd.» Med dessa ord hänsyftade Brage därpå, att om gudarne
lyckades lägga Loke i bojor, skulle han få en häst af skarpkantiga
hällar att rida och Brages svärd stäldt med udden i hans rygg. Loke
förstod ej antydningen, men begrep väl, att det låg hån i den, och han
öfveröste först Brage, därefter
samtliga\protect\hypertarget{lb1625905.xhtmlux5cux23start173}{}{}\protect\hypertarget{lb1625905.xhtmlux5cux23start173-a}{}{}\protect\hypertarget{lb1625905.xhtmlux5cux23start173-b}{}{}\protect\hypertarget{lb1625905.xhtmlux5cux23start173-c}{}{}\protect\hypertarget{lb1625905.xhtmlux5cux23start173-d}{}{}
närvarande gudar och gudinnor, den tyste Vidar dock undantagen, med de
värsta smädelser, och skröt med sina ogärningar. Därmed fortfor han
ännu, när Tor steg in i salen. Äfven denne fick några stickord; men då
Tor vardt vred, tystnade Loke och gick ut, sedan han önskat värden på
stället allt ondt.

Han kom oantastad från denna färd för vägfridens skull. Men några dagar
därefter började man efterleta honom. Från Lidskjalf iakttog man, att
högt uppe i Svitiod det kalla var på ett bärg en egendomlig boning
uppförd, som dit intills icke varit sedd i den ödemarken, och man beslöt
taga reda på hvem nybyggaren var. Det var Loke, som uppfört ett hus med
fyra dörrar, så att han kunde se ut därifrån åt alla håll. Stället var
så tillvida väl valdt, att strax därbredvid brusade en djup och strid
älf, Franangers fors, dit Loke, om fara nalkades, kunde taga sin
tillflykt. Han tillbragte sina dagar med att i laxskepnad fara ned och
uppför forsen, när han icke satt vid härdens eld och grubblade eller
förehade något handarbete. Hvarjehanda saker, såsom lin och garn och
virke, hade han fört dit till ämne för tidsfördrif. Så kom han en dag på
den tanken: »Huru skulle jag själf bära mig åt, för att kunna fånga
någon, som gjort sig till lax och simmade i forsen?» Han satt en stund
eftertänkande; därefter tog han garn och band det första nät, som
blifvit gjordt. Under det han sysslade med nätet, kommo Tor och några
andra af asarne så hastigt öfver honom, att han nätt och jämt hann kasta
nätet i elden och springa ut och ned i älfven. Då asarne kommit in i
huset och sågo på härden de rutor af aska, som det uppbrända nätet
bildade samt garnet, som låg därbredvid, förstodo de, att Loke här gjort
en uppfinning, hvarmed han själf kunde fångas, och de satte sig genast
och bundo ett ganska drygt nät efter det mönster de hade på härden. När
det var färdigt, gingo de ut, kastade nätet i forsen och drogo det ned
efter älfven; Tor gick och drog vid ena älfkanten; några af asarne vid
den andra. Medan de drogo, simmade Loke i laxskepnad framför nätet,
tills han
\protect\hypertarget{lb1625905.xhtmlux5cux23start174}{}{}\protect\hypertarget{lb1625905.xhtmlux5cux23start174-a}{}{}\protect\hypertarget{lb1625905.xhtmlux5cux23start174-b}{}{}\protect\hypertarget{lb1625905.xhtmlux5cux23start174-c}{}{}\protect\hypertarget{lb1625905.xhtmlux5cux23start174-d}{}{}
kom till två nära hvarandra varande stenar. Mällan dem stannade han, och
det gick som han beräknat: att nätet gled öfver och förbi honom. Gudarne
märkte emällertid, att nätet rört vid något lefvande. De vände då om,
bundo tyngder vid nätet, så att ingenting kunde slinka under det, och
drogo det på nytt ned för älfven. Loke simmade åter framför nätet. På
detta sätt kommo de ned till älfmynningen; men som Loke ej ville ut i
saltvattnet, vände han sig, hoppade öfver nätets öfversta kant och
arbetade sig åter uppför forsen. För andra gången gingo då asarne
tillbaka. De delade sig nu i två hopar, som å hvar sin sida drogo nätet,
medan Tor vadade midt i forsen, och så kommo de igän till älfmynningen.
Då vände Loke och gjorde ett raskt hopp öfver nätet; Tor grep efter
honom och fick honom om lifvet, men han gled i handen på honom och
skulle kommit undan, om icke Tor fått bättre håll om stjärten. Loke var
nu fången, och väl kommen ur älfven, återfick han sin vanliga skepnad
med de listiga och fräcka anletsdragen och den af Sindres syl märkta
munnen. Gudarne förde honom till den håla å Lyngveholmen i Amsvartners
haf, som ligger innanför den, där Fenrer är fängslad. Strax efter att
han blifvit fångad, kom en jättekvinna, Sigyn, som var hans laggilla
hustru och honom i mycket olik, och bad, att hon måtte få dela Lokes
öde, huru än gudarne bestämde det. Då de nu rodde till Lyngveholmen, var
fördenskull Sigyn med i båten, och när de rodde därifrån, satt hon
bredvid den olycklige i hans straffhåla. Där ligger Loke på tre å kant
ställda hällar, en under hans skuldror, den andre under hans länder, den
tredje under hans knäveck. Bojorna, som fängsla honom, äro gjorda af en
ulfs tarmar. Ulfven var en af Lokes söner, som vardt söndersliten af en
annan ulf, hans broder. Ett svärd är af Brage stäldt med udden i Lokes
rygg, till hämd för att han bortröfvade Idun, och öfver hans ansikte
fäste Skade en giftorm, som sprutar sitt etter ned mot hans mun, detta
till straff för att Loke vållade Valands död. Men Sigyn håller en skål
under etlerdropparne. När skålen är full och hon slår ut
\protect\hypertarget{lb1625905.xhtmlux5cux23start175}{}{}\protect\hypertarget{lb1625905.xhtmlux5cux23start175-a}{}{}\protect\hypertarget{lb1625905.xhtmlux5cux23start175-b}{}{}\protect\hypertarget{lb1625905.xhtmlux5cux23start175-c}{}{}\protect\hypertarget{lb1625905.xhtmlux5cux23start175-d}{}{}
ettret, kommer ormgiftet i hans mun. Då vrider sig Loke af smärta så
våldsamt, att jorden skälfver. Så skall han ligga intill Ragnarök. Under
tiden förvandla sig håren å hans hufvud till hornhårda spjut, som gömma
stinkande pästgift i sig. Att afrycka ett sådant hårstrå vore farligt.

Äfven Gullveig fångades af gudarne. Hon förvisades till Järnskogen och
förblifver där, af mäktiga galder tvingad, till Ragnarök. Hon uppsökte
Eggter, som i Järnskogen förvarar Valands segersvärd och vallar
ulfhjordarne, hennes och Lokes barn. Järnskogen är uppfylld af trolldom
och fasor. Dalarne mällan de svarta, stormpiskade, vildt splittrade
bärgen fyllas af nästan ogenomträngliga skogar eller af träsk, hvari
vidunderliga giftdjur vältra. Vindarnes oafbrutna tjut i de tusenåriga
trädens malmhårda, dolkhvassa blad fyller hjärtat med ångest och
förvirrar sinnet. Nattetid likna de från fjällen nedstörtande vattnen
eldforsar, och giftiga flammor fladdra öfver marker, där ingen blomma
trifves.

{\_\_\_\_\_\_\_\_}

Sedan ondskans värste främjare blifvit sålunda oskadliggjorda, råder
endrägt i gudavärlden, och i världslifvet all den ordning, som under det
nuvarande tidehvarfvet är möjlig. De besjungna stora urtidshändelsernas
ålder, »slägtenas morgontid» (de mytiska händelsernas ålder) är därmed
afslutad. De fleste hade sitt upphof i Lokes och Gullveigs ränker och i
Jotunheims maktutveckling. Men jättarnes makt är för lång tid bruten
genom hunkrigets utgång, Loke ligger i länkar, Gullveig är bannlyst från
himmelen och Midgard. Oden förfogar öfver Mimers ord och Mimers tankar,
och om mällan de olika gudaslägterna eller enskilda gudar en tvist
uppstår, kommer den ej till människors kännedom, ty tvisten har inga
följder för Midgard. Forsete, Balders son, förliker alla sådana mål.
Gudarne ha haft en lärotid, äfven de. Den var nu fulländad.

Sällan förete sig gudarne numera för människornas ögon. Förbindelsen dem
emällan är dock alldrig afbruten. Den
\protect\hypertarget{lb1625905.xhtmlux5cux23start176}{}{}\protect\hypertarget{lb1625905.xhtmlux5cux23start176-a}{}{}\protect\hypertarget{lb1625905.xhtmlux5cux23start176-b}{}{}\protect\hypertarget{lb1625905.xhtmlux5cux23start176-c}{}{}\protect\hypertarget{lb1625905.xhtmlux5cux23start176-d}{}{}
upprätthålles genom böner, offer och tämpeltjänst, men framför allt
genom en lefnad, som ställer sig Urds och gudarnes stadgar till
efterrättelse. Dock händer någon gång, att när en germanhär rycker till
strid, deras siare skåda Oden rida framför fylkingarna, och än oftare
hör man hans röst i sköldsången. Den här-uppställning, som han lärde
Hamal, iakttages ständigt: alltid fylka sig germanernas skaror i
kilform, emedan den ordningen är dem helig. De välska folken bakom Ren
och de höga bärgen hafva lärt Lokes sluga krigskonst, men ehuru många
germanhöfdingar känna denna slagordning väl och själfve anfört välska
härar, använda de den alldrig hos sina landsmän.

Stundom hör man Oden fara genom rymden, följd af asar och asynjor och
Asgards ulfhundar. Då är storm i luftkretsen, som Oden ränsar från
sjukdomsandar och andra skadliga väsen.

Än oftare höres Tor åka i molnen och ses hans blixtar, kastade mot
Bergelmers ättlingar. De skulle åter varda för många, minskade han dem
icke med sin ljungeld. Stundom lär han i mänsklig skepnad vandra i
Midgard och tillse landtmännens arbete. Tor tycker om de odlare, som
arbeta ej blott för eget bästa, utan för kommande slägtens. Den som
bygger endast med tanke på att det skall stå i hans tid, och den som ej
vill lägga ett frö eller sätta en telning, emedan han själf ej får njuta
trädets skugga och frukt, han gynnas ej af Tor, och straffet drabbar
honom, den själfviske, eller hans ättlingar.

Alla jättar, som bosatte sig i Midgard under fimbulvinterns dagar, hafva
icke återvändt till Jotunheim. Här och där bor någon kvar, särdeles i
vilda fjällbygder eller djupa skogar. Någon gång kan vandraren höra
deras betande hjordars pinglor från trakter, som sällan beträdas af
människofot. Mällan desse jättar och deras mänskliga grannar består ett
tyst fördrag: om grannen icke oroar jätten eller hans boskap, oroar han
ej häller grannen och hans egendom. Någon gång kan till och med välvilja
och tjenstaktighet dem emällan råda. Oftare
\protect\hypertarget{lb1625905.xhtmlux5cux23start177}{}{}\protect\hypertarget{lb1625905.xhtmlux5cux23start177-a}{}{}\protect\hypertarget{lb1625905.xhtmlux5cux23start177-b}{}{}\protect\hypertarget{lb1625905.xhtmlux5cux23start177-c}{}{}\protect\hypertarget{lb1625905.xhtmlux5cux23start177-d}{}{}
händer dock, att jättar hemligen göra intrång på fördraget: att vackra
människodöttrar bärgtagas af dem; att de förvända syn på ensamma
vandrare och narra dem på villostigar, och att ifall de fått ett
sjukligt barn, de smyga till en människoboning och lägga det i vaggan
och bortföra det friska och välbildade människobarnet. Sådana
bortbytingar äta otroligt mycket. Är den sitt eget barn beröfvade modern
medlidsam mot bortbytingen, inträffar det, att jättemodern däraf röres,
samt återlämnar det tagna barnet, lägger guld till det i vaggan och
förer bortbytingen tillbaka till bärget.

Mångfaldiga väsen af alfernas och dvärgarnes slägter syssla nu som
fordom i världshushållningen och främja under Fröjs tillsyn grodd och
växtlighet. Alferna äro sköna och välvilliga, men snarstuckna och, om de
förnärmats, hämdlystna. De hämnas med att afskjuta osynliga pilar
(»älfskott»), som förorsaka sjukdom. Eller om någon är för närgången mot
dem, då de ej vilja störas, såsom när de i månljusa nätter tråda
ringdans på ängarna, kan det hända, att den alf, som han kommit för
nära, andas på honom; han får då »älfvablästen». Alfdiserna äro
obeskrifligt väna att åse.

I somliga bärg hålla dvärgar till, som skapa malmådror därinne och
smida.

Ett slags dvärgar hålla sig till människogårdarne och främja årsväxten
icke på själfva marken, utan inne i ladorna, där de föröka den bärgade
grödan och göra henne dryg. Till hvarje nybyggd tomt kommer en sådan
dvärg och stannar där, om han finner vistelsen behaglig. Och behaglig är
den honom, om där råder endrägt, god vilja, flit och renlighet, samt
vänlighet mot husdjuren. För sitt arbete fordrar han blott ett ringa
mjölkoffer hvarje jul. För öfrigt ser han hälst, att man icke talar till
honom eller låtsar om honom.

När hungersnöd eller farsot förestår, säges det, att några kunna se
Leikin vara ute, ridande på en trebent häst.

Jägare och kolare må akta, att de icke låta bedraga sig af syner, som de
få i skogen. En och annan af Gullveigs fränkor kan hålla till under
furors och granars skymning och
\protect\hypertarget{lb1625905.xhtmlux5cux23start178}{}{}\protect\hypertarget{lb1625905.xhtmlux5cux23start178-a}{}{}\protect\hypertarget{lb1625905.xhtmlux5cux23start178-b}{}{}\protect\hypertarget{lb1625905.xhtmlux5cux23start178-c}{}{}\protect\hypertarget{lb1625905.xhtmlux5cux23start178-d}{}{}
framträda därur, för att locka med skenfagra behag. Framtill äro de
vackra att skåda; men ryggsidan är ihålig som ett tråg. De kallas
skogs-nufvor och skogs-rån.

Dock äro dessa farliga väsen på långt när icke så många som de
välvilliga i naturen eller de oskadliga. Många sådana hafva uppkommit ur
de lifsfrön, hvarmed Audumla mättat skapelsen. Allt har lif, ehuru
lifvet är af många slag. Så lefva väsen inne i träden. De gamla
vårdträden, som växa på gårdplanerna, hafva själar, som känna med de
människor, som de sett födas och växa upp i närheten af sin skugga. Bor
en slägt länge på samma gård, och har vårdträdets själ sett många barn
af den slägten leka under dess krona och växa upp till dugande
människor, då uppstår ett förtroligt förhållande mällan vårdträdet och
slägten. När den senare blomstrar, frodas vårdträdet, äfven i sin höga
ålder, och vid vinterfästen, då det står utan blad, smyckas det af
gårdfolket med brokiga band. Aldrig sjunga fåglarne vackrare än i
vårdträdet, och när en man, som varit länge borta på sjöfärder eller i
krigsäfventyr, kommer hem, då hälsar honom en susning i dess krona, som
väcker hans käraste barndomsminnen.

Äfven häraderna hafva sina vårdträd, under hvilka rätt skipas, och
folkstammarne hafva sina, där de samlas till tings och rådgöra om krig
och fred.

På hafvet i stormen hör man necken slå sin harpa, och från älfvar och
forsar förnimmes stundom i sommarnätterna strömkarlens strängaspel.

Ej sällan händer, att två af Järnskogens farligaste Lokesöner, ulfvarne
Skall och Hate, lämna sitt tillhåll och jaga efter solen och månen under
deras färder på himmelen. Asagudarne hafva hitintills alltid lyckats
frälsa dessa undan deras förföljare, ehuru de ofta kommit dem så nära,
att de blifvit skymda för människornas ögon (vid sol- och
månförmörkelser). Fullt trygga känna sig sol- och månehästarnes länkare
först, när de hunnit till varnernas skog i aftonrodnadsalfen Billings
land. Varnerna äro aftonrodnadsalfens stridsmän. Sedan solen
\protect\hypertarget{lb1625905.xhtmlux5cux23start179}{}{}\protect\hypertarget{lb1625905.xhtmlux5cux23start179-a}{}{}\protect\hypertarget{lb1625905.xhtmlux5cux23start179-b}{}{}\protect\hypertarget{lb1625905.xhtmlux5cux23start179-c}{}{}\protect\hypertarget{lb1625905.xhtmlux5cux23start179-d}{}{}
gått till hvila, vaka de ännu öfver henne »med burna ljus och tända
facklor», som kasta sitt sken på himmelen och förgylla kvällens skyar.

Till minne af det gamla fäderneslandet, af urfädernas grafvar och af de
vigtiga händelser, som i urtiden timade på den skandiska halfön, bedja
de söder om hafvet boende germanerna med anletet vändt emot Norden.

\paragraph{XXXVIII.}

\paragraph{NORNORNA. DOMEN ÖFVER DE DÖDE. SALIGHETS- OCH STRAFFORTERNA.}

Urd, ödets dis, är också dödens. Emedan hon bestämmer hvarje människas
lefnadsöden och lefnadslängd, bestämmer hon äfven hennes dödsstund. Hon,
som lägger lifvets lotter, lägger dödens. Hon, som med sina systrar
råder öfver det förflutna, närvarande och tillkommande, råder öfver och
samlar under sitt rikes spira det förflutnas, närvarandes och
tillkommandes alla slägten. Såsom dödens dis och herrskarinna i
underjorden kallas hon äfven Hel. Hel är namnet såväl på
lycksalighetsriket i underjorden som på dess drottning.

Urd har oräkneliga hjälparinnor och tjenarinnor, emedan de varelser, för
hvilka hon bestämmer inträde i lifvet, lefnadsställning och död, äro
oräkneliga. De bilda två stora flockar: den ena är värksam i hennes
tjänst med afseende på lifvet, den andra med afseende på döden.

Närmast under henne äro hennes systrar Verdande och Skuld. De tre äro de
förnämsta nornorna. De rådgöra tillsammans och döma gemensamt.

Det är förut omtaladt, att hvarje människa har en skyddsande, som kallas
hennes fylgia eller haminga. Fylgiorna äro Urds tjenarinnor. När ett
barn födes till världen, utser Urd åt det en fylgia, som då stiger upp
ur underjorden vid västerns rand, sväfvar öfver hafvet och uppsöker sin
skyddsling i Midgard och i osynlig måtto följer honom genom
\protect\hypertarget{lb1625905.xhtmlux5cux23start180}{}{}\protect\hypertarget{lb1625905.xhtmlux5cux23start180-a}{}{}\protect\hypertarget{lb1625905.xhtmlux5cux23start180-b}{}{}\protect\hypertarget{lb1625905.xhtmlux5cux23start180-c}{}{}\protect\hypertarget{lb1625905.xhtmlux5cux23start180-d}{}{}
lifvet, ser hans tankar, hviskar i hans samvete, manar och varnar honom
i hans drömmar. Hvarje dag sväfva tusentals fylgior öfver vattnen till
Midgard, och hvarje dag återvända tusental, för att bereda sina
skyddslingar boningar i salighetsriket, om de förtjenat det. En fylgia
lämnar icke den, som anförtrotts åt hennes vård, förr än kort före hans
död, och hon gör det då, för att återförenas med honom i
sällhetsvärlden. Dock händer, att hon för alltid öfvergifver honom, om
han blifvit en niding. Den som öfvergifvits af sin fylgia är en förlorad
man.

Ett annat slags nornor äro de diser, som utvälja mödrar åt de barn, som
framom jordelifvets tröskel vänta på att födas till världen.

Dessa alla äro Urds lefnadstjenarinnor. Den andra flocken af tjenande
diser fullgöra hennes dödsdomar och föra de dödas själar till
underjorden.

Förnämst bland dem äro valkyriorna, sköna ungmör med tankfulla anleten.
Där en drabbning står, där infinna de sig till häst, fullt rustade, och
utpeka med spjutskaften de kämpar, som Oden eller Fröja kårat till sina
salar, samt föra de fallne till underjorden och därifrån på Bifrost till
Asgard. När de icke äro ute i dessa ärenden, fylla de dryckeshornen åt
valfader och einheriarne i Valhall och Sessrymner. Urds syster Skuld är
valkyriornas herrskarinna och själf valkyria.

Till de Midgards innebyggare, som ej äro bestämda att falla för svärdet,
sänder Urd andra tjenarinnor af mycket olika skepnad allt efter arten af
deras död. Till dem, som svigta under årens börda, kommer den dis, som
är »de böjdes och lutandes hjälparinna». Barnen ha sina
dödsledsagarinnor, moderligt ömma och vänliga. Till dem, som bortryckas
af päst eller andra farsoter, komma Leikin och Nifelhelsväsen, som likna
henne; och de, som dö af andra sjukdomar, bortföras af de motsvarande
sjukdomsandarne.

Till en början vandra alla hädangångna en gemensam väg. En och samma
färdeled är föreskrifven dem, och samma Helport öppnar sig dagligen för
skaror af själar, som väntas
\protect\hypertarget{lb1625905.xhtmlux5cux23start181}{}{}\protect\hypertarget{lb1625905.xhtmlux5cux23start181-a}{}{}\protect\hypertarget{lb1625905.xhtmlux5cux23start181-b}{}{}\protect\hypertarget{lb1625905.xhtmlux5cux23start181-c}{}{}\protect\hypertarget{lb1625905.xhtmlux5cux23start181-d}{}{}
af olika lotter. Kvinnor och barn, ynglingar, män och ålderstigne, de
som sysslat i fredliga värf och de som blodat vapnen, de som lefvat i
enlighet med nornornas och gudarnes stadgar och de som brutit dem --
alla hafva de att taga samma kosa. De komma till fots och till häst,
anförda af skönrustade valkyrior, af den milda ålderdomshjälperskan, de
vänliga barnledsagarinnorna, eller af den blåhvita Leikin, de dystra
sjukdomsandarne. De samlas utanför österns Helport. Och när alla de
kommit, som den dagen väntats, kringvrides nyckeln Gilling i låset,
porten gnisslar på sina hakar och slås upp.

De dödes allfarväg i underjorden går först i västlig riktning genom
djupa och mörka dalar. På ett ställe har man att gå öfver en milsbred
törnbevuxen hed utan stigar. Då är det godt att hafva Hel-skor till
skydd för fötterna. Fördenskull försumma ej en afliden människas fränder
eller vänner att binda Hel-skor på liket, innan det jordas. Det är
visserligen sannt, att dessa skor, likasom allt annat, som liket medför
i griften, såsom kläder, vapen och smycken, stanna i grafven; men allt i
skapelsen, äfven de af människor slöjdade tingen, har ett inre ämne och
en inre form, och det är de graflagda tingens inre väsen, som följer den
döde till underjorden. De efterlefvandes omsorg om de döda räknas dessa
till godo, och ha de Hel-skor, komma de med välbevarade fötter öfver
törneheden. Ha de det icke, och om de i lifstiden varit obarmhärtiga mot
dem, som få vandra jordelifvets törniga stigar, då komma de ej utan
rifna och blödande fötter däröfver. Men åt de barmhärtiga, som sakna
Hel-skor, räckas sådana från ett träd, som växer där törnevandringen
börjar.

Därefter hinna de döda till en älf med forsande vatten, i hvars hvirflar
det är fullt af egghvassa järn. Fotsbreda ribbor flyta där, men bro
finnes icke. Ribborna bära, när de barmhärtigas fötter stiga på dem, och
föra dem oskadda öfver älfven. För de obarmhärtigas fötter slinta de
undan; dessa falla i älfven och måste under svåra kval genomvada
\protect\hypertarget{lb1625905.xhtmlux5cux23start182}{}{}\protect\hypertarget{lb1625905.xhtmlux5cux23start182-a}{}{}\protect\hypertarget{lb1625905.xhtmlux5cux23start182-b}{}{}\protect\hypertarget{lb1625905.xhtmlux5cux23start182-c}{}{}\protect\hypertarget{lb1625905.xhtmlux5cux23start182-d}{}{}
den. Ehuru de förskräckligt sargas af järnen, synes dock däraf intet
spår, när de kommit upp på stranden.

På andra sidan denna älf börjar det ljusna och i gryningsdager ligga
gröna ängder, genomströmmade af floden Gjall. Öfver den förer en
guldlagd bro, Gjallarbron, och på andra sidan den är vägskäl. En väg går
norrut till Mimers rike; en söderut till Urds brunn. Det är på den leden
de hädangångna fortsätta färden, ty deras slutmål är näjden kring Urds
brunn. Här är tingsstaden, där dom förkunnas öfver alla döda.

Hela denna vandring har skett under oafbruten tystnad. De dödas tungor
äro kalla och stela och frambringa icke ett ljud. Ej häller deras steg
höras. Deras hästar, när de komma å sådana, sätta sina hofvar ljudlöst
på dödsrikets mark. Endast under valkyriehästarnes hofvar dånar den
guldlagda bron öfver Gjall.

När de döda anländt till tingsstaden, sättas de i långa bänkrader
framför den heliga kretsen af domarestenar. Här äro de inväntade af sina
fylgior, som gått före dem till dödsriket och nu ställa sig, enhvar
invid sin skyddsling. Olycklig den, som ingen fylgia har å tingsstaden
vid Urds brunn, där de domar afkunnas, som ha evig giltighet!

Asagudarne äro domare här. De hafva två tingsplatser: en uppe i Asgard,
där ärenden afgöras, som vidkomma gudavärlden; en vid Urds brunn, där de
döma de döda. En gång om dagen rida de på Bifrost dit ner och komma
genom den södra Helporten in i Urds rike och öfver åtskilliga älfvar
till färdens mål. Hästarne, som de rida, äro Sleipner, Glad, Gyller,
Gler, Skeidbrimer, Silfvertopp, Siner, Gils, Falhofner, Gulltopp och
Lättfote. Tor rider aldrig, och på denna resa får han ej häller åka, ty
hans åskande char kunde skada Bifrost. Han går fördenskull och har på
vägen att öfvervada fyra älfvar, Karmt och Armt och två med namnet
Kerlaug. När de andre asarne stiga af sina hästar, är äfven han framme,
och aldrig har det å denna tingsstad händt, att rättens ledamöter låta
vänta på sig.

\protect\hypertarget{lb1625905.xhtmlux5cux23start183}{}{}\protect\hypertarget{lb1625905.xhtmlux5cux23start183-a}{}{}\protect\hypertarget{lb1625905.xhtmlux5cux23start183-b}{}{}\protect\hypertarget{lb1625905.xhtmlux5cux23start183-c}{}{}\protect\hypertarget{lb1625905.xhtmlux5cux23start183-d}{}{}

Oden sätter sig i högsätet och de andre asarne å rådstolar å ömse sidor.
Framför dem sitta å sina bänkrader de döda, bleka och med märkena af den
död de lidit. Tigande hafva de att åhöra rättsförhandlingarna och
emottaga domen, såvida de icke ega målrunor, som ge dem kraft att tala
och själfva försvara sig mot anklagelser. Ytterst sällan händer, att
någon eger dessa runor; om så är, får han stiga upp i en talarestol,
byggd för detta ändamål, och anföra hvad han kan till sitt försvar. Men
ingen annan gör det än den som blifvit öfvergifven af sin fylgia och
fördenskull ej har någon sakförare vid tinget. De andra behöfva icke
tala, lika litet som de mäkta det, ty hvarje fylgia försvarar sin
skyddsling; hon är ett välvilligt vittne för honom och tillika det
pålitligaste inför domstolen, ty hon känner alla hans tankar,
bevekelsegrunder och handlingar. Sällan kräfves, att hon talar; ty
hennes närvaro invid den döde är i och för sig ett bevis, att han ej
gjort sig skyldig till någon oförlåtlig synd.

Utanför tingskretsen vänta skaror af straff-andar på
domstolsförhandlingens slut, för att ledsaga de till osalighet dömda och
öfverantvarda dem åt deras yttersta bestämmelse. Bland dessa
straff-andar äro heiptorna, som äro väpnade med törnegissel.

Till tinget vid Urds brunn böra de döda komma väl klädda och prydda.
Stridsmännen medföra sina försvars- och anfallsvapen; kvinnorna och
barnen smycken, som varit dem kära. Bilder af de saker, som fränder och
vänner nedlagt i grafhögarne, medfölja de döda till vittnesbörd inför
domarne, att de åtnjutit de efterlefvandes aktning och tillgifvenhet.
Den åsyn de till tinget samlade förete ådagalägger i hvad mån de
efterlefvande iakttaga den lag, som bjuder vördnad för döden och omsorg
om de hädangångnes stoft.

Många dö under förhållanden, som omöjliggöra för fränder att iakttaga
denna omsorgs plikter. Då böra främlingar träda i fränders ställe. Det
skick, hvari dessa döda anlända till tinget, visar bäst, om fromt
sinnelag är rådande i Midgard. Ty ädla sinnen behjärta de råd, som så
lyda: »Visa
\protect\hypertarget{lb1625905.xhtmlux5cux23start184}{}{}\protect\hypertarget{lb1625905.xhtmlux5cux23start184-a}{}{}\protect\hypertarget{lb1625905.xhtmlux5cux23start184-b}{}{}\protect\hypertarget{lb1625905.xhtmlux5cux23start184-c}{}{}\protect\hypertarget{lb1625905.xhtmlux5cux23start184-d}{}{}
de lik du finner på marken den sista tjänsten, de må vara sotdöda eller
drunknade eller vapenfallna! Du skall göra bad åt dem, som äro döda, två
deras händer och hufvud och kamma och torka dem, innan du lägger dem i
kista och bjuder dem säll sömn.»

Deras naglar böra vara klippta. Det finns tjenstgörande andar, som å
tingsstaden undersöka, om de äro det, och i annat fall klippa dem.
Straff-andar samla det klippta affallet, taga det med sig och förskaffa
det till Lyngveholmen i Amsvartners haf, där osynliga händer arbeta på
ett skepp, som kallas Nagelfar och bygges af de dödes å tingsstaden
samlade naglar. När skeppet är färdigt, förestår världsförstörelsen. Då
brista Lokes, Fenrers och de andra världsfördärfvets söners fjättrar; de
stiga ombord å Nagelfar och segla med detta skepp till kamp mot gudarne.
Enhvar, som försummar de lefvandes plikter mot de döda, påskyndar
således denna världs undergång och främjar det ondas sak i kampen mot
det goda.

Mänskliga fel och svagheter dömas af gudarne mildt. Under sin lärotid
hafva de ju själfve felat. De tingförda ha att vänta en god dom, om de
gått genom lifvet svekfria, hederliga, hjälpsamma och utan dödsfruktan
-- om de iakttagit vördnad för gudarne och tämplen, för frändeplikter
och för de döda. Men lögn, om därmed afsetts att skada andra, får en
genom långa tider räckande vedergällning; mened, smygmord,
äktenskapsbrott, tämpelskändning, kummelbrytning, förräderi och
nidingsdåd straffas med onämneliga fasor.

De, som af domstolen förklarats salighet värdiga, få, innan de lämna
tingsstaden, smaka en dryck, som utplånar hvarje märke den lekamliga
döden efterlämnat, återställer deras lifsvärma, lossar deras tungor,
förhöjer deras lefnadskraft, ökar deras styrka och förlänar dem
sorgeglömska, dock utan att bortplåna kära minnen eller medföra
förgätenhet af det, som man kan erinra sig utan saknad eller ängslan.
Drycken kallas »krafternas dryck» och är en blandning af safter ur de
tre källor, som underhålla Yggdrasils lif: ur Urds och Mimers källor och
ur Hvergelmer. Därför säges om
densamma\protect\hypertarget{lb1625905.xhtmlux5cux23start185}{}{}\protect\hypertarget{lb1625905.xhtmlux5cux23start185-a}{}{}\protect\hypertarget{lb1625905.xhtmlux5cux23start185-b}{}{}\protect\hypertarget{lb1625905.xhtmlux5cux23start185-c}{}{}\protect\hypertarget{lb1625905.xhtmlux5cux23start185-d}{}{},
att den består af »Urds kraft, köldkall sjö och Sons vätska». (Den
köldkalla sjön är Hvergelmer och Son är ett af namnen på Mimers källa.)

Denna dryck räckes dem i ett guldsmidt uroxehorn, kring hvars rand
bilden af en orm är ristad. Om en obefogad hand vidrör hornet, får ormen
lif och dräper gärningsmannen. Mångahanda andra ristningar pryda hornet:
bilder af drakar, som vakta underjordsportarne, bilder af örter, som
växa på salighetsrikets evigt gröna ängar, samt välsignelsebringande
runor.

Äfven för de till osalighet dömda är en dryck bestämd, men den är
blandad med etter, kvalfull att smaka och ryslig i sina värkningar. De
dö genom den drycken en andra död, som består däri, att anden, som de
före sin födelse fått af Oden och är den ädlaste beståndsdelen i
människan, flyr bort ifrån dem. Med den flyr också den af fina
grundämnen bildade inre lekamen, som Lodur gifvit hvarje människa, den
lekamen, som är bildad efter gudarnes skepnad och ger det yttre
kroppsliga höljet den form, som detta bär i jordelifvet. I och med det
att denna inre lekamen lämnat dem, får den fördömda själen ett annat
hölje, hvars utseende afbildar själens ondska och alltid är styggt, ofta
vidunderligt fult att skåda.

När de, som fått dom till salighet öfver sig afkunnad, lämna tinget,
ledsagas de af sina fylgior till de sköna hem, som dessa åt sina
skyddslingar iordningställt i »gudars gröna världar», på underjordens
lycksalighetsfält. Ifriga äro de att se de härliga ängdernas många under
och att besöka fränder och vänner, som gått före dem till den yttersta
bestämmelsen. Fylgian ledsagar sin skyddsling på »glädjestigar» genom
fält, som äro »honungsskeppens (blommornas) hemstad». Här kan den
vetgirige söka och samtala med förfäder och urfäder och få sin slägts,
ja forntidens alla märkligare öden förtalda af dem, som sett det de
förtälja.

De svärdfallne, som valkyriorna ledsaga, göra halt på tingsplatsen,
mötas där af sina fylgior och emottaga
»krafternas\protect\hypertarget{lb1625905.xhtmlux5cux23start186}{}{}\protect\hypertarget{lb1625905.xhtmlux5cux23start186-a}{}{}\protect\hypertarget{lb1625905.xhtmlux5cux23start186-b}{}{}\protect\hypertarget{lb1625905.xhtmlux5cux23start186-c}{}{}\protect\hypertarget{lb1625905.xhtmlux5cux23start186-d}{}{}
dryck». Saknar en svärdfallen sin fylgia, måste han stiga ur sadeln och
sätta sig på de dödas bänk; han är då visserligen en niding, och kan han
ej med målrunor försvara sig, dömes han till pinovärldens kval. De
redlige stridsmännen begifva sig efter undfången dryck från tinget,
besöka sina fränder på salighetsfälten och beskåda märkvärdigheterna
där, tills tid är inne, att de fortsätta färden till Asgard. Då äro
asarne där före dem, och när de höra Bifrosts dån under de anländande
ryttarne, skickar Oden Brage och Svipdag eller andra bordvänner att med
välkomstbägaren möta dem i gafvelporten till Valhall. Där tillbringa de
sina dagar i umgänge med gudarne och roa sig med stridslekar, när de
icke sitta vid dryckeshornet i glada samtal eller lyssna till Brages
sång och harposlag.

Till Valhall komma äfven sotdöde hjältar och höfdingar.

När domen öfver dem, som begått dödssynder, afkunnats, hafva de att
vandra till mötes sin förfärliga bestämmelse. Deras forna fylgior gråta,
när de se deras affärd. Fly är dem icke möjligt, nornornas bojor fängsla
dem, och de drifvas sin väg fram af heiptorna, väpnade med spöknippor af
törne, som obarmhärtigt gissla dem på dröjande hälar. Deras väg från
Urds källa går norrut genom Mimers rike. Det är så anordnadt, att de
före sin ankomst till kvalens värld skola hafva sett lycksalighetens
ängder. De få sålunda veta hvad de hafva förspillt. Så leder dem deras
kosa öfver älfven Leipt, vid hvars skinande, klara, högheliga vatten
dyra eder pläga sväras, och som flyter mällan »glansfält» med blomster,
som aldrig vissna, och skördar, som aldrig skäras -- öfver denna älf
förbi Breidablik, den strålande borgen, där Balder och Nanna bo med
Leiftraser och Lif; förbi Hoddgoda, en af flere underjordsfloder
omslingrad borg, där Mimer samlat skatter för en kommande världsålder;
förbi Mimers med sjudubbel guldinfattning sirade källa, i hvilken
Yggdrasils nedersta bladrika grenknippa speglar sig, och hvari dess
mällersta rot nedsänker sina silfverhvita rottrådar; förbi de sofvande
sju Mimersönernas tysta borg och förbi Natts och hennes
\protect\hypertarget{lb1625905.xhtmlux5cux23start187}{}{}\protect\hypertarget{lb1625905.xhtmlux5cux23start187-a}{}{}\protect\hypertarget{lb1625905.xhtmlux5cux23start187-b}{}{}\protect\hypertarget{lb1625905.xhtmlux5cux23start187-c}{}{}\protect\hypertarget{lb1625905.xhtmlux5cux23start187-d}{}{}
disers salar. Allt starkare förnimmes dånet af Hvergelmerkällans brus
och världskvarnens gång, ty man nalkas Nidafjällets södra sluttning.
Tåget går upp i denna fjälltrakt genom dälder och kjusor, där de från
Hvergelmer söderut rinnande älfvarna leta sina vägar. Det lämnar
Hvergelmer och världskvarnen bakom sig och sätter öfver gränsvattnet
Raun (underjordens Elivågor). Där bakom resa sig Nifelhels svarta
brådstupande fjällmurar. Stegar eller trappor leda öfver svindlande djup
till portar, som kallas likportarne (»nagrindarne»), emedan de, som i
underjorden dött den andra döden och sålunda blifvit för andra gången
lik, föras genom dem till sitt mål. Tjut och skall från de portvaktande
Nifelhelshundarne båda de fördömdas ankomst. Då ila i täta flockar
bevingade vidunder, Nifelhels roffågelskaror, världsträdets gnagare
Nidhögg, örnarne Are och Räsvälg (»liksväljaren») och deras likar
söderut och slå sig ned kring likportarnes klippor. Dessa öppna sig på
rasslande hakar, och när de fördömda kommit genom dem, störta de vingade
demonerna öfver de åt dem utsedda offren, trycka dem under sin
dolkhvassa fjäderklädnad och flyga med gräsliga skrän genom Nifelhels
töckenrymder till de för dem bestämda pinorummen. De trakter, öfver
hvilka demonskaran flyger, äro desamma, som Svipdag skildrade för Gerd,
då han hotade att med Valandssvärdet försända henne till dödsriket. Det
är Nifelhel, rimtursarnes, de döde underjordsjättarnes och
sjukdomsandarnes hem. Det är där som Ymers fötters afföda bor, de
vidunderligt födde och födande urjättarne, eller snarare deras själar,
höljda i en spöklekamen, liknande deras vanskapliga jordiska. De tala
icke; de endast tjuta och stirra med vilda ögon. De bo tillsammans i en
stor hall, medan de till Nifelhel förflyttade medlemmarne af den yngre
jätteslägten bo i gårdar, spridda öfver den stinkande träskiga marken,
genom hvilken den från Hvergelmer norrut flytande älfven Slid söker i
dyig bädd sin väg. Det är där den rastlösa orons, själavåndans, den
skakande gråtens och vanvettets demoner hafva sin hemstad; där som
farsoternas och sjukdomarnes andar
\protect\hypertarget{lb1625905.xhtmlux5cux23start188}{}{}\protect\hypertarget{lb1625905.xhtmlux5cux23start188-a}{}{}\protect\hypertarget{lb1625905.xhtmlux5cux23start188-b}{}{}\protect\hypertarget{lb1625905.xhtmlux5cux23start188-c}{}{}\protect\hypertarget{lb1625905.xhtmlux5cux23start188-d}{}{}
bo med sin drottning, Lokedottern Leikin, hvars tröskel är fallförsåt
och hvars bädd är sjukläger. Rymden är ständigt uppfylld af töcken.

Detta dystra land är dock endast förgården till de egentliga kvalhemmen.
Ett svalg förer därifrån ned till nio under Nifelhel belägna ofantliga
pinohålor. Ur svalget uppstiga vidriga ångor, och älfven Slid vräker
sina svarta slammiga vattenmassor ned för dess branter. I detta svalg är
det som Nidhögg och de andra flygande demonerna störta sig med sina
offer. Innan de aflämna dem, borra de sina näbbar, käftar och klor i
deras lemmar och slita dem i trasor; men lemmarne växa åter tillsammans:
en tredje död ges för de fördömda icke. Därefter fördelas dessa mällan
pinohålorna i enlighet med de dödssynder, som de begått. De nio
straffvärldarne bestå af nio oerhördt vidsträckta fjällgrottor, förenade
med hvarandra genom öppningar, brutna i bärgväggarna och stängda med
portar, utanför hvilka stå väktare, som till skepnad och uppförande äro
vartecken af de synder, hvilkas föröfvare de bevaka. Den längst i norr
belägna straffhålan kallas »Likstränderna» (Nastränderna), emedan man
genom en port i dess norra fjällvägg kan komma till Amsvartners haf. I
en förgård utanför porten vakta svartalfer, hvilka underhålla en eld,
hvarifrån röken hvirflar in i den oöfverskådligt långa sal, som är byggd
i pinohålan. Byggnadsämnet består af lefvande ormar. Väggarna utgöras af
deras flätade ryggar; genom taköppningarna spruta ormarnes hufvud gift i
störtande skurar. Under taket är upphängd en rad af järnbänkar,
öfverspänd med ett nätvärk af bly. På bänkarne ligga menedare och
lönnmördare och emottaga mällan krampaktigt öppnade läppar ormgiftet.
Under denna bänkrad är en annan, och under den andra en tredje och
flere, och förbrytarne på hvarje öfre bänkrad äro »etterrännor», genom
hvilka ormgiftet forsar och öfversköljer de under dem liggande.

Liknande straffhålor äro inrättade äfven inom Lyngveholmens bärg i
Amsvartners haf för Lokes och Fenrers närmaste fränder,
»världsfördärfvets söner» (Muspels söner).

\protect\hypertarget{lb1625905.xhtmlux5cux23start189}{}{}\protect\hypertarget{lb1625905.xhtmlux5cux23start189-a}{}{}\protect\hypertarget{lb1625905.xhtmlux5cux23start189-b}{}{}\protect\hypertarget{lb1625905.xhtmlux5cux23start189-c}{}{}\protect\hypertarget{lb1625905.xhtmlux5cux23start189-d}{}{}

\paragraph{XXXIX.}

\paragraph{HÖGBEBYGGARNE.}

Människan dör på samma gång »till Hel» och »till hög», till underjorden
och till grafkullen. Hennes egentliga jag flyttar till underjorden; men
hon har till en tid en dubbelgångare, ett andra jag i sin grift.

Detta har sin förklaring i hennes naturs beståndsdelar. Så länge hon
lefver på jorden, äro dessa säx till antalet: \emph{anden}, som är Odens
gåfva; \emph{själen}, som är Höners gåfva; \emph{den efter gudaskepnaden
bildade inre lekamen}, som är Lodurs skänk; \emph{blodet}, som äfven är
skänkt af Lodur, samt \emph{växtkraft} och \emph{jordämne.} De båda
sistnämda funnos i Ask och Embla, medan de ännu voro träd, och de finnas
i de på världsträdet vuxna frukter, som bäras af Höners bevingade
tjenare till dem, som skola varda mödrar.

»Lit» är af gammalt det namn, hvarmed den inre lekamen betecknas. Af
»litens» form beror kroppens utseende. Är liten skön, så är kroppen det
äfven, och om liten förändras, förändras kroppen. Det finns människor,
som för en kort tid kunna byta lit med hvarandra, den ene får då den
andres utseende, utan att de fördenskull förändras till själ och ande.

Den jordiska döden består däri, att jordämnet, växtkraften och blodet
skiljas från människans ypperligare beståndsdelar och kvarstanna på
jorden. Den döde, som färdas till underjorden, utgör fördenskull en
förening af ande, själ och lit. Dömes han i underjorden att dö en andra
död, skiljes från honom anden och den gudaformade liten. Det återstår då
hans själ, och denna får, såsom redan är nämdt, en lit, som
öfverensstämmer med själens tillstånd.

De i grafhögen gömda beståndsdelarna af den aflidne fortsätta en längre
tid sin växelvärkan med hvarandra och bilda ett slags enhet, som bevarar
något af hans personlighet och egenskaper, emedan de i jordelifvet varit
genomträngda\protect\hypertarget{lb1625905.xhtmlux5cux23start190}{}{}\protect\hypertarget{lb1625905.xhtmlux5cux23start190-a}{}{}\protect\hypertarget{lb1625905.xhtmlux5cux23start190-b}{}{}\protect\hypertarget{lb1625905.xhtmlux5cux23start190-c}{}{}\protect\hypertarget{lb1625905.xhtmlux5cux23start190-d}{}{}
af hans ande och hans själ. Grafhögen kommer på detta sätt att innesluta
en dubbelgängare af den till dödsriket nedstigne. Dubbelgängaren kallas
»högbo», högbebyggare och draug. Draug betyder egentligen en från sin
rot afhuggen trädstam; högbebyggaren kallas så, emedan han är skild från
sin lifsrot, själen, och efter hand, om än mycket långsamt, skattar åt
förgängelsen och går sin upplösning till mötes. Högbebyggaren liknar
till skaplynnet den själ, af hvilken hans beståndsdelar varit
genomträngda. En afliden nidings dubbelgängare är ond och farlig; de
hederligas och välvilligas dubbelgängare äro goda att hafva i sin närhet
och värka framför allt till sin slägts bästa. Fördenskull vill man gärna
hafva slägtens grafhögar nära intill hemmen. Vanligen sofver
högbebyggaren om dagen i sin grift; men han kan nattetid vakna eller
väckas genom bönens kraft, såsom då Svipdag väckte Groa. Förestå vigtiga
händelser i slägten, samlas dess högbebyggare och rådgöra sins emällan.

Om en innerligt älskad frände eller vän för häftigt sörjer den döde,
stör dess sorg hans sällhet i dödsriket och mäktar draga honom tillbaka
till jorden. Då uppsöker han sin grafhög, och han och hans andra jag,
högbebyggaren, varda för en stund till ett. Det berättas, att så hände
en gång med Halfdan. Sedan han fallit för Valandssvärdet, sörjde honom
hans maka Alvig så djupt och oaflåtligt, att han från Valhall återvände
till jorden och red till sin grift, där Alvig kom till honom och ville
hvila på hans arm. Men sedan Halfdan sagt henne, att hennes tårar föllo
urkalla på hans bröst och genomträngde det med smärta, samt lofvat, att
hon en gång tillsammans med honom skall dricka underjordens
sorgestillande safter, red han åter bort, för att, innan hanen gol, vara
tillbaka bland de hädangångne hjältarne. Det finnes, utom sorgens kraft
och bönens, en tredje, som kan återkalla en död till jordelifvet. Det är
besvärjelsens; men en dödsbesvärjelse är, såsom Hardgreips öde visat, en
svår synd, som gör föröfvaren hemfallen åt straffandarne.

\protect\hypertarget{lb1625905.xhtmlux5cux23start191}{}{}\protect\hypertarget{lb1625905.xhtmlux5cux23start191-a}{}{}\protect\hypertarget{lb1625905.xhtmlux5cux23start191-b}{}{}\protect\hypertarget{lb1625905.xhtmlux5cux23start191-c}{}{}\protect\hypertarget{lb1625905.xhtmlux5cux23start191-d}{}{}

\paragraph{XL.}

\paragraph{RAGNARÖK. VÄRLDSFÖRNYELSEN.}

Alla fel hafva långt räckande följder. Ondt aflar ondt. De fördärfvets
frön, som Gullveig i tidernas morgon sådde i människornas sinnen, bära
skördar genom århundradena, och med hvarje århundrade allt rikligare.
Sedan gudarne genomgått sin lärotid, har världsstyrelsen varit präglad
af visdom, men deras dessförinnan begångna misstag kunna ej till sina
värkningar upphäfvas. Mimers källa förblifver i denna världsålder
sjunken djupt under sina bräddar. Den rot, ur hvilken den store asken
Yggdrasil suger andlig lifssaft, vattnas otillräckligt, och dess vårdare
äro borta. Asken åldras och förtorkar. Gudarne kunna ej hjälpa det.
Människorna förvärras, slägte efter slägte, och ej häller det mäkta
gudarne hindra; ty han, som skulle kunnat skänka det goda större
växtkraft än ogräset -- han, Balder -- är inom underjordens Breidabliks
stängda murar, och spiran är tagen från den milde Höner. Den erbjöds
honom åter efter asarnes och vanernas försoning, men han vägrade. Han
känner sig för svag att föra henne i detta järnhårda tidehvarf. Först i
världsförnyelsen omfattas hon ånyo af hans hand, för att aldrig falla
därur.

Men hvad godt som är kvar lefver under gudarnes värnande hand, och det
skall lända dem till ära, när på den sista stridens stora dag det godas
härar möta fördärfvets med jämnvägande styrka.

När detta tidehvarfs ände nalkas, båda tecken hvad komma skall. Solens
ljus och värme minskas sommar efter sommar; de tyglar makterna haft på
vindarne slitas, och genom stormens dån förnimmes Fenrers tjut från
Gnipahålan (»de stupande branternas håla») å Lyngveholmen, och Fenrers
fränders från Järnskogen. Bland människorna råder förvildning. De döfva
sin ångest i utsväfningar och blodbad. Sedernas band äro afkastade,
urtidens heliga runor förgätna. Löften
\protect\hypertarget{lb1625905.xhtmlux5cux23start192}{}{}\protect\hypertarget{lb1625905.xhtmlux5cux23start192-a}{}{}\protect\hypertarget{lb1625905.xhtmlux5cux23start192-b}{}{}\protect\hypertarget{lb1625905.xhtmlux5cux23start192-c}{}{}\protect\hypertarget{lb1625905.xhtmlux5cux23start192-d}{}{}
och eder aktas icke, makar svika hvarandra, broder varder broders bane,
och systrars söner utgjuta hvarandras blod.

Från Järnskogen gör Hate ströftåg in i Midgard. Skaror af baningar och
ulfvar följa honom. Midgards höfdingar ställa sig till motvärn; landen
fyllas med slagfält, och deras borgar sköljas röda af sårsaft. Yxan,
svärdet och dolken härja. Bygder varda ödemarker, de döde äro för många
att jordas, och ulfvarne fråssa i kapp med förruttnelsens Nidhögg på
otaliga lik.

Nu kommer den andre fimbulvintern. Väl stiger solen, som vanligt, efter
den kortaste dagen, högre på himmelen, och dagens längd tilltager; men
det allt svagare ljuset återkallar en allt svagare grönska.
Frostjättarne rycka öfver Elivågor in i Midgard; isjöklarne växa,
snöfälten utbreda sig och smälta icke. Asks och Emblas alla afkomlingar
på jordens yta bortryckas af vapen, farsoter, köld och hunger. Hate i
ulfham slukar månen. Öfver den förödda jorden brusa vilda etterkalla
vindar. Yggdrasil våndas och kvider.

I Järnskogen slår stormjätten Eggter, vaktaren af Gullveigs ulfhjordar,
sin harpa och kallar med skadegladt strängaspel på eldjätten Fjalar, ty
tiden är inne, när denne skall få hämd. Fjalar kommer i den röde hanens
skepnad, och ur trollträsket räcker honom Gullveig, »den gamla i
Järnskogen», Valands svärd. Fjalar flyger med det ned i de mörka
»djupdalarne», där de i sina hålor gömda världsfördärfvets eldar bereda
sig till utbrott, och han räcker vapnet till sin fader Surt.

Yggdrasil skälfver från roten till kronan. Då bäfvar Asgard, jord och
underjord. Då faller gjallarhornet, där det hängde i helig skymning
bland Yggdrasils löfvärk, och kommer i Heimdalls hand. Asgards väktare
lyfter det, och dess åskande toner genomtränga världen. Mimers sju
söner, väckte af hornskallet, springa upp ur århundradens sömn. De fatta
sina vapen och stiga till häst, för att deltaga i striden mot det onda.
För dem gäller det främst att skydda underjordens gröna riken mot
Nifelhels vidunderskaror. Dvärgarne
\protect\hypertarget{lb1625905.xhtmlux5cux23start193}{}{}\protect\hypertarget{lb1625905.xhtmlux5cux23start193-a}{}{}\protect\hypertarget{lb1625905.xhtmlux5cux23start193-b}{}{}\protect\hypertarget{lb1625905.xhtmlux5cux23start193-c}{}{}\protect\hypertarget{lb1625905.xhtmlux5cux23start193-d}{}{}
komma ut ur sina fjällboningar och stå ångestfulle vid sina bärgväggar.
Lycksalighetsfältens fredliga innebyggare äro gripna af förskräckelse.
Pinohålornas fångar komma lösa och storma, i förening med Nifelhels
vidunder, mot Hvergelmer och Nidafjället, för att härja Mimers och Urds
riken, eller de störta ut genom den norra porten åt Amsvartners haf, där
de invänta Loke. Då Yggdrasil skälfde, lossnade hans och Fenrers och de
andre Lyngvefångarnes länkar, och de ilade till skeppet Nagelfar, som,
nu färdigbygdt, lossas från stranden och sätter ut. Lokes broder
Helblinde, vattendjupens jätte, följer i kölvattnet det väldiga skeppet,
vid hvars roder Loke står. Han styr till Likstränderna och hämtar de
fördömde, som bida honom där. Han tager därefter kosan österut till
stränderna af Järnskogen, där hela hans ulfyngel skall hämtas till
striden. Allt hvad skogar, bärg och vatten hysa af trollska väsen far i
förvirring öfver markerna. Öfver hela Jotunheim hörs det vildaste gny.
Dess jättar samlas under den andre Lokebrodern Byleist-Ryms ledning.
Världshafvet vältrar höga böljor, ty Midgardsormen vaknade, då Yggdrasil
skalf, och slingrar sig i jotunvrede. Hans rysliga hufvud lyfter sig
redan öfver vattenbrynet med blodtörstiga ögon.

I Asgard råder lugn. Oden har för sista gången talat med Mimers hufvud,
och han vet hvad som skall komma. Asar, vaner och alfer äro i full
rustning samlade å tinget, och deras hästar stå sadlade, väntande på
sina ryttare. Einheriarne, anförde af Svipdag och andra urtidshjältar,
af Sköld- Borgars ättlingar och af Ivaldes, af ynglingarnes,
budlungarnes, hildingarnes, ylfingarnes, amalernas hjältar, sitta redan
till häst i ordnade leder. Så äfven valkyriorna.

Å gudatinget rådslås icke om frälsning, utan om gudarnes platser i
striden. Vanerna med Sessrymners einheriar skola möta Surt och
Suttungssönerna. Njord har begifvit sig till Vanaheim; men Fröj, hans
son, har stannat i Asgard, och han påyrkar att få sin plats mot Surt
till bot för den dårskap, som bragte Valandssvärdet i jättevåld.

\protect\hypertarget{lb1625905.xhtmlux5cux23start194}{}{}\protect\hypertarget{lb1625905.xhtmlux5cux23start194-a}{}{}\protect\hypertarget{lb1625905.xhtmlux5cux23start194-b}{}{}\protect\hypertarget{lb1625905.xhtmlux5cux23start194-c}{}{}\protect\hypertarget{lb1625905.xhtmlux5cux23start194-d}{}{}

Efter slutadt ting tacka hvarandra gudarne för snart tilländalupen
sammanlefnad och taga afsked af sina asynjor och diser.

Dessa skåda, medan de afvakta döden, ut från Lidskjalf, för att intill
det yttersta hafva för sina ögon dem de älska.

Asgards krigarskaror dela sig och rycka å ömse sidor ned för Bifrost.
Vanerna med sina einheriar rida ned på den södra halfbågen. Där brister
Bifrost under ryttareskarornas tyngd, och simmande i lufthafvet
tillryggalägga de den återstående vägen till Oskopners slätter. Asarne
med sina einheriar tåga ned på den norra halfbågen, för att möta
Jotunheimsskarornas anlopp. Striden kommer att stå längs Jormungrunds
hela randbälte, där Oskopners slätter och Vigrids sammansluta sig till
ett enda stort slagfält.

Angrepp äro att vänta från alla håll. Mot norr mörknar synkretsen af
snö- och hagelbyar öfver frost- och stormjättarnes framryckande massor.
De hålla sköldarne framför sig och vråla under dem sin orkanvilda
stridssång.

Östern svartnar af Järnskogens svärmar. Gullveigs och Eggters
vidunderhjordar, anförda af Hate, baningarne och Nifelhels fördömda
innebyggare, anförda af Loke, »världsfördärfvets söner» (Muspels söner)
komma där i oöfverskådliga skaror.

Söderns synkrets rödfärgas som af lågor. Surt och Fjalar med »Suttungs
söner» anrycka därifrån. Surt bär Valandssvärdet i sin hand. Det är som
om solen åter skene med sin forna glans öfver världen; men skenet kommer
från svärdet, ej från den tynande »Alfrödul».

I väster höjer Midgardsormen sitt hufvud högre öfver hafvet.

Vidar den tyste var ej å tinget. På sin egen mark, det ensliga och
gräsbevuxna Vide, sadlade han sin häst och iklädde sig sin rustning. Där
band han på vänstra foten en märkvärdig sko, på hvilken osynliga händer
arbetat under århundradenas lopp. Det förhåller sig med denna på
följande sätt: Förfäderna hafva stadgat, att när skor göras,
\protect\hypertarget{lb1625905.xhtmlux5cux23start195}{}{}\protect\hypertarget{lb1625905.xhtmlux5cux23start195-a}{}{}\protect\hypertarget{lb1625905.xhtmlux5cux23start195-b}{}{}\protect\hypertarget{lb1625905.xhtmlux5cux23start195-c}{}{}\protect\hypertarget{lb1625905.xhtmlux5cux23start195-d}{}{}
skola stycken af lädret afskäras vid tå och häl och förvaras åt
fattigman. Den det gör, han ger osynligt ämne åt den sko, som skall
skydda Vidars fot, när han ställer den i Fenrers öppna käftar. Af skons
styrka och ogenomtränglighet för Fenrers tänder, gift och etter, beror
om Vidar skall gå med lifvet ur striden eller icke. Minsta
barmhärtighetsgärd kommer sålunda gudarne och deras sak till godo på
slutuppgörelsens dag.

När striden börjar, har Vidar intagit sitt rum i fylkingen och rider sin
fader Oden nära.

Anförde af gudar, valkyrior och urtidshjältar, spränga einheriarne fram
i täta leder, som bryta sig, här mot Jotunheims fylkingar, där mot
baningarnes och mot Järnskogens odjurshorder, där mot Suttungssönernas
skaror. Öfver allt strid och död -- tursar, vidunder och eldjättar
fallande under einheriarnes spjut och svärd; einheriar störtande, man
och häst, till jorden under jättarnes klubbor, odjurens käftar,
baningarnes pästpilar och Suttungssönernas gnistrande vapen.

Tyr, som svänger sitt svärd i vänster hand, träffar Hate i
vapenträngseln. Tvekamp dem emällan. Hate faller. Men äfven den
ridderlige asaguden, som, dödligt sårad, dignar ur sadeln.

Heimdall och Loke, de gamle fienderna, söka fram mot hvarandra. Loke är
afskyvärd att se. Århundradens pinor, hat och hämdekänslor drifva honom
fram till orädd kamp. Heimdall kämpar en svår strid; han segrar och
skiljer med svärdet Lokes hufvud från hans kropp. Men lemmarne af
Midgardsormens fader lefva äfven i styckadt skick; det ohyggliga
hufvudet, fullsatt med pästladdade, spjutlika horn, återstudsar från
marken och borrar sig väg genom Heimdalls bröst. Den rene eldens ädle
gud stupar. Då slocknar solen, och stjärnorna falla från himmelen.

Fenrer rasar i einheriarnes leder. Oden rider på Sleipner mot honom, men
dignar framför odjurets ettersprutande gap och försvinner däri. Då
kastar sig Vidar af sin häst, ställer vänstra foten på Fenrers underkäk,
griper om hans
\protect\hypertarget{lb1625905.xhtmlux5cux23start196}{}{}\protect\hypertarget{lb1625905.xhtmlux5cux23start196-a}{}{}\protect\hypertarget{lb1625905.xhtmlux5cux23start196-b}{}{}\protect\hypertarget{lb1625905.xhtmlux5cux23start196-c}{}{}\protect\hypertarget{lb1625905.xhtmlux5cux23start196-d}{}{}
öfre med väldig hand och stöter svärdet i hans hjärta. Där håller guden
vapnets udd kvar, tills ulfven tumlar på sidan och ligger död. Så hämnar
Vidar den tyste sin fader. Oskadd går den starke och blygsamme asasonen
ur Ragnarökstriden.

Midgardsormen har under striden sträckt hufvudet upp öfver Vigrids slätt
och börjar spruta etter öfver gudarnes fylkingar. Tor har blicken fäst
på honom och har aldrig kastat sin hammare med den asakraft som nu.
Ormen störtar ned med krossadt hufvud; men förgiftad af hans etter,
vacklar segervinnaren nio steg tillbaka och faller död.

Fröj på sin ypperlige häst Blodughofve stormar genom Suttungssönernas
svärmar mot Surt. Då höjer denne det förfärliga Valandssvärdet, och Fröj
stupar under hugget af sin fosterfaders smide, brudköpet för Gerd; men i
samma stund är Jotunheims öde fullbordadt. Svärdet, svängdt af en jätte,
skulle tillintetgöra jättevärlden. Det ödet hade Valand smidt in i dess
klinga. Himlahvalf efter himlahvalf rämnar. Fjällen, som i Surts dalar
höllo djupens lågor fängslade, brista; flammorna omhvärfva slagfältet,
tillintetgöra jättarnes, vidundrens och de fördömdes skaror och slå upp
genom de sprängda himlarne. Genom rök och lågor rida Vidar, Vale och
Tors söner Mode och Magne bort från det brinnande slagfältet och in i
underjordens lycksalighetsrike till Mimers lund, dit döden och
förgängelsen icke nå.

Den gamla syndbefläckade jorden, härjad först af fimbulvintern, sen af
Surtbranden, sjunker i hafvet och upplöser sig i slagg och aska. Lågorna
slockna. Luften är renad af dem och hvälfver sig klarare blå än
någonsin. Och upp ur hafvet stiger en annan jord med härlig grönska. Det
är Mimers och Urds lycksalighetsriken. Det är de tre världskällornas
jord med Mimers lund och Breidablik, bostaden för Balder, Nanna, Had och
Leiftraser och Lif, som skola varda föräldrarna till den nya
världsålderns människoslägte. Forsar falla från Nidafjällen, som nu
öfverspännas af en högre himmel och öfver dem flyger, spanande efter
fisk, det örnpar,
\protect\hypertarget{lb1625905.xhtmlux5cux23start197}{}{}\protect\hypertarget{lb1625905.xhtmlux5cux23start197-a}{}{}\protect\hypertarget{lb1625905.xhtmlux5cux23start197-b}{}{}\protect\hypertarget{lb1625905.xhtmlux5cux23start197-c}{}{}\protect\hypertarget{lb1625905.xhtmlux5cux23start197-d}{}{}
som, i likhet med många andra djurslag, blifvit i Mimers rike förvaradt
undan förstörelsen.

Gudar återfinna hvarandra på Idaslätten. Höner, Vidar, Vale, Mode och
Magne och Mimers söner samlas där kring Balder, Nanna och Had. De
samtala om det härliga världsträdet, som öfverlefvat förödelsen och
skall blomstra i förnyad kraft, ty visdomskällan har åter stigit till
sin brädd, och Mimers söner skola vårda det. De påminna hvarandra om
alla öden de genomlefvat och om urtidens heliga runor. Och i Idaslättens
oförvissneliga gräs återfinna de det underbara tafvelspel, hvarmed
gudarne lekte i tidens morgon. Sorgfrihetens dagar hafva återvändt.

På den nyuppståndna jorden skola osådda åkrar bära skördar, och dygdiga
slägten samlas till oförgänglig lycka i Gimles guldtäckta salar, som
skina härligare än solen.

Vid kedjan af de heliga sånger, som diktades i urtiden och blifvit ärfda
från slägte till slägte, har således nornan fäst det dyrbaraste af alla
smycken, då hon sjöng, att

{»all brist skall bättras}\\
{och Balder komma.»}\\

Ja, hon har siat, att en mäktigare än han skall komma, den Gud, hvars
tjenarinna Urd är, han, hvars ande genom visdomsbrunnen mängde sig in i
skapelsen, en allsmäktig Gud och en fridens Gud tillika, som då

{»stadgar en dyrkan,}\\
{som stånda skall.»}\\


\bye
