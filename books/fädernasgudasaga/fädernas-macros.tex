% Macros för fädernas.tex

%%
%% Personal plain TeX macros.
%% Can only be used with LuaTeX, requires luaotfload.sty in a searchable path.
%%


%%%%%%%%% F O N T S %%%%%%%%%%%%%%%%%%%%%%%%%%%%%%%%%%%%%%%%%%%%%%%%%%%%%%%%%%%%

% Use font loader from ConTeXt
\input luaotfload.sty

% Font names

\font\twelverm 	"DanteMTStd-Regular":+onum;+pnum;+tlig; at 12pt
\font\twelveit 	"DanteMTStd-Italic" at 12pt
\font\twelvesc 	"DanteMTStd-Regular":+smcp;+c2sc;letterspace=8; at 12pt
\font\twelvetit "DanteMTStd-Regular":-titl;+upper;+lnum;letterspace=8; at 12.5pt
\font\twelvetitb "DanteMTStd-Regular":+onum;-titl;letterspace=8; at 12pt


\font\thirteenrm "DanteMTStd-Regular":+lnum;+pnum;letterspace=10; at 14pt
\font\thirteenit "DanteMTStd-Italic" at 13pt
\font\thirteensc "DanteMTStd-Regular":+onum;+pnum;+smcp;+c2sc;letterspace=10; at 13pt

\font\fourteentit "DanteMTStd-Regular":+onum;-titl;letterspace=10; at 14pt
\font\fourteensc "DanteMTStd-Regular":+smcp;+onum;letterspace=10; at 14pt
\font\fourteenit "DanteMTStd-Italic":+onum;+pnum; at 14pt

\font\fiftyfourtit "DanteMTStd-Regular":+titl at 54pt
\font\dcquotefont "DanteMTStd-Regular":+titl at 25pt

\def\dccapfont{\fiftyfourtit\red}

\def\secIfontA{\thirteenrm}
\def\secIfontB{\eighteenrm}

\def\secIIfont{\twelvetit}

% Semantic names for direct use

\def\normalsize{%
	\gdef\rm{\twelverm}%
	\gdef\it{\twelveit}%
	\gdef\sc{\twelvesc}%
	\gdef\tit{\twelvetit}%
}

% Use italics for emphasis
\def\emph{\it}

% Start with normalsize roman
\normalsize\rm


%%%%%%%%% P A R A G R A P H   S E T T I N G S %%%%%%%%%%%%%%%%%%%%%%%%%%%%%%%%%%

% Default distance between lines
\baselineskip=14.06pt

% Static height of first line
\topskip=\baselineskip

% No extra space after periods.
\frenchspacing

% Vertical distance between paragraphs
\parskip=0pt

% Paragraph indentation
\parindent=1em

% Lineskip & lineskiplimit
\lineskiplimit=-\maxdimen
\lineskip0pt

% Badness
\emergencystretch=0pt
\pretolerance=150
\tolerance=9999
\hbadness=150
\hfuzz=0pt

% \firstnoindent -- Don't indent next paragraph
\def\firstnoindent{\global\everypar={\wipeeverypar\setbox7=\lastbox}}
\def\wipeeverypar{\global\everypar={}}

% No penalties for widows and orphans, these are dealt with using \adjustpage
\widowpenalty=0
\clubpenalty=0

%%%%%%%%% P A G E   L A Y O U T %%%%%%%%%%%%%%%%%%%%%%%%%%%%%%%%%%%%%%%%%%%%%%%%

% Paper size
\pagewidth=150mm	
\pageheight=225mm

% Undo legacy 1in origin
\pdfvariable horigin 0in	
\pdfvariable vorigin 0in

% Typeset area distance from top left corner
\hoffset=\dimexpr(\pagewidth/9)\relax	
\voffset=\dimexpr(\pageheight/9)\relax
\advance\voffset by -\topskip
\advance\voffset by 1ex


% Ensure that first baseline starts at the same distance from the top
\topskip=\baselineskip % I had an idea with this but forgot, leaving it in for now.

% Typeset area size
\hsize=\dimexpr(6\pagewidth/9)
\vsize=\topskip \advance\vsize by 30\baselineskip

% Backup original vsize 
\newdimen\originalvsize \originalvsize=\vsize

% Defines a macro \ap:<pageno>, expands to how many lines should be adjusted.
\def\adjustpage#1#2{%
	\expandafter\xdef\csname ap:#1\endcsname{#2}
	\ifnum #1=1
		\setvsize
	\fi
}
% Example: \adjustpage{15}{+2} adjusts page 15 to have 2 extra lines

% This macro gets expanded last in the output routine, don't use directly.
\def\setvsize{%
	\global\vsize=\originalvsize
	\ifcsname ap:\the\pageno\endcsname
		\global\advance \vsize by \csname ap:\the\pageno\endcsname \baselineskip
	\fi
}


%%%%%%%%% H E A D E R %%%%%%%%%%%%%%%%%%%%%%%%%%%%%%%%%%%%%%%%%%%%%%%%%%%%%%%%%%

%\def\makeheadlineplain{%
%	\vbox to0pt{%
%		\vskip-22.5pt
% 		\line{%
%  			\vbox to8.5pt{}%
%  			\the\headline
%  		}%
%  		\vss
%  	}
%	\nointerlineskip
%}

\def\makeheadline{%
    \vbox to0pt{%
        \vss
        \hbox to\hsize{%
            \the\headline
        }%
        \vskip-\topskip
        \vskip0.5\baselineskip % raise or lower from default
        \hbox{}%
    }%
    \nointerlineskip
}

\newif\ifheadline \headlinetrue

\headline{%
	\ifheadline
		\ifodd\pageno
			\hss \normalsize\sc \defaultcolor 
			\lowercase{Name of the book} \hss% recto
		\else
			\hss \normalsize\sc \defaultcolor 
			\lowercase{Author Authorsson}\hss% verso	
		\fi
	\else
		\hss
		\global\headlinetrue	
	\fi
}

\headline{}

% Turn off headline on current page
\def\thispagenoheadline{\global\headlinefalse}


%%%%%%%%% F O O T E R %%%%%%%%%%%%%%%%%%%%%%%%%%%%%%%%%%%%%%%%%%%%%%%%%%%%%%%%%%

\def\makefootline{%
	\baselineskip=14pt
	\lineskiplimit=0pt
	\line{%
		\the\footline
	}%
}

\newif\iffootline \footlinetrue

\footline{%
	\iffootline
		\ifodd\pageno
			\hss{\normalsize\rm \defaultcolor \folio\quad}% recto
		\else
			{\quad \normalsize\rm \defaultcolor \folio}\hss% verso	
		\fi
	\else
		\hss
		\global\footlinetrue	
	\fi
}

% Turn off footline on current page
\def\thispagenofootline{\global\footlinefalse}


%%%%%%%%%% C O L O R S %%%%%%%%%%%%%%%%%%%%%%%%%%%%%%%%%%%%%%%%%%%%%%%%%%%%%%%%%

% Token for \aftergroup to reset color
\def\colorstackpop{\pdfextension colorstack 0 pop}

% Switch to CMYK color in current group. Expects four values between 0 and 1
\def\setcmykcolor#1{%
   	\pdfextension colorstack 0 push {#1 k #1 K}% 
   	\aftergroup\colorstackpop
}

% Switch to RGB color in current group. Expects three values between 0 and 1
\def\setrgbcolor#1{%
   	\pdfextension colorstack 0 push {#1 rg #1 RG}% 
   	\aftergroup\colorstackpop
}

% Color names
\def\black{\setcmykcolor{0 0 0 1}}
\def\blue{\setcmykcolor{1 1 0 0}}
\def\red{\setrgbcolor{0.8 0 0}}

% Semantic names for direct use
\def\defaultcolor{\black}
\def\hyperlinkcolor{\blue}
\def\rubricationcolor{\red}

% Example: {\hyperlinkcolor This is blue text}


%%%%%%%%% I M A G E S %%%%%%%%%%%%%%%%%%%%%%%%%%%%%%%%%%%%%%%%%%%%%%%%%%%%%%%%%%

\newdimen\imagewidth    \imagewidth=0pt 	% 0pt gives natural size of image
\newdimen\imageheight   \imageheight=0pt
\newtoks\imagedir		\imagedir{} 		% Must end with /, e.g. mypics/

\def\image#1{%
	\hbox{%
		\saveimageresource 
			\ifdim\imagewidth=0pt \else width\imagewidth\fi 
            \ifdim\imageheight=0pt \else height\imageheight\fi 
            {\the\imagedir#1}%
      	\useimageresource\lastsavedimageresourceindex
	}%
}

% Example: {\imageheight=4cm \image{mypicture.png}}

%%%%%%%%% O U T P U T   R O U T I N E %%%%%%%%%%%%%%%%%%%%%%%%%%%%%%%%%%%%%%%%%%



\output{\beforeoutput \plainoutput \afteroutput}

% For two page layout, \shiftoffset is how much verso pages should move
\newdimen\shiftoffset
\shiftoffset = \pagewidth
\advance \shiftoffset by -\hsize
\advance \shiftoffset by -2\hoffset

\def\beforeoutput{%
	% Two page layout
	\ifodd\pageno
	\else
		\advance \hoffset by \shiftoffset
	\fi
}

\def\prepghook{\vbox to0pt{\kern-\voffset 
   \hbox to0pt{\kern-\hoffset\background\hss}\vss}% 
   \nointerlineskip 
} 

\def\background{%
    \saveimageresource width\pagewidth height\pageheight {grid.pdf}%
    \useimageresource\lastsavedimageresourceindex
}

% Turn off background image
\def\background{}

\def\plainoutput{\shipout\vbox{\prepghook\makeheadline\pagebody\makefootline}%
  \advancepageno
  \ifnum\outputpenalty>-20000 \else\dosupereject\fi}

\def\afteroutput{%
	% Manually overriding vsize to solve widows through \adjustpage
	\setvsize
}

%%%%%%%%% S E C T I O N I N G %%%%%%%%%%%%%%%%%%%%%%%%%%%%%%%%%%%%%%%%%%%%%%%%%%

\def\secI#1#2{%
	\vfil\break
	\ifodd\pageno
	\else
		\null\thispagenofootline
		\vfill\eject
	\fi
	\centerline{\noindent\fourteensc \lowercase{#1}}
	\vskip0.8\baselineskip
	\centerline{\noindent\fourteentit \uppercase{#2}}
	\vskip2.2\baselineskip
}


\def\secII#1#2{%
	\vskip1\baselineskip
	%\vskip1ex
	{\twelvetit\red%\secIIfont\red
	 \noindent 
	 \if 😀#1😀\else
	 	Kapitel #1\space\char"00B7 \space
	 \fi
	 \noindent #2\par%
	}%
	%\vskip1\baselineskip
	%\vskip-1ex
	\firstnoindent
}

%\def\secII#1#2{%
%	\vskip\baselineskip
%	%\vskip-1ex
%	{\twelvetit\red
%	 \noindent 
%	 \if 😀#1😀\else
%	 	Kapitel #1\space$\cdot$\space
%	 \fi
%	 \noindent #2\par%
%	}%
%	%\vskip1\baselineskip
%	%\vskip1ex
%	\firstnoindent
%}

%%%%%%%%% T A B L E   O F   C O N T E N T S %%%%%%%%%%%%%%%%%%%%%%%%%%%%%%%%%%%%

\def\ToCPartLine#1{%
	\vskip0.5\baselineskip
    {\twelvesc\uppercase{#1}}
    \vskip0.5\baselineskip
}

% Fake page numbers for now
\newcount\testpg \testpg=1

\def\TocChapLine#1#2{{%
	\advance\leftskip by 2em
	\noindent\leavevmode
	\rm
	\llap{%
		\hbox to 1em{\hfil #1}\quad
	}%
	\global\advance\testpg by 3\relax
	#2\hfill \the\testpg\par
}}

%%%%%%%%% D R O P C A P S %%%%%%%%%%%%%%%%%%%%%%%%%%%%%%%%%%%%%%%%%%%%%%%%%%%%%%

\def\dropcapD{%
\vskip2\baselineskip
\smash{%
    \hskip-1pt
    \raise 21pt \llap{\dcquotefont \hskip0pt}%
    {\dccapfont D}%
}%
\vskip-3\baselineskip
\dimen0=39pt
\dimen1=43pt
\dimen2=43pt
\parshape 4
    \dimen0 \dimexpr(\hsize-\dimen0)
    \dimen1 \dimexpr(\hsize-\dimen1)
    \dimen2 \dimexpr(\hsize-\dimen2)
    0pt \hsize
\noindent
}

\def\dropcapF{%
\vskip2\baselineskip
\smash{%
    \hskip-1.5pt
    \raise 21pt \llap{\dcquotefont \hskip0pt}%
    {\dccapfont F}%
}%
\vskip-3\baselineskip
\dimen0=29pt
\dimen1=32pt
\dimen2=32pt
\parshape 4
    \dimen0 \dimexpr(\hsize-\dimen0)
    \dimen1 \dimexpr(\hsize-\dimen1)
    \dimen2 \dimexpr(\hsize-\dimen2)
    0pt \hsize
\noindent
}

\def\dropcapN{%
\vskip2\baselineskip
\smash{%
    \hskip-1pt
    \raise 21pt \llap{\dcquotefont \hskip0pt}%
    {\dccapfont N}%
}%
\vskip-3\baselineskip
\dimen0=42pt
\dimen1=45pt
\dimen2=45pt
\parshape 4
    \dimen0 \dimexpr(\hsize-\dimen0)
    \dimen1 \dimexpr(\hsize-\dimen1)
    \dimen2 \dimexpr(\hsize-\dimen2)
    0pt \hsize
\noindent
}

\def\dropcapO{%
\vskip2\baselineskip
\smash{%
    \hskip-2pt
    \raise 21pt \llap{\dcquotefont \hskip0pt}%
    {\dccapfont O}%
}%
\vskip-3\baselineskip
\dimen0=38pt
\dimen1=45pt
\dimen2=45pt
\parshape 4
    \dimen0 \dimexpr(\hsize-\dimen0)
    \dimen1 \dimexpr(\hsize-\dimen1)
    \dimen2 \dimexpr(\hsize-\dimen2)
    0pt \hsize
\noindent
}

\def\dropcapS{%
\vskip2\baselineskip
\smash{%
    \hskip-2pt
    \raise 21pt \llap{\dcquotefont \hskip0pt}%
    {\dccapfont S}%
}%
\vskip-3\baselineskip
\dimen0=23pt
\dimen1=29pt
\dimen2=29pt
\parshape 4
    \dimen0 \dimexpr(\hsize-\dimen0)
    \dimen1 \dimexpr(\hsize-\dimen1)
    \dimen2 \dimexpr(\hsize-\dimen2)
    0pt \hsize
\noindent
}

\def\dropcapT{%
\vskip2\baselineskip
\smash{%
    \hskip-2pt
    \raise 21pt \llap{\dcquotefont \hskip0pt}%
    {\dccapfont T}%
}%
\vskip-3\baselineskip
\dimen0=33.5pt
\dimen1=33.5pt
\dimen2=33.5pt
\parshape 4
    \dimen0 \dimexpr(\hsize-\dimen0)
    \dimen1 \dimexpr(\hsize-\dimen1)
    \dimen2 \dimexpr(\hsize-\dimen2)
    0pt \hsize
\noindent
}

\def\dropcapU{%
\vskip2\baselineskip
\smash{%
    \hskip-5pt
    \raise 21pt \llap{\dcquotefont \hskip0pt}%
    {\dccapfont U}%
}%
\vskip-3\baselineskip
\dimen0=34pt
\dimen1=36pt
\dimen2=36pt
\parshape 4
    \dimen0 \dimexpr(\hsize-\dimen0)
    \dimen1 \dimexpr(\hsize-\dimen1)
    \dimen2 \dimexpr(\hsize-\dimen2)
    0pt \hsize
\noindent
}


